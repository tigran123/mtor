\documentclass[12pt]{book}
\usepackage{mtor}
\begin{document}

\MainMatter

\thispagestyle{empty}
\BookMark{0}{Title Page}
\vspace*{1ex}
\begin{center}
\parttitlefont
\booktitlefontsize THE MATHEMATICAL THEORY\\
\medskip
\large OF\\
\medskip
\booktitlefontsize RELATIVITY\\
\bigskip
\titlefont
\normalsize BY \\
\authortitlefontsize A. S. EDDINGTON, M.A., \textsc{M.Sc.}, F.R.S.\\
\medskip
\scriptsize PLUMIAN PROFESSOR OF ASTRONOMY AND EXPERIMENTAL\\
PHILOSOPHY IN THE UNIVERSITY OF CAMBRIDGE\\
\vspace*{\stretch{0.1}}
{\normalsize\itshape With a Foreword by Tigran Aivazian}\\
\vspace*{\stretch{0.1}}
\small THIRD EDITION\\
\vspace*{\stretch{0.4}}
\titlesepbig\\[2ex]
\Large Bibles.org.uk, 2022\\
\end{center}
\newpage\thispagestyle{empty}
\vspace*{\stretch{0.3}}
\hfill
\parbox{8.5cm}{%
\begin{center}
\textit{First Edition, Cambridge University Press}\hfill1923\\
\medskip
\textit{Second Edition, Cambridge University Press}\hfill1924\\
\medskip
\textit{Third Edition, Bibles.org.uk}\hfill2022\\
\end{center}
}
\hfill
\vspace*{\stretch{0.7}}
\begin{center}
Please send all comments to: {\makeatletter\upshape\bfseries aivazian.tigran@gmail.com\makeatother}\\[1ex]
\tux\ Typeset with \XeLaTeX\ of \TeX\ Live 2021 under Linux.\\
PDF version/date: \tunemarkup{pgnexus10}{\textcolor{red}{Colour} 10"}\tunemarkup{pgauraone}{B\&W 8"}, \textbf{\today{}}\\
Source Code: \myurl{https://github.com/tigran123/mtor}
\end{center}

\TableofContents

\Matter{Editor's Foreword}

\lettrine{\textcolor{lettrinecolour}{I}}{s} it right and proper today, in the year 2022 A.D., standing as we are on the brink of the
\emph{New Epoch} of brotherhood of all mankind,
to publish a monograph on Special and General Relativity, written by a physicist,
who flourished nearly a century ago?
It is indeed.
Sir Arthur Stanley Eddington (1882--1944) was not ``merely'' the father of modern stellar astrophysics, who started with the
scraps of uncoordinated patchwork of guesses about the internal constitution of stars, which only marginally
differed from the idle speculations of antiquity, and proceeded to build the entire body of
observationally verified knowledge, which serves as the basis of all stellar astronomy today.
Neither was he ``merely'' a lone genius physicist, who for the first time in history predicted
theoretically the value of the mass ratio of protons and electrons, as well as that of the fine
structure constant---a feat standing unrivalled to this day.
No, he was, first and foremost, the greatest \emph{God\hyp{}knowing} physicist who ever lived on this planet.

Theoretical physics of the XX century, alas, was not immune to the dangers besetting the economico\hyp{}political
structure of the world at the time.
As is usually common in the poisoned atmosphere permeated by the ``spirit of democracy'',
which, as we know, fosters the election of base and ignorant rulers and glorifies mediocrity by placing the
universal suffrage in the hands of uneducated and indolent majorities, the science was forced to pursue the false
materialistic goals of a few mediocre (relatively speaking) men like Niels Bohr and Max Born, forgetting for a season
the direction of the brilliant contributions made by the real lovers of truth---men like Arthur Eddington,
James Jeans, Louis de Broglie in theoretical physics,
Robert Millikan in experimental physics, William Sadler in medicine and psychology and others.
And so, through its neglect of unbiased approach to the truth, theoretical physics has wallowed
in the mire of purely materialistic attempts of the ``interpretation of Quantum Mechanics'',
which inevitably led to the greatest crisis since the days of Max Planck in 1900.
This crisis is indicated by the fact that since 1930s there have been no major discoveries in the field of
theoretical physics---the ``progress'' usually boasted of consisting almost entirely of technological advances.
No new substantial knowledge of the material reality was attained since the advent of General Relativity (1915)
and Quantum Mechanics (1927) and because of this we now have no choice but to go back to those glorious days
of the true discoveries and peruse the words uttered by the truth discoverers.

The chief value today of the work of men like Eddington and Jeans is revealed by the recently established
(see~\cite{Block1} and~\cite{Aivazian1}) fact, that these people happened to be the pivotal individuals
in the synthesis of all humanly accessible factual knowledge as well as the highest spiritual strivings,---
the synthesis, which culminated in the publication of the \emph{Fifth Epochal Revelation,} known also as
\emph{The Urantia Papers} \cite{Aivazian1}.
The very thoughts of those men, as well as their printed words, were used in compiling and presentation of this
great Revelation, which undoubtedly will for a long time serve universally as the common basis of
elementary education for all peoples and nations destined to survive into the \emph{New Epoch}.
It devolves upon them to build the new civilisation on the ruins made by those who arrived at an impasse,
which is an inexorable outcome of evolutionary processes, whenever beauty is substituted with ugliness,
truth with falsehood and goodness with evil and selfish deceit.

The monumental 1928--1939 trilogy on the philosophy of science (\cite{Eddington1}, \cite{Eddington2}, \cite{Eddington3}),
written by A.S~Eddington, if re\hyp{}issued today, might be bound in a single volume and furnished with
a not inappropriate title ``Becoming a Spirit''.
Having studied practically all of Eddington's published books and papers, as well as most books written about him,
I was almost tempted to write such a book myself.
However, being admonished by Ecclesiastes~12:12 (\emph{``of making many books there is no end''}) I chose to exercise the
Editor's prerogative and attempted to make a clarification and qualification of the essence of Eddington's
fundamental research in this Foreword.

A flesh and bones mortal, speaking authoritatively on the spirit state of being or how to become one, will no doubt
attract at least the following two kinds of caustic criticisms.

The first critic is an unbelieving and unfriendly type, who, on the first hearing of the word ``spirit''
can think of nothing else but the desire to touch you and see if you are made of flesh and bones.
And upon such ``empirical confirmation'' he will gleefully declare you an impostor, who violated the,
in his opinion immutable, authority of the Scriptures as expressed in Luke~24:35 (\emph{``\ldots\ for a spirit hath not flesh and bones''}).
He is quick to ridicule with disdain the very notion of a possibility for a mortal to assert anything useful on the ``matters spiritual''
and no matter how hard he tries, every possible formulation of his objections will betray how alien are the things of the spirit
to the wretched and doomed mind subservient to matter.
To such a critic I have nothing to answer, except---``come and see!''

The second type is a learned and cautious scholar, who first studies the matter diligently, albeit not without deeply ingrained
preconceptions of the scope and utility of the ``scientific method'', and then proceeds to make the following tactful enquiry:
``We know of the reality of the spirit world and, moreover, that the Universe is basically spiritual, the material aspect
thereof being merely a shadow of the more abiding spirit reality.
But may we point out that during the long career of the ascent of a typical evolutionary mortal of animal origin, the attainment
of even the first stage of spirit is only possible after the sojourn on the mansion worlds and on the other spheres of the morontia
realm, that being the intermediate stage between matter and spirit?
Would it not, therefore, be a little premature to speak of this `becoming a spirit' now, while we are not yet even morontia beings,
let alone spiritual?''
Such honest doubts and sincere questionings are not to be despised and are not evil \Foreign{per se}.
Though they may delay the progressive journey towards perfection attainment, they can never inhibit it.
What follows may be considered an answer to such a friendly critic.

Those who see in Eddington merely a physicist---even a genius---completely miss both the potential goal of his life and the
actually and literally attained levels of spirit insight as revealed in his writings.
I admit that I myself, at first, had just such an attitude when I approached the task of studying Eddington's works more than
a decade ago.
There is a certain preordained path of studying one's environment and what makes the contribution of Eddington (and a few others)
unique is that he has walked this path to the triumphant end. And the path is this---

\emph{Analysis of the material reality can be pursued until it disappears from the sensory mechanism, yet remains real to the mind.
One can then continue this analysis by ``mind alone'' and arrive at the epistemological basis of the fundamental laws of science,
but at some point the reality placed under scruity vanishes to the (material or morontia) mind also and yet remains perceptible to the insight
of the spirit in the form of the supreme values of entirely spiritual nature.
And this is the meaning in which it can be said, that ``the Universe is basically spiritual''.}

Well, the critic says, this may all be well and good, relating strictly to the abovementioned three philosophical books,
but what does it have to do with the purely technical monograph on General Relativity that is supposed to be discussed here?
The division of knowledge into \emph{technical} and \emph{non\hyp{}technical} along the lines this is done today will become obsolete
in the \emph{New Epoch} of brotherhood.
Not because the distinction will be blurred and the two kinds of knowledge somehow merge into one, but for precisely the opposite reason:
there is a clear demarcation line between the \emph{structural} knowledge and the knowledge of the \emph{substance}.
The so\hyp{}called \emph{technical} knowledge, that is expressible in mathematical language, is nothing other than the knowledge
of \emph{structure,} i.e. of the relations between entities, and is entirely abstracted from the knowledge of the nature or essence of
the entities themselves,---their substance.
Understanding this fundamental difference neither precludes the possibility nor obviates the need for specialists, who
are skilled in one particular type of knowledge, more than in others, but it certainly does remove all illusion of
self\hyp{}sufficiency, which was giving origin to the arrogance on the part of members of one profession towards the others.
And who is better qualified to teach us about one particular kind of knowledge, than the one who has discovered that this
is not the \emph{only} kind and that there are others?

The possibility of dissecting the two types of knowledge should not be considered as something ephemeral, because it has immediate practical implications.
In fact, the very reason I moved to the United Kingdom in 1994 (from Armenia, then in a state of ruin after the destruction of Soviet Union) was due
to one such application, made by myself (then a postgraduate theoretical physicist) independently and many years \emph{prior} to learning about Eddington's research.
Namely, I have attempted and successfully performed a dissection of the formalism of Quantum Mechanics into ``information dynamics'' (or \emph{infodynamics}) and the specific features
of the microscopic world---these two being logically independent.
Having separated the purely \emph{infodynamical} aspect of Quantum Mechanics as a self\hyp{}consistent set of rules for manipulation of structural knowledge, I then proceeded
to apply it to a completely unrelated domain of human activity---economics.
Given the functions of demand $D(x,t)$ and supply $S(x,t)$, which regulate the actual price of a commodity $x(t)$ according to $\dot x = D(x,t) - S(x,t)$ I constructed
a ``price momentum'' variable $p(t)$ conjugated to $x(t)$ and the corresponding $p$-linear Hamiltonian, which upon canonical quantisantion yielded a model for prediction of the evolution
of the \emph{probability distribution} $|\psi(x,t)|^2$ of $x$ as opposed to the ``classical'' actual \emph{fixed value} of $x$ at the moment of time $t$.
Moreover, this model was free of all artificial constants (i.e. the ``Planck constant'' equivalent) due to the $p$-linearity of the Hamiltonian.
I have also pointed out that this approach is different from the well\hyp{}known ``Quantum Economics'' due to John von~Neumann.
The resulting ``New Quantum Economics'' scheme was described in a very brief paper, which I called ``On the New Method of Price Forecasting'' and showed it to the two economists from the UK,
who visited Armenia and they were sufficiently impressed to invite me to continue my studies here in the UK.


I was pleasantly surprised when I discovered that a newly typeset version of Eddington's
``The Mathematical Theory of Relativity'' has been produced recently by Andrew~D.~Hwang,
a professor of mathematics at College of the Holy Cross, who made the fruits of his labours of love
available under Public Domain as part of the \emph{Gutenberg Project} \cite{Hwang1}.
However, seeing that Dr~Hwang used the early edition (1923) of this excellent book as the basis of his work,
I have decided to update his sources to the latest edition.
During the course of editing, I decided to incorporate the material of the \emph{Supplementary Notes} into the
body of the main text.
This is similar to the way it is done in the Russian translation of this book as published in 1934.
In the previous English editions this material was delegated to the end in order to preserve the pagination
of the rest of the book.
As there is obviously no need to preserve the old pagination in a freshly retypeset modern edition,
moving the material into the main body of text seemed desirable to prevent interrupting the reading flow.

It is my hope that this edition will help those who wish to understand the new and revolutionary concepts of
time and space as contained in both the General Relativity of Albert Einstein and in its unification with the
electromagnetism by Hermann Weyl, masterfully presented by A.S.~Eddington.

\Signature{Tigran Aivazian.}{6 \emph{March} 2022.}

\begin{thebibliography}{100}
\bibitem{Block1}
Matthew~Block.
{``The Urantia Book Sources'' (online).}
{\myurl{https://urantiabooksources.com}}
\bibitem{Aivazian1}
Tigran~Aivazian.
{\em ``The British Study Edition of the Urantia Papers''.}
{Freely available at \myurl{http://www.bibles.org.uk/study-edition.html}}
\bibitem{Hwang1}
Andrew~D.~Hwang.
{\em ``The Mathematical Theory of Relativity'' by Sir Arthur Stanley Eddington.}
{Freely available at \myurl{https://www.gutenberg.org/ebooks/59248}}
\bibitem{Eddington1}
A.S.~Eddington.
{\em ``The Nature of the Physical World''.}
{Cambridge University Press,} 1928.
\bibitem{Eddington2}
A.S.~Eddington.
{\em ``New Pathways in Science''.}
{Cambridge University Press,} 1935.
\bibitem{Eddington3}
A.S.~Eddington.
{\em ``The Philosophy of Physical Science''.}
{Cambridge University Press,} 1939.
\end{thebibliography}

\Matter{Preface}

\lettrine{\textcolor{lettrinecolour}{A}}{first} draft of this book was published in 1921 as a mathematical supplement
to the French Edition of \Title{Space, Time and Gravitation}. During
the ensuing eighteen months I have pursued my intention of developing it
into a more systematic and comprehensive treatise on the mathematical
theory of Relativity. The matter has been rewritten, the sequence of the argument
rearranged in many places, and numerous additions made throughout;
so that the work is now expanded to three times its former size. It is hoped
that, as now enlarged, it may meet the needs of those who wish to enter fully
into these problems of reconstruction of theoretical physics.

The reader is expected to have a general acquaintance with the less
technical discussion of the theory given in \Title{Space, Time and Gravitation,}
although there is not often occasion to make direct reference to it. But it is
eminently desirable to have a general grasp of the revolution of thought
associated with the theory of Relativity before approaching it along the
narrow lines of strict mathematical deduction. In the former work we explained
how the older conceptions of physics had become untenable, and traced
the gradual ascent to the ideas which must supplant them. Here our task is
to formulate mathematically this new conception of the world and to follow
out the consequences to the fullest extent.

The present widespread interest in the theory arose from the verification
of certain minute deviations from Newtonian laws. To those who are still
hesitating and reluctant to leave the old faith, these deviations will remain
the chief centre of interest; but for those who have caught the spirit of the
new ideas the observational predictions form only a minor part of the subject.
It is claimed for the theory that it leads to an understanding of the world of
physics clearer and more penetrating than that previously attained, and it
has been my aim to develop the theory in a form which throws most light
on the origin and significance of the great laws of physics.

It is hoped that difficulties which are merely analytical have been minimised
by giving rather fully the intermediate steps in all the proofs with
abundant cross\hyp{}references to the auxiliary formulae used.

For those who do not read the book consecutively attention may be called
to the following points in the notation. The summation convention (\SecRef{22})
is used. German letters always denote the product of the corresponding
English letter by~$\sqrt{-g}$ (\SecRef{49}). $\Ham$~is the symbol for ``Hamiltonian differentiation''
introduced on \SecRef{60}. An asterisk is prefixed to symbols generalised
so as to be independent of or covariant with the gauge (\SecRef{86}).

A selected list of original papers on the subject is given in the Bibliography
at the end, and many of these are sources (either directly or at
second\hyp{}hand) of the developments here set forth. To fit these into a continuous
chain of deduction has involved considerable modifications from their
original form, so that it has not generally been found practicable to indicate
the sources of the separate sections. A frequent cause of deviation in treatment
is the fact that in the view of most contemporary writers the Principle
of Stationary Action is the final governing law of the world; for reasons
explained in the text I am unwilling to accord it so exalted a position. After
the original papers of Einstein, and those of de~Sitter from which I first
acquired an interest in the theory, I am most indebted to Weyl's \Title{Raum, Zeit,
Materie}. Weyl's influence will be especially traced in \SecRefs{49}, \SecNum{58}, \SecNum{59}, \SecNum{61}, \SecNum{63}, as
well as in the sections referring to his own theory.

I am under great obligations to the officers and staff of the University
Press for their help and care in the intricate printing.

\Signature{A. S. E.}{10 \emph{August} 1922.}

\Matter{Introduction}

\index{Differentiation|seealso{Derivative}}
\index{Distance|see{Length}}
\index{Gravitation|seealso{Einstein's law}}
\index{Ponderomotive force|see{Mechanical force}}
\index{Proper-|see{Invariant mass \emph{and} Density}}

\lettrine{\textcolor{lettrinecolour}{T}}{he} subject of this mathematical treatise is not pure mathematics but
\index{Mathematics contrasted with physics}%
physics. The vocabulary of the physicist comprises a number of words such
as length, angle, velocity, force, work, potential, current, etc., which we shall
\index{Length!definition of}%
call briefly ``physical quantities.'' Some of these terms occur in pure mathematics
\index{Physical quantities}%
also; in that subject they may have a generalised meaning which does
not concern us here. The pure mathematician deals with ideal quantities
defined as having the properties which he deliberately assigns to them. But
in an experimental science we have to discover properties not to assign them;
and physical quantities are defined primarily according to the way in which
we recognise them when confronted by them in our observation of the world
around us.

Consider, for example, a length or distance between two points. It is
a numerical quantity associated with the two points; and we all know the
procedure followed in practice in assigning this numerical quantity to two
points in nature. A definition of distance will be obtained by stating the
exact procedure; that clearly must be the primary definition if we are to
make sure of using the word in the sense familiar to everybody. The pure
mathematician proceeds differently; he defines distance as an attribute of
the two points which obeys certain laws---the axioms of the geometry which
he happens to have chosen---and he is not concerned with the question how
this ``distance'' would exhibit itself in practical observation. So far as his own
investigations are concerned, he takes care to use the word self\hyp{}consistently;
but it does not necessarily denote the thing which the rest of mankind are
accustomed to recognise as the distance of the two points.

To find out any physical quantity we perform certain practical operations
followed by calculations; the operations are called experiments or observations
according as the conditions are more or less closely under our control. The
physical quantity so discovered is primarily the result of the operations and
calculations; it is, so to speak, \emph{a manufactured article}---manufactured by
\index{Manufacture of physical quantities}%
our operations. But the physicist is not generally content to believe that the
quantity he arrives at is something whose nature is inseparable from the kind
of operations which led to it; he has an idea that if he could become a god
contemplating the external world, he would see his manufactured physical
quantity forming a distinct feature of the picture. By finding that he can
lay $x$~unit measuring\hyp{}rods in a line between two points, he has manufactured
the quantity~$x$ which he calls the distance between the points; but he believes
that that distance~$x$ is something already existing in the picture of the world---a
gulf which would be apprehended by a superior intelligence as existing
in itself without reference to the notion of operations with measuring\hyp{}rods.
Yet he makes curious and apparently illogical discriminations. The parallax
of a star is found by a well-known series of operations and calculations; the
distance across the room is found by operations with a tape\hyp{}measure. Both
parallax and distance are quantities manufactured by our operations; but
for some reason we do not expect parallax to appear as a distinct element in
the true picture of nature in the same way that distance does. Or again,
instead of cutting short the astronomical calculations when we reach the
parallax, we might go on to take the cube of the result, and so obtain another
manufactured quantity, a ``cubic parallax.'' For some obscure reason we
expect to see distance appearing plainly as a gulf in the true world\hyp{}picture;
parallax does not appear directly, though it can be exhibited as an angle by
a comparatively simple construction; and cubic parallax is not in the picture
at all. The physicist would say that he \emph{finds} a length, and \emph{manufactures} a
cubic parallax; but it is only because he has inherited a preconceived theory
of the world that he makes the distinction. We shall venture to challenge
this distinction.

Distance, parallax and cubic parallax have the same kind of potential
existence even when the operations of measurement are not actually made---\emph{if}
you will move sideways you will be able to determine the angular shift, \emph{if}
you will lay measuring\hyp{}rods in a line to the object you will be able to count
their number. Any one of the three is an indication to us of some existent
condition or relation in the world outside us---a condition not created by our
operations. But there seems no reason to conclude that this world\hyp{}condition
\emph{resembles} distance any more closely than it resembles parallax or cubic
parallax. Indeed any notion of ``resemblance'' between physical quantities
and the world\hyp{}conditions underlying them seems to be inappropriate. If the
length~$AB$ is double the length~$CD$, the parallax of~$B$ from~$A$ is half the parallax
of~$D$ from~$C$; there is undoubtedly some world\hyp{}relation which is different
for $AB$ and~$CD$, but there is no reason to regard the world\hyp{}relation of~$AB$ as
being better represented by double than by half the world\hyp{}relation of~$CD$.

The connection of manufactured physical quantities with the existent
world\hyp{}condition can be expressed by saying that the physical quantities are
\emph{measure\hyp{}numbers} of the world\hyp{}condition. Measure\hyp{}numbers may be assigned
\index{Measure\hyp{}code}%
according to any code, the only requirement being that the same measure\hyp{}number
always indicates the same world\hyp{}condition and that different world\hyp{}conditions
receive different measure\hyp{}numbers. Two or more physical quantities
may thus be measure\hyp{}numbers of the same world\hyp{}condition, \emph{but in different
codes,} e.g.\ parallax and distance; mass and energy; stellar magnitude and luminosity.
The constant formulae connecting these pairs of physical quantities
give the relation between the respective codes. But in admitting that physical
quantities can be used as measure\hyp{}numbers of world\hyp{}conditions existing
independently of our operations, we do not alter their status as manufactured
quantities. The same series of operations will naturally manufacture the
same result when world\hyp{}conditions are the same, and different results when
they are different. (Differences of world\hyp{}conditions which do not influence
the results of experiment and observation are \Foreign{ipso facto} excluded from the
domain of physical knowledge.) The size to which a crystal grows may be a
measure\hyp{}number of the temperature of the mother\hyp{}liquor; but it is none the
less a manufactured size, and we do not conclude that the true nature of size
is caloric.

The study of physical quantities, although they are the results of our
\index{Physical quantities!definition of}%
own operations (actual or potential), gives us some kind of knowledge of the
world\hyp{}conditions, since the same operations will give different results in
different world\hyp{}conditions. It seems that this indirect knowledge is all that
we can ever attain, and that it is only through its influences on such operations
that we can represent to ourselves a ``condition of the world.'' Any
\index{Condition of the world}%
attempt to describe a condition of the world otherwise is either mathematical
symbolism or meaningless jargon. To grasp a condition of the world as
completely as it is in our power to grasp it, we must have in our minds a
symbol which comprehends at the same time its influence on the results of
all possible kinds of operations. Or, what comes to the same thing, we must
contemplate its measures according to all possible measure\hyp{}codes---of course,
without confusing the different codes. It might well seem impossible to
realise so comprehensive an outlook; but we shall find that the mathematical
calculus of tensors does represent and deal with world\hyp{}conditions precisely in
this way. A tensor expresses simultaneously the whole group of measure\hyp{}numbers
associated with any world\hyp{}condition; and machinery is provided for
keeping the various codes distinct. For this reason the somewhat difficult
tensor calculus is not to be regarded as an evil necessity in this subject, which
ought if possible to be replaced by simpler analytical devices; our knowledge
of conditions in the external world, as it comes to us through observation and
experiment, is precisely of the kind which can be expressed by a tensor and
not otherwise. And, just as in arithmetic we can deal freely with a billion
objects without trying to visualise the enormous collection; so the tensor
calculus enables us to deal with the world\hyp{}condition in the totality of its
aspects without attempting to picture it.

Having regard to this distinction between physical quantities and world\hyp{}conditions,
we shall not define a physical quantity as though it were a feature
in the world\hyp{}picture which had to be sought out. \emph{A physical quantity is
defined by the series of operations and calculations of which it is the result.}
The tendency to this kind of definition had progressed far even in pre\hyp{}relativity
physics. Force had become ``$\text{mass} \times \text{acceleration}$,'' and was no longer an invisible
agent in the world\hyp{}picture, at least so far as its definition was concerned.
Mass is defined by experiments on inertial properties, no longer as ``quantity
of matter.'' But for some terms the older kind of definition (or lack of
definition) has been obstinately adhered to; and for these the relativity
theory must find new definitions. In most cases there is no great difficulty
in framing them. We do not need to ask the physicist what conception
he attaches to ``length''; we watch him measuring length, and frame our
definition according to the operations he performs. There may sometimes be
cases in which theory outruns experiment and requires us to decide between
two definitions, either of which would be consistent with present experimental
practice; but usually we can foresee which of them corresponds to the ideal
which the experimentalist has set before himself. For example, until recently
the practical man was never confronted with problems of non\hyp{}Euclidean space,
and it might be suggested that he would be uncertain how to construct a
straight line when so confronted; but as a matter of fact he showed no
hesitation, and the eclipse observers measured without ambiguity the bending
of light from the ``straight line.'' The appropriate practical definition was so
obvious that there was never any danger of different people meaning different
loci by this term. Our guiding rule will be that a physical quantity must be
defined by prescribing operations and calculations which will lead to an
unambiguous result, and that due heed must be paid to existing practice;
the last clause should secure that everyone uses the term to denote the same
\emph{quantity,} however much disagreement there may be as to the \emph{conception}
attached to it.

When defined in this way, there can be no question as to whether the
operations give us the real physical quantity or whether some theoretical
correction (not mentioned in the definition) is needed. The physical quantity
is the measure\hyp{}number of a world\hyp{}condition in some code; we cannot assert
that a code is right or wrong, or that a measure\hyp{}number is real or unreal;
what we require is that the code should be the accepted code, and the measure\hyp{}number
the number in current use. For example, what is the real difference
of time between two events at distant places? The operation of determining
time has been entrusted to astronomers, who (perhaps for mistaken reasons)
have elaborated a regular procedure. If the times of the two events are found
in accordance with this procedure, the difference must be the real difference
of time; the phrase has no other meaning. But there is a certain generalisation
to be noticed. In cataloguing the operations of the astronomers, so as to
obtain a definition of time, we remark that one condition is adhered to in
practice evidently from necessity and not from design---the observer and his
apparatus are placed on the earth and move with the earth. This condition
is so accidental and parochial that we are reluctant to insist on it in our
definition of time; yet it so happens that the motion of the apparatus makes
an important difference in the measurement, and without this restriction the
operations lead to no definite result and cannot define anything. We adopt
what seems to be the commonsense solution of the difficulty. We decide that
time is \emph{relative to an observer}; that is to say, we admit that an observer on
another star, who carries out all the rest of the operations and calculations
as specified in our definition, is also measuring time---not our time, but a
time relative to himself. The same relativity affects the great majority of
elementary physical quantities\footnotemark;\footnotetext
  {The most important exceptions are number (of discrete entities), action, and entropy.\index{Absolute change!physical quantities@physical quantities|indexfn}}
the description of the operations is insufficient
to lead to a unique answer unless we arbitrarily prescribe a particular
motion of the observer and his apparatus.

In this example we have had a typical illustration of ``relativity,'' the
\index{Relativity of physical quantities}%
recognition of which has had far\hyp{}reaching results revolutionising the outlook
of physics. Any operation of measurement involves a comparison between
a measuring\hyp{}appliance and the thing measured. Both play an equal part in
the comparison and are theoretically, and indeed often practically, interchangeable;
for example, the result of an observation with the meridian circle
gives the right ascension of the star or the error of the clock indifferently,
and we can regard either the clock or the star as the instrument or the
object of measurement. Remembering that physical quantities are results of
comparisons of this kind, it is clear that they cannot be considered to belong
solely to one partner in the comparison. It is true that we standardise the
measuring appliance as far as possible (the method of standardisation being
explained or implied in the definition of the physical quantity) so that in
general the variability of the measurement can only indicate a variability of
the object measured. To that extent there is no great practical harm in
regarding the measurement as belonging solely to the second partner in
the relation. But even so we have often puzzled ourselves needlessly over
paradoxes, which disappear when we realise that the physical quantities are
not properties of certain external objects but are relations between these
objects and something else. Moreover, we have seen that the standardisation
of the measuring\hyp{}appliance is usually left incomplete, as regards the specification
of its motion; and rather than complete it in a way which would be
arbitrary and pernicious, we prefer to recognise explicitly that our physical
quantities belong not solely to the objects measured but have reference also
to the particular frame of motion that we choose.

The principle of relativity goes still further. Even if the measuring\hyp{}appliances
were standardised completely, the physical quantities would still
involve the properties of the constant standard. We have seen that the
world\hyp{}condition or object which is surveyed can only be apprehended in our
knowledge as the sum total of all the measurements in which it can be
concerned; any \emph{intrinsic} property of the object must appear as a uniformity
or law in these measures. When one partner in the comparison is fixed and
the other partner varied widely, whatever is common to all the measurements
may be ascribed exclusively to the first partner and regarded as an intrinsic
property of it. Let us apply this to the converse comparison; that is to say,
keep the measuring\hyp{}appliance constant or standardised, and vary as widely
as possible the objects measured---or, in simpler terms, make a particular
kind of measurement in all parts of the field. Intrinsic properties of the
measuring\hyp{}appliance should appear as uniformities or laws in these measures.
We are familiar with several such uniformities; but we have not generally
recognised them as properties of the measuring\hyp{}appliance. We have called
them \emph{laws of nature}!

The development of physics is progressive, and as the theories of the
external world become crystallised, we often tend to replace the elementary
physical quantities defined through operations of measurement by theoretical
quantities believed to have a more fundamental significance in the external
world. Thus the \Foreign{vis viva} $mv^2$, which is immediately determinable by experiment,
becomes replaced by a generalised energy, virtually defined by having
the property of conservation; and our problem becomes inverted---we have
not to discover the properties of a thing which we have recognised in nature,
but to discover how to recognise in nature a thing whose properties we have
assigned. This development seems to be inevitable; but it has grave drawbacks
especially when theories have to be reconstructed. Fuller knowledge
may show that there is nothing in nature having precisely the properties
assigned; or it may turn out that the thing having these properties has
entirely lost its importance when the new theoretical standpoint is adopted\footnotemark.\footnotetext
  {We shall see in \SecRef{59} that this has happened in the case of energy. The dead\hyp{}hand of a
  superseded theory continues to embarrass us, because in this case the recognised terminology
  still has implicit reference to it. This, however, is only a slight drawback to set off against the
  many advantages obtained from the classical generalisation of energy as a step towards the more
  complete theory.}%
\tunemarkup{pgnexus10}{\linebreak}When we decide to throw the older theories into the melting\hyp{}pot and make
a clean start, it is best to relegate to the background terminology associated
with special hypotheses of physics. Physical quantities defined by operations
of measurement are independent of theory, and form the proper starting\hyp{}point
for any new theoretical development.

Now that we have explained how physical quantities are to be defined,
the reader may be surprised that we do not proceed to give the definitions of
the leading physical quantities. But to catalogue all the precautions and
provisos in the operation of determining even so simple a thing as length, is
a task which we shirk. We might take refuge in the statement that the task
though laborious is straightforward, and that the practical physicist knows
the whole procedure without our writing it down for him. But it is better to
be more cautious. I should be puzzled to say off-hand what is the series of
operations and calculations involved in measuring a length of $10^{-15}$~cm.;
nevertheless I shall refer to such a length when necessary as though it were
a quantity of which the definition is obvious. We cannot be forever examining
our foundations; we look particularly to those places where it is reported to
us that they are insecure. I may be laying myself open to the charge that
I am doing the very thing I criticise in the older physics---using terms that
have no definite observational meaning, and mingling with my physical
quantities things which are not the results of any conceivable experimental
operation. I would reply---

By all means explore this criticism if you regard it as a promising field
of inquiry. I here assume that you will probably find me a justification for
my $10^{-15}$~cm.; but you may find that there is an insurmountable ambiguity
in defining it. In the latter event you may be on the track of something
which will give a new insight into the fundamental nature of the world.
Indeed it has been suspected that the perplexities of quantum phenomena
may arise from the tacit assumption that the notions of length and duration,
acquired primarily from experiences in which the average effects of large
numbers of quanta are involved, are applicable in the study of individual
quanta. There may need to be much more excavation before we have brought
to light all that is of value in this critical consideration of experimental
knowledge. Meanwhile I want to set before you the treasure which has
already been unearthed in this field.

\Chapter{I}{Elementary Principles}

\Section{1.}{Indeterminateness of the space-time frame}

\lettrine{\textcolor{lettrinecolour}{I}}{t} has been explained in the early chapters of
\Title{Space, Time and Gravitation} that observers with different motions use different
reckonings of space and time, and that no one of these reckonings is more fundamental than another.
Our problem is to construct a method of description of the world in which
this indeterminateness of the space-time frame of reference is formally recognised.

Prior to Einstein's researches no doubt was entertained that there existed
a ``true even\hyp{}flowing time'' which was unique and universal. The moving
observer, who adopts a time\hyp{}reckoning different from the unique true time,
must have been deluded into accepting a fictitious time with a fictitious
space\hyp{}reckoning modified to correspond. The compensating behaviour of
electromagnetic forces and of matter is so perfect that, so far as present
knowledge extends, there is no test which will distinguish the true time from
the fictitious. But since there are many fictitious times and, according to
this view, only one true time, some kind of distinction is implied although its
nature is not indicated.

Those who still insist on the existence of a unique ``true time'' generally
rely on the possibility that the resources of experiment are not yet exhausted
and that some day a discriminating test may be found. But the off-chance
that a future generation may discover a significance in our utterances is
scarcely an excuse for making meaningless noises.

Thus in the phrase \emph{true time,} ``true'' is an epithet whose meaning has yet
to be discovered. It is a blank label. We do not know what is to be written
on the label, nor to which of the apparently indistinguishable time\hyp{}reckonings
it ought to be attached. There is no way of progress here. We return to
firmer ground, and note that in the mass of experimental knowledge which
has accumulated, the words \emph{time} and \emph{space} refer to one of the ``fictitious''
times and spaces---primarily that adopted by an observer travelling with the
earth, or with the sun---and our theory will deal directly with these space-time
frames of reference, which are admittedly fictitious or, in the more usual
phrase, \emph{relative to an observer with particular motion}.

The observers are studying the same external events, notwithstanding
their different space-time frames. The space-time frame is therefore something
overlaid by the observer on the external world; the partitions representing
his space and time reckonings are imaginary surfaces drawn in the
world like the lines of latitude and longitude drawn on the earth. They do
not follow the natural lines of structure of the world, any more than the
meridians follow the lines of geological structure of the earth. Such a mesh\hyp{}system
\index{Mesh\hyp{}system}%
is of great utility and convenience in describing phenomena, and we
shall continue to employ it; but we must endeavour not to lose sight of its
fictitious and arbitrary nature.

It is evident from experience that a four-fold mesh\hyp{}system must be used;
and accordingly an event is located by four coordinates, generally taken as
\index{Coordinates}%
$x$,~$y$, $z$,~$t$. To understand the significance of this location, we first consider
the simple case of two dimensions. If we describe the points of a plane figure
by their rectangular coordinates $x$,~$y$, the description of the figure is complete
and would enable anyone to construct it; but it is also more than complete,
because it specifies an arbitrary element, the orientation, which is irrelevant
to the intrinsic properties of the figure and ought to be cast aside from
a description of those properties. Alternatively we can describe the figure by
stating the distances between the various pairs of points in it; this description
is also complete, and it has the merit that it does not prescribe the
orientation or contain anything else irrelevant to the intrinsic properties of
the figure. The drawback is that it is usually too cumbersome to use in
practice for any but the simplest figures.

Similarly our four coordinates $x$,~$y$, $z$,~$t$ may be expected to contain an
arbitrary element, analogous to an orientation, which has nothing to do with
the properties of the configuration of events. A different set of values of
$x$,~$y$, $z$,~$t$ may be chosen in which this arbitrary element of the description is
altered, but the configuration of events remains unchanged. It is this
arbitrariness in coordinate specification which appears as the indeterminateness
of the space-time frame. The other method of description, by giving the
distances between every pair of events (or rather certain relations between
pairs of events which are analogous to distance), contains all that is relevant
to the configuration of events and nothing that is irrelevant. By adopting
this latter method we can strip away the arbitrary part of the description,
leaving only that which has an exact counterpart in the configuration of the
external world.

To put the contrast in another form, in our common outlook the idea of
\index{Extension and location}%
\index{Location and extension}%
position or \emph{location} seems to be fundamental. From it we derive distance or
\emph{extension} as a subsidiary notion, which covers part but not all of the conceptions
which we associate with location. Position is looked upon as the
physical fact---a coincidence with what is vaguely conceived of as an
identifiable point of space---whereas distance is looked upon as an abstraction
or a computational result calculable when the positions are known. The view
which we are going to adopt reverses this. Extension (distance, interval) is
now fundamental; and the location of an object is a computational result
summarising the physical fact that it is at certain intervals from the other
objects in the world. Any idea contained in the concept location which is not
expressible by reference to distances from other objects, must be dismissed
from our minds. Our ultimate analysis of space leads us not to a ``here'' and
a ``there,'' but to an extension such as that which relates ``here'' and ``there.''
To put the conclusion rather crudely---space is not a lot of points close
together; it is a lot of distances interlocked.

Accordingly our fundamental hypothesis is that---

\emph{Everything connected with location which enters into observational knowledge---everything
we can know about the configuration of events---is contained
\index{Configuration of events}%
in a relation of extension between pairs of events.}

This relation is called the \emph{interval,} and its measure is denoted by~$ds$.%
\index{Interval}%

If we have a system~$S$ consisting of events $A$,~$B$, $C$, $D$,~\dots, and a system~$S'$
consisting of events $A'$,~$B'$, $C'$, $D'$,~\dots, then the fundamental hypothesis implies
that the two systems will be exactly alike observationally if, and only if, all
pairs of corresponding intervals in the two systems are equal, $AB = A'B'$,
$AC = A'C'$,~\dots. In that case if $S$~and $S'$ are material systems they will appear
to us as precisely similar bodies or mechanisms; or if $S$~and $S'$ correspond to
the same material body at different times, it will appear that the body has
not undergone any change detectable by observation. But the position,
motion, or orientation of the body may be different; that is a change detectable
by observation, not of the system~$S$, but of a wider system comprising $S$
and surrounding bodies.

Again let the systems $S$ and $S'$ be abstract coordinate\hyp{}frames of reference,
the events being the corners of the meshes; if all corresponding intervals in
the two systems are equal, we shall recognise that the coordinate\hyp{}frames are
of precisely the same kind---rectangular, polar, unaccelerated, rotating, etc.

\Section{2.}{The fundamental quadratic form}

We have to keep side by side the two methods of describing the configurations
of events by coordinates and by the mutual intervals, respectively---the
first for its conciseness, and the second for its immediate absolute
significance. It is therefore necessary to connect the two modes of description
by a formula which will enable us to pass readily from one to the other. The
particular formula will depend on the coordinates chosen as well as on the
absolute properties of the region of the world considered; but it appears that
in all cases the formula is included in the following general form---

The interval $ds$ between two neighbouring events with coordinates
\index{Quadratic formula for interval}%
$(x_1, x_2, x_3, x_4)$ and $(x_1 + dx_1, x_2 + dx_2, x_3 + dx_3, x_4 + dx_4)$ in any coordinate\hyp{}system
is given by
\begin{multline*}
  ds^2 = g_{11}\, dx_1^2 + g_{22}\, dx_2^2 + g_{33}\, dx_3^2 + g_{44}\, dx_4^2 \\
  \begin{aligned}[b]
    &+ 2g_{12}\, dx_1\, dx_2 + 2g_{13}\, dx_1\, dx_3 + 2g_{14}\, dx_1\, dx_4 \\
    &+ 2g_{23}\, dx_2\, dx_3 + 2g_{24}\, dx_2\, dx_4 + 2g_{34}\, dx_3\, dx_4,
  \end{aligned}
\Tag{(2.1)}
\end{multline*}
where the coefficients $g_{11}$, etc. are functions of $x_1$,~$x_2$, $x_3$,~$x_4$. That is to say,
$ds^2$~is some quadratic function of the differences of coordinates.

This is, of course, not the most general case conceivable; for example, we
might have a world in which the interval depended on a general quartic
function of the~$dx$'s. But, as we shall presently see, the quadratic form~\Eq{(2.1)} is
definitely indicated by observation as applying to the actual world. Moreover
near the end of our task (\SecRef{97}) we shall find in the general theory of relation\hyp{}structure
a precise reason why a quadratic function of the coordinate\hyp{}differences
should have this paramount importance.

Whilst the form of the right-hand side of~\Eq{(2.1)} is that required by
observation, the insertion of~$ds^2$ on the left, rather than some other function
of~$ds$, is merely a convention. The quantity~$ds$ is a measure of the interval.
\index{Length!measurement of}%
\index{Measure of interval}%
It is necessary to consider carefully how measure\hyp{}numbers are to be affixed
to the different intervals occurring in nature. We have seen in the last
section that equality of intervals can be tested observationally; but so far
as we have yet gone, intervals are merely either equal or unequal, and their
differences have not been further particularised. Just as wind-strength may
be measured by velocity, or by pressure, or by a number on the Beaufort
scale, so the relation of extension between two events could be expressed
numerically according to many different plans. To conform to~\Eq{(2.1)} a
particular code of measure\hyp{}numbers must be adopted; the nature and
advantages of this code will be explained in the next section.

The pure geometry associated with the general formula~\Eq{(2.1)} was studied
by Riemann, and is generally called Riemannian geometry. It includes
\index{Geometry, Riemannian}%
\index{Riemannian geometry}%
Euclidean geometry as a special case.

\Section{3.}{Measurement of intervals}

Consider the operation of proving by measurement that a distance~$AB$ is
equal to a distance~$CD$. We take a configuration of events $LMNOP$\dots, viz.\ a
measuring\hyp{}scale, and lay it over~$AB$, and observe that $A$~and $B$ coincide with
two particular events $P$,~$Q$ (scale\hyp{}divisions) of the configuration. We find
that the same configuration\footnote
  {The logical point may be noticed that the measuring\hyp{}scale in two positions (necessarily at
  different times) represents the same \emph{configuration} of events, not the same events.}
can also be arranged so that $C$~and $D$ coincide
with $P$~and $Q$~respectively. Further we apply all possible tests to the
measuring\hyp{}scale to see if it has ``changed'' between the two measurements;
and we are only satisfied that the measures are correct if no observable
difference can be detected. According to our fundamental axiom, the absence
of any observable difference between the two configurations (the structure of
the measuring\hyp{}scale in its two positions) signifies that the intervals are unchanged;
in particular the interval between $P$ and~$Q$ is unchanged. It follows
that the interval $A$ to~$B$ is equal to the interval $C$ to~$D$. We consider that the
experiment proves equality of distance; but it is primarily a test of equality
of interval.

In this experiment time is not involved; and we conclude that in space
considered apart from time the test of equality of distance is equality of
interval. There is thus a one-to-one correspondence of distances and intervals.
We may therefore adopt the same measure\hyp{}number for the interval as is in
general use for the distance, thus settling our plan of affixing measure\hyp{}numbers
to intervals. It follows that, when time is not involved, the interval
reduces to the distance.

It is for this reason that the quadratic form~\Eq{(2.1)} is needed in order to
agree with observation, for it is well known that in three dimensions the
square of the distance between two neighbouring points is a quadratic
function of their infinitesimal coordinate\hyp{}differences---a result depending
ultimately on the experimental law expressed by Euclid \Vol{I},~47.

When time is involved other appliances are used for measuring intervals.
If we have a mechanism capable of cyclic motion, its cycles will measure
equal intervals provided the mechanism, its laws of behaviour, and all relevant
surrounding circumstances, remain precisely similar. For the phrase ``precisely
similar'' means that no observable differences can be detected in the mechanism
or its behaviour; and that, as we have seen, requires that all corresponding
intervals should be equal. In particular the interval between the events
marking the beginning and end of the cycle is unaltered. Thus a clock
primarily measures equal intervals; it is only under more restricted conditions
that it also measures the time\hyp{}coordinate~$t$.

In general any repetition of an operation under similar conditions, but for
a different time, place, orientation and velocity (attendant circumstances
which have a relative but not an absolute significance\footnotemark),\footnotetext
  {They express relations to events which are not concerned in the test, e.g.\ to the sun and
  stars.}
tests, equality of
interval.

It is obvious from common experience that intervals which can be
measured with a clock cannot be measured with a scale, and \Foreign{vice versa}. We
have thus two varieties of intervals, which are provided for in the formula~\Eq{(2.1)},
since $ds^2$~may be positive or negative and the measure of the interval
will accordingly be expressed by a real or an imaginary number. The
\index{Imaginary intervals}%
abbreviated phrase ``imaginary interval'' must not be allowed to mislead;
there is nothing imaginary in the corresponding relation; it is merely that in
our arbitrary code an imaginary number is assigned as its measure\hyp{}number.
We might have adopted a different code, and have taken, for example, the
antilogarithm of~$ds^2$ as the measure of the interval; in that case space\hyp{}intervals
would have received code\hyp{}numbers from~$1$ to~$\infty$, and time\hyp{}intervals
numbers from~$0$ to~$1$. When we encounter~$\sqrt{-1}$ in our investigations, we
must remember that it has been introduced by our choice of measure\hyp{}code,
and must not think of it as occurring with some mystical significance in the
external world.

\Section{4.}{Rectangular coordinates and time}
\index{Rectangular coordinates and time}%
\index{Coordinate\hyp{}systems!rectangular}%

Suppose that we have a small region of the world throughout which the
$g$'s can be treated as constants\footnotemark.\footnotetext
  {It will be shown in \SecRef{36} that it is always possible to transform the coordinates so that the
  first derivatives of the~$g$'s vanish at a selected point. We shall suppose that this preliminary
  transformation has already been made, in order that the constancy of the~$g$'s may be a valid
  approximation through as large a region as possible round the selected point.}
In that case the right\hyp{}hand side of~\Eq{(2.1)} can
be broken up into the sum of four squares, admitting imaginary coefficients
if necessary. Thus writing
\begin{align*}
  y_1 &= a_1 x_1 + a_2 x_2 + a_3 x_3 + a_4 x_4, \\
  y_2 &= b_1 x_1 + b_2 x_2 + b_3 x_3 + b_4 x_4,\quad\text{etc.,} \\
\intertext{so that}
  dy_1 &= a_1\, dx_1 + a_2\, dx_2 + a_3\, dx_3 + a_4\, dx_4,\quad\text{etc.,}
\end{align*}
we can choose the constants $a_1$, $b_1$,~\dots\ so that \Eq{(2.1)}~becomes
\[
ds^2 = dy_1^2 + dy_2^2 + dy_3^2 + dy_4^2.
\Tag{(4.1)}
\]
For, substituting for the $dy$'s and comparing coefficients with~\Eq{(2.1)}, we have
only $10$~equations to be satisfied by the $16$~constants. There are thus many
ways of making the reduction. Note, however, that the reduction to the sum
of four squares of complete differentials is not in general possible for a \emph{large}
region, where the $g$'s have to be treated as functions, not constants.

Consider all the events for which $y_4$~has some specified value. These will
form a three\hyp{}dimensional world. Since $dy_4$~is zero for every pair of these
events, their mutual intervals are given by
\[
ds^2 = dy_1^2 + dy_2^2 + dy_3^2.
\Tag{(4.2)}
\]
But this is exactly like familiar space in which the interval (which we have
shown to be the same as the distance for space without time) is given by
\[
ds^2 = dx^2 + dy^2 + dz^2,
\Tag{(4.3)}
\]
where $x$, $y$, $z$ are rectangular coordinates.

Hence a section of the world by $y_4 = \text{const.}$ will appear to us as space, and
$y_1$,~$y_2$,~$y_3$ will appear to us as rectangular coordinates. The coordinate\hyp{}frames
$y_1$,~$y_2$,~$y_3$, and $x$,~$y$,~$z$, are examples of the systems $S$ and~$S'$ of \SecRef{1}, for which
the intervals between corresponding pairs of mesh\hyp{}corners are equal. The
two systems are therefore exactly alike observationally; and if one appears
to us to be a rectangular frame in space, so also must the other. One proviso
must be noted; the coordinates $y_1$,~$y_2$,~$y_3$ for real events must be real, as in
familiar space, otherwise the resemblance would be only formal.

Granting this proviso, we have reduced the general expression to
\[
ds^2 = dx^2 + dy^2 + dz^2 + dy_4^2,
\Tag{(4.4)}
\]
where $x$, $y$, $z$ will be recognised by us as rectangular coordinates in space.
Clearly $y_4$~must involve the time, otherwise our location of events by the four
coordinates would be incomplete; but we must not too hastily identify it
with the time~$t$.

I suppose that the following would be generally accepted as a satisfactory
(pre\hyp{}relativity) definition of equal time\hyp{}intervals:---if we have a mechanism
\index{Time!definition of}%
capable of cyclic motion, its cycles will measure equal durations of time
\emph{anywhere} and \emph{anywhen,} provided the mechanism, its laws of behaviour, and
all outside influences remain precisely similar. To this the relativist would
add the condition that the mechanism (as a whole) must be at rest in the
space-time frame considered, because it is now known that a clock in motion
goes slow in comparison with a fixed clock. The non\hyp{}relativist does not disagree
in fact, though he takes a slightly different view; he regards the proviso
that the mechanism must be at rest as already included in his enunciation,
because for him motion involves progress through the aether, which (he
considers) directly affects the behaviour of the clock, and is one of those
``outside influences'' which have to be kept ``precisely similar.''

Since then it is agreed that the mechanism as a whole is to be at rest,
and the moving parts return to the same positions after a complete cycle, we
shall have for the two events marking the beginning and end of the cycle
\[
dx,\ dy,\ dz = 0.
\]
Accordingly \Eq{(4.4)} gives for this case
\[
ds^2 = dy_4^2.
\]
We have seen in \SecRef{3} that the cycles of the mechanism in all cases correspond
to equal intervals~$ds$; hence they correspond to equal values of~$dy_4$. But by
the above definition of time they also correspond to equal lapses of time~$dt$;
hence we must have $dy_4$~proportional to~$dt$, and we express this proportionality
by writing
\[
dy_4 = ic\, dt,
\Tag{(4.5)}
\]
where $i = \sqrt{-1}$, and $c$~is a constant. It is, of course, possible that $c$~may be
an imaginary number, but provisionally we shall suppose it real. Then \Eq{(4.4)}
becomes
\[
ds^2 = dx^2 + dy^2 + dz^2 - c^2 dt^2.
\Tag{(4.6)}
\]

A further discussion is necessary before it is permissible to conclude that
\Eq{(4.6)}~is the most general possible form for~$ds^2$ in terms of ordinary space and
time coordinates. If we had reduced~\Eq{(2.1)} to the rather more general form
\[
ds^2 = dx^2 + dy^2 + dz^2 - c^2 dt^2 - 2c\alpha\, dx\, dt - 2c\beta\, dy\, dt - 2c\gamma\, dz\, dt,
\Tag{(4.7)}
\]
this would have agreed with~\Eq{(4.6)} in the only two cases yet discussed, viz.\
(1)~when $dt = 0$, and (2)~when $dx$,~$dy$, $dz = 0$. To show that this more general
form is inadmissible we must examine pairs of events which differ both in
time and place.

In the preceding pre\hyp{}relativity definition of~$t$ our clocks had to remain
stationary and were therefore of no use for comparing time at different places.
What did the pre\hyp{}relativity physicist mean by the difference of time~$dt$
between two events at different places? I do not think that we can attach
any meaning to his hazy conception of what $dt$~signified; but we know one
or two ways in which he was accustomed to determine it. One method which
he used was that of transport of chronometers. Let us examine then what
happens when we move a clock from $(x_1, 0, 0)$ at the time~$t_1$ to another place
\index{Clocks, transport of}%
\index{Time!convention in reckoning}%
$(x_2, 0, 0)$ at the time~$t_2$.

We have seen that the clock, whether at rest or in motion, provided it
remains a precisely similar mechanism, records equal \emph{intervals}; hence the
difference of the clock\hyp{}readings at the beginning and end of the journey will
be proportional to the integrated interval
\[
\int_{1}^{2} ds.
\Tag{(4.81)}
\]
If the transport is made in the direct line ($dy = 0$, $dz = 0$), we shall have
according to~\Eq{(4.7)}
\[
-ds^2 = c^2\, dt^2 + 2c\alpha\, dx\, dt - dx^2 = c^2\, dt^2\left\{1 + \frac{2\alpha}{c}\, \frac{dx}{dt} - \frac{1}{c^2} \left(\frac{dx}{dt}\right)^{2}\right\}.
\]
Hence the difference of the clock\hyp{}readings \Eq{(4.81)} is proportional to
\[
\int\limits_{t_1}^{t_2} dt \sqrt{1 + \frac{2\alpha u}{c} - \frac{u^2}{c^2}},
\Tag{(4.82)}
\]
where $u = dx/dt$, i.e. the velocity of the clock. The integral will not in general
reduce to $t_{2} - t_{1}$; so that the difference of time at the two places is not given
correctly by the reading of the clock. Even when $\alpha = 0$, the moving clock
does not record correct time.

Now introduce the condition that the velocity~$u$ is very small, remembering
that $t_{2} - t_{1}$ will then become very large. Neglecting $u^2/c^2$, \Eq{(4.82)}~becomes approximately
\[
\int\limits_{t_1}^{t_2} dt \left(1 + \frac{\alpha}{c}\, \frac{dx}{dt}\right) = (t_2 - t_1) + \frac{\alpha}{c} (x_2 - x_1).
\]
The clock, if moved sufficiently slowly, will record the correct time\hyp{}difference
if, and only if, $\alpha = 0$. Moving it in other directions, we must have, similarly,
$\beta = 0$, $\gamma = 0$. Thus \Eq{(4.6)}~is the most general formula for the interval, when
the time at different places is compared by slow transport of clocks from one
\index{Transport of clocks}%
place to another.

I do not know how far the reader will be prepared to accept the condition
that it must be possible to correlate the times at different places by moving
a clock from one to the other with infinitesimal velocity. The method
employed in accurate work is to send an electromagnetic signal from one to
the other, and we shall see in \SecRef{11} that this leads to the same formulae. We
can scarcely consider that either of these methods of comparing time at
different places is an essential part of our primitive notion of time in the
same way that measurement at one place by a cyclic mechanism is; therefore
they are best regarded as conventional. Let it be understood, however, that
although the relativity theory has formulated the convention explicitly, the
usage of the word \emph{time\hyp{}difference} for the quantity fixed by this convention is
\index{Retardation of moving clocks}%
in accordance with the long established practice in experimental physics and
astronomy.

Setting $\alpha = 0$ in~\Eq{(4.82)}, we see that the accurate formula for the clock\hyp{}reading
will be
\[
\int\limits_{t_1}^{t_2} dt \sqrt{1-\frac{u^2}{c^2}} = \sqrt{1-\frac{u^2}{c^2}} (t_2 - t_1)
\Tag{(4.9)}
\]
for a uniform velocity~$u$. Thus a clock travelling with finite velocity gives
too small a reading---the clock goes slow compared with the time\hyp{}reckoning
conventionally adopted.

To sum up the results of this section, if we choose coordinates such that
the general quadratic form reduces to
\[
ds^2 = dy_1^2 + dy_2^2 + dy_3^2 + dy_4^2,
\Tag{(4.95)}
\]
then $y_1$, $y_2$, $y_3$ and $y_4 \sqrt{-1}$ will represent ordinary rectangular coordinates and
time. If we choose coordinates for which
\[
  ds^2 = dy_1^2 + dy_2^2 + dy_3^2 + dy_4^2
  + 2\alpha\, dy_1\, dy_4 + 2\beta\, dy_2\, dy_4 + 2\gamma\, dy_3\, dy_4,
\Tag{(4.96)}
\]
these coordinates also will agree with rectangular coordinates and time so far
as the more primitive notions of time are concerned; but the reckoning by
this formula of differences of time at different places will not agree with the
reckoning adopted in physics and astronomy according to long established
practice. For this reason it would only introduce confusion to admit these
coordinates as a permissible space and time system.

We who regard all coordinate\hyp{}frames as equally fictitious structures have
no special interest in ruling out the more general form~\Eq{(4.96)}. It is not a
question of ascribing greater significance to one frame than to another, but
of discovering which frame corresponds to the space and time reckoning
generally accepted and used in standard works such as the Nautical Almanac.

As far as \SecRef{14} our work will be subject to the condition that we are dealing
with a region of the world in which the $g$'s are constant, or approximately
constant. A region having this property is called \emph{flat}. The theory of this
\index{Flat space-time}%
\index{Special theory of relativity}%
case is called the ``special'' theory of relativity; it was discussed by Einstein
in 1905---some ten years before the general theory. But it becomes much
simpler when regarded as a special case of the general theory, because it is
no longer necessary to defend the conditions for its validity as being essential
properties of space-time. For a given region these conditions may hold, or
they may not. The special theory applies only if they hold; other cases must
be referred to the general theory.

\Section{5.}{The Lorentz transformation}
\index{Lorentz transformation}%

Make the following transformation of coordinates
\index{Transformation of coordinates!Lorentz}%
\begin{gather*}
  x = \beta(x' - ut'),\
  y = y',\
  z = z',\
  t = \beta (t' - ux'/c^2),
  \Tag{(5.1)} \\
  \beta = \frac{1}{\sqrt{1-\frac{u^2}{c^2}}},
\end{gather*}
where $u$~is any real constant not greater than~$c$.

We have by~\Eq{(5.1)}
\begin{align*}
  dx^2 - c^2\, dt^2
  &= \beta^2 \bigl\{(dx' - u\, dt')^{2} - c^2 (dt' - u\, dx'/c^2)^{2}\bigr\}\displaybreak[0] \\
  &= \beta^2 \left\{\left(1 - \frac{u^2}{c^2}\right) dx'^{2} - (c^2 - u^2)\, dt'^{2}\right\} \\
  &= dx'^{2} - c^2\, dt'^{2}.
\end{align*}
Hence from \Eq{(4.6)}
\[
ds^2 = dx^2 + dy^2 + dz^2 - c^2 dt^2 = dx'^2 + dy'^2 + dz'^2 - c^2 dt'^2.
\Tag{(5.2)}
\]

The accented and unaccented coordinates give the same formula for the
interval, so that the intervals between corresponding pairs of mesh\hyp{}corners
will be equal, and therefore in all observable respects they will be alike. We
shall recognise $x'$,~$y'$,~$z'$ as rectangular coordinates in space, and $t'$~as the
associated time. We have thus arrived at another possible way of reckoning
space and time---another fictitious space-time frame, equivalent in all its
properties to the original one. For convenience we say that the first reckoning
is that of an observer~$S$ and the second that of an observer~$S'$, both observers
being at rest in their respective spaces\footnotemark.\footnotetext
  {This is partly a matter of nomenclature. A sentient observer can force himself to ``recollect
  that he is moving'' and so adopt a space in which he is not at rest; but he does not so readily
  adopt the time which properly corresponds; unless he uses the space-time frame in which he is
  at rest, he is likely to adopt a hybrid space-time which leads to inconsistencies. There is no
  ambiguity if the ``observer'' is regarded as merely an involuntary measuring apparatus, which by
  the principles of \SecRef{4} naturally partitions a space and time with respect to which it is at rest.}

The constant~$u$ is easily interpreted. Since $S$~is at rest in his own space,
his location is given by $x = \text{const}$. By~\Eq{(5.1)} this becomes, in $S'$'s~coordinates,
$x' - ut' = \text{const.}$; that is to say, $S$~is travelling in the $x'$-direction with velocity~$u$.
Accordingly the constant~$u$ is interpreted as the velocity of~$S$ relative to~$S'$.

It does not follow immediately that the velocity of~$S'$ relative to~$S$ is~$-u$;
but this can be proved by algebraical solution of the equations~\Eq{(5.1)} to
determine $x'$,~$y'$, $z'$,~$t'$. We find
\[
x' = \beta(x + ut),\quad
y' = y,\quad
z' = z,\quad
t' = \beta(t + ux/c^2),
\Tag{(5.3)}
\]
showing that an interchange of $S$ and~$S'$ merely reverses the sign of~$u$.

The essential property of the foregoing transformation is that it leaves
the formula for~$ds^2$ unaltered~\Eq{(5.2)}, so that the coordinate\hyp{}systems which it
connects are alike in their properties. Looking at the matter more generally,
we have already noted that the reduction to the sum of four squares can be
made in many ways, so that we can have
\[
ds^2 = dy_1^2 + dy_2^2 + dy_3^2 + dy_4^2 = dy_1'^2 + dy_2'^2 + dy_3'^2 + dy_4'^2.
\Tag{(5.4)}
\]
The determination of the necessary connection between any two sets of
coordinates satisfying this equation is a problem of pure mathematics; we
can use freely the conceptions of four\hyp{}dimensional geometry and imaginary
rotations to find this connection, whether the conceptions have any physical
significance or not. We see from~\Eq{(5.4)} that $ds$~is the distance between two
points in four\hyp{}dimensional Euclidean space, the coordinates $(y_1, y_2, y_3, y_4)$ and
$(y_1', y_2', y_3', y_4')$ being rectangular systems (real or imaginary) in that space.
Accordingly these coordinates are related by the general transformations from
one set of rectangular axes to another in four dimensions, viz.\ translations
and rotations. Translation, or change of origin, need not detain us; nor need
a rotation of the space-axes $(y_1, y_2, y_3)$ leaving time unaffected. The interesting
case is a rotation in which $y_4$~is involved, typified by
\[
y_1 = y_1' \cos\theta - y_4' \sin\theta,\quad
y_4 = y_1' \sin\theta + y_4' \cos\theta.
\]
Writing $u = ic \tan\theta$, so that $\beta = \cos\theta$, this leads to the Lorentz transformation~\Eq{(5.1)}.

Thus, apart from obvious trivial changes of axes, the Lorentz transformations
are the only ones which leave the form~\Eq{(4.6)} unaltered.

Historically this transformation was first obtained for the particular case
of electromagnetic equations. Its more general character was pointed out by
Einstein in 1905.

\Section{6.}{The velocity of light}
\index{Addition of velocities}%

Consider a point moving along the $x$-axis whose velocity measured by~$S'$
is~$v'$, so that
\[
v' = \frac{dx'}{dt'}.
\Tag{(6.1)}
\]
Then by \Eq{(5.1)} its velocity measured by~$S$ is
\begin{align*}
  v = \frac{dx}{dt}
  &= \frac{\beta(dx' - u\, dt')}{\beta(dt' - u\, dx'/c^2)} \\
  &= \frac{v' - u}{1 - u v'/c^2}\quad\text{by \Eq{(6.1)}.}
  \Tag{(6.2)}
\end{align*}
In non\hyp{}relativity kinematics we should have taken it as axiomatic that
$v = v' - u$.

If two points move relatively to~$S'$ with equal velocities in opposite
directions $+v'$ and~$-v'$, their velocities relative to~$S$ are
\[
\frac{v' - u}{1 - u v'/c^2}\quad\text{and}\quad
-\frac{v' + u}{1 + u v'/c^2}.
\]
As we should expect, these speeds are usually unequal; but there is an exceptional
case when $v' = c$. The speeds relative to~$S$ are then also equal, both
in fact being equal to~$c$.

Again it follows from~\Eq{(5.2)} that when
\[
\left(\frac{dx'}{dt'}\right)^{2} + \left(\frac{dy'}{dt'}\right)^{2} + \left(\frac{dz'}{dt'}\right)^{2} = c^2,
\]
$ds = 0$, and hence
\[
\left(\frac{dx}{dt}\right)^{2} + \left(\frac{dy}{dt}\right)^{2} + \left(\frac{dz}{dt}\right)^{2} = c^2.
\]
Thus when the resultant velocity relative to~$S'$ is~$c$, the velocity relative to~$S$
is also~$c$, whatever the direction. We see that the velocity~$c$ has a unique
and very remarkable property.

According to the older views of absolute time this result appears incredible.
Moreover we have not yet shown that the formulae have practical significance,
since $c$~might be imaginary. But experiment has revealed a real velocity
with this remarkable property, viz.\ 299,860~km.\ per~sec. We shall call this
the \emph{fundamental velocity}.
\index{Fundamental velocity}%
\index{Velocity, fundamental}%

By good fortune there is an entity---light---which travels with the fundamental
\index{Light!velocity of}%
velocity. It would be a mistake to suppose that the existence of such
an entity is responsible for the prominence accorded to the fundamental velocity~$c$
in our scheme; but it is helpful in rendering it more directly accessible to
experiment. The Michelson\hyp{}Morley experiment detected no difference in the
\index{Michelson\hyp{}Morley experiment}%
velocity of light in two directions at right angles. Six months later the earth's
\index{Velocity of light}%
orbital motion had altered the observer's velocity by 60~km.\ per~sec., corresponding
to the change from~$S'$ to~$S$, and there was still no difference. Hence
the velocity of light has the distinctive property of the fundamental velocity.

Strictly speaking the Michelson\hyp{}Morley experiment did not prove directly
that the velocity of light was constant in all directions, but that the average
to-and-fro velocity was constant in all directions. The experiment compared
the times of a journey ``there-and-back.'' If $v(\theta)$~is the velocity of light in
the direction~$\theta$, the experimental result is
\[
\left.
\begin{alignedat}{2}
  \frac{1}{v(\theta)} + \frac{1}{v(\theta + \pi)} &= \text{const.} &&= C \\
  \frac{1}{v'(\theta)} + \frac{1}{v'(\theta + \pi)} &= \text{const.} &&= C'
\end{alignedat}
\right\}
\Tag{(6.3)}
\]
for all values of~$\theta$. The constancy has been established to about $1$~part in~$10^{10}$.

It is exceedingly unlikely that the first equation could hold unless
\[
v(\theta) = v(\theta + \pi) = \text{const.};
\]
and it is fairly obvious that the existence of the second equation excludes the
possibility altogether. However, on account of the great importance of the
identification of the fundamental velocity with the velocity of light, we give
a formal proof.

Let a ray travelling with velocity~$v$ traverse a distance~$R$ in a direction~$\theta$,
so that
\[
dt = \frac{R}{v},\quad
dx = R \cos\theta,\quad
dy = R \sin \theta.
\]
Let the relative velocity of~$S$ and~$S'$ be small so that $u^2/c^2$~is neglected. Then
by~\Eq{(5.3)}
\[
dt' = dt + u\, dx/c^2,\quad
dx' = dx + u\, dt,\quad
dy' = dy.
\]
Writing $\delta R$, $\delta\theta$, $\delta v$ for the change in $R$,~$\theta$,~$v$ when a transformation is made
to $S'$'s~system, we obtain
\begin{align*}
  \delta(\frac{R}{v}) &= dt' - dt = \frac{uR \cos \theta}{c^2}, \\
  \delta(R \cos\theta) &= dx' - dx = \frac{uR}{v}, \\
  \delta(R \sin\theta) &= dy' - dy = 0.
  \intertext{Whence the values of $\delta R$, $\delta\theta$, $\delta (1/v)$ are found as follows:}
  \delta R &= \frac{uR \cos \theta}{v}, \\
  \delta\theta &= -\frac{u \sin \theta}{v}, \\
  \delta\left(\frac{1}{v}\right) &= u \cos\theta \left(\frac{1}{c^2} - \frac{1}{v^2}\right).
\end{align*}

Here $\delta(1/v)$ refers to a comparison of velocities in the directions~$\theta$ in~$S$'s
system and $\theta'$~in $S'$'s~system. Writing $\Delta(1/v)$ for a comparison when the
direction is~$\theta$ in both systems
\begin{align*}
  \Delta\left(\frac{1}{v}\right)
  &= \delta\left(\frac{1}{v}\right) - \frac{\dd}{\dd\theta}\left(\frac{1}{v}\right)\cdot\delta\theta\\
  &= \frac{u}{c^2} \cos\theta - \frac{u}{v^2} \cos\theta + \frac{u\sin\theta}{v}\, \frac{\dd}{\dd\theta}\left(\frac{1}{v}\right) \\
  &= \frac{u}{c^2} \cos\theta + \tfrac{1}{2} u \sin^{3}\theta\, \frac{\dd}{\dd\theta} \left(\frac{1}{v^2 \sin^2\theta}\right).
\end{align*}
Hence
\[
\Delta\left(\frac{1}{v(\theta)} + \frac{1}{v(\theta + \pi)}\right)
= \tfrac{1}{2} u\sin^{3}\theta\, \frac{\dd}{\dd\theta}\left\{\frac{1}{\sin^2\theta}\left(\frac{1}{v^2(\theta)} - \frac{1}{v^2(\theta + \pi)}\right)\right\}.
\]
By~\Eq{(6.3)} the left-hand side is independent of~$\theta$, and equal to the constant
$C' - C$. We obtain on integration
\begin{align*}
  \frac{1}{v^2(\theta)} - \frac{1}{v^2(\theta + \pi)}
  &= \frac{C' - C}{u} (\sin^{2}\theta\cdot\log\tan \tfrac{1}{2}\theta - \cos\theta), \\
\intertext{or}
  \frac{1}{v(\theta)} - \frac{1}{v(\theta + \pi)}
  &= \frac{C' - C}{C} \cdot \frac{1}{u} (\sin^{2}\theta \cdot \log\tan \tfrac{1}{2}\theta - \cos\theta).
\end{align*}
It is clearly impossible that the difference of~$1/v$ in opposite directions should
be a function of~$\theta$ of this form; because the origin of~$\theta$ is merely the direction
of relative motion of $S$ and~$S'$, which may be changed at will in different
experiments, and has nothing to do with the propagation of light relative to~$S$.
Hence $C' - C = 0$, and $v(\theta) = v(\theta + \pi)$. Accordingly by~\Eq{(6.3)} $v(\theta)$~is independent
of~$\theta$; and similarly $v'(\theta)$~is independent of~$\theta$. Thus the velocity of
light is uniform in all directions for both observers and is therefore to be
identified with the fundamental velocity.

When this proof is compared with the statement commonly (and correctly)
made that the equality of the forward and backward velocity of light cannot
be deduced from experiment, regard must be paid to the context. The use
of the Michelson\hyp{}Morley experiment to fill a particular gap in a generally
deductive argument must not be confused with its use (e.g.\ in \Title{Space, Time
and Gravitation}) as the basis of a pure induction from experiment. Here we
have not even used the fact that it is a second\hyp{}order experiment. We have
deduced the Lorentz transformation from the fundamental hypothesis of \SecRef{1},
and have already introduced a conventional system of time\hyp{}reckoning explained
in \SecRef{4}. The present argument shows that the convention that time is defined
by the slow transport of chronometers is equivalent to the convention that
the forward velocity of light is equal to the backward velocity. The proof of
\index{Velocity of light!in moving matter}%
this equivalence is mainly deductive except for one hiatus---the connection
of the propagation of light and the fundamental velocity---and for that step
appeal is made to the Michelson\hyp{}Morley experiment.

The law of composition of velocities~\Eq{(6.2)} is well illustrated by Fizeau's
\index{Addition of velocities}%
\index{Composition of velocities}%
experiment on the propagation of light along a moving stream of water. Let
the observer~$S'$ travel with the stream of water, and let $S$~be a fixed observer.
The water is at rest relatively to~$S'$ and the velocity of the light relative to
him will thus be the ordinary velocity of propagation in still water, viz.\
$v' = c/\mu$, where $\mu$~is the refractive index. The velocity of the stream being~$w$,
$-w$~is the velocity of~$S$ relative to~$S'$; hence by~\Eq{(6.2)} the velocity~$v$ of the
light relative to~$S$ is
\[
  v = \frac{v' + w}{1 + wv'/c^2} = \frac{c/\mu + w}{1 + w/\mu c} \approx c/\mu + w(1 - 1/\mu^2)
\]
neglecting the square of~$w/c$.

Accordingly the velocity of the light is not increased by the full velocity
of the stream in which it is propagated, but by the fraction $(1 - 1/\mu^2) w$. For
water this is about $0.44 w$. The effect can be measured by dividing a beam
of light into two parts which are sent in opposite directions round a circulating
stream of water. The factor $(1 - 1/\mu^2)$ is known as Fresnel's convection\hyp{}coefficient;
\index{Fresnel's convection\hyp{}coefficient}%
it was confirmed experimentally by Fizeau in~1851.

If the velocity of light \Foreign{in vacuo} were a constant~$c'$ differing from the
fundamental velocity~$c$, the foregoing calculation would give for Fresnel's
convection\hyp{}coefficient
\[
1 - \frac{c'^{2}}{c^2} \cdot \frac{1}{\mu^2}.
\]
Thus Fizeau's experiment provides independent evidence that the fundamental
\index{Fizeau's experiment}%
velocity is at least approximately the same as the velocity of light. In the most
recent repetitions of this experiment made by Zeeman\footnote
  {\Title{Amsterdam Proceedings}, vol.~\Vol{XVIII}, pp.~398 and 1240.}
the agreement between
theory and observation is such that $c'$~cannot differ from~$c$ by more than 1 part in 500.

\Section{7.}{Timelike and spacelike intervals}
\index{Spacelike intervals}%
\index{Timelike intervals}%

We make a slight change of notation, the quantity hitherto denoted by~$ds^2$
being in all subsequent formulae replaced by $-ds^2$, so that \Eq{(4.6)}~becomes
\[
ds^2 = c^2 dt^2 - dx^2 - dy^2 - dz^2.
\Tag{(7.1)}
\]
There is no particular advantage in this change of sign; it is made in order
to conform to the customary notation.

The formula may give either positive or negative values of~$ds^2$, so that the
interval between real events may be a real or an imaginary number. We call
real intervals timelike, and imaginary intervals spacelike.

From \Eq{(7.1)}
\begin{align*}
  \left(\frac{ds}{dt}\right)^{2}
  &= c^2 - \left(\frac{dx}{dt}\right)^{2} - \left(\frac{dy}{dt}\right)^{2} - \left(\frac{dz}{dt}\right)^{2} \\
  &= c^2 - v^2,
  \Tag{(7.2)}
\end{align*}
where $v$~is the velocity of a point describing the track along which the interval
lies. The interval is thus real or imaginary according as $v$~is less than or
greater than~$c$. Assuming that a material particle cannot travel faster than
light, the intervals along its track must be timelike. We ourselves are limited
by material bodies and therefore can only have direct experience of timelike
intervals. We are immediately aware of the passage of time without the use
of our external senses; but we have to infer from our sense perceptions the
existence of spacelike intervals outside us.

From any event $x$,~$y$, $z$,~$t$, intervals radiate in all directions\footnotemark\footnotetext
  {It should be noted that a four\hyp{}dimensional ``direction'' corresponds to velocity
   in space $xyz$, for such a direction can be defined by the expression~$dx:dy:dz:dt$, or~$u:v:w:1$.}
to other events;
and the real and imaginary intervals are separated by the cone
\[
0 = c^2\, dt^{2} - dx^{2} - dy^{2} - dz^{2},
\]
which is called the \emph{null-cone}. Since light travels with velocity~$c$, the track of
\index{Null-cone}%
any light-pulse proceeding from the event lies on the null-cone. When the
$g$'s are not constants and the fundamental quadratic form is not reducible to~\Eq{(7.1)},
there is still a null\hyp{}surface, given by $ds = 0$ in~\Eq{(2.1)}, which separates the
timelike and spacelike intervals. There can be little doubt that in this case
also the light-tracks lie on the null\hyp{}surface, but the property is perhaps scarcely
self\hyp{}evident, and we shall have to justify it in more detail later.

The formula \Eq{(6.2)} for the composition of velocities in the same straight
\index{Addition of velocities}%
\index{Composition of velocities}%
line may be written
\[
\tanh^{-1} v/c = \tanh^{-1} v'/c - \tanh^{-1} u/c.
\Tag{(7.3)}
\]
The quantity $\tanh^{-1} v/c$ has been called by Robb the \emph{rapidity} corresponding
\index{Rapidity}%
to the velocity~$v$. Thus \Eq{(7.3)}~shows that relative rapidities in the same direction
compound according to the simple addition\hyp{}law. Since $\tanh^{-1} 1 = \infty$ , the
velocity of light corresponds to infinite rapidity. We cannot reach infinite
rapidity by adding any finite number of finite rapidities; therefore we cannot
reach the velocity of light by compounding any finite number of relative
velocities less than that of light.

There is an essential discontinuity between speeds greater than and less
than that of light which is illustrated by the following example. If two points
\index{Light!velocity of}%
move in the same direction with velocities
\[
v_{1} = c + \epsilon,\quad
v_{2} = c - \epsilon
\]
respectively, their relative velocity is by~\Eq{(6.2)}
\[
\frac{v_{1} - v_{2}}{1 - v_{1}v_{2}/c^2}
= \frac{2\epsilon}{1 - (c^2 - \epsilon^{2})/c^2}
= \frac{2c^2}{\epsilon},
\]
which tends to infinity as $\epsilon$ is made infinitely small! If the fundamental
velocity is exactly 300,000~km.\ per~sec., and two points move in the same
direction with speeds of 300,001 and 299,999~km.\ per~sec., the speed of one
relative to the other is 180,000,000,000 km.\ per~sec. The barrier at 300,000~km.\
per~sec.\ is not to be crossed by approaching it. A particle which is aiming to
reach a speed of 300,001~km.\ per~sec.\ might naturally hope to attain its object
by continually increasing its speed; but when it has reached 299,999~km.\ per~sec.,
and takes stock of the position, it sees its goal very much farther off than
when it started.

A particle of matter is a structure whose linear extension is timelike. We
might perhaps imagine an analogous structure ranged along a spacelike track.
That would be an attempt to picture a particle travelling with a velocity
greater than that of light; but since the structure would differ fundamentally
from matter as known to us, there seems no reason to think that it would be
recognised by us as a particle of matter, even if its existence were possible.
For a suitably chosen observer a spacelike track can lie wholly in an instantaneous
space. The structure would exist along a line in space at one moment;
at preceding and succeeding moments it would be non\hyp{}existent. Such instantaneous
intrusions must profoundly modify the continuity of evolution from
past to future. In default of any evidence of the existence of these spacelike
particles we shall assume that they are impossible structures.

\Section{8.}{Immediate consciousness of time}
\index{Time!immediate consciousness of}%

Our minds are immediately aware of a ``flight of time'' without the intervention
of external senses. Presumably there are more or less cyclic processes
occurring in the brain, which play the part of a material clock, whose indications
the mind can read. The rough measures of duration made by the internal
time-sense are of little use for scientific purposes, and physics is accustomed
to base time\hyp{}reckoning on more precise external mechanisms. It is, however,
desirable to examine the relation of this more primitive notion of time to the
scheme developed in physics.

Much confusion has arisen from a failure to realise that time as currently
used in physics and astronomy deviates widely from the time recognised by
the primitive time-sense. In fact the time of which we are immediately conscious
is not in general physical time, but the more fundamental quantity
which we have called interval (confined, however, to timelike intervals).

Our time-sense is not concerned with events outside our brains; it relates
only to the linear chain of events along our own track through the world. We
may learn from another observer similar information as to the time\hyp{}succession
of events along his track. Further we have inanimate observers---clocks---from
which we may obtain similar information as to their local time\hyp{}successions.
The combination of these linear successions along different tracks into a complete
ordering of the events in relation to one another is a problem that
requires careful analysis, and is not correctly solved by the haphazard intuitions
of pre\hyp{}relativity physics. Recognising that both clocks and time-sense measure~$ds$
between pairs of events along their respective tracks, we see that the
problem reduces to that which we have already been studying, viz.\ to pass
from a description in terms of intervals between pairs of events to a description
in terms of coordinates.

The external events which we see appear to fall into our own local
time\hyp{}succession; but in reality it is not the events themselves, but the
sense\hyp{}impressions to which they indirectly give rise, which take place in the
time\hyp{}succession of our consciousness. The popular outlook does not trouble to
discriminate between the external events themselves and the events constituted
by their light\hyp{}impressions on our brains; and hence events throughout the
universe are crudely located in our private time\hyp{}sequence. Through this confusion
the idea has arisen that the instants of which we are conscious extend
so as to include external events, and are world-wide; and the enduring universe
is supposed to consist of a succession of instantaneous states. This crude view
was disproved in 1675 by Römer's celebrated discussion of the eclipses of
Jupiter's satellites; and we are no longer permitted to locate external events
in the instant of our visual perception of them. The whole foundation of the
idea of world-wide instants was destroyed 250~years ago, and it seems strange
that it should still survive in current physics. But, as so often happens, the
theory was patched up although its original \Foreign{raison d'être}\footnotemark\footnotetext
   {ground for existence. (Ed.)}
had vanished.
Obsessed with the idea that the external events had to be put somehow into the
instants of our private consciousness, the physicist succeeded in removing
the pressing difficulties by placing them not in the instant of visual perception
but in a suitable preceding instant. Physics borrowed the idea of world-wide
instants from the rejected theory, and constructed mathematical continuations
of the instants in the consciousness of the observer, making in this way time\hyp{}partitions
throughout the four\hyp{}dimensional world. We need have no quarrel
with this very useful construction which gives physical time. We only insist
that its artificial nature should be recognised, and that the original demand
for a \emph{world-wide} time arose through a mistake. We should probably have
had to invent universal time\hyp{}partitions in any case in order to obtain a complete
mesh\hyp{}system; but it might have saved confusion if we had arrived at it
as a deliberate invention instead of an inherited misconception. If it is found
that physical time has properties which would ordinarily be regarded as contrary
to common sense, no surprise need be felt; this highly technical construct
of physics is not to be confounded with the time of common sense. It is important
for us to discover the exact properties of physical time; but those
properties were put into it by the astronomers who invented it.

It is clear from current debates on the relativity theory that the distinction between the
time of consciousness and the scheme of time in physical and astronomical reckoning is not
always appreciated.
The word ``time'' is in common use for two distinct quantities which are translated into
mathematical language by different symbols $dt$ and $ds$.
They present an important contrast, viz.
\begin{align*}
&\text{$ds$ is an invariant;} & \text{$dt$ is not;}\\
&\text{$dt$ is a perfect differential;} & \text{$ds$ is not;}
\end{align*}
Naturally confusion will arise when we try to answer such ambiguious questions as whether
\emph{time} is absolute or whether two observers have necessarily existed for the same \emph{time}
between two meetings.

Great prominence has been given to the following deduction from the theory, which is an example
of equation~\Eq{(4.9)}.
An observer~$B$ leaves the earth with a velocity about 15~km.\ per~sec. less than the velocity of light;
after a while his motion is suddenly reversed and he returns to the earth.
His journey has lasted 1 year as judged by his consciousness, his physiological growth, or by a chronometer
travelling with him; but he finds that an observer~$A$, who has remained on the earth, has aged 100 years
as judged by similar criteria.
So far there is no real difficulty.
Proper\hyp{}time or ``time lived'' is $ds$; the time of physics and astronomy or ``time represented'' is $dt$.
The world-lines of~$A$ and~$B$ are different tracks which intersect at the beginning and end of the journey,
say at~$P_1$ and~$P_2$.
Since $ds$ is not a perfect differential, $\int_1^2 ds$ will be different for the two tracks, i.e.\ the time
lived will be different.
Moreover since the world-line of an undisturbed observer is such that this integral is a maximum~\Eq{(15.7)}
the time lived by~$A$ is greater than that lived by~$B$ whose motion was disturbed by reversal.
On the other hand~$\int_1^2 dt$ is the same for
both\footnotemark,\footnotetext
    {I.e., for both as \emph{objects observed,} not as \emph{observers} (since $dt$ is not invariant).}
and physical time was purposely introduced in order to have a reckoning which would secure this
consistency.

It is urged, however, that~$B$ is entitled to regard himself as at rest all the time,
and that he will observe~$A$ to have a large velocity relative to him which undergoes a sudden reversal.
From his point of view~$A$ is the disturbed person and ought to have lived the shorter time.
We cannot admit this; disturbance (in the sense here used) is not a question of point of view;
it is absolute.
$B$ could if he wished detect the molecular bombardment or electromagnetic pressure which reversed his motion;
he can learn observationally that it is he who has been disturbed, not~$A$.
But if~$B$ knows that he has undergone an absolute disturbance, is he still entitled to regard himself as at rest?
I do not think we can forbid him, since he is following our own example.
On the surface of the earth we are disturbed by molecular bombardment of the ground,
yet we consider ourselves at rest; whereas an undisturbed stone is considered to be accelerated.
Thus~$B$ may consider~$A$ to be accelerated, but he may not consider him to be disturbed.
It is because the kinematical \emph{acceleration} is not generally coordinated with the physical \emph{disturbance}
that acceleration is relative; if the two were coordinated the disturbance would become an absolute acceleration.

The problem may be modified by supposing that~$B$ reverses his motion by travelling like a comet round a massive star.
In that case both~$A$ and~$B$ have ``undisturbed'' tracks (geodesies), and we cannot immediately predict which will
have lived the longer proper\hyp{}time.
There is, however, no reason to expect their lives to be equal; in particular, there is no support for the idea
that~$B$ must live through the lost 99 years in the brief time ($dt$) occupied by the reversal of his motion.
It is easy to deduce from~\Eq{(38.8)} that the proper time for~$B$ is not appreciably altered by substituting
a gravitational field for a supernatural reversal, so that the conclusions of the elementary theory as to the
respective ages of~$A$ and~$B$ are upheld.

\Section{9.}{The ``$3 + 1$ dimensional'' world}

The constant $c^2$ in~\Eq{(7.1)} is positive according to experiments made in
regions of the world accessible to us. The $3$~minus signs with $1$~plus sign
particularise the world in a way which we could scarcely have predicted from
first principles. H.~Weyl expresses this specialisation by saying that the world
is $3 + 1$ dimensional. Some entertainment may be derived by considering the
properties of a $2 + 2$ or a $4 + 0$ dimensional world. A more serious question
is, Can the world change its type? Is it possible that in making the reduction
of~\Eq{(2.1)} to the sum or difference of squares for some region remote in space or
time, we might have $4$~minus signs? I think not; because if the region exists
it must be separated from our $3 + 1$ dimensional region by some boundary.
On one side of the boundary we have
\[
ds^2 = -dx^2 - dy^2 - dz^2 + c_1^2 dt^2,
\]
and on the other side
\[
ds^2 = -dx^2 - dy^2 - dz^2 - c_2^2 dt^2.
\]
The transition can only occur through a boundary where
\[
ds^2 = -dx^2 - dy^2 - dz^2 + 0\, dt^2,
\]
so that the fundamental velocity is zero. Nothing can move at the boundary,
and no influence can pass from one side to another. The supposed region
beyond is thus not in any spatio\hyp{}temporal relation to our own universe---which
is a somewhat pedantic way of saying that it does not exist.

This barrier is more formidable than that which stops the passage of light
round the world in de~Sitter's spherical space-time (\Title{Space, Time and Gravitation,}
p.~160). The latter stoppage was relative to the space and time of a
distant observer; but everything went on normally with respect to the space
and time of an observer at the region itself. But here we are contemplating
a barrier which does not recede as it is approached.

The passage to a $2 + 2$ dimensional world would occur through a transition
region where
\[
ds^2 = -dx^2 - dy^2 + 0\, dz^2 + c^2 dt^2.
\]
Space here reduces to two dimensions, but there does not appear to be any
\index{Dimensions, world of $3 + 1$}%
barrier. The conditions on the far side, where time becomes two\hyp{}dimensional,
defy imagination.

\Section{10.}{The FitzGerald contraction}
\index{Contraction, FitzGerald}%
\index{FitzGerald contraction}%

We shall now consider some of the consequences deducible from the
Lorentz transformation.
\index{Lorentz transformation}%

The first equation of~\Eq{(5.3)} may be written
\[
x'/\beta = x + ut,
\]
which shows that~$S$, besides making the allowance~$ut$ for the motion of his
origin, divides by~$\beta$ all lengths in the $x$-direction measured by~$S'$. On the
other hand the equation $y' = y$ shows that $S$~accepts $S'$'s~measures in directions
transverse to their relative motion. Let $S'$~take his standard metre
(at rest relative to him, and therefore moving relative to~$S$) and point it first
in the transverse direction~$y'$ and then in the longitudinal direction~$x'$. For~$S'$
its length is $1$~metre in each position, since it is his standard; for~$S$ the
length is $1$~metre in the transverse position and $1/\beta$~metres in the longitudinal
position. Thus $S$~finds that a moving rod contracts when turned from the
transverse to the longitudinal position.

The question remains, How does the length of this moving rod compare
with the length of a similarly constituted rod at rest relative to~$S$? The
answer is that the transverse dimensions are the same whilst the longitudinal
dimensions are contracted. We can prove this by a \Foreign{reductio ad absurdum}.
For suppose that a rod moving transversely were longer than a similar rod at
rest. Take two similar transverse rods $A$ and~$A'$ at rest relatively to~$S$ and
$S'$~respectively. Then $S$~must regard $A'$ as the longer, since it is moving
relatively to him; and $S'$~must regard $A$ as the longer, since it is moving
relatively to him. But this is impossible since, according to the equation
$y = y'$, $S$~and $S'$ agree as to transverse measures.

We see that the Lorentz transformation~\Eq{(5.1)} requires that $(x, y, z, t)$ and
$(x', y', z', t')$ should be measured with standards of identical material constitution,
but moving respectively with $S$ and~$S'$. This was really implicit in our
deduction of the transformation, because the property of the two systems
is that they give the same formula~\Eq{(5.2)} for the interval; and the test of
complete similarity of the standards is equality of all corresponding intervals
occurring in them.

The fourth equation of~\Eq{(5.1)} is
\index{Retardation of moving clocks}%
\[
t = \beta (t' - ux'/c^2).
\]
Consider a clock recording the time~$t'$, which accordingly is at rest in $S'$'s
system ($x' = \text{const.}$). Then for any time-lapse by this clock, we have
\[
\delta t = \beta\, \delta t',
\]
since $\delta x' = 0$. That is to say, $S$~does not accept the time as recorded by this
moving clock, but multiplies its readings by~$\beta$, as though the clock were
going slow. This agrees with the result already found in~\Eq{(4.9)}.

It may seem strange that we should be able to deduce the contraction of
a material rod and the retardation of a material clock from the general
geometry of space and time. But it must be remembered that the contraction
and retardation do not imply any absolute change in the rod and clock. The
``configuration of events'' constituting the four\hyp{}dimensional structure which
we call a rod is unaltered; all that happens is that the observer's space and
time partitions cross it in a different direction.

Further we make no prediction as to what would happen to the rod set
in motion in an actual experiment. There may or may not be an absolute
change of the configuration according to the circumstances by which it is set
in motion. Our results apply to the case in which the rod after being set in
motion is (according to all experimental tests) found to be similar to the rod
in its original state of rest\footnotemark.\footnotetext
  {It may be impossible to change the motion of a rod without causing a rise of temperature.
  Our conclusions will then not apply until the temperature has fallen again, i.e.\ until the temperature\hyp{}test
  shows that the rod is precisely similar to the rod before the change of motion.}

When a number of phenomena are connected together it becomes somewhat
arbitrary to decide which is to be regarded as the explanation of the
others. To many it will seem easier to regard the strange property of
the fundamental velocity as \emph{explained} by these differences of behaviour of
the observers' clocks and scales. They would say that the observers arrive
at the same value of the velocity of light because they omit the corrections
which would allow for the different behaviour of their measuring\hyp{}appliances.
That is the relative point of view, in which the relative quantities, length,
time, etc., are taken as fundamental. From the absolute point of view, which
has regard to intervals only, the standards of the two observers are equal and
behave similarly; the so-called \emph{explanations} of the invariance of the velocity
of light only lead us away from the root of the matter.

Moreover the recognition of the FitzGerald contraction does not enable
us to avoid paradox. From~\Eq{(5.3)} we found that $S'$'s~longitudinal measuring\hyp{}rods
were contracted relatively to those of~$S$. From~\Eq{(5.1)} we can show similarly
that $S$'s~rods are contracted relatively to those of~$S'$. There is complete
reciprocity between $S$ and~$S'$. This paradox is discussed more fully in
\Title{Space, Time and Gravitation,} p.~55.

\Section{11.}{Simultaneity at different places}
\index{Simultaneity at different places}%

It will be seen from the fourth equation of~\Eq{(5.1)}, viz.\
\[
t = \beta (t' - ux'/c^2),
\]
that events at different places which are simultaneous for~$S'$ are not in general
simultaneous for~$S$. In fact, if $dt' = 0$,
\[
dt = -\beta u\, dx'/c^2.
\Tag{(11.1)}
\]

It is of some interest to examine in detail how this difference of reckoning
of simultaneity arises. It has been explained in \SecRef{4} that by convention the
time at two places is compared by transporting a clock from one to the other
\index{Clocks, transport of}%
\index{Transport of clocks}%
with infinitesimal velocity. Our formulae are based on this convention; and,
of course, \Eq{(11.1)}~will only be true if the convention is adhered to. The fact
that infinitesimal velocity relative to~$S'$ is not the same as infinitesimal
velocity relative to~$S$, leaves room for the discrepancy of reckoning of simultaneity
to creep in. Consider two points $A$ and~$B$ at rest relative to~$S'$, and
distant~$x'$ apart. Take a clock at~$A$ and move it gently to~$B$ by giving it an
\index{Electromagnetic action!signals}%
infinitesimal velocity~$du'$ for a time~$x'/du'$. Owing to the motion, the clock
will by~\Eq{(4.9)} be retarded in the ratio $(1 - du'^{2}/c^2)^{-\frac{1}{2}}$; this continues for a time~$x'/du'$
and the total loss is thus
\[
\bigl\{1 - (1 - du'^{2}/c^2)^{\frac{1}{2}}\bigr\} x'/du',
\]
which tends to zero when $du'$~is infinitely small. $S'$~may accordingly accept
the result of the comparison without applying any correction for the motion
of the clock.

Now consider $S$'s~view of this experiment. For him the clock had already
a velocity~$u$, and accordingly the time indicated by the clock is only $(1 - u^2/c^2)^{\frac{1}{2}}$
of the true time for~$S$. By differentiation, an additional velocity~$du$\footnote
  {Note that $du$ will not be equal to~$du'$.}
causes
a supplementary loss
\[
(1 - u^2/c^2)^{-\frac{1}{2}} u\, du/c^2 \text{ clock seconds}
\Tag{(11.2)}
\]
per true second. Owing to the FitzGerald contraction of the length~$AB$, the
distance to be travelled is~$x'/\beta$, and the journey will occupy a time
\[
x'/\beta\, du \text{ true seconds}.
\Tag{(11.3)}
\]
Multiplying \Eq{(11.2)} and \Eq{(11.3)}, the total loss due to the journey is
\[
ux'/c^2 \text{ clock seconds,}
\]
or
\[
\beta ux'/c^2 \text{ true seconds for~$S$.}
\Tag{(11.4)}
\]

Thus, whilst $S'$~accepts the uncorrected result of the comparison, $S$~has to
apply a correction~$\beta ux'/c^2$ for the disturbance of the chronometer through
transport. This is precisely the difference of their reckonings of simultaneity
given by~\Eq{(11.1)}.

In practice an accurate comparison of time at different places is made,
not by transporting chronometers, but by electromagnetic signals---usually
wireless time\hyp{}signals for places on the earth, and light\hyp{}signals for places in
the solar system or stellar universe. Take two clocks at $A$ and~$B$, respectively.
Let a signal leave~$A$ at clock-time~$t_{1}$, reach~$B$ at time~$t_{B}$ by the clock at~$B$,
and be reflected to reach $A$ again at time~$t_{2}$. The observer~$S'$, who is at rest
relatively to the clocks, will conclude that the instant~$t_{B}$ at~$B$ was simultaneous
with the instant $\frac{1}{2}(t_{1} + t_{2})$ at~$A$, because he assumes that the forward
velocity of light is equal to the backward velocity. But for~$S$ the two clocks
are moving with velocity~$u$; therefore he calculates that the outward journey
will occupy a time $x/(c - u)$ and the homeward journey a time $x/(c + u)$. Now
\begin{align*}
  \frac{x}{c - u} &= \frac{x(c + u)}{c^2 - u^2} = \frac{\beta^2x}{c^2}(c + u),\displaybreak[0] \\
  \frac{x}{c + u} &= \frac{x(c - u)}{c^2 - u^2} = \frac{\beta^2x}{c^2}(c + u).
  \end{align*}
Thus the instant~$t_{B}$ of arrival at~$B$ must be taken as $\beta^2xu/c^2$ later than the
half-way instant $\frac{1}{2}(t_{1} + t_{2})$. This correction applied by~$S$, but not by~$S'$, agrees
with~\Eq{(11.4)} when we remember that owing to the FitzGerald contraction
$x = x'/\beta$.

Thus the same difference in the reckoning of simultaneity by $S$ and~$S'$
appears whether we use the method of transport of clocks or of light\hyp{}signals.
In either case a convention is introduced as to the reckoning of time\hyp{}differences
\index{Time!convention in reckoning}%
at different places; this convention takes in the two methods the alternative
forms---

(1) A clock moved with infinitesimal velocity from one place to another
continues to read the correct time at its new station, \emph{or}

(2) The forward velocity of light along any line is equal to the backward
velocity\footnotemark.\footnotetext
  {The chief case in which we require for practical purposes an accurate convention as to the
  reckoning of time at places distant from the earth, is in calculating the elements and mean
  places of planets and comets. In these computations the velocity of light in any direction is taken
  to be 300,000~km.\ per~sec., an assumption which rests on the convention~(2). All experimental
methods of measuring the velocity of light determine only an average to-and-fro velocity.}

Neither statement is by itself a statement of observable fact, nor does it
refer to any intrinsic property of clocks or of light; it is simply an announcement
of the rule by which we propose to extend fictitious time\hyp{}partitions
through the world. But the mutual agreement of the two statements is a fact
which could be tested by observation, though owing to the obvious practical
difficulties it has not been possible to verify it directly. We have here given
a theoretical proof of the agreement, depending on the truth of the fundamental
axiom of \SecRef{1}.

The two alternative forms of the convention are closely connected. In
general, in any system of time\hyp{}reckoning, a change~$du$ in the velocity of a
clock involves a change of rate proportional to~$du$, but there is a certain
turning\hyp{}point for which the change of rate is proportional to~$du^2$. In adopting
a time\hyp{}reckoning such that this stationary point corresponds to his own
motion, the observer is imposing a symmetry on space and time with respect
to himself, which may be compared with the symmetry imposed in assuming
a constant velocity of light in all directions. Analytically we imposed the
same general symmetry by adopting \Eq{(4.6)} instead of~\Eq{(4.7)} as the form for~$ds^2$,
making our space-time reckoning symmetrical with respect to the interval
and therefore with respect to all observational criteria.

\Section{12.}{Momentum and mass}
\index{Momentum!elementary treatment}%

Besides possessing extension in space and time, matter possesses inertia.
\index{Inertia!elementary treatment}%
We shall show in due course that \emph{inertia, like extension, is expressible in terms
of the interval relation}; but that is a development belonging to a later stage
of our theory. Meanwhile we give an elementary treatment based on the
empirical laws of conservation of momentum and energy rather than on any
deep-seated theory of the nature of inertia.

For the discussion of space and time we have made use of certain ideal
apparatus which can only be imperfectly realised in practice---rigid scales and
perfect cyclic mechanisms or clocks, which always remain similar configurations
from the absolute point of view. Similarly for the discussion of inertia
we require some ideal material object, say a perfectly elastic billiard ball, whose
condition as regards inertial properties remains constant from an absolute
point of view. The difficulty that actual billiard balls are not perfectly elastic
must be surmounted in the same way as the difficulty that actual scales are
not rigid. To the ideal billiard ball we can affix a constant number, called
the \emph{invariant mass}\footnotemark,\footnotetext
  {Or \emph{proper\hyp{}mass}.}
which will denote its absolute inertial properties; and
\index{Invariant}%
\index{Invariant mass}%
\index{Mass!invariant and relative}%
\index{Mass!variation with velocity}%
this number is supposed to remain unaltered throughout the vicissitudes of
its history, or, if temporarily disturbed during a collision, is restored at the
times when we have to examine the state of the body.

With the customary definition of momentum, the components
\[
M\, \frac{dx}{dt},\quad
M\, \frac{dy}{dt},\quad
M\, \frac{dz}{dt}
\Tag{(12.1)}
\]
cannot satisfy a general law of conservation of momentum unless the mass~$M$
\index{Conservation!of momentum and mass}%
is allowed to vary with the velocity. But with the slightly modified definition
\[
m\, \frac{dx}{ds},\quad
m\, \frac{dy}{ds},\quad
m\, \frac{dz}{ds}
\Tag{(12.2)}
\]
the law of conservation can be satisfied simultaneously in all space-time
systems, $m$~being an invariant number. This was shown in \Title{Space, Time and Gravitation,} p.~142.

Comparing \Eq{(12.1)} and \Eq{(12.2)}, we have
\[
M = m\, \frac{dt}{ds}.
\Tag{(12.3)}
\]
We call $m$ the \emph{invariant mass,} and $M$ the \emph{relative mass,} or simply the \emph{mass}.

The term ``invariant'' signifies unchanged for any transformation of
coordinates, and, in particular, the same for all observers; constancy during
the life\hyp{}history of the body is an additional property of~$m$ attributed to our
ideal billiard balls, but not assumed to be true for matter in general.

Choosing units of length and time so that the velocity of light is unity,
we have by~\Eq{(7.2)}
\[
\frac{ds}{dt} = (1 - v^2)^{\frac{1}{2}}.
\]
Hence by \Eq{(12.3)}
\[
M = m(1 - v^2)^{-\frac{1}{2}}.
\Tag{(12.4)}
\]
The mass increases with the velocity by the same factor as that which gives
the FitzGerald contraction; and when $v = 0$, $M = m$. The invariant mass is
thus equal to the mass at rest.

It is natural to extend~\Eq{(12.2)} by adding a fourth component, thus
\[
m\, \frac{dx}{ds},\quad
m\, \frac{dy}{ds},\quad
m\, \frac{dz}{ds},\quad
m\, \frac{dt}{ds}.
\Tag{(12.5)}
\]
By \Eq{(12.3)} the fourth component is equal to~$M$. Thus the momenta and mass
(relative mass) form together a symmetrical expression, the momenta being
space\hyp{}components, and the mass the time\hyp{}component. We shall see later that
the expression~\Eq{(12.5)} constitutes a vector, and the laws of conservation of
momentum and mass assert the conservation of this vector.

The following is an analytical proof of the law of variation of mass with
velocity directly from the principle of conservation of mass and momentum.
Let $M_{1}$,~$M_{1}'$ be the mass of a body as measured by $S$ and $S'$ respectively,
$v_{1}$,~$v_{1}'$ being its velocity in the $x$-direction. Writing
\[
\beta_{1} = (1 - v_{1}^{2}/c^2)^{-\frac{1}{2}},\quad
\beta_{1}' = (1 - v_{1}'^{2}/c^2)^{-\frac{1}{2}},\quad
\beta = (1 - u^2/c^2)^{-\frac{1}{2}},
\]
we can easily verify from~\Eq{(6.2)} that
\[
\beta_{1}v_{1} = \beta\beta_{1}'(v_{1}' - u).
\Tag{(12.6)}
\]

Let a number of such particles be moving in a straight line subject to the
conservation of mass and momentum as measured by~$S'$, viz.\
\[
\sum M_{1}' \quad\text{and}\quad \sum M_{1}' v_{1}' \quad\text{are conserved.}
\]
Since $\beta$ and $u$ are constants it follows that
\[
\sum M_{1}' \beta (v_{1}' - u)\quad\text{is conserved.}
\]
Therefore by \Eq{(12.6)}
\[
\sum M_{1}' \beta_{1} v_{1}/\beta_{1}'\quad\text{is conserved.}
\Tag{(12.71)}
\]
But since momentum must also be conserved for the observer~$S$
\[
\sum M_{1} v_{1}\quad\text{is conserved.}
\Tag{(12.72)}
\]
The results \Eq{(12.71)} and \Eq{(12.72)} will agree if
\[
M_{1}/\beta_{1} = M_{1}'/\beta_{1}',
\]
and it is easy to see that there can be no other general solution. Hence for
different values of~$v_{1}$, $M_{1}$~is proportional to~$\beta_{1}$, or
\[
M = m(1 - v^2/c^2)^{-\frac{1}{2}},
\]
where $m$~is a constant for the body.

It requires a greater impulse to produce a given change of velocity~$\delta v$ in
the original direction of motion than to produce an equal change~$\delta w$ at right
angles to it. For the momenta in the two directions are initially
\[
mv(1 - v^2/c^2)^{-\frac{1}{2}},\quad 0,
\]
and after a change $\delta v$, $\delta w$, they become
%[** TN: Not broken in the original]
\begin{gather*}
m(v + \delta v) \bigl[1 - \bigl\{(v + \delta v)^{2} + (\delta w)^{2}\bigr\}/c^2\bigr]^{-\frac{1}{2}},\displaybreak[0] \\
m\, \delta w \bigl[1 - \bigl\{(v + \delta v)^{2} + (\delta w)^{2}\bigr\}/c^2\bigr]^{-\frac{1}{2}}.
\end{gather*}
Hence to the first order in $\delta v$, $\delta w$ the changes of momentum are
\[
m(1 - v^2/c^2)^{-\frac{3}{2}}\, \delta v,\quad
m(1 - v^2/c^2)^{-\frac{1}{2}}\, \delta w,
\]
or
\[
M\beta^2\, \delta v,\quad
M\, \delta w,
\]
where $\beta$~is the FitzGerald factor for velocity~$v$. The coefficient $M\beta^2$ was
formerly called the \emph{longitudinal mass,} $M$~being the \emph{transverse mass}; but the
\index{Longitudinal mass}%
longitudinal mass is of no particular importance in the general theory, and
the term is dropping out of use.

\Section{13.}{Energy}

When the units are such that $c = 1$, we have
\begin{align*}
  M &= m(1 - v^2)^{-\frac{1}{2}} \\
    &= m + \tfrac{1}{2} mv^2 \text{ approximately,}
  \Tag{(13.1)}
\end{align*}
if the speed is small compared with the velocity of light. The second term is
the kinetic energy, so that the change of mass is the same as the change of
\index{Energy, identified with mass}%
\index{Mass!identified with energy}%
energy, when the velocity alters. This suggests the identification of mass with
energy. It may be recalled that in mechanics the total energy of a system
is left vague to the extent of an arbitrary additive constant, since only changes
of energy are defined. In identifying energy with mass we fix the additive
constant~$m$ for each body, and $m$~may be regarded as the internal energy of
constitution of the body.

The approximation used in~\Eq{(13.1)} does not invalidate the argument.
Consider two ideal billiard balls colliding. The conservation of mass (relative
\index{Conservation!of energy}%
mass) states that
\[
\sum m (1 - v^2)^{-\frac{1}{2}} \text{ is unaltered.}
\]
The conservation of energy states that
\[
\sum m (1 + \tfrac{1}{2}v^2) \text{ is unaltered.}
\]
But if both statements were exactly true we should have two equations
determining unique values of the speeds of the two balls; so that these speeds
could not be altered by the collision. The two laws are not independent, but
one is an approximation to the other. The first is the accurate law since it is
independent of the space-time frame of reference. Accordingly the expression
$\frac{1}{2}mv^2$ for the kinetic energy in elementary mechanics is only an approximation
in which terms in~$v^{4}$, etc.\ are neglected.

When the units of length and time are not restricted by the condition
$c = 1$, the relation between the mass~$M$ and the energy~$E$ is
\[
M = E/c^2.
\Tag{(13.2)}
\]
Thus the energy corresponding to a gram is $9 \cdot 10^{20}$ ergs. This does not
affect the identity of mass and energy---that both are measures of the same
world\hyp{}condition. A world\hyp{}condition can be examined by different kinds of
experimental tests, and the units gram and erg are associated with different
tests of the mass\hyp{}energy condition. But when once the measure has been
made it is of no consequence to us which of the experimental methods was
chosen, and grams or ergs can be used indiscriminately as the unit of mass.
In fact, measures made by energy\hyp{}tests and by mass-tests are convertible like
measures made with a yard-rule and a metre-rule.

The principle of conservation of mass has thus become merged in the
principle of conservation of energy. But there is another independent phenomenon
which perhaps corresponds more nearly to the original idea of Lavoisier
when he enunciated the law of conservation of matter. I refer to the permanence
of \emph{invariant mass} attributed to our ideal billiard balls but not
supposed to be a general property of matter. The conservation of~$m$ is an
\index{Conservation!of matter}%
\index{Matter!conservation of}%
accidental property like rigidity; the conservation of~$M$ is an invariable law
of nature.

When radiant heat falls on a billiard ball so that its temperature rises,
the increased energy of motion of the molecules causes an increase of mass~$M$.
The invariant mass~$m$ also increases since it is equal to~$M$ for a body at rest.
There is no violation of the conservation of~$M$, because the radiant heat has
mass~$M$ which it transfers to the ball; but we shall show later that the
electromagnetic waves have no invariant mass, and the addition to~$m$ is
created out of nothing. Thus invariant mass is not conserved in general.

To some extent we can avoid this failure by taking the microscopic point
of view. The billiard ball can be analysed into a very large number of constituents---electrons
and protons---each of which is believed to preserve the
same invariant mass for life. But the invariant mass of the billiard ball is
not exactly equal to the sum of the invariant masses of its constituents\footnotemark.\footnotetext
  {This is because the invariant mass of each electron is its relative mass referred to axes
  moving with it; the invariant mass of the billiard ball is the relative mass referred to axes at rest
  in the billiard ball as a whole.}
The permanence and permanent similarity of all electrons seems to be the
modern equivalent of Lavoisier's ``conservation of matter.'' It is still uncertain
whether it expresses a universal law of nature; and we are willing to contemplate
the possibility that occasionally a positive and negative electron
may coalesce and annul one another. In that case the mass~$M$ would pass
into the electromagnetic waves generated by the catastrophe, whereas the
invariant mass~$m$ would disappear altogether. Again if ever we are able to
synthesise helium out of hydrogen, 0.8\% of the invariant mass will
be annihilated, whilst the corresponding proportion of relative mass will be
liberated as radiant energy.

It will thus be seen that although in the special problems considered the
quantity~$m$ is usually supposed to be permanent, its conservation belongs to
an altogether different order of ideas from the universal conservation of~$M$.

\Section{14.}{Density and temperature}
\index{Density!Lorentz transformation of}%

Consider a volume of space delimited in some invariant way, e.g.\ the
content of a material box. The counting of a number of discrete particles
continually within (i.e.\ moving with) the box is an absolute operation; let
the absolute number be~$N$. The volume~$V$ of the box will depend on the
space\hyp{}reckoning, being decreased in the ratio~$\beta$ for an observer moving
relatively to the box and particles, owing to the FitzGerald contraction of one
of the dimensions of the box. Accordingly the particle\hyp{}density $\sigma = N/V$
satisfies
\[
\sigma' = \sigma\beta,
\Tag{(14.1)}
\]
where $\sigma'$~is the particle\hyp{}density for an observer in relative motion, and $\sigma$~the
particle\hyp{}density for an observer at rest relative to the particles.

It follows that the mass\hyp{}density~$\rho$ obeys the equation
\[
\rho' = \rho\beta^2,
\Tag{(14.2)}
\]
since the mass of each particle is increased for the moving observer in the
ratio~$\beta$.

Quantities referred to the space-time system of an observer moving with
\index{Proper-(prefix)}%
the body considered are often distinguished by the prefix \emph{proper-} (German,
\emph{Eigen-}), e.g.\ proper\hyp{}length, proper\hyp{}volume, proper\hyp{}density, proper\hyp{}mass $=$~invariant
mass.

The transformation of temperature for a moving observer does not often
\index{Temperature}%
concern us. In general the word obviously means proper\hyp{}temperature, and
the motion of the observer does not enter into consideration. In thermometry
and in the theory of gases it is essential to take a standard with respect to
which the matter is at rest on the average, since the indication of a thermometer
moving rapidly through a fluid is of no practical interest. But
thermodynamical temperature is defined by
\[
dS = dM/T,
\Tag{(14.3)}
\]
where $dS$~is the change of entropy for a change of energy~$dM$. The temperature~$T$
\index{Entropy}%
defined by this equation will depend on the observer's frame of
reference. Entropy is clearly meant to be an invariant, since it depends on
the probability of the statistical state of the system compared with other
states which might exist. Hence $T$~must be altered by motion in the same
way as~$dM$, that is to say
\[
T' = \beta T.
\Tag{(14.4)}
\]
But it would be useless to apply such a transformation to the adiabatic gas\hyp{}equation
\[
T = k\rho^{\gamma-1},
\]
for, in that case, $T$~is evidently intended to signify the proper\hyp{}temperature and
$\rho$~the proper\hyp{}density.

In general it is unprofitable to apply the Lorentz transformation to the
\emph{constitutive equations} of a material medium and to coefficients occurring in
\index{Constitutive equations}%
them (permeability, specific inductive capacity, elasticity, velocity of sound).
Such equations naturally take a simpler and more significant form for axes
moving with the matter. The transformation to moving axes introduces great
complications without any evident advantages, and is of little interest except
as an analytical exercise.

\Section{15.}{General transformations of coordinates}
\index{Coordinates!general transformation of}%

We obtain a transformation of coordinates by taking new coordinates
\index{Transformation of coordinates!general}%
$x_{1}'$, $x_{2}'$, $x_{3}'$, $x_{4}'$ which are any four functions of the old coordinates $x_{1}$, $x_{2}$, $x_{3}$, $x_{4}$.
Conversely, $x_{1}$, $x_{2}$, $x_{3}$, $x_{4}$ are functions of $x_{1}'$, $x_{2}'$, $x_{3}'$,~$x_{4}'$. It is assumed that
multiple values are excluded, at least in the region considered, so that values
of $(x_{1}, x_{2}, x_{3}, x_{4})$ and $(x_{1}', x_{2}', x_{3}', x_{4}')$ correspond one to one.

If
%[** TN: First group not broken in the original]
\begin{gather*}
  x_{1} = f_{1}(x_{1}', x_{2}', x_{3}', x_{4}');\quad
  x_{2} = f_{2}(x_{1}', x_{2}', x_{3}', x_{4}'); \text{ etc.,} \\
  dx_{1} = \frac{\dd f_{1}}{\dd x_{1}'}\, dx_{1}'
        + \frac{\dd f_{1}}{\dd x_{2}'}\, dx_{2}'
        + \frac{\dd f_{1}}{\dd x_{3}'}\, dx_{3}'
        + \frac{\dd f_{1}}{\dd x_{4}'}\, dx_{4}'; \text{ etc.,}
  \Tag{(15.1)}
\end{gather*}
or it may be written simply,
\[
dx_{1} = \frac{\dd x_{1}}{\dd x_{1}'}\, dx_{1}'
  + \frac{\dd x_{1}}{\dd x_{2}'}\, dx_{2}'
  + \frac{\dd x_{1}}{\dd x_{3}'}\, dx_{3}'
  + \frac{\dd x_{1}}{\dd x_{4}'}\, dx_{4}'; \text{ etc.,}
  \Tag{(15.2)}
\]
Substituting from \Eq{(15.2)} in~\Eq{(2.1)} we see that $ds^2$~will be a homogeneous
quadratic function of the differentials of the new coordinates; and the new
coefficients $g_{11}'$, $g_{22}'$, etc.\ could be written down in terms of the old, if desired.

For an example consider the usual transformation to axes revolving with
\index{Rotating axes, quadratic form for}%
constant angular velocity~$\omega$, viz.
\[
\left.
\begin{aligned}
  x &= x_{1}' \cos \omega x_{4}' - x_{2}' \sin \omega x_{4}' \\
  y &= x_{1}' \sin \omega x_{4}' + x_{2}' \cos \omega x_{4}' \\
  z &= x_{3}' \\
  t &= x_{4}'
\end{aligned}
\right\}.
\Tag{(15.3)}
\]
Hence
\begin{align*}
  dx &= dx_1' \cos \omega x_4' - dx_2' \sin \omega x_4'
  + \omega(-x_1' \sin \omega x_4' - x_2' \cos \omega x_4')\, dx_4', \\
  dy &= dx_1' \sin \omega x_4' + dx_2' \cos \omega x_4'
  + \omega( x_1' \cos \omega x_4' - x_2' \sin \omega x_4')\, dx_4', \\
  dz &= dx_3', \\
  dt &= dx_4'.
\end{align*}
Taking units of space and time so that $c = 1$, we have for our original fixed
coordinates by~\Eq{(7.1)}
\[
ds^2 = -dx^2 - dy^2 - dz^2 + dt^2.
\]
Hence, substituting the values found above,
\begin{multline*}
  ds^2 = - dx_1'^2 - dx_2'^2 - dx_3'^2 + \bigl\{1 - \omega^2 (x_1'^2 + x_2'^2)\bigr\}\, dx_4'^2 \\
  + 2\omega x_2'\, dx_1'\, dx_4' - 2\omega x_1'\, dx_2'\, dx_4'.
  \Tag{(15.4)}
\end{multline*}

Remembering that all observational differences of coordinate\hyp{}systems must
arise \Foreign{via} the interval, this formula must comprise everything which distinguishes
the rotating system from a fixed system of coordinates.

In the transformation~\Eq{(15.3)} we have paid no attention to any contraction
of the standards of length or retardation of clocks due to motion with the
rotating axes. The formulae of transformation are those of elementary
kinematics, so that $x_1'$, $x_2'$, $x_3'$, $x_4'$ are quite strictly the coordinates used in
the ordinary theory of rotating axes. But it may be suggested that elementary
kinematics is now seen to be rather crude, and that it would be worth while
to touch up the formulae~\Eq{(15.3)} so as to take account of these small changes
of the standards. A little consideration shows that the suggestion is impracticable.
It was shown in \SecRef{4} that if $x_1'$, $x_2'$, $x_3'$, $x_4'$ represent rectangular
coordinates and time as partitioned by direct readings of scales and clocks, then
\[
ds^2 = -dx_1'^2 - dx_2'^2 - dx_3'^2 + c^2 dx_4'^2,
\Tag{(15.45)}
\]
so that coordinates which give any other formula for the interval cannot
represent the immediate indications of scales and clocks. As shown at the
end of \SecRef{5}, the only transformations which give \Eq{(15.45)} are Lorentz transformations.
If we wish to make a transformation of a more general kind, such
as that of~\Eq{(15.3)}, we must necessarily abandon the association of the coordinate\hyp{}system
with uncorrected scale and clock readings. It is useless to try to
``improve'' the transformation to rotating axes, because the supposed improvement
could only lead us back to a coordinate\hyp{}system similar to the fixed
axes with which we started.

The inappropriateness of rotating axes to scale and clock measurements
can be regarded from a physical point of view. We cannot keep a scale or
clock at rest in the rotating system unless we constrain it, i.e.\ subject it to
molecular bombardment---an ``outside influence'' whose effect on the measurements
must not be ignored.

In the $x$, $y$, $z$, $t$ system of coordinates the scale and clock are the natural
equipment for exploration. In other systems they will, if unconstrained, continue
to measure~$ds$; but the reading of~$ds$ is no longer related in a simple
way to the differences of coordinates which we wish to determine; it depends
on the more complicated calculations involved in~\Eq{(2.1)}. The scale and clock
to some extent lose their pre\hyp{}eminence, and since they are rather elaborate
appliances it may be better to refer to some simpler means of exploration.
We consider then two simpler test\hyp{}objects---the moving particle and the
\index{Particle!motion of}%
\index{Track of moving particle and light-pulse}%
light-pulse.

In ordinary rectangular coordinates and time $x$,~$y$,~$z$, $t$ an undisturbed
particle moves with uniform velocity, so that its track is given by the
equations
\[
x = a + bt,\quad
y = c + dt,\quad
z = e + ft,
\Tag{(15.5)}
\]
i.e.\ the equations of a straight line in four dimensions. By substituting from~\Eq{(15.3)}
we could find the equations of the track in rotating coordinates; or by
substituting from~\Eq{(15.2)} we could obtain the differential equations for any
desired coordinates. But there is another way of proceeding. The differential
equations of the track may be written
\[
\frac{d^2 x}{ds^2},\quad
\frac{d^2 y}{ds^2},\quad
\frac{d^2 z}{ds^2},\quad
\frac{d^2 t}{ds^2} = 0,
\Tag{(15.6)}
\]
which on integration, having regard to the condition~\Eq{(7.1)}, give equations~\Eq{(15.5)}.

The equations~\Eq{(15.6)} are comprised in the single statement
\[
\int ds \text{ is stationary}
\Tag{(15.7)}
\]
for all arbitrary small variations of the track which vanish at the initial and
final limits---a well-known property of the straight line.

In arriving at~\Eq{(15.7)} we use freely the geometry of the $x$, $y$, $z$, $t$ system
\index{Abstract geometry and natural geometry}%
\index{Geometry, Riemannian!abstract and natural}%
given by~\Eq{(7.1)}; but the final result does not allude to coordinates at all, and
must be unaltered whatever system of coordinates we are using. To obtain
explicit equations for the track in any desired system of coordinates, we
substitute in~\Eq{(15.7)} the appropriate expression~\Eq{(2.1)} for~$ds$ and apply the
calculus of variations. The actual analysis will be given in \SecRef{28}.

The track of a light-pulse, being a straight line in four dimensions, will
\index{Light-pulse!equation of track}%
also satisfy~\Eq{(15.7)}; but the light-pulse has the special velocity~$c$ which gives
the additional condition obtained in \SecRef{7}, viz.
\[
ds = 0.
\Tag{(15.8)}
\]
Here again there is no reference to any coordinates in the final result.

We have thus obtained equations \Eq{(15.7)} and \Eq{(15.8)} for the behaviour of
the moving particle and light-pulse which must hold good whatever the
coordinate\hyp{}system chosen. The indications of our two new test\hyp{}bodies are
connected with the interval, just as in \SecRef{3} the indications of the scale and
clock were connected with the interval. It should be noticed however that
whereas the use of the older test\hyp{}bodies depends only on the truth of the
fundamental axiom, the use of the new test\hyp{}bodies depends on the truth of the
empirical laws of motion and of light\hyp{}propagation. In a deductive theory this
appeal to empirical laws is a blemish which we must seek to remove later.

\Section{16.}{Fields of force}
\index{Fields of force}%

Suppose that an observer has chosen a definite system of space\hyp{}coordinates
and of time\hyp{}reckoning $(x_1, x_2, x_3, x_4)$ and that the geometry of these is given by
\[
ds^2 = g_{11}\, dx_1^2 + g_{22}\, dx_2^2 + \ldots + 2g_{12}\, dx_1\, dx_2 + \ldots.
\Tag{(16.1)}
\]
Let him be under the mistaken impression that the geometry is
\[
ds_0^2 = -dx_1^2 - dx_2^2 - dx_3^2 + dx_4^2,
\Tag{(16.2)}
\]
that being the geometry with which he is most familiar in pure mathematics.
We use~$ds_{0}$ to distinguish his mistaken value of the interval. Since intervals
can be compared by experimental methods, he ought soon to discover that his~$ds$
cannot be reconciled with observational results, and so realise his mistake.
But the mind does not so readily get rid of an obsession. It is more likely
that our observer will continue in his opinion, and attribute the discrepancy
of the observations to some influence which is present and affects the behaviour
of his test\hyp{}bodies. He will, so to speak, introduce a supernatural agency
which he can blame for the consequences of his mistake. Let us examine
what name he would apply to this agency.

Of the four test\hyp{}bodies considered the moving particle is in general the
most sensitive to small changes of geometry, and it would be by this test that
the observer would first discover discrepancies. The path laid down for it by
our observer is
\[
\int ds_0 \text{ is stationary,}
\]
i.e.\ a straight line in the coordinates $(x_1, x_2, x_3, x_4)$. The particle, of course,
pays no heed to this, and moves in the different track
\[
\int ds \text{ is stationary.}
\]
Although apparently undisturbed it deviates from ``uniform motion in a
straight line.'' The name given to any agency which causes deviation from
uniform motion in a straight line is \emph{force} according to the Newtonian definition
of force. Hence the agency invoked through our observer's mistake is described
as a ``field of force.''

The field of force is not always introduced by inadvertence as in the foregoing
illustration. It is sometimes introduced deliberately by the mathematician,
e.g.\ when he introduces the centrifugal force. There would be little
\index{Centrifugal force}%
advantage and many disadvantages in banishing the phrase ``field of force''
from our vocabulary. We shall therefore regularise the procedure which our
observer has adopted. We call~\Eq{(16.2)} the \emph{abstract geometry} of the system of
coordinates $(x_1, x_2, x_3, x_4)$; it may be chosen arbitrarily by the observer. The
\index{Natural coordinates!geometry}%
\emph{natural geometry} is~\Eq{(16.1)}.

\emph{A field of force represents the discrepancy between the natural geometry of
a coordinate\hyp{}system and the abstract geometry arbitrarily ascribed to it.}

A field of force thus arises from an attitude of mind. If we do not take
our coordinate\hyp{}system to be something different from that which it really is,
there is no field of force. If we do not regard our rotating axes as though
they were non\hyp{}rotating, there is no centrifugal force.

Coordinates for which the natural geometry is
\[
ds^2 = - dx_1^2 - dx_2^2 - dx_3^2 + dx_4^2
\]
are called Galilean coordinates. They are the same as those we have hitherto
\index{Coordinate\hyp{}systems!Galilean}%
\index{Galilean coordinates}%
called ordinary rectangular coordinates and time (the velocity of light being
unity). Since this geometry is familiar to us, and enters largely into current
conceptions of space, time and mechanics, we usually choose Galilean geometry
when we have to ascribe an abstract geometry. Or we may use slight modifications
of it, e.g.\ substitute polar for rectangular coordinates.

It has been shown in \SecRef{4} that when the $g$'s are constants, coordinates can
be chosen so that Galilean geometry is actually the natural geometry. There
is then no need to introduce a field of force in order to enjoy our accustomed
outlook; and if we deliberately choose non\hyp{}Galilean coordinates and attribute
to them abstract Galilean geometry, we recognise the artificial character of
the field of force introduced to compensate the discrepancy. But in the more
general case it is not possible to make the reduction of \SecRef{4} accurately throughout
the region explored by our experiments; and no Galilean coordinates
exist. In that case it has been usual to select some system (preferably an
approximation to a Galilean system) and ascribe to it the abstract geometry
of the Galilean system. The field of force so introduced is called ``Gravitation.''
\index{Gravitation}%

It should be noticed that the rectangular coordinates and time in current
\index{Time!extended meaning}%
use can scarcely be regarded as a close approximation to the Galilean system,
since the powerful force of terrestrial gravitation is needed to compensate
the error.

The naming of coordinates (e.g.\ time) usually follows the \emph{abstract geometry}
attributed to the system. In general the natural geometry is of some complicated
kind for which no detailed nomenclature is recognised. Thus when we
call a coordinate~$t$ the ``time,'' we may either mean that it fulfils the
observational conditions discussed in \SecRef{4}, or we may mean that any departure
from those conditions will be ascribed to the interference of a field of force.
In the latter case ``time'' is an arbitrary name, useful because it fixes a
consequential nomenclature of velocity, acceleration, etc.

To take a special example, an observer at a station on the earth has found
a particular set of coordinates $x_1$, $x_2$, $x_3$, $x_4$ best suited to his needs. He calls
them $x$, $y$, $z$, $t$ in the belief that they are actually rectangular coordinates and
time, and his terminology---straight line, circle, density, uniform velocity, etc.---follows
from this identification. But, as shown in \SecRef{4}, this nomenclature can
only agree with the measures made by clocks and scales provided \Eq{(16.2)}~is
satisfied; and if \Eq{(16.2)}~is satisfied, the tracks of undisturbed particles must be
straight lines. Experiment immediately shows that this is not the case; the
tracks of undisturbed particles are parabolas. But instead of accepting the
verdict of experiment and admitting that $x_1$, $x_2$, $x_3$, $x_4$ are not what he supposed
they were, our observer introduces a field of force to explain why his
test is not fulfilled. A certain part of this field of force might have been
avoided if he had taken originally a different set of coordinates (not rotating
with the earth); and in so far as the field of force arises on this account it is
generally recognised that it is a mathematical fiction---the centrifugal force.
But there is a residuum which cannot be got rid of by any choice of coordinates;
there exists no extensive coordinate\hyp{}system having the simple
properties which were ascribed to $x$, $y$, $z$,~$t$. The intrinsic nature of space\hyp{}time
near the earth is not of the kind which admits coordinates with Galilean
geometry. This irreducible field of force constitutes the field of terrestrial
gravitation. The statement that space\hyp{}time round the earth is ``curved''---that
is to say, that it is not of the kind which admits Galilean coordinates---is
not an hypothesis; it is an equivalent expression of the observed fact that
an irreducible field of force is present, having regard to the Newtonian
definition of force. It is this fact of observation which demands the introduction
of non\hyp{}Galilean space\hyp{}time and non\hyp{}Euclidean space into the theory.

\Section{17.}{The Principle of Equivalence}

In \SecRef{15} we have stated the laws of motion of undisturbed material particles
and of light\hyp{}pulses in a form independent of the coordinates chosen. Since
a great deal will depend upon the truth of these laws it is desirable to
consider what justification there is for believing them to be both accurate
and universal. Three courses are open:

\Item{(a)} It will be shown in Chapters \ChapNum{IV} and \ChapNum{VI} that these laws follow
rigorously from a more fundamental discussion of the nature of matter and
of electromagnetic fields; that is to say, the hypotheses underlying them may
be pushed a stage further back.

\Item{(b)} The track of a moving particle or light-pulse under specified initial
conditions is unique, and it does not seem to be possible to specify any
unique tracks in terms of intervals only other than those given by equations
\Eq{(15.7)} and~\Eq{(15.8)}.

\Item{(c)} We may arrive at these laws by induction from experiment.

If we rely solely on experimental evidence we cannot claim exactness for
the laws. It goes without saying that there always remains a possibility of
small amendments of the laws too slight to affect any observational tests yet
tried. Belief in the perfect accuracy of \Eq{(15.7)} and \Eq{(15.8)} can only be justified
on the theoretical grounds \Item{(a)} or~\Item{(b)}. But the more important consideration
is the universality, rather than the accuracy, of the experimental laws; we
have to guard against a spurious generalisation extended to conditions
intrinsically dissimilar from those for which the laws have been established
observationally.

We derived~\Eq{(15.7)} from the equations~\Eq{(15.5)} which describe the observed
behaviour of a particle moving under no field of force. We assume that the
result holds in all circumstances. The risky point in the generalisation is not
in introducing a field of force, because that may be due to an attitude of
mind of which the particle has no cognizance. The risk is in passing from
regions of the world where Galilean coordinates $(x, y, z, t)$ are possible to
intrinsically dissimilar regions where no such coordinates exist---from flat
space-time to space-time which is not flat.

The \emph{Principle of Equivalence} asserts the legitimacy of this generalisation.
It is essentially an hypothesis to be tested by experiment as opportunity
offers. Moreover it is to be regarded as a suggestion, rather than a dogma
admitting of no exceptions. It is likely that some of the phenomena will be
determined by comparatively simple equations in which the components of
curvature of the world do not appear; such equations will be the same for
a curved region as for a flat region. It is to these that the Principle of
Equivalence applies. It is a plausible suggestion that the undisturbed motion
of a particle and the propagation of light are governed by laws of this specially
simple type; and accordingly \Eq{(15.7)} and \Eq{(15.8)} will apply in all circumstances.
But there are more complex phenomena governed by equations in which the
curvatures of the world are involved; terms containing these curvatures will
vanish in the equations summarising experiments made in a flat region, and
would have to be reinstated in passing to the general equations. Clearly
there must be some phenomena of this kind which discriminate between
a flat world and a curved world; otherwise we could have no knowledge of
world\hyp{}curvature. For these the Principle of Equivalence breaks down.
\index{Equivalence, Principle of}%
\index{Principle!of equivalence}%

The Principle of Equivalence thus asserts that \emph{some} of the chief differential
equations of physics are the same for a curved region of the world as for an
osculating flat region\footnotemark.\footnotetext
  {The correct equations for a curved world will necessarily include as a special case those
  already obtained for a flat world. The practical point on which we seek the guidance of the
  Principle of Equivalence is whether the equations already obtained for a flat world will serve as
  they stand or will require generalisation.}
There can be no infallible rule for generalising
experimental laws; but the Principle of Equivalence offers a suggestion for
trial, which may be expected to succeed sometimes, and fail sometimes.

The Principle of Equivalence has played a great part as a guide in the
original building up of the generalised relativity theory; but now that we
have reached the new view of the nature of the world it has become less
necessary. Our present exposition is in the main deductive. We start with
a general theory of world\hyp{}structure and work down to the experimental
consequences, so that our progress is from the general to the special laws,
instead of \Foreign{vice versa}.

\Section{18.}{Retrospect}

The investigation of the external world in physics is a quest for \emph{structure}
rather than \emph{substance}. A structure can best be represented as a complex of
relations and relata; and in conformity with this we endeavour to reduce the
phenomena to their expressions in terms of the relations which we call
intervals and the relata which we call events.

If two bodies are of identical structure as regards the complex of interval
relations, they will be exactly similar as regards observational properties\footnotemark,\footnotetext
  {At present this is limited to extensional properties (in both space and time). It will be
  shown later that all mechanical properties are also included. Electromagnetic properties require
  separate consideration.}
if
our fundamental hypothesis is true. By this we show that experimental
measurements of lengths and duration are equivalent to measurements of the
interval relation.

To the events we assign four identification\hyp{}numbers or coordinates
according to a plan which is arbitrary within wide limits. The connection
between our physical measurements of interval and the system of identification\hyp{}numbers
is expressed by the general quadratic form~\Eq{(2.1)}. In the particular
case when these identification\hyp{}numbers can be so assigned that the product
terms in the quadratic form disappear leaving only the four squares, the
coordinates have the metrical properties belonging to rectangular coordinates
and time, and are accordingly so identified. If any such system exists an
infinite number of others exist connected with it by the Lorentz transformation,
so that there is no unique space-time frame. The relations of
these different space-time reckonings have been considered in detail. It is
shown that there must be a particular speed which has the remarkable
property that its value is the same for all these systems; and by appeal to
the Michelson\hyp{}Morley experiment or to Fizeau's experiment it is found that
this is a distinctive property of the speed of light.

But it is not possible throughout the world to choose coordinates fulfilling
the current definitions of rectangular coordinates and time. In such cases we
usually relax the definitions, and attribute the failure of fulfilment to a field
of force pervading the region. We have now no definite guide in selecting
what coordinates to take as rectangular coordinates and time; for whatever
the discrepancy, it can always be ascribed to a suitable field of force. The
field of force will vary according to the system of coordinates selected; but in
the general case it is not possible to get rid of it altogether (in a large region)
by any choice of coordinates. This irreducible field of force is ascribed to
gravitation. It should be noticed that the gravitational influence of a massive
body is not properly expressed by a definite field of force, but by the property
of irreducibility of the field of force. We shall find later that the irreducibility
of the field of force is equivalent to what in geometrical nomenclature is
called a curvature of the continuum of space-time.

For the fuller study of these problems we require a special mathematical
calculus which will now be developed \Foreign{ab initio}.

\Chapter{II}{The Tensor Calculus}

\Section{19.}{Contravariant and covariant vectors}
\index{Contravariant vectors}%
\index{Transformation of coordinates!general}%

\lettrine{\textcolor{lettrinecolour}{W}}{e} consider the transformation from one system of coordinates $x_{1}$, $x_{2}$, $x_{3}$, $x_{4}$
\index{Coordinates!general transformation of}%
to another system $x_{1}'$, $x_{2}'$, $x_{3}'$,~$x_{4}'$.

The differentials $(dx_{1}, dx_{2}, dx_{3}, dx_{4})$ are transformed according to the
equations~\Eq{(15.2)}, viz.\
\[
dx_{1}' = \frac{\dd x_{1}'}{\dd x_{1}}\, dx_{1}
  + \frac{\dd x_{1}'}{\dd x_{2}}\, dx_{2}
  + \frac{\dd x_{1}'}{\dd x_{3}}\, dx_{3}
  + \frac{\dd x_{1}'}{\dd x_{4}}\, dx_{4}; \text{ etc.,}
\]
which may be written shortly
\[
dx_{\mu}' = \sum_{\alpha=1}^{4} \frac{\dd x_{\mu}'}{\dd x_{\alpha}}\, dx_{\alpha},
\]
four equations being obtained by taking $\mu = 1$, $2$, $3$, $4$, successively.

Any set of four quantities transformed according to this law is called
a \emph{contravariant vector}. Thus if $(A^{1}, A^{2}, A^{3}, A^{4})$ becomes $(A'^{1}, A'^{2}, A'^{3}, A'^{4})$ in
\index{Vector}%
the new coordinate\hyp{}system, where
\[
A'^{\mu} =  \sum_{\alpha=1}^{4} \frac{\dd x_{\mu}'}{\dd x_{\alpha}}\, A^{\alpha},
\Tag{(19.1)}
\]
then $(A^{1}, A^{2}, A^{3}, A^{4})$, denoted briefly as~$A^{\mu}$, is a contravariant vector. The
upper position of the suffix (which is, of course, not an exponent) is reserved
to indicate contravariant vectors.

If $\phi$~is an invariant function of position, i.e.\ if it has a fixed value at each
point independent of the coordinate\hyp{}system employed, the four quantities
\[
\left(\frac{\dd\phi}{\dd x_{1}},
      \frac{\dd\phi}{\dd x_{2}},
      \frac{\dd\phi}{\dd x_{3}},
      \frac{\dd\phi}{\dd x_{4}}\right)
\]
are transformed according to the equations
\[
\frac{\dd\phi}{\dd x_{1}'}
= \frac{\dd x_{1}}{\dd x_{1}'}\, \frac{\dd\phi}{\dd x_{1}}
+ \frac{\dd x_{2}}{\dd x_{1}'}\, \frac{\dd\phi}{\dd x_{2}}
+ \frac{\dd x_{3}}{\dd x_{1}'}\, \frac{\dd\phi}{\dd x_{3}}
+ \frac{\dd x_{4}}{\dd x_{1}'}\, \frac{\dd\phi}{\dd x_{4}}; \text{ etc.}
\]
which may be written shortly
\[
\frac{\dd\phi}{\dd x_{\mu}'}
= \sum_{\alpha=1}^{4} \frac{\dd x_{\alpha}}{\dd x_{\mu}'}\, \frac{\dd\phi}{\dd x_{\alpha}}.
\]
Any set of four quantities transformed according to this law is called a
\emph{covariant vector}. Thus if $A_{\mu}$~is a covariant vector, its transformation law is
\index{Covariant vector}%
\[
A_{\mu}' =  \sum_{\alpha=1}^{4} \frac{\dd x_{\alpha}}{\dd x_{\mu}'}\, A_{\alpha}.
\Tag{(19.2)}
\]

We have thus two varieties of vectors which we distinguish by the upper
or lower position of the suffix. The first illustration of a contravariant vector,
\index{Contravariant vectors}%
\index{Vector!mathematical notion of}%
$dx_{\mu}$, forms rather an awkward exception to the rule that a lower suffix indicates
covariance and an upper suffix contravariance. There is no other
exception likely to mislead the reader, so that it is not difficult to keep in
mind this peculiarity of~$dx_{\mu}$; but we shall sometimes find it convenient to
indicate its contravariance explicitly by writing
\[
dx_{\mu} \equiv (dx)^{\mu}.
\Tag{(19.3)}
\]

A vector may either be a single set of four quantities associated with
a special point in space-time, or it may be a set of four functions varying
continuously with position. Thus we can have an ``isolated vector'' or a
``vector\hyp{}field.''

For an illustration of a covariant vector we considered the gradient of an
invariant, $\dd\phi/\dd x_{\mu}$; but a covariant vector is not necessarily the gradient of an
invariant.

The reader will probably be already familiar with the term vector, but
the distinction of covariant and contravariant vectors will be new to him.
This is because in the elementary analysis only rectangular coordinates are
contemplated, and for transformations from one rectangular system to another
the laws \Eq{(19.1)} and \Eq{(19.2)} are equivalent to one another. From the geometrical
point of view, the \emph{contravariant vector} is the vector with which everyone is
familiar; this is because a displacement, or directed distance between two
points, is regarded as representing $(dx_{1}, dx_{2}, dx_{3})$\footnote
  {The customary resolution of a displacement into components in oblique directions assumes
  this.}
which, as we have seen, is
contravariant. The covariant vector is a new conception which does not so
easily lend itself to graphical illustration.

\Section{20.}{The mathematical notion of a vector}

The formal definitions in the preceding section do not help much to
an understanding of what the notion of a vector really is. We shall try to
explain this more fully, taking first the mathematical notion of a vector (with
which we are most directly concerned) and leaving the more difficult physical
notion to follow.

We have a set of four numbers $(A_{1}, A_{2}, A_{3}, A_{4})$ which we associate with
some point $(x_{1}, x_{2}, x_{3}, x_{4})$ and with a certain system of coordinates. We make
a change of the coordinate\hyp{}system, and we ask, What will these numbers
become in the new coordinates? The question is meaningless; they do not
automatically ``become'' anything. Unless we interfere with them they stay
as they were. But the mathematician may say ``When I am using the
coordinates $x_{1}$, $x_{2}$, $x_{3}$, $x_{4}$, I want to talk about the numbers $A_{1}$, $A_{2}$, $A_{3}$, $A_{4}$;
and when I am using $x_{1}'$, $x_{2}'$, $x_{3}'$, $x_{4}'$, I find that at the corresponding stage of
my work I shall want to talk about four different numbers $A_{1}'$, $A_{2}'$, $A_{3}'$, $A_{4}'$.
%[** TN: Nine fraktur "A"s set \textgoth in the original]
So for brevity I propose to call both sets of numbers by the same symbol~$\mf{A}$.''
We reply ``That will be all right, provided that you tell us just what numbers
will be denoted by~$\mf{A}$ for \emph{each} of the coordinate\hyp{}systems you intend to use.
Unless you do this we shall not know what you are talking about.''

Accordingly the mathematician begins by giving us a list of the numbers
that $\mf{A}$~will signify in the different coordinate\hyp{}systems. We here denote these
numbers by letters. $\mf{A}$~will mean\footnote
  {For convenience I take a three\hyp{}dimensional illustration.}
\begin{align*}
&\text{$X$, $Y$, $Z$ for certain rectangular coordinates $x$, $y$, $z$,} \\
&\text{$R$, $\Theta$, $\Phi$ for certain polar coordinates $r$, $\theta$, $\phi$,} \\
  &\text{$\Lambda$, $M$, $N$ for certain ellipsoidal coordinates $\lambda$, $\mu$, $\nu$.}
  \end{align*}
``But,'' says the mathematician, ``I shall never finish at this rate. There are
an infinite number of coordinate\hyp{}systems which I want to use. I see that
I must alter my plan. I will give you a general rule to find the new values
of~$\mf{A}$ when you pass from one coordinate\hyp{}system to another; so that it is only
necessary for me to give you one set of values and you can find all the others
for yourselves.''

In mentioning a \emph{rule} the mathematician gives up his arbitrary power of
making~$\mf{A}$ mean anything according to his fancy at the moment. He binds
himself down to some kind of regularity. Indeed we might have suspected
that our orderly\hyp{}minded friend would have some principle in his assignment
of different meanings to~$\mf{A}$. But even so, can we make any guess at the rule
he is likely to adopt unless we have some idea of the problem he is working
at in which~$\mf{A}$ occurs? I think we can; it is not necessary to know anything
about the nature of his problem, whether it relates to the world of physics or
to something purely conceptual; it is sufficient that we know a little about
the nature of a mathematician.

What kind of rule could he adopt? Let us examine the quantities which
can enter into it. There are first the two sets of numbers to be connected,
say, $X$, $Y$, $Z$ and $R$, $\Theta$, $\Phi$. Nothing has been said as to these being analytical
functions of any kind; so far as we know they are isolated numbers. Therefore
there can be no question of introducing their derivatives. They are regarded
as located at some point of space $(x, y, z)$ and $(r, \theta, \phi)$, otherwise the question
of coordinates could scarcely arise. They are changed because the coordinate\hyp{}system
has changed \emph{at this point,} and that change is defined by quantities like
$\dfrac{\dd r}{\dd x}$, $\dfrac{\dd^{2}\theta}{\dd x\, \dd y}$, and so on. The integral coordinates themselves, $x$,~$y$,~$z$, $r$,~$\theta$,~$\phi$,
cannot be involved; because they express relations to a distant origin, whereas
we are concerned only with changes at the spot where $(X, Y, Z)$ is located.
Thus the rule must involve only the numbers $X$,~$Y$,~$Z$, $R$,~$\Theta$,~$\Phi$ combined
with the mutual derivatives of $x$,~$y$,~$z$, $r$,~$\theta$,~$\phi$.

One such rule would be
\[
\left.
\begin{alignedat}{3}
  R &= \frac{\dd r}{\dd x}\, X &&+ \frac{\dd r}{\dd y}\, Y &&+ \frac{\dd r}{\dd z}\, Z \\
  \Theta &= \frac{\dd \theta}{\dd x}\, X &&+ \frac{\dd \theta}{\dd y}\, Y &&+ \frac{\dd \theta}{\dd z}\, Z \\
  \Phi &= \frac{\dd \phi}{\dd x}\, X &&+ \frac{\dd \phi}{\dd y}\, Y &&+ \frac{\dd \phi}{\dd z}\, Z
  \end{alignedat}\right\}
\Tag{(20.1)}
\]
Applying the same rule to the transformation from $(r, \theta, \phi)$ to $(\lambda, \mu, \nu)$
we have
\[
\Lambda = \frac{\dd \lambda}{\dd r}\, R + \frac{\dd \lambda}{\dd \theta}\, \Theta + \frac{\dd \lambda}{\dd \phi}\, \Phi,
\Tag{(20.2)}
\]
whence, substituting for $R$, $\Theta$, $\Phi$ from~\Eq{(20.1)} and collecting terms,
\begin{align*}
  \Lambda &= \left(\frac{\dd \lambda}{\dd r}\, \frac{\dd r}{\dd x} + \frac{\dd \lambda}{\dd \theta}\, \frac{\dd \theta}{\dd x} + \frac{\dd \lambda}{\dd \phi}\, \frac{\dd \phi}{\dd x}\right) X
  + \left(\frac{\dd \lambda}{\dd r}\, \frac{\dd r}{\dd y} + \frac{\dd \lambda}{\dd \theta}\, \frac{\dd \theta}{\dd y} + \frac{\dd \lambda}{\dd \phi}\, \frac{\dd \phi}{\dd y}\right) Y \\
  &\qquad\qquad+ \left(\frac{\dd \lambda}{\dd r}\, \frac{\dd r}{\dd z} + \frac{\dd \lambda}{\dd \theta}\, \frac{\dd \theta}{\dd z} + \frac{\dd \lambda}{\dd \phi}\, \frac{\dd \phi}{\dd z}\right) Z \\
  &= \frac{\dd\lambda}{\dd x}\, X + \frac{\dd\lambda}{\dd y}\, Y + \frac{\dd\lambda}{\dd z}\, Z,
  \Tag{(20.3)}
\end{align*}
which is the same formula as we should have obtained by applying the rule
to the direct transformation from $(x, y, z)$ to $(\lambda, \mu, \nu)$. The rule is thus self\hyp{}consistent.
But this is a happy accident, pertaining to this particular rule,
and depending on the formula
\[
\frac{\dd\lambda}{\dd x}
= \frac{\dd \lambda}{\dd r}\, \frac{\dd r}{\dd x}
+ \frac{\dd \lambda}{\dd \theta}\, \frac{\dd \theta}{\dd x}
+ \frac{\dd \lambda}{\dd \phi}\, \frac{\dd \phi}{\dd x},
\]
and amid the apparently infinite choice of formulae it will not be easy to find
others which have this self\hyp{}consistency.

The above rule is that already given for the contravariant vector~\Eq{(19.1)}.
The rule for the covariant vector is also self\hyp{}consistent. There do not appear
to be any other self\hyp{}consistent rules for the transformation of a set of three
numbers (or four numbers for four coordinates)\footnotemark.\footnotetext
  {Except that we may in addition multiply by any power of the Jacobian of the transformation.
  This is self\hyp{}consistent because
  \[
  \frac{\dd(x, y, z)}{\dd(r, \theta, \phi)} \cdot \frac{\dd(r, \theta, \phi)}{\dd(\lambda, \mu, \nu)}
  = \frac{\dd(x, y, z)}{\dd(\lambda, \mu, \nu)}.
  \]
  Sets of numbers transformed with this additional multiplication are degenerate cases of tensors
  of higher rank considered later. See \SecRefs{48}, \SecNum{49}.}

We see then that unless the mathematician disregards the need for self\hyp{}consistency
in his rule, he must inevitably make his quantity~$\mf{A}$ either a
contravariant or a covariant vector. The choice between these is entirely at
his discretion. He might obtain a wider choice by disregarding the property
of self\hyp{}consistency---by selecting a particular coordinate\hyp{}system, $x$,~$y$,~$z$, and
insisting that values in other coordinate\hyp{}systems must always be obtained by
applying the rule immediately to $X$,~$Y$,~$Z$, and not permitting intermediate
transformations. In practice he does not do this, perhaps because he can
never make up his mind that any particular coordinates are deserving of this
special distinction.

We see now that a mathematical vector is a common name for an infinite
\index{Vector!physical notion of}%
number of sets of quantities, each set being associated with one of an infinite
number of systems of coordinates. The arbitrariness in the association is
removed by postulating that some method is followed, and that no one
system of coordinates is singled out for special distinction. In technical
language the transformations must form a \emph{Group. The quantity $(R, \Theta, \Phi)$
\index{Group}%
is in no sense the same quantity as $(X, Y, Z)$}; they have a common name and
a certain analytical connection, but the idea of anything like identity is
entirely excluded from the mathematical notion of a vector.

\Section{21.}{The physical notion of a vector}

The components of a force $(X, Y, Z)$, $(X', Y', Z')$, etc. in different systems
of Cartesian coordinates, rectangular or oblique, form a contravariant vector.
This is evident because in elementary mechanics a force is resolved into
components according to the parallelogram law just as a displacement~$dx_{\mu}$ is
resolved, and we have seen that $dx_{\mu}$~is a contravariant vector. So far as the
mathematical notion of the vector is concerned, the quantities $(X, Y, Z)$ and
$(X', Y', Z')$ are not to be regarded as in any way identical; but in physics
we conceive that both quantities express some kind of condition or relation
of the world, and this condition is the same whether expressed by $(X, Y, Z)$
or by $(X', Y', Z')$. The physical vector is this vaguely conceived entity, which
is independent of the coordinate\hyp{}system, and is at the back of our measurements
of force.

A world\hyp{}condition cannot appear directly in a mathematical equation;
only the \emph{measure} of the world\hyp{}condition can appear. Any number or set of
numbers which can serve to specify uniquely a condition of the world may
\index{Condition of the world}%
be called a measure of it. In using the phrase ``condition of the world''
I intend to be as non\hyp{}committal as possible; whatever in the external world
determines the values of the physical quantities which we observe, will be
included in the phrase.

The simplest case is when the condition of the world under consideration
can be indicated by a single measure\hyp{}number. Take two such conditions
underlying respectively the wave-length~$\lambda$ and period~$T$ of a light-wave. We
have the equation
\[
\lambda = 3 \cdot 10^{10}\, T.
\Tag{(21.1)}
\]
This equation holds only for one particular plan of assigning measure\hyp{}numbers
(the \CGS\ system). But it may be written in the more general form
\[
\lambda = cT,
\Tag{(21.2)}
\]
where $c$~is a velocity having the value $3 \cdot 10^{10}$ in the \CGS\ system. This
comprises any number of particular equations of the form~\Eq{(21.1)}. For each
measure\hyp{}plan, or system of units, $c$~has a different numerical value. The
method of determining the necessary change of~$c$ when a new measure\hyp{}plan
is adopted, is well known; we assign to it the \emph{dimensions} length $\div$ time, and
by a simple rule we know how it must be changed when the units of~$\lambda$ and~$T$
are changed. For any general equation the total dimensions of every term
ought to be the same.

The tensor calculus extends this \emph{principle of dimensions} to changes of
\index{Dimensions, principle of}%
\index{Principle!of dimensions}%
measure\hyp{}code much more general than mere changes of units. There are
\index{Measure\hyp{}code}%
\index{Unit!change of}%
conditions of the world which cannot be specified by a single measure\hyp{}number;
some require~$4$, some~$16$, some~$64$, etc., measure\hyp{}numbers. Their variety is
such that they cannot be arranged in a single serial order. Consider then an
equation between the measure\hyp{}numbers of two conditions of the world which
require $4$~measure\hyp{}numbers. The equation, if it is of the necessary general type,
must hold for every possible measure\hyp{}code; this will be the case if, when we
transform the measure\hyp{}code, both sides of the equation are transformed in
the same way, i.e.\ if we have to perform the same series of mathematical
operations on both sides.

We can here make use of the mathematical vector of \SecRef{20}. Let our equation
in some measure\hyp{}code be
\[
A_{1}, A_{2}, A_{3}, A_{4} = B_{1}, B_{2}, B_{3}, B_{4}.
\Tag{(21.3)}
\]
Now let us change the code so that the left\hyp{}hand side becomes \emph{any} four
numbers $A_{1}'$, $A_{2}'$, $A_{3}'$, $A_{4}'$. We identify this with the transformation of a covariant
vector by associating with the change of measure\hyp{}code the corresponding
transformation of coordinates from~$x_{\mu}$ to~$x_{\mu}'$ as in~\Eq{(19.2)}. But since \Eq{(21.3)}~is
to hold in all measure\hyp{}codes, the transformation of the right-hand side must
involve the same set of operations; and the change from $B_{1}$, $B_{2}$, $B_{3}$, $B_{4}$ to $B_{1}'$,
$B_{2}'$, $B_{3}'$, $B_{4}'$ will also be the transformation of a covariant vector associated
with the \emph{same} transformation of coordinates from~$x_{\mu}$ to~$x_{\mu}'$.

We thus arrive at the result that in an equation which is independent
of the measure\hyp{}plan both sides must be covariant or both contravariant
vectors. We shall extend this later to conditions expressed by $16$, $64$,~\dots,
measure\hyp{}numbers; the general rule is that both sides of the equation must
have the same elements of covariance and contravariance. Covariance and
contravariance are a kind of generalised dimension, showing how the measure
of one condition of the world is changed when the measure of another condition
is changed. The ordinary theory of change of units is merely an
elementary case of this.

Coordinates are the identification\hyp{}numbers of the points of space-time.
There is no fundamental distinction between measure\hyp{}numbers and identification\hyp{}numbers,
so that we may regard the change of coordinates as part of the
general change applied to all measure\hyp{}numbers. The change of coordinates
no longer leads the way, as it did in \SecRef{20}; it is placed on the same level with
the other changes of measure.

When we applied a change of measure\hyp{}code to~\Eq{(21.3)} we associated with
it a change of coordinates; but it is to be noted that the change of coordinates
\index{Coordinates!representation of displacement}%
was then ambiguous, since the two sides of the equation might have been
taken as both contravariant instead of both covariant; and further the change
did not refer explicitly to coordinates \emph{in} the world---it was a mere entry in
the mathematician's note-book in order that he might have the satisfaction
of calling $A_{\mu}$ and~$B_{\mu}$ vectors consistently with his definition. Now if the
measure\hyp{}plan of a condition~$A_{\mu}$ is changed the measures of other conditions
and relations associated with it will be changed. Among these is a certain
relation of two events which we may call the \emph{aspect}\footnote
  {The relation of \emph{aspect} (or in its graphical conception \emph{displacement}) with four measure\hyp{}numbers
\index{Aspect, relation of}%
\index{Displacement}%
  seems to be derived from the relation of \emph{interval} with one measure\hyp{}number, by taking
  account not only of the mutual interval between the two events but also of their intervals from
  all surrounding events.}
of one from the other;
and this relation requires four measure\hyp{}numbers to specify it. Somewhat
arbitrarily we decide to make the aspect a contravariant vector, and the
measure\hyp{}numbers assigned to it are denoted by~$(dx)^{\mu}$. That settles the ambiguity
once for all. For obscure psychological reasons the mind has singled
out this transcendental relation of aspect for graphical representation, so that
it is conceived by us as a \emph{displacement} or difference of location in a frame of
space-time. Its measure\hyp{}numbers $(dx)^{\mu}$ are represented graphically as coordinate\hyp{}differences~$dx_{\mu}$,
and so for each measure\hyp{}code of aspect we get a corresponding
coordinate\hyp{}frame of location. This ``real'' coordinate\hyp{}frame can now
replace the abstract frame in the mathematician's note-book, because as we
have seen in~\Eq{(19.1)} the actual transformation of coordinates \emph{resulting} in a
change of~$dx_{\mu}$ is the same as the transformation associated with the change of~$dx_{\mu}$
according to the law of a contravariant vector.

I do not think it is too extravagant to claim that the method of the tensor
\index{Tensor equations}%
calculus, which presents all physical equations in a form independent of the
choice of measure\hyp{}code, is the only possible means of studying the conditions
of the world which are at the basis of physical phenomena. The physicist is
accustomed to insist (sometimes quite unnecessarily) that all equations should
be stated in a form independent of the units employed. Whether this is
desirable depends on the purpose of the formulae. But whatever additional
insight into underlying causes is gained by stating equations in a form independent
of units, must be gained to a far greater degree by stating them in
a form altogether independent of the measure\hyp{}code. An equation of this
general form is called a \emph{tensor equation}.

When the physicist is attacking the everyday problems of his subject, he
may use any form of the equations---any specialised measure\hyp{}plan---which
will shorten the labour of calculation; for in these problems he is concerned
with the outward significance rather than the inward significance of his
formulae. But once in a while he turns to consider their inward significance---to
consider that relation of things in the world\hyp{}structure which is the
origin of his formulae. The only intelligible idea we can form of such a
structural relation is that it exists between the world\hyp{}conditions themselves
and not between the measure\hyp{}numbers of a particular code. A law of nature
resolves itself into a constant relation, or even an identity, of the two world\hyp{}conditions
to which the different classes of observed quantities forming the
two sides of the equation are traceable. Such a constant relation independent
of measure\hyp{}code is only to be expressed by a tensor equation.

It may be remarked that if we take a force $(X, Y, Z)$ and transform it to
\index{Force, covariant and contravariant components}%
polar coordinates, whether as a covariant or a contravariant vector, in neither
case do we obtain the quantities called polar components in elementary
mechanics. The latter are not in our view the true polar components; they
are merely rectangular components in three new directions, viz.\ radial and
transverse. In general the elementary definitions of physical quantities do
not contemplate other than rectangular components, and they may need to
be supplemented before we can decide whether the physical vector is covariant
or contravariant. Thus if we define force as ``mass $\times$ acceleration,'' the force
turns out to be contravariant; but if we define it by ``work $=$~force $\times$ displacement,''
the force is covariant. With the latter definition, however, we have
to abandon the method of resolution into \emph{oblique} components adopted in
elementary mechanics.

In what follows it is generally sufficient to confine attention to the mathematical
notion of a vector. Some idea of the physical notion will probably
give greater insight, but is not necessary for the formal proofs.

\Section{22.}{The summation convention}
\index{Summation convention}%

We shall adopt the convention that whenever a literal suffix appears twice
in a term that term is to be summed for values of the suffix $1$, $2$, $3$,~$4$. For
example, \Eq{(2.1)}~will be written
\[
ds^{2} = g_{\mu\nu}\, dx_{\mu}\, dx_{\nu}\qquad (g_{\nu\mu} = g_{\mu\nu}).
\Tag{(22.1)}
\]
Here, since $\mu$ and $\nu$ each appear twice, the summation
\[
\sum_{\mu=1}^{4} \sum_{\nu=1}^{4}
\]
is indicated; and the result written out in full gives~\Eq{(2.1)}.

Again, in the equation
\[
A_{\mu}' = \frac{\dd x_{\alpha}}{\dd x_{\mu}'}\, A_{\alpha},
\]
the summation on the right is with respect to $\alpha$~only ($\mu$~appearing only once).
The equation is equivalent to~\Eq{(19.2)}.

The convention is not merely an abbreviation but an immense aid to the
analysis, giving it an impetus which is nearly always in a profitable direction.
Summations occur in our investigations without waiting for our tardy approval.

A useful rule may be noted---

Any literal suffix appearing twice in a term is a dummy suffix, which may
be changed freely to any other letter not already appropriated in that term.
Two or more dummy suffixes can be interchanged\footnotemark.\footnotetext
  {At first we shall call attention to such changes when we employ them; but the reader will
  be expected gradually to become familiar with the device as a common process of manipulation.}
For example
\index{Dummy suffixes}%
\[
g_{\alpha\beta}\, \frac{\dd^{2} x_{\alpha}}{\dd x_{\mu}'\, \dd x_{\nu}'}\, \frac{\dd x_{\beta}}{\dd x_{\lambda}'}
= g_{\alpha\beta}\, \frac{\dd^{2} x_{\beta}}{\dd x_{\mu}'\, \dd x_{\nu}'}\, \frac{\dd x_{\alpha}}{\dd x_{\lambda}'}
\Tag{(22.2)}
\]
{\Loosen by interchanging the dummy suffixes $\alpha$ and~$\beta$, remembering that $g_{\beta\alpha} = g_{\alpha\beta}$.}

For a further illustration we shall prove that
\[
\left.
\begin{aligned}
  \frac{\dd x_{\mu}}{\dd x_{\alpha}'}\, \frac{\dd x_{\alpha}'}{\dd x_{\nu}}
  = \frac{\dd x_{\mu}}{\dd x_{\nu}}
  &= 0,\quad \text{if $\mu \neq \nu$} \\
  &= 1,\quad \text{if $\mu = \nu$}
\end{aligned}
\right\}.
\Tag{(22.3)}
\]
The left-hand side written in full is
\[
\frac{\dd x_{\mu}}{\dd x_{1}'}\, \frac{\dd x_{1}'}{\dd x_{\nu}}
+ \frac{\dd x_{\mu}}{\dd x_{2}'}\, \frac{\dd x_{2}'}{\dd x_{\nu}}
+ \frac{\dd x_{\mu}}{\dd x_{3}'}\, \frac{\dd x_{3}'}{\dd x_{\nu}}
+ \frac{\dd x_{\mu}}{\dd x_{4}'}\, \frac{\dd x_{4}'}{\dd x_{\nu}},
\]
which by the usual theory gives the change~$dx_{\mu}$ consequent on a change~$dx_{\nu}$.
But $x_{\mu}$~and $x_{\nu}$ are coordinates of the same system, so that their variations are
independent; hence $dx_{\mu}$~is zero unless $x_{\mu}$~and $x_{\nu}$ are the same coordinate, in
which case, of course, $dx_{\mu} = dx_{\nu}$. Thus the theorem is proved.

The multiplier $\dfrac{\dd x_{\mu}}{\dd x_{\alpha}'}\, \dfrac{\dd x_{\alpha}'}{\dd x_{\nu}}$ acts as a \emph{substitution\hyp{}operator}. That is to say if
\index{Substitution\hyp{}operator}%
$A(\mu)$~is any expression involving the suffix~$\mu$
\[
\frac{\dd x_{\mu}}{\dd x_{\alpha}'}\, \frac{\dd x_{\alpha}'}{\dd x_{\nu}}\, A(\mu) = A(\nu).
\Tag{(22.4)}
\]
For on the left the summation with respect to~$\mu$ gives four terms corresponding
to the values $1$,~$2$, $3$, $4$ of~$\mu$. One of these values will agree with~$\nu$.
Denote the other three values by $\sigma$, $\tau$,~$\rho$. Then by~\Eq{(22.3)} the result is
\begin{align*}
  1 \cdot &A(\nu) + 0 \cdot A(\sigma) + 0 \cdot A(\tau) + 0 \cdot A(\rho) \\
  {}={} & A(\nu).
  \end{align*}
The multiplier accordingly has the effect of substituting~$\nu$ for~$\mu$ in the multiplicand.

\Section{23.}{Tensors}
\index{Tensor}%

The two laws of transformation given in \SecRef{19} are now written---

Contravariant vectors
\begin{align*}
  A'^{\mu} &= \frac{\dd x_{\mu}'}{\dd x_{\alpha}}\, A^{\alpha}.
  \Tag{(23.11)} \\
  \intertext{\hspace*{\parindent}Covariant vectors}
  A_{\mu}' &= \frac{\dd x_{\alpha}}{\dd x_{\mu}'}\, A_{\alpha}.
  \Tag{(23.12)}
\end{align*}

We can denote by~$A_{\mu\nu}$ a quantity with $16$~components obtained by giving
$\mu$~and $\nu$ the values from~$1$ to~$4$ independently. Similarly $A_{\mu\nu\sigma}$~has $64$~components.
By a generalisation of the foregoing transformation laws we classify
quantities of this kind as follows---

Contravariant tensors
\index{Contravariant vectors!tensors}%
\begin{align*}
  A'^{\mu\nu} &= \frac{\dd x_{\mu}'}{\dd x_{\alpha}}\, \frac{\dd x_{\nu}'}{\dd x_{\beta}}\, A^{\alpha\beta}.
  \Tag{(23.21)} \\
  \intertext{\hspace*{\parindent}Covariant tensors}
\index{Covariant vector!tensor}%
  A_{\mu\nu}' &= \frac{\dd x_{\alpha}}{\dd x_{\mu}'}\, \frac{\dd x_{\beta}}{\dd x_{\nu}'}\, A_{\alpha\beta}.
  \Tag{(23.22)}
  \intertext{\hspace*{\parindent}Mixed tensors}
\index{Mixed tensors}%
  A_{\mu}'^{\nu} &= \frac{\dd x_{\alpha}}{\dd x_{\mu}'}\, \frac{\dd x_{\nu}'}{\dd x_{\beta}}\, A_{\alpha}^{\beta}.
  \Tag{(23.23)}
\end{align*}
The above are called tensors of the second rank. We have similar laws for
tensors of higher ranks. E.g.\
\[
A_{\mu\nu\sigma}'^{\tau}
= \frac{\dd x_{\alpha}}{\dd x_{\mu}'}\,
  \frac{\dd x_{\beta}}{\dd x_{\nu}'}\,
  \frac{\dd x_{\gamma}}{\dd x_{\sigma}'}\,
  \frac{\dd x_{\tau}'}{\dd x_{\delta}}\, A_{\alpha\beta\gamma}^{\delta}.
\Tag{(23.3)}
\]
It may be worth while to remind the reader that \Eq{(23.3)}~typifies $256$~distinct
equations each with a sum of $256$~terms on the right-hand side.

It is easily shown that these transformation laws fulfil the condition of
self\hyp{}consistency explained in \SecRef{20}, and it is for this reason that quantities
governed by them are selected for special nomenclature.

If a tensor vanishes, i.e.\ if all its components vanish, in one system of
coordinates, it will continue to vanish when any other system of coordinates
is substituted. This is clear from the linearity of the above transformation
laws.

Evidently the sum of two tensors of the same covariant or contravariant
character is a tensor. Hence a law expressed by the vanishing of the sum of
a number of tensors, or by the equality of two tensors of the same kind, will
be independent of the coordinate\hyp{}system used.

The product of two tensors such as $A_{\mu\nu}$ and $B_{\sigma}^{\tau}$ is a tensor of the character
indicated by~$A_{\mu\nu\sigma}^{\tau}$. This is proved by showing that the transformation law of
the product is the same as~\Eq{(23.3)}.

The general term \emph{tensor} includes vectors (tensors of the first rank) and
invariants or scalars\footnote
  {Scalar is a synonym for invariant. I generally use the latter word as the more self\hyp{}explanatory.}
\index{Scalar}%
(tensors of zero rank).

A tensor of the second or higher rank need not be expressible as a product
of two tensors of lower rank.

A simple example of an expression of the second rank is afforded by the
stresses in a solid or viscous fluid. The component of stress denoted by~$p_{xy}$
is the traction in the $y$-direction across an interface perpendicular to the
$x$-direction. Each component is thus associated with two directions.

%[** TN: Shortened running head]
\Section[Inner multiplication and contraction]{24.}{Inner multiplication and contraction. The quotient law}

If we multiply $A_{\mu}$ by $B^{\nu}$ we obtain sixteen quantities $A_{1}B^{1}$, $A_{1}B^{2}$, $A_{2}B^{1}$,~\dots
constituting a mixed tensor. Suppose that we wish to consider the four
``diagonal'' terms $A_{1}B^{1}$, $A_{2}B^{2}$, $A_{3}B^{3}$, $A_{4}B^{4}$; we naturally try to abbreviate
these by writing them~$A_{\mu}B^{\mu}$. But by the summation convention $A_{\mu}B^{\mu}$~stands
for the sum of the four quantities. The convention is right. We have no use
for them individually since they do not form a vector; but the sum is of great
importance.

$A_{\mu}B^{\mu}$~is called the \emph{inner product} of the two vectors, in contrast to the
\index{Contraction of tensors}%
\index{Inner multiplication}%
\index{Multiplication, inner and outer}%
\index{Product, inner and outer}%
ordinary or \emph{outer product}~$A_{\mu}B^{\nu}$.

In rectangular coordinates the inner product coincides with the \emph{scalar\hyp{}product}
defined in the well\hyp{}known elementary theory of vectors; but the outer
product is not the so-called \emph{vector\hyp{}product} of the elementary theory.

By a similar process we can form from any mixed tensor~$A_{\mu\nu\sigma}^{\tau}$ a ``contracted\footnotemark''\footnotetext
  {German, \Foreign{verjüngt}.}%
tensor~$A_{\mu\nu\sigma}^{\sigma}$, which is two ranks lower since $\sigma$~has now become a
dummy suffix. To prove that $A_{\mu\nu\sigma}^{\sigma}$~is a tensor, we set $\tau = \sigma$ in~\Eq{(23.3)},
\[
A_{\mu\nu\sigma}'^{\sigma}
= \frac{\dd x_{\alpha}}{\dd x_{\mu}'}\,
  \frac{\dd x_{\beta}}{\dd x_{\nu}'}\,
  \frac{\dd x_{\gamma}}{\dd x_{\sigma}'}\,
  \frac{\dd x_{\sigma}'}{\dd x_{\delta}}\, A_{\alpha\beta\gamma}^{\delta}.
\]
The substitution operator $\dfrac{\dd x_{\gamma}}{\dd x_{\sigma}'}\, \dfrac{\dd x_{\sigma}'}{\dd x_{\delta}}$ changes~$\delta$ to~$\gamma$ in~$A_{\alpha\beta\gamma}^{\delta}$ by~\Eq{(22.4)}. Hence
\[
A_{\mu\nu\sigma}'^{\sigma}
= \frac{\dd x_{\alpha}}{\dd x_{\mu}'}\,
  \frac{\dd x_{\beta}}{\dd x_{\nu}'}\, A_{\alpha\beta\gamma}^{\gamma}.
\]
Comparing with the transformation law~\Eq{(23.22)} we see that $A_{\mu\nu\sigma}^{\sigma}$~is a covariant
tensor of the second rank. Of course, the dummy suffixes $\gamma$ and~$\sigma$ are equivalent.

Similarly, setting $\nu = \mu$ in~\Eq{(23.23)},
\[
A_{\mu}'^{\mu}
= \frac{\dd x_{\alpha}}{\dd x_{\mu}'}\,
  \frac{\dd x_{\mu}'}{\dd x_{\beta}}\, A_{\alpha}^{\beta}
= A_{\alpha}^{\alpha} = A_{\mu}^{\mu},
\]
that is to say $A_{\mu}^{\mu}$~is unaltered by a transformation of coordinates. Hence it
is an invariant.

By the same method we can show that $A_{\mu}B^{\mu}$, $A_{\mu\nu}^{\mu\nu}$, $A_{\mu}^{\nu}B_{\nu}^{\mu}$ are invariants.
In general when an upper and lower suffix are the same the corresponding
covariant and contravariant qualities cancel out. If all suffixes cancel out in
this way, the expression must be invariant. The identified suffixes must be
of opposite characters; the expression $A_{\mu\sigma\sigma}^{\tau}$ is not a tensor, and no interest
is attached to it.

We see that the suffixes keep a tally of what we have called the generalised
dimensions of the terms in our equations. After cancelling out any suffixes
which appear in both upper and lower positions, the remaining suffixes must
appear in the same position in each term of an equation. When that is
satisfied each term will undergo the same set of operations when a transformation
of coordinates is made, and the equation will continue to hold in all
systems of coordinates. This may be compared with the well-known condition
that each term must have the same physical dimensions, so that it undergoes
multiplication by the same factor when a change of units is made and the
equation continues to hold in all systems of units.

Just as we can infer the physical dimensions of some novel entity entering
\index{Dimensions, principle of}%
\index{Principle!of dimensions}%
into a physical equation, so we can infer the contravariant and covariant
dimensions of an expression whose character was hitherto unknown. For
example, if the equation is
\[
A(\mu\nu) B_{\nu\sigma} = C_{\mu\sigma},
\Tag{(24.1)}
\]
where the nature of $A(\mu\nu)$ is not known initially, we see that $A(\mu\nu)$~must
be a tensor of the character~$A_{\mu}^{\nu}$, so as to give
\[
A_{\mu}^{\nu} B_{\nu\sigma} = C_{\mu\sigma},
\]
which makes the covariant dimensions on both sides consistent.

The equation~\Eq{(24.1)} may be written symbolically
\[
A(\mu\nu) = C_{\mu\sigma}/B_{\nu\sigma},
\]
and the conclusion is that not only the product but also the (symbolic)
quotient of two tensors is a tensor. Of course, the operation here indicated
is not that of ordinary division.

This quotient law is a useful aid in detecting the tensor\hyp{}character of
\index{Quotient law}%
expressions. It is not claimed that the general argument here given amounts
to a strict mathematical proof. In most cases we can supply the proof required
by one or more applications of the following rigorous theorem---

A quantity which on inner multiplication by \emph{any} covariant (alternatively,
by \emph{any} contravariant) vector always gives a tensor, is itself a tensor.

For suppose that
\[
A(\mu\nu) B^{\nu}
\]
is always a covariant vector for any choice of the contravariant vector~$B^{\nu}$.
Then by~\Eq{(23.12)}
\[
\{A'(\mu\nu) B'^{\nu}\}
  = \frac{\dd x_{\alpha}}{\dd x_{\mu}'}\, \{A(\alpha\beta) B^{\beta}\}.
\Tag{(24.2)}
\]
But by \Eq{(23.11)} applied to the reverse transformation from accented to unaccented
coordinates
\[
B^{\beta} = \frac{\dd x_{\beta}}{\dd x_{\nu}'}\, B'^{\nu}.
\]
Hence, substituting for~$B^{\beta}$ in~\Eq{(24.2)},
\[
B'^{\nu} \left(A'(\mu\nu) - \frac{\dd x_{\alpha}}{\dd x_{\mu}'}\, \frac{\dd x_{\beta}}{\dd x_{\nu}'}\, A(\alpha\beta)\right) = 0.
\]
Since $B'^{\nu}$~is arbitrary the quantity in the bracket must vanish. This shows
that $A(\mu\nu)$~is a covariant tensor obeying the transformation law~\Eq{(23.22)}.

We shall cite this theorem as the ``rigorous quotient theorem.''

With reference to the statement that an equation such as~\Eq{(24.1)} does not afford a \emph{rigorous} proof of
the tensor character of~$A(\mu\nu)$, it is desirable to give an example of failure.
Let~$F(\mu\nu)$ be any expression antisymmetrical in~$\mu$ and~$\nu$, and let~$G^{\mu\nu}$ be a symmetrical tensor,
so that
\[
F(\mu\nu) = -F(\mu\nu) \quad G^{\mu\nu} = G^{\nu\mu}.
\]
Then
\begin{align*}
F(\mu\nu)G^{\mu\nu} & = - F(\nu\mu)G^{\nu\mu}\\
                    & = - F(\mu\nu)G^{\mu\nu}
\end{align*}
by interchanging the dummy suffixes. Hence
\[
F(\mu\nu)G^{\mu\nu} = 0.
\]
Thus the product of~$F(\mu\nu)$ and~$G^{\mu\nu}$ is invariant; but it is fallacious to argue from this
that~$F(\mu\nu)$ must be a covariant tensor, since we have seen that any antisymmetrical expression will have
this property.

An equation,~$A(\mu\nu)G^{\mu\nu} = \text{invariant}$, only allows us to infer that the
symmetrical part of~$A(\mu\nu)$ is a tensor; the antisymmetrical part is artjitrary.
Similarly if~$G^{\mu\nu}$ is an antisymmetrical tensor, the inference is that the antisymmetrical part
of~$A(\mu\nu)$ is a tensor.

Thus when in~\SecRef{29} we find that~$A_{\mu\nu}$ multiplied by the symmetrical tensor~$dx_\mu/ds \cdot dx_\nu/ds$
is an invariant, the proper deduction is that the symmetrical part of~$A_{\mu\nu}$ is a tensor.
To complete the proof of~\Eq{(29.3)} it is necessary to show that the antisymmetrical part,
viz.~$\frac{1}{2}(\dd A_\mu/{\dd x_\nu} - \dd A_\nu/{\dd x_\mu})$, is also a tensor.
The reader will easily verify this by determining its transformation law, using~\Eq{(23.12)}.

Similarly the proof that~$g_{\mu\nu}$ is a tensor at the beginning of~\SecRef{25} is not rigorous.
Any antisymmetrical expression could be added to~$g_{\mu\nu}$ without altering $ds^2$, and the proof should take
account of the fact that~$g_{\mu\nu}$ is defined as a symmetrical expression.
A rigorous proof is easily supplied by finding the transformation law as suggested in~\SecRef{15}.

Although the chance of a breakdown of the general deduction from covariant and contravariant dimensions is somewhat
greater than I originally realised, I do not regret having employed the method extensively in this book.
It is desirable that the student's course of reading should train him instinctively to ``spot'' tensors in this way,
and there is never any serious difficulty in confirming his discoveries by more rigorous tests.
Although cases of failure are easily constructed artificially, I have yet to hear of a natural instance of this
happening.

\Section{25.}{The fundamental tensors}
\index{Fundamental velocity!tensors}%

It is convenient to write \Eq{(22.1)} as
\[
ds^{2} = g_{\mu\nu} (dx)^{\mu} (dx)^{\nu}
\]
in order to show explicitly the contravariant character of $dx_{\mu} = (dx)^{\mu}$. Since
$ds^{2}$~is independent of the coordinate\hyp{}system it is an invariant or tensor
of zero rank. The equation shows that $g_{\mu\nu} (dx)^{\mu}$ multiplied by an arbitrarily
chosen contravariant vector $(dx)^{\nu}$ always gives a tensor of zero rank; hence
$g_{\mu\nu} (dx)^{\mu}$ is a vector. Again, we see that $g_{\mu\nu}$~multiplied by an arbitrary contravariant
vector~$(dx)^{\mu}$ always gives a vector; hence $g_{\mu\nu}$~is a tensor. This
double application of the rigorous quotient theorem shows that $g_{\mu\nu}$~is a
tensor; and it is evidently covariant as the notation has anticipated.

Let $g$~stand for the determinant
\[
\left\lvert
\begin{array}{@{}cccc@{}}
  g_{11} & g_{12} & g_{13} & g_{14} \\
  g_{21} & g_{22} & g_{23} & g_{24} \\
  g_{31} & g_{32} & g_{33} & g_{34} \\
  g_{21} & g_{42} & g_{43} & g_{44} \\
\end{array}\right\rvert.
\]

Let $g^{\mu\nu}$~be defined as the minor of~$g_{\mu\nu}$ in this determinant, divided by~$g$\footnotemark.\footnotetext
  {The notation anticipates the result proved later that $g^{\mu\nu}$~is a contravariant tensor.}

Consider the inner product $g_{\mu\sigma} g^{\nu\sigma}$. We see that $\mu$~and $\nu$ select two rows
in the determinant; we have to take each element in turn from the $\mu$~row,
multiply by the minor of the corresponding element of the $\nu$~row, add
together, and divide the result by~$g$. This is equivalent to substituting the
$\mu$~row for the $\nu$~row and dividing the resulting determinant by~$g$. If $\mu$~is not
the same as~$\nu$ this gives a determinant with two rows identical, and the
result is~$0$. If $\mu$~is the same as~$\nu$ we reproduce the determinant~$g$ divided by
itself, and the result is~$1$. We write
\[
\left.
\begin{aligned}
  g_{\mu}^{\nu} &= g_{\mu\sigma} g^{\nu\sigma} \\
  &= 0 \quad\text{if $\mu \neq \nu$} \\
  &= 1 \quad\text{if $\mu = \nu$}
\end{aligned}
\right\}.
\Tag{(25.1)}
\]

Thus $g_{\mu}^{\nu}$~has the same property of a substitution\hyp{}operator that we found
\index{Substitution\hyp{}operator}%
for $\dfrac{\dd x_{\mu}}{\dd x_{\alpha}'}\, \dfrac{\dd x_{\alpha}'}{\dd x_{\nu}}$ in~\Eq{(22.4)}. For example\footnotemark,\footnotetext
  {Note that $g_{\mu}^{\nu}$~will act as a substitution\hyp{}operator on \emph{any} expression and is not restricted to
  operating on tensors.}
\[
g_{\mu}^{\nu} A^{\mu} = A^{\nu} + 0 + 0 + 0.
\Tag{(25.2)}
\]

Note that $g_{\nu}^{\mu}$~has not the same meaning as~$g_{\mu}^{\nu}$ with $\mu = \nu$, because a
summation is implied. Evidently
\[
g_{\nu}^{\nu} = 1 + 1 + 1 +1 = 4.
\Tag{(25.3)}
\]

The equation~\Eq{(25.2)} shows that $g_{\mu}^{\nu}$~multiplied by any contravariant vector
always gives a vector. Hence $g_{\mu}^{\nu}$~is a tensor. It is a very exceptional tensor
since its components are the same in all coordinate\hyp{}systems.

Again since $g_{\mu\sigma}g^{\nu\sigma}$~is a tensor we can infer that $g^{\nu\sigma}$~is a tensor. This is
proved rigorously by remarking that $g_{\mu\sigma} A^{\mu}$~is a covariant vector, arbitrary
on account of the free choice of~$A^{\mu}$. Multiplying this vector by~$g^{\nu\sigma}$ we have
\[
g_{\mu\sigma} g^{\nu\sigma} A^{\mu} = g_{\mu}^{\nu} A^{\mu} = A^{\nu},
\]
so that the product is always a vector. Hence the rigorous quotient theorem
applies.

The tensor character of~$g^{\mu\nu}$ may also be demonstrated by a method which
shows more clearly the reason for its definition as the minor of~$g_{\mu\nu}$ divided by~$g$.
Since $g_{\mu\nu} A^{\nu}$~is a covariant vector, we can denote it by~$B_{\mu}$. Thus
\[
g_{11} A^{1} + g_{12} A^{2} + g_{13} A^{3} + g_{14} A^{4} = B_{1}; \text{ etc.}
\]
Solving these four linear equations for $A^{1}$, $A^{2}$, $A^{3}$, $A^{4}$ by the usual method of
determinants, the result is
\[
A^{1} = g^{11} B_{1} + g^{12} B_{2} + g^{13} B_{3} + g^{14} B_{4}; \text{ etc.,}
\]
so that
\[
A^{\mu} = g^{\mu\nu} B_{\nu}.
\]
Whence by the rigorous quotient theorem $g^{\mu\nu}$~is a tensor.

We have thus defined three fundamental tensors
\[
g_{\mu\nu},\quad
g_{\mu}^{\nu},\quad
g^{\mu\nu}
\]
of covariant, mixed, and contravariant characters respectively.

\Section{26.}{Associated tensors}
\index{Associated tensors}%

We now define the operation of raising or lowering a suffix. Raising the
\index{Suffixes, raising and lowering of}%
suffix of a vector is defined by the equation
\[
A^{\mu} = g^{\mu\nu} A_{\nu},
\]
and lowering by the equation
\[
A_{\mu} = g_{\mu\nu} A^{\nu}.
\]
For a more general tensor such as~$A_{\alpha\beta\mu}^{\gamma\delta}$, the operation of raising~$\mu$ is defined
in the same way, viz.\
\[
A_{\alpha\beta}^{\gamma\delta\mu} = g^{\mu\nu} A_{\alpha\beta\nu}^{\gamma\delta},
\Tag{(26.1)}
\]
and for lowering
\[
A_{\alpha\beta\mu}^{\gamma\delta} = g_{\mu\nu} A_{\alpha\beta}^{\gamma\delta\nu}.
\Tag{(26.2)}
\]

These definitions are consistent, since if we raise a suffix and then lower
it we reproduce the original tensor. Thus if in~\Eq{(26.1)} we multiply by~$g_{\mu\sigma}$ in
order to lower the suffix on the left, we have
\begin{align*}
  g_{\mu\sigma} A_{\alpha\beta}^{\gamma\delta\mu}
  &= g_{\mu\sigma} g^{\mu\nu} A_{\alpha\beta\nu}^{\gamma\delta} \\
  &= g_{\sigma}^{\nu} A_{\alpha\beta\nu}^{\gamma\delta} \\
  &= A_{\alpha\beta\sigma}^{\gamma\delta}\qquad\text{by \Eq{(25.2)},}
\end{align*}
which is the rule expressed by~\Eq{(26.2)}.

It will be noticed that the raising of a suffix~$\nu$ by means of~$g^{\mu\nu}$ is accompanied
by the substitution of~$\mu$ for~$\nu$. The whole operation is closely akin to
the plain substitution of~$\mu$ for~$\nu$ by means of~$g_{\mu}^{\nu}$. Thus
\begin{align*}
  &\text{multiplication by $g^{\mu\nu}$ gives substitution with raising,} \\
  &\text{multiplication by $g_{\mu}^{\nu}$ gives plain substitution,} \\
  &\text{multiplication by $g_{\mu\nu}$ gives substitution with lowering.}
\end{align*}

In the case of non\hyp{}symmetrical tensors it may be necessary to distinguish
the place from which the raised suffix has been brought, e.g.\ to distinguish
between ${A_{\mu}}^{\nu}$ and~${A^{\nu}}_{\mu}$.

It is easily seen that this rule of association between tensors with suffixes
in different positions is fulfilled in the case of $g^{\mu\nu}$, $g_{\mu}^{\nu}$, $g_{\mu\nu}$; in fact the definition
of~$g_{\mu}^{\nu}$ in~\Eq{(25.1)} is a special case of~\Eq{(26.1)}.

For rectangular coordinates the raising or lowering of a suffix leaves the
components unaltered in three\hyp{}dimensional space\footnotemark;\footnotetext
  {If $ds^{2} = dx_{1}^{2} + dx_{2}^{2} + dx_{3}^{2}$, $g_{\mu\nu} = g^{\mu\nu} = g_{\mu}^{\nu}$ so that all three tensors are merely substitution\hyp{}operators.}
and it merely reverses
\index{Components, covariant and contravariant}%
the signs of some of the components for Galilean coordinates in four\hyp{}dimensional
space-time. Since the elementary definitions of physical
quantities refer to rectangular axes and time, we can generally use any one
of the associated tensors to represent a physical entity without infringing
pre\hyp{}relativity definitions. This leads to a somewhat enlarged view of a tensor
as having in itself no particular covariant or contravariant character, but
having \emph{components} of various degrees of covariance or contravariance represented
by the whole system of associated tensors. That is to say, the raising
or lowering of suffixes will not be regarded as altering the individuality of
the tensor; and reference to a tensor~$A_{\mu\nu}$ may (if the context permits) be
taken to include the associated tensors~$A_{\mu}^{\nu}$ and~$A^{\mu\nu}$.

It is useful to notice that dummy suffixes have a certain freedom of movement
between the tensor\hyp{}factors of an expression. Thus
\[
A_{\alpha\beta} B^{\alpha\beta} = A^{\alpha\beta} B_{\alpha\beta},\quad
A_{\mu\alpha} B^{\nu\alpha} = {A_{\mu}}^{\alpha} {B^{\nu}}_{\alpha}.
\Tag{(26.3)}
\]
The suffix may be raised in one term provided it is lowered in the other.
The proof follows easily from \Eq{(26.1)} and~\Eq{(26.2)}.

In the elementary vector theory two vectors are said to be \emph{perpendicular}
\index{Length of a vector}%
\index{Perpendicularity of vectors}%
\index{Self\hyp{}perpendicular vector}%
if their scalar\hyp{}product vanishes; and the square of the \emph{length} of the vector is
its scalar\hyp{}product into itself. Corresponding definitions are adopted in the
tensor calculus.

The vectors $A_{\mu}$ and~$B_{\mu}$ are said to be \emph{perpendicular} if
\[
A_{\mu} B^{\mu} = 0.
\Tag{(26.4)}
\]

If $l$~is the \emph{length} of~$A_{\mu}$ (or~$A^{\mu}$)
\[
l^{2} = A_{\mu} A^{\mu}.
\Tag{(26.5)}
\]
A vector is self\hyp{}perpendicular if its length vanishes.

The interval is the length of the corresponding displacement~$dx_{\mu}$ because
\begin{align*}
  ds^{2} &= g_{\mu\nu}\, (dx)^{\mu} \cdot (dx)^{\nu} \\
  &= (dx)_{\nu} (dx)^{\nu}
\end{align*}
by~\Eq{(26.2)}. A displacement is thus self\hyp{}perpendicular when it is along a
light-track, $ds = 0$.

If a vector~$A_{\mu}$ receives an infinitesimal increment~$dA_{\mu}$ perpendicular to
itself, its length is unaltered to the first order; for by~\Eq{(26.5)}
\begin{align*}
  (l + dl)^{2}
  &= (A_{\mu} + dA_{\mu}) (A^{\mu} + dA^{\mu}) \\
  &= A_{\mu} A^{\mu} + A^{\mu}\, dA_{\mu} + A_{\mu}\, dA^{\mu}\quad\text{to the first order} \\
  &= l^{2} + 2A_{\mu}\, dA^{\mu} \quad\text{by~\Eq{(26.3)},}
\end{align*}
and $A_{\mu}\, dA^{\mu} = 0$ by the condition of perpendicularity~\Eq{(26.4)}.

In the elementary vector theory, the scalar\hyp{}product of two vectors is
equal to the product of their lengths multiplied by the cosine of the angle
\index{Angle between two vectors}%
between them. Accordingly in the general theory the angle~$\theta$ between two
vectors $A_{\mu}$ and $B_{\mu}$ is defined by
\[
\cos\theta = \frac{A_{\mu} B^{\mu}}{\sqrt{(A_{\alpha} A^{\alpha}) (B_{\beta} B^{\beta})}}.
\Tag{(26.6)}
\]
Clearly the angle so defined is an invariant, and agrees with the usual
\index{Invariant!formation of}%
definition when the coordinates are rectangular. In determining the angle
between two intersecting lines it makes no difference whether the world is
curved or flat, since only the initial directions are concerned and these in any
case lie in the tangent plane. The angle~$\theta$ (if it is real) has thus the usual
geometrical meaning even in non\hyp{}Euclidean space. It must not, however, be
inferred that ordinary angles are invariant for the Lorentz transformation;
naturally an angle in three dimensions is invariant only for transformations
in three dimensions, and the angle which is invariant for Lorentz transformations
is a four\hyp{}dimensional angle.

From a tensor of even rank we can construct an invariant by bringing
half the suffixes to the upper and half to the lower position and contracting.
Thus from~$A_{\mu\nu\sigma\tau}$ we form~$A_{\mu\nu}^{\sigma\tau}$ and contract, obtaining $A = A_{\mu\nu}^{\mu\nu}$. This invariant
will be called the \emph{spur}\footnotemark.\footnotetext
  {Originally the German word \Foreign{Spur}.}
Another invariant is the square of the
\index{Spur}%
length $A_{\mu\nu\sigma\tau} A^{\mu\nu\sigma\tau}$. There may also be intermediate invariants such as
$A_{\mu\nu\alpha}^{\alpha} A_{\beta}^{\mu\nu\beta}$.

\Section{27.}{Christoffel's $3$-index symbols}
\index{Christoffel's $3$-index symbols}%
\index{Three-index symbol}%

We introduce two expressions (not tensors) of great importance throughout
our subsequent work, namely
\begin{align*}
  [\mu\nu, \sigma]
  &= \tfrac{1}{2} \left(\frac{\dd g_{\mu\sigma}}{\dd x_{\nu}} + \frac{\dd g_{\nu\sigma}}{\dd x_{\mu}} - \frac{\dd g_{\mu\nu}}{\dd x_{\sigma}}\right),
  \Tag{(27.1)} \\
  \{\mu\nu, \sigma\}
  &= \tfrac{1}{2} g^{\sigma\lambda} \left(\frac{\dd g_{\mu\lambda}}{\dd x_{\nu}} + \frac{\dd g_{\nu\lambda}}{\dd x_{\mu}} - \frac{\dd g_{\mu\nu}}{\dd x_{\lambda}}\right).
\Tag{(27.2)}
\end{align*}
We have
\begin{align*}
  \{\mu\nu, \sigma\} &= g^{\sigma\lambda}\, [\mu\nu, \lambda],
  \Tag{(27.3)} \\
  [\mu\nu, \sigma] &= g_{\sigma\lambda}\, \{\mu\nu, \lambda\}.
  \Tag{(27.4)}
\end{align*}
The result~\Eq{(27.3)} is obvious from the definitions. To prove~\Eq{(27.4)}, multiply \Eq{(27.3)}
by~$g_{\sigma\alpha}$; then
\begin{align*}
  g_{\sigma\alpha}\, \{\mu\nu, \sigma\}
  &= g_{\sigma\alpha} g^{\sigma\lambda}\, [\mu\nu, \lambda] \\
  &= g_{\alpha}^{\lambda}\, [\mu\nu, \lambda] \\
  &= [\mu\nu, \alpha],
\end{align*}
which is equivalent to~\Eq{(27.4)}.

Comparing with \Eq{(26.1)} and~\Eq{(26.2)} we see that the passage from the
``square'' to the ``curly'' symbol, and \Foreign{vice versa,} is the same process as raising
and lowering a suffix. It might be convenient to use a notation in which
this was made evident, e.g.\
\[
\Gamma_{\mu\nu, \sigma} = [\mu\nu, \sigma],\quad
\Gamma_{\mu\nu}^{\sigma} = \{\mu\nu, \sigma\},
\]
but we shall adhere to the more usual notation.

From \Eq{(27.1)} it is found that
\[
[\mu\nu, \sigma] + [\sigma\nu, \mu] = \frac{\dd g_{\mu\sigma}}{\dd x_{\nu}}.
\Tag{(27.5)}
\]

There are $40$~different $3$-index symbols of each kind. It may here be
explained that the~$g_{\mu\nu}$ are components of a generalised \emph{potential,} and the
\index{Potential!gravitational}%
$3$-index symbols components of a generalised \emph{force} in the gravitational
theory (see \SecRef{55}).

\Section{28.}{Equations of a geodesic}

We shall now determine the equations of a geodesic or path between two
points for which
\[
\int ds \text{ is stationary.}
\]
This absolute track is of fundamental importance in dynamics, but at the
moment we are concerned with it only as an aid in the development of the
tensor calculus\footnotemark.\footnotetext
  {Our ultimate goal is equation~\Eq{(29.3)}. An alternative proof (which does not introduce the
  calculus of variations) is given in \SecRef{31}.}

Keeping the beginning and end of the path fixed, we give every intermediate
point an arbitrary infinitesimal displacement~$\delta x_{\sigma}$ so as to deform the
path. Since
\begin{align*}
  ds^{2} &= g_{\mu\nu}\, dx_{\mu}\, dx_{\nu}, \\
  2\, ds\, \delta(ds)
  &= dx_{\mu}\, dx_{\nu} \, \delta g_{\mu\nu}
  + g_{\mu\nu}\, dx_{\mu}\, \delta(dx_{\nu})
  + g_{\mu\nu}\, dx_{\nu}\, \delta(dx_{\mu}) \\
  &= dx_{\mu}\, dx_{\nu}\, \frac{\dd g_{\mu\nu}}{\dd x_{\sigma}}\, \delta x_{\sigma}
  + g_{\mu\nu}\, dx_{\mu}\, d(\delta x_{\nu})
  + g_{\mu\nu}\, dx_{\nu}\, d(\delta x_{\mu}).
  \Tag{(28.1)}
\end{align*}
The stationary condition is
\[
\int \delta(ds) = 0,
\Tag{(28.2)}
\]
which becomes by~\Eq{(28.1)}
\[
  \frac{1}{2} \int \biggl\{
  \frac{dx_{\mu}}{ds}\, \frac{dx_{\nu}}{ds}\, \frac{\dd g_{\mu\nu}}{\dd x_{\sigma}}\, \delta x_{\sigma}
  + g_{\mu\nu}\, \frac{dx_{\mu}}{ds}\, \frac{d}{ds}(\delta x_{\nu})
  + g_{\mu\nu}\, \frac{dx_{\nu}}{ds}\, \frac{d}{ds}(\delta x_{\mu})\biggr\}\, ds = 0,
\]
or, changing dummy suffixes in the last two terms,
\[
\frac{1}{2} \int \biggl\{
\frac{dx_{\mu}}{ds}\, \frac{dx_{\nu}}{ds}\, \frac{\dd g_{\mu\nu}}{\dd x_{\sigma}}\, \delta x_{\sigma}
+ \biggl(g_{\mu\sigma}\, \frac{dx_{\mu}}{ds} + g_{\sigma\nu}\, \frac{dx_{\nu}}{ds}\biggr)\, \frac{d}{ds}(\delta x_{\sigma})\biggr\}\, ds = 0.
\]
Applying the usual method of partial integration, and rejecting the integrated
part since $\delta x_{\sigma}$~vanishes at both limits,
\[
\frac{1}{2} \int \biggl\{
\frac{dx_{\mu}}{ds}\, \frac{dx_{\nu}}{ds}\, \frac{\dd g_{\mu\nu}}{\dd x_{\sigma}}
  - \frac{d}{ds} \biggl(g_{\mu\sigma}\, \frac{dx_{\mu}}{ds} + g_{\sigma\nu}\, \frac{dx_{\nu}}{ds}\biggr)\, \delta x_{\sigma}\, ds = 0.
\]

This must hold for all values of the arbitrary displacements~$\delta x_{\sigma}$ at all
\index{Geodesic, equations of}%
points, hence the coefficient in the integrand must vanish at all points on the
path. Thus
\[
\frac{1}{2} \frac{dx_{\mu}}{ds}\, \frac{dx_{\nu}}{ds}\, \frac{\dd g_{\mu\nu}}{\dd x_{\sigma}}
  - \frac{1}{2} \frac{dg_{\mu\sigma}}{ds}\, \frac{dx_{\mu}}{ds}
  - \frac{1}{2} \frac{dg_{\sigma\nu}}{ds}\, \frac{dx_{\nu}}{ds}
  - \frac{1}{2} g_{\mu\sigma}\, \frac{d^{2}x_{\mu}}{ds^{2}}
  - \frac{1}{2} g_{\sigma\nu}\, \frac{d^{2}x_{\nu}}{ds^{2}} = 0.
\]
Now\footnote
  {These simple formulae are noteworthy as illustrating the great value of the summation
  convention. The law of total differentiation for four coordinates becomes formally the same as for
  one coordinate.}
\[
\frac{dg_{\mu\sigma}}{ds} = \frac{\dd g_{\mu\sigma}}{\dd x_{\nu}}\, \frac{dx_{\nu}}{ds}
\quad\text{and}\quad
\frac{dg_{\sigma\nu}}{ds} = \frac{\dd g_{\sigma\nu}}{\dd x_{\mu}}\, \frac{dx_{\mu}}{ds}.
\]
Also in the last two terms we replace the dummy suffixes $\mu$ and~$\nu$ by~$\epsilon$. The
equation then becomes
\[
\frac{1}{2}\, \frac{dx_{\mu}}{ds}\, \frac{dx_{\nu}}{ds}\,
\biggl(\frac{\dd g_{\mu\nu}}{\dd x_{\sigma}} - \frac{\dd g_{\mu\sigma}}{\dd x_{\nu}} - \frac{\dd g_{\nu\sigma}}{\dd x_{\mu}}\biggr)
- g_{\epsilon\sigma}\, \frac{d^{2} x_{\epsilon}}{ds^{2}} = 0.
\Tag{(28.3)}
\]

We can get rid of the factor~$g_{\epsilon\sigma}$ by multiplying through by~$g^{\sigma\alpha}$ so as to
form the substitution operator~$g_{\epsilon}^{\alpha}$. Thus
\[
\frac{1}{2}\, \frac{dx_{\mu}}{ds}\, \frac{dx_{\nu}}{ds}\,
g^{\sigma\alpha} \biggl(\frac{\dd g_{\mu\sigma}}{\dd x_{\nu}} + \frac{\dd g_{\nu\sigma}}{\dd x_{\mu}} - \frac{\dd g_{\mu\nu}}{\dd x_{\sigma}}\biggr)
+ \frac{d^{2} x_{\alpha}}{ds^{2}} = 0,
\Tag{(28.4)}
\]
or, by~\Eq{(27.2)}
\[
\frac{d^{2} x_{\alpha}}{ds^{2}} + \{\mu\nu, \alpha\}\, \frac{dx_{\mu}}{ds}\, \frac{dx_{\nu}}{ds} = 0.
\Tag{(28.5)}
\]

For $\alpha = 1$, $2$, $3$, $4$ this gives the four equations determining a geodesic.

\Section{29.}{Covariant derivative of a vector}
\index{Covariant derivative of vector}%
\index{Derivative!covariant}%

The derivative of an invariant is a covariant vector (\SecRef{19}), but the
derivative of a vector is not a tensor. We proceed to find certain tensors
which are used in this calculus in place of the ordinary derivatives of vectors.

Since $dx_{\mu}$~is contravariant and $ds$~invariant, a ``velocity'' $dx_{\mu}/ds$ is a
contravariant vector. Hence if $A_{\mu}$~is any covariant vector the inner product
\[
A_{\mu}\, \frac{dx_{\mu}}{ds} \text{ is invariant.}
\]
The rate of change of this expression per unit interval along any assigned
curve must also be independent of the coordinate\hyp{}system, i.e.\
\[
\frac{d}{ds} \left(A_{\mu}\, \frac{dx_{\mu}}{ds}\right)
\text{ is invariant.}
\Tag{(29.1)}
\]
This assumes that we keep to the same absolute curve however the coordinate\hyp{}system
is varied. The result~\Eq{(29.1)} is therefore only of practical use if it is
applied to a curve which is defined independently of the coordinate\hyp{}system.
We shall accordingly apply it to a geodesic. Performing the differentiation,
\[
\frac{\dd A_{\mu}}{\dd x_{\nu}}\, \frac{dx_{\nu}}{ds}\, \frac{dx_{\mu}}{ds}
+ A_{\mu}\, \frac{d^{2}x_{\mu}}{ds^{2}} \text{ is invariant along a geodesic.}
\Tag{(29.2)}
\]
From \Eq{(28.5)} we have that along a geodesic
\[
A_{\mu}\, \frac{d^{2}x_{\mu}}{ds^{2}} = A_{\alpha}\, \frac{d^{2}x_{\alpha}}{ds^{2}}
= -A_{\alpha}\, \{\mu\nu, \alpha\}\, \frac{dx_{\mu}}{ds}\, \frac{dx_{\nu}}{ds}.
\]
Hence \Eq{(29.2)}~gives
\[
\frac{dx_{\mu}}{ds}\, \frac{dx_{\nu}}{ds} \left(\frac{\dd A_{\mu}}{\dd x_{\nu}} - A_{\alpha} \{\mu\nu, \alpha\}\right) \text{ is invariant.}
\]
The result is now general since the curvature (which distinguishes the
geodesic) has been eliminated by using the equations~\Eq{(28.5)} and only the
gradient of the curve ($dx_{\mu}/ds$ and $dx_{\nu}/ds$) has been left in the expression.

Since $dx_{\mu}/ds$ and $dx_{\nu}/ds$ are contravariant vectors, their co\hyp{}factor is a
covariant tensor of the second rank. We therefore write
\[
A_{\mu\nu} = \frac{\dd A_{\mu}}{\dd x_{\nu}} - \{\mu\nu, \alpha\}\, A_{\alpha},
\Tag{(29.3)}
\]
and the tensor~$A_{\mu\nu}$ is called the \emph{covariant derivative} of~$A_{\mu}$.

By raising a suffix we obtain two associated tensors ${A^{\mu}}_{\nu}$ and ${A_{\mu}}^{\nu}$ which
must be distinguished since the two suffixes are not symmetrical. The first
of these is the most important, and is to be understood when the tensor
is written simply as~$A_{\nu}^{\mu}$ without distinction of original position.

Since
\[
A_{\sigma} = g_{\sigma\epsilon} A^{\epsilon},
\]
we have by~\Eq{(29.3)}
\begin{align*}
  A_{\sigma\nu}
  &= \frac{\dd}{\dd x_{\nu}} (g_{\sigma\epsilon} A^{\epsilon}) - \{\sigma\nu, \alpha\}\, (g_{\alpha\epsilon} A^{\epsilon}) \\
  &= g_{\sigma\epsilon}\, \frac{\dd A^{\epsilon}}{\dd x_{\nu}} + A^{\epsilon}\, \frac{\dd g_{\sigma\epsilon}}{\dd x_{\nu}} - [\sigma\nu, \epsilon]\, A^{\epsilon} \text{ by~\Eq{(27.4)}} \\
  &= g_{\sigma\epsilon}\, \frac{\dd A^{\epsilon}}{\dd x_{\nu}} + [\epsilon\nu, \sigma]\, A^{\epsilon} \text{ by~\Eq{(27.5)}.}
\end{align*}
Hence multiplying through by~$g^{\mu\sigma}$, and remembering that $g^{\mu\sigma}g_{\sigma\epsilon}$~is a
substitution\hyp{}operator, we have
\[
{A^{\mu}}_{\nu} = \frac{\dd A^{\mu}}{\dd x_{\nu}} + \{\epsilon\nu, \mu\}\, A^{\epsilon}.
\Tag{(29.4)}
\]
This is called the covariant derivative of~$A^{\mu}$. The considerable differences
\index{Covariant derivative of vector!of tensor}%
\index{Derivative!covariant}%
\index{Derivative!contravariant}%
between the formulae \Eq{(29.3)} and \Eq{(29.4)} should be carefully noted.

The tensors ${A_{\mu}}^{\nu}$ and~$A^{\mu\nu}$, obtained from \Eq{(29.3)} and~\Eq{(29.4)} by raising the
second suffix, are called the \emph{contravariant derivatives} of $A_{\mu}$ and~$A^{\mu}$. We shall
\index{Contravariant vectors!derivatives}%
not have much occasion to refer to contravariant derivatives.

\Section{30.}{Covariant derivative of a tensor}

The covariant derivatives of tensors of the second rank are formed as
follows---
\begin{align*}
  A_{\sigma}^{\mu\nu} &= \frac{\dd A^{\mu\nu}}{\dd x_{\sigma}}
    + \{\alpha\sigma, \mu\}\, A^{\alpha\nu} + \{\alpha\sigma, \nu\}\, A^{\mu\alpha},
\Tag{(30.1)} \\
  A_{\mu\sigma}^{\nu} &= \frac{\dd A_{\mu}^{\nu}}{\dd x_{\sigma}}
    - \{\mu\sigma, \alpha\}\, A_{\alpha}^{\nu} + \{\alpha\sigma, \nu\}\, A_{\mu}^{\alpha},
\Tag{(30.2)} \\
  A_{\mu\nu\sigma} &= \frac{\dd A_{\mu\nu}}{\dd x_{\sigma}}
    - \{\mu\sigma, \alpha\}\, A_{\alpha\nu} - \{\nu\sigma, \alpha\}\, A_{\mu\alpha}.
\Tag{(30.3)}
\end{align*}
And the general rule for covariant differentiation with respect to~$x_{\sigma}$ is
illustrated by the example
%[** TN: Not broken in the original]
\begin{multline*}
  A_{\lambda\mu\nu\sigma}^{\rho}
  = \frac{\dd}{\dd x_{\sigma}}\, A_{\lambda\mu\nu}^{\rho}
  - \{\lambda\sigma, \alpha\}\, A_{\alpha\mu\nu}^{\rho}
  - \{\mu\sigma, \alpha\}\, A_{\lambda\alpha\nu}^{\rho} \\
  - \{\nu\sigma, \alpha\}\, A_{\lambda\mu\alpha}^{\rho}
  + \{\alpha\sigma, \rho\}\, A_{\lambda\mu\nu}^{\alpha}.
  \Tag{(30.4)}
\end{multline*}

The above formulae are primarily definitions; but we have to prove that
the quantities on the right are actually tensors. This is done by an obvious
generalisation of the method of the preceding section. Thus if in place of~\Eq{(29.1)}
we use
\[
\frac{d}{ds} \left(A_{\mu\nu}\, \frac{dx_{\mu}}{ds}\, \frac{dx_{\nu}}{ds}\right)
\text{ is invariant along a geodesic,}
\]
we obtain
\[
\frac{\dd A_{\mu\nu}}{\dd x_{\sigma}}\, \frac{dx_{\sigma}}{ds}\, \frac{dx_{\mu}}{ds}\, \frac{dx_{\nu}}{ds}
+ A_{\mu\nu}\, \frac{dx_{\nu}}{ds}\, \frac{d^{2}x_{\mu}}{ds^{2}}
+ A_{\mu\nu}\, \frac{dx_{\mu}}{ds}\, \frac{d^{2}x_{\nu}}{ds^{2}}.
\]
Then substituting for the second derivatives from~\Eq{(28.5)} the expression
reduces to
\[
A_{\mu\nu\sigma}\, \frac{dx_{\mu}}{ds}\, \frac{dx_{\nu}}{ds}\, \frac{dx_{\sigma}}{ds}
\text{ is invariant,}
\]
showing that $A_{\mu\nu\sigma}$~is a tensor.

The formulae \Eq{(30.1)}~and \Eq{(30.2)} are obtained by raising the suffixes $\nu$ and~$\mu$,
the details of the work being the same as in deducing \Eq{(29.4)} from~\Eq{(29.3)}.

Consider the expression
\[
B_{\mu\sigma} C_{\nu} + B_{\mu} C_{\nu\sigma},
\]
the $\sigma$ denoting covariant differentiation. By~\Eq{(29.3)} this is equal to
\begin{multline*}
\left(\frac{\dd B_{\mu}}{\dd x_{\sigma}} - \{\mu\sigma, \alpha\}\, B_{\alpha}\right) C_{\nu}
+ B_{\mu} \left(\frac{\dd C_{\nu}}{\dd x_{\sigma}} - \{\nu\sigma, \alpha\}\, C_{\alpha}\right) \\
= \frac{\dd}{\dd x_{\sigma}}\, (B_{\mu} C_{\nu})
  - \{\mu\sigma, \alpha\}\, (B_{\alpha} C_{\nu})
  - \{\nu\sigma, \alpha\}\, (B_{\mu} C_{\alpha}).
\end{multline*}
But comparing with~\Eq{(30.3)} we see that this is the covariant derivative of the
\index{Covariant derivative of vector!of invariant}%
\index{Covariant derivative of vector!utility of}%
tensor of the second rank~$(B_{\mu}C_{\nu})$. Hence
\[
(B_{\mu} C_{\nu})_{\sigma} = B_{\mu\sigma} C_{\nu} + B_{\mu} C_{\nu\sigma}.
\Tag{(30.5)}
\]
Thus in covariant differentiation of a product the distributive rule used in
ordinary differentiation holds good.

Applying~\Eq{(30.3)} to the fundamental tensor, we have
\begin{align*}
  g_{\mu\nu\sigma}
  &= \frac{\dd g_{\mu\nu}}{\dd x_{\sigma}} - \{\mu\sigma, \nu\}\, g_{\alpha\nu} - \{\nu\sigma, \alpha\}\, g_{\mu\alpha} \\
  &= \frac{\dd g_{\mu\nu}}{\dd x_{\sigma}} - [\mu\sigma, \nu] - [\nu\sigma, \mu] \\
  &= 0 \text{ by \Eq{(27.5)}.}
\end{align*}

Hence the covariant derivatives of the fundamental tensors vanish identically,
and the fundamental tensors can be treated as \emph{constants} in covariant
differentiation. It is thus immaterial whether a suffix is raised before or after
the differentiation, as our definitions have already postulated.

If $I$~is an invariant, $IA_{\mu}$~is a covariant vector; hence its covariant
derivative is
\begin{align*}
  (IA_{\mu})_{\nu} &= \frac{\dd}{\dd x_{\nu}}(IA_{\mu}) - \{\mu\nu, \alpha\}\, IA_{\alpha} \\
  &= A_{\mu}\, \frac{\dd I}{\dd x_{\nu}} + IA_{\mu\nu}.
\end{align*}
But by the rule for differentiating a product~\Eq{(30.5)}
\[
(IA_{\mu})_{\nu} = I_{\nu} A_{\mu} + IA_{\mu\nu},
\]
so that
\[
I_{\nu} = \frac{\dd I}{\dd x_{\nu}}.
\]
Hence the covariant derivative of an invariant is the same as its ordinary
derivative.

It is, of course, impossible to reserve the notation~$A_{\mu\nu}$ exclusively for the
covariant derivative of~$A_{\mu}$, and the concluding suffix does not denote differentiation
unless expressly stated. In case of doubt we may indicate the covariant
and contravariant derivatives by $(A_{\mu})_{\nu}$ and~$(A_{\mu})^{\nu}$.

The utility of the covariant derivative arises largely from the fact that, when
the~$g_{\mu\nu}$ are constants, the $3$-index symbols vanish and the covariant derivative
reduces to the ordinary derivative. Now in general our physical equations
have been stated for the case of Galilean coordinates in which the~$g_{\mu\nu}$ are
constants; and we may in Galilean equations replace the ordinary derivative
by the covariant derivative without altering anything. This is a necessary
step in reducing such equations to the general tensor form which holds true
for all coordinate\hyp{}systems.

As an illustration suppose we wish to find the general equation of propagation
\index{Propagation with unit velocity}%
of a potential with the velocity of light. In Galilean coordinates the
equation is of the well-known form
\[
\Wave\phi
  = \frac{\dd^{2} \phi}{\dd t^{2}}
  - \frac{\dd^{2} \phi}{\dd x^{2}}
  - \frac{\dd^{2} \phi}{\dd y^{2}}
  - \frac{\dd^{2} \phi}{\dd z^{2}} = 0.
\Tag{(30.6)}
\]

The Galilean values of~$g^{\mu\nu}$ are $g^{44} = 1$, $g^{11} = g^{22} = g^{33} = -1$, and the other
components vanish. Hence \Eq{(30.6)}~can be written
\[
g^{\mu\nu}\, \frac{\dd^{2} \phi}{\dd x_{\mu}\, \dd x_{\nu}} = 0.
\Tag{(30.65)}
\]
The potential~$\phi$ being an invariant, its ordinary derivative is a covariant
vector $\phi_{\mu} = \dd\phi/\dd x_{\mu}$; and since the coordinates are Galilean we may insert
the covariant derivative~$\phi_{\mu\nu}$ instead of~$\dd \phi_{\mu}/\dd x_{\nu}$. Hence the equation becomes
\[
g^{\mu\nu} \phi_{\mu\nu} = 0.
\Tag{(30.7)}
\]
Up to this point Galilean coordinates are essential; but now, by examining the
covariant dimensions of~\Eq{(30.7)}, we notice that the left-hand side is an invariant,
and therefore its value is unchanged by any transformation of coordinates.
Hence \Eq{(30.7)}~holds for all coordinate\hyp{}systems, if it holds for any. Using~\Eq{(29.3)}
we can write it more fully
\[
g^{\mu\nu} \left(\frac{\dd^{2}\phi}{\dd x_{\mu}\, \dd x_{\nu}} - \{\mu\nu, \alpha\}\, \frac{\dd\phi}{\dd x_{\alpha}}\right) = 0.
\Tag{(30.8)}
\]
This formula may be used for transforming Laplace's equation into curvilinear
coordinates, etc.

It must be remembered that a transformation of coordinates does not alter
the kind of space. Thus if we know by experiment that a potential~$\phi$ is
propagated according to the law~\Eq{(30.6)} in Galilean coordinates, it follows
rigorously that it is propagated according to the law~\Eq{(30.8)} in any system of
coordinates in flat space-time; but it does not follow rigorously that it will
be propagated according to~\Eq{(30.8)} when an irreducible gravitational field is
present which alters the kind of space-time. It is, however, a plausible
suggestion that \Eq{(30.8)}~may be the general law of propagation of~$\phi$ in any kind
of space-time; that is the suggestion which the principle of equivalence makes.
Like all generalisations which are only tested experimentally in a particular
case, it must be received with caution.

The operator~$\Wave$ will frequently be referred to. In general coordinates it
\index{Contracted derivative (divergence)!second derivative ($\Wave$)}%
\index{Operators!W@$\Wave$}%
is to be taken as defined by
\[
\Wave A_{\mu\nu\cdots} = g^{\alpha\beta} (A_{\mu\nu\cdots})_{\alpha\beta}.
\Tag{(30.9)}
\]
Or we may write it in the form
\[
\Wave = \bigl((\cdots)_{\alpha}\bigr)^{\alpha},
\]
i.e.\ we perform a covariant and contravariant differentiation and contract
them.

\Summary{Summary of Rules for Covariant Differentiation.}%
\index{Covariant derivative of vector!of tensor}%
\index{Differentiation!covariant, rules for}%

1. To obtain the covariant derivative of any tensor~$A_{...}^{...}$ with respect to~$x_{\sigma}$,
\index{Derivative!covariant}%
we take first the ordinary derivative
\[
\frac{\dd}{\dd x_{\sigma}} A_{...}^{...}\,;
\]
and for \emph{each} covariant suffix~$A_{.\mu.}^{...}$, we add a term
\[
- \{\mu\sigma, \alpha\} A_{.\alpha.}^{...};
\]
and for \emph{each} contravariant suffix~$A_{...}^{.\mu.}$, we add a term
\[
+ \{\alpha\sigma, \mu\} A_{...}^{.\alpha.}.
\]

2. The covariant derivative of a product is formed by covariant differentiation
of each factor in turn, by the same rule as in ordinary differentiation.

3. The fundamental tensor~$g_{\mu\nu}$ or~$g^{\mu\nu}$ behaves as though it were a constant
in covariant differentiation.

4. The covariant derivative of an invariant is its ordinary derivative.

5. In taking second, third or higher derivatives, the order of differentiation
is not interchangeable\footnotemark.\footnotetext
  {This is inserted here for completeness; it is discussed later.}

\Section{31.}{Alternative discussion of the covariant derivative}

By~\Eq{(23.22)}
\[
g_{\mu\nu}' = \frac{\dd x_{\alpha}}{\dd x_{\mu}'}\, \frac{\dd x_{\beta}}{\dd x_{\nu}'}\, g_{\alpha\beta}.
\]
Hence differentiating
\[
\frac{\dd g_{\mu\nu}'}{\dd x_{\lambda}'}
= g_{\alpha\beta} \left\{
    \frac{\dd^{2} x_{\alpha}}{\dd x_{\lambda}'\, \dd x_{\mu}'}\, \frac{\dd x_{\beta}}{\dd x_{\nu}'}
  + \frac{\dd^{2} x_{\alpha}}{\dd x_{\lambda}'\, \dd x_{\nu}'}\, \frac{\dd x_{\beta}}{\dd x_{\mu}'}\right\}
  + \frac{\dd x_{\alpha}}{\dd x_{\mu}'}\,
    \frac{\dd x_{\beta}}{\dd x_{\nu}'}\,
    \frac{\dd x_{\gamma}}{\dd x_{\lambda}'}\,
    \frac{\dd g_{\alpha\beta}}{\dd x_{\gamma}}.
\Tag{(31.11)}
\]

Here we have used
\[
\frac{\dd g_{\alpha\beta}}{\dd x_{\lambda}'}
= \frac{\dd g_{\alpha\beta}}{\dd x_{\gamma}}\, \frac{\dd x_{\gamma}}{\dd x_{\lambda}'},
\]
and further we have interchanged the dummy suffixes $\alpha$ and~$\beta$ in the second
term in the bracket. Similarly
\begin{align*}
  \frac{\dd g_{\nu\lambda}'}{\dd x_{\mu}'}
  &= g_{\alpha\beta} \left\{
    \frac{\dd^{2} x_{\alpha}}{\dd x_{\mu}'\, \dd x_{\nu}'}\, \frac{\dd x_{\beta}}{\dd x_{\lambda}'}
  + \frac{\dd^{2} x_{\alpha}}{\dd x_{\mu}'\, \dd x_{\lambda}'}\, \frac{\dd x_{\beta}}{\dd x_{\nu}'}\right\}
  + \frac{\dd x_{\alpha}}{\dd x_{\mu}'}\,
    \frac{\dd x_{\beta}}{\dd x_{\nu}'}\,
    \frac{\dd x_{\gamma}}{\dd x_{\lambda}'}\,
    \frac{\dd g_{\beta\gamma}}{\dd x_{\alpha}},
  \Tag{(31.12)} \\
  \frac{\dd g_{\nu\lambda}'}{\dd x_{\nu}'}
  &= g_{\alpha\beta} \left\{
    \frac{\dd^{2} x_{\alpha}}{\dd x_{\nu}'\, \dd x_{\mu}'}\, \frac{\dd x_{\beta}}{\dd x_{\lambda}'}
  + \frac{\dd^{2} x_{\alpha}}{\dd x_{\nu}'\, \dd x_{\lambda}'}\, \frac{\dd x_{\beta}}{\dd x_{\mu}'}\right\}
  + \frac{\dd x_{\alpha}}{\dd x_{\mu}'}\,
    \frac{\dd x_{\beta}}{\dd x_{\nu}'}\,
    \frac{\dd x_{\gamma}}{\dd x_{\lambda}'}\,
    \frac{\dd g_{\alpha\gamma}}{\dd x_{\beta}}.
\Tag{(31.13)}
\end{align*}
% [** TN: [sic] Add and subtract]
Add \Eq{(31.12)} and \Eq{(31.13)} and subtract~\Eq{(31.11)}, we obtain by~\Eq{(27.1)}

\[
[\mu\nu, \lambda]'
= g_{\alpha\beta}\, \frac{\dd x_{\alpha}}{\dd x_{\mu}'\, \dd x_{\nu}'}\, \frac{\dd x_{\beta}}{\dd x_{\lambda}'}
+ \frac{\dd x_{\alpha}}{\dd x_{\mu}'}\,
  \frac{\dd x_{\beta}}{\dd x_{\nu}'}\,
  \frac{\dd x_{\gamma}}{\dd x_{\lambda}'}\, [\alpha\beta, \gamma].
\Tag{(31.2)}
\]
%[** TN: [sic] Multiply]
Multiply through by $g'^{\lambda\rho}\, \dfrac{\dd x_{\epsilon}}{\dd x_{\rho}'}$, we have by~\Eq{(27.3)}
%[** TN: Added break in first line]
\begin{align*}
  \{\mu\nu, \rho\}\, \frac{\dd x_{\epsilon}}{\dd x_{\rho}'}
  &= g_{\alpha\beta}\, \frac{\dd^{2} x_{\alpha}}{\dd x_{\mu}'\, \dd x_{\nu}'}
  \cdot g'^{\lambda\rho}\, \frac{\dd x_{\beta}}{\dd x_{\lambda}'}\, \frac{\dd x_{\epsilon}}{\dd x_{\rho}'} \\
  &\qquad\qquad+ g'^{\lambda\rho}\, \frac{\dd x_{\gamma}}{\dd x_{\lambda}'}\, \frac{\dd x_{\epsilon}}{\dd x_{\rho}'}
  \cdot \frac{\dd x_{\alpha}}{\dd x_{\mu}'}\, \frac{\dd x_{\beta}}{\dd x_{\nu}'}\, [\alpha\beta, \gamma]\displaybreak[0] \\
  &= g_{\alpha\beta} g^{\beta\epsilon}\, \frac{\dd^{2} x_{\alpha}}{\dd x_{\mu}'\, \dd x_{\nu}'}
  + \frac{\dd x_{\alpha}}{\dd x_{\mu}'}\, \frac{\dd x_{\beta}}{\dd x_{\nu}'}\, g^{\gamma\epsilon}\, [\alpha\beta, \gamma]
  \qquad\text{by \Eq{(23.21)}}\displaybreak[0] \\
  &= \frac{\dd^{2} x_{\epsilon}}{\dd x_{\mu}'\, \dd x_{\nu}'}
  + \frac{\dd x_{\alpha}}{\dd x_{\mu}'}\, \frac{\dd x_{\beta}}{\dd x_{\nu}'}\, \{\alpha\beta, \epsilon\},
  \Tag{(31.3)}
\end{align*}
a formula which determines the second derivative $\dd^{2} x_{\epsilon}/\dd x_{\mu'}'\, \dd x_{\nu}'$ in terms of the
first derivatives.

By~\Eq{(23.12)}
\[
A_{\mu}' = \frac{\dd x_{\epsilon}}{\dd x_{\mu}'}\, A_{\epsilon}.
\Tag{(31.4)}
\]
Hence differentiating
\begin{align*}
  \frac{\dd A_{\mu}'}{\dd x_{\nu}'}
  &= \frac{\dd^{2} x_{\epsilon}}{\dd x_{\mu}'\, \dd x_{\nu}'}\, A_{\epsilon}
  + \frac{\dd x_{\epsilon}}{\dd x_{\mu}'}\,
    \frac{\dd x_{\delta}}{\dd x_{\nu}'}\,
    \frac{\dd A_{\epsilon}}{\dd x_{\delta}} \\
  &= \left(\{\mu\nu, \rho\}'\, \frac{\dd x_{\epsilon}}{\dd x_{\rho}'}
    - \frac{\dd x_{\alpha}}{\dd x_{\mu}'}\, \frac{\dd x_{\beta}}{\dd x_{\nu}'}\, [\alpha\beta, \epsilon]\right) A_{\epsilon}
    + \frac{\dd x_{\alpha}}{\dd x_{\mu}'}\, \frac{\dd x_{\beta}}{\dd x_{\nu}'}\, \frac{\dd A_{\alpha}}{\dd x_{\beta}}
\Tag{(31.5)}
\end{align*}
by \Eq{(31.3)} and changing the dummy suffixes in the last term.

Also by~\Eq{(23.12)}
\[
A_{\epsilon}\, \frac{\dd x_{\epsilon}}{\dd x_{\rho}'} = A_{\rho}'.
\]
Hence \Eq{(31.5)}~becomes
\[
\frac{\dd A_{\mu}'}{\dd x_{\nu}'} - \{\mu\nu, \rho\}' A_{\rho}'
= \frac{\dd x_{\alpha}}{\dd x_{\mu}'}\, \frac{\dd x_{\beta}}{\dd x_{\nu}'} \left(\frac{\dd A_{\alpha}}{\dd x_{\beta}} - \{\alpha\beta, \epsilon\}\, A_{\epsilon}\right)\!,
\Tag{(31.6)}
\]
showing that
\[
\frac{\dd A_{\mu}}{\dd x_{\nu}} - \{\mu\nu, \rho\}\, A_{\rho}
\]
obeys the law of transformation of a covariant tensor. We thus reach the
result~\Eq{(29.3)} by an alternative method.

A tensor of the second or higher rank may be taken instead of~$A_{\mu}$ in~\Eq{(31.4)},
and its covariant derivative will be found by the same method.

\Section{32.}{Surface\hyp{}elements and Stokes's theorem}

{\Loosen Consider the outer product~$\Sigma^{\mu\nu}$ of two different displacements $dx_{\mu}$ and $\delta x_{\nu}$.
The tensor $\Sigma^{\mu\nu}$ will be unsymmetrical in $\mu$ and~$\nu$. We can decompose any
such tensor into the sum of a symmetrical part $\frac{1}{2}(\Sigma^{\mu\nu} + \Sigma^{\nu\mu})$ and an antisymmetrical
part $\frac{1}{2}(\Sigma^{\mu\nu} - \Sigma^{\nu\mu})$.}

Double\footnote
  {The doubling of the natural expression is avenged by the appearance of the factor~$\frac{1}{2}$ in most
  formulae containing~$dS^{\mu\nu}$.}
the antisymmetrical part of the product~$dx_{\mu}\, \delta x_{\nu}$ is called the
\emph{surface\hyp{}element} contained by the two displacements, and is denoted by~$dS^{\mu\nu}$.
\index{Surface\hyp{}element}%
We have accordingly
\begin{align*}
  dS^{\mu\nu} &= dx_{\mu}\, \delta x_{\nu} - dx_{\nu}\, \delta x_{\mu}
\Tag{(32.1)} \\
  &= \left\lvert\begin{array}{@{}cc@{}}
  dx_{\mu} & dx_{\nu} \\
  \delta x_{\mu} & \delta x_{\nu} \\
\end{array}\right\rvert.
\end{align*}
In rectangular coordinates this determinant represents the area of the projection
on the $\mu\nu$~plane of the parallelogram contained by the two displacements;
thus the components of the tensor are the projections of the
parallelogram on the six coordinate planes. In the tensor~$dS^{\mu\nu}$ these are
repeated twice, once with positive and once with negative sign (corresponding
perhaps to the two sides of the surface). The four components $dS^{11}$, $dS^{22}$, etc.\
vanish, as must happen in every antisymmetrical tensor. The appropriateness
\index{Antisymmetrical tensors}%
of the name ``surface\hyp{}element'' is evident in rectangular coordinates; the
geometrical meaning becomes more obscure in other systems.

The surface\hyp{}element is always a tensor of the second rank whatever the
number of dimensions of space; but in \emph{three} dimensions there is an alternative
representation of a surface area by a simple \emph{vector} at right angles to the
surface and of length proportional to the area; indeed it is customary in three
dimensions to represent any antisymmetrical tensor by an adjoint vector.
Happily in four dimensions it is not possible to introduce this source of
confusion.

The invariant
\[
\tfrac{1}{2} A_{\mu\nu}\, dS^{\mu\nu}
\]
is called the \emph{flux} of the tensor~$A_{\mu\nu}$ through the surface\hyp{}element. The flux
\index{Flux}%
involves only the antisymmetrical part of~$A_{\mu\nu}$, since the inner product of a
symmetrical and an antisymmetrical tensor evidently vanishes.

Some of the chief antisymmetrical tensors arise from the operation of
\emph{curling}. If $K_{\mu\nu}$~is the covariant derivative of~$K_{\mu}$, we find from~\Eq{(29.3)} that
\[
K_{\mu\nu} - K_{\nu\mu} = \frac{\dd K_{\mu}}{\dd x_{\nu}} - \frac{\dd K_{\nu}}{\dd x_{\mu}}
\Tag{(32.2)}
\]
since the $3$-index symbols cancel out. Since the left-hand side is a tensor, the
right-hand side is also a tensor. The right-hand side will be recognised as the
``curl'' of elementary vector theory, except that we have apparently reversed
\index{Curl}%
the sign. Strictly speaking, however, we should note that the curl in the
elementary three\hyp{}dimensional theory is a vector, whereas our curl is a tensor;
and comparison of the sign attributed is impossible.

The result that the covariant curl is the same as the ordinary curl does
not apply to contravariant vectors or to tensors of higher rank:
\[
{K^{\mu}}_{\nu} - {K^{\nu}}_{\mu} \neq \frac{\dd K^{\mu}}{\dd x_{\nu}} - \frac{\dd K^{\nu}}{\dd x_{\mu}}.
\]

In tensor notation the famous theorem of Stokes becomes
\index{Stokes's theorem}%
\[
\int K_{\mu}\, dx_{\mu} = -\frac{1}{2} \iint \left(\frac{\dd K_{\mu}}{\dd x_{\nu}} - \frac{\dd K_{\nu}}{\dd x_{\mu}}\right) dS^{\mu\nu},
\Tag{(32.3)}
\]
the double integral being taken over any surface bounded by the path of the
single integral. The factor~$\frac{1}{2}$ is needed because each surface\hyp{}element occurs
twice, e.g.\ as $dS^{12}$ and $-dS^{21}$. The theorem can be proved as follows---

Since both sides of the equation are invariants it is sufficient to prove the
equation for any one system of coordinates. Choose coordinates so that the
surface is on one of the fundamental partitions $x_{2} = \text{const.}$, $x_{4} = \text{const.}$, and so
that the contour consists of four parts given successively by $x_{1} = \alpha$, $x_{2} = \beta$,
$x_{1} = \gamma$, $x_{2} = \delta$; the rest of the mesh\hyp{}system may be filled up arbitrarily. For
an elementary mesh the containing vectors are $(dx_{1}, 0, 0, 0)$ and $(0, dx_{2} , 0, 0)$,
so that by~\Eq{(32.1)}
\[
dS^{12} = dx_{1}\, dx_{2} = -dS^{21}.
\]
Hence the right-hand side of~\Eq{(32.3)} becomes
\begin{multline*}
  -\int_{\alpha}^{\gamma} \int_{\beta}^{\delta} \left(\frac{\dd K_{1}}{\dd x_{2}} - \frac{\dd K_{2}}{\dd x_{1}}\right) dx_{1}\, dx_{2} \\
  = -\int_{\alpha}^{\gamma} \bigl\{[K_{1}]^{\delta} - [K_{1}]^{\beta}\bigr\}\, dx_{1}
   + \int_{\beta}^{\delta} \bigl\{[K_{2}]^{\gamma} - [K_{2}]^{\alpha}\bigr\}\, dx_{2},
\end{multline*}
which consists of four terms giving $\int K_{\mu}\, dx_{\mu}$ for the four parts of the contour.

This proof affords a good illustration of the methods of the tensor calculus.
The relation to be established is between two quantities which (by examination
of their covariant dimensions) are seen to be invariants, viz.\ $K_{\mu} (dx)^{\mu}$ and
\index{Covariant derivative of vector!significance of}%
\index{Derivative!significance of}%
$(K_{\mu\nu} - K_{\nu\mu})\, dS^{\mu\nu}$, the latter having been simplified by~\Eq{(32.2)}. Accordingly it
is a relation which does not depend on any particular choice of coordinates,
although in~\Eq{(32.3)} it is expressed as it would appear when referred to a
coordinate\hyp{}system. In proving the relation of the two invariants once for all,
we naturally choose for the occasion coordinates which simplify the analysis;
and the work is greatly shortened by drawing our curved meshes so that four
partition\hyp{}lines make up the contour.

\Section{33.}{Significance of covariant differentiation}

Suppose that we wish to discuss from the physical point of view how a
field of force varies from point to point. If polar coordinates are being used,
a change of the $r$-component does not necessarily indicate a want of uniformity
in the field of force; it is at least partly attributable to the inclination between
the $r$-directions at different points. Similarly when rotating axes are used,
the rate of change of momentum~$h$ is given not by $dh_{1}/dt$, etc., but by
\[
dh_{1}/dt - \omega_{3}/h_{2} + \omega_{2}/h_{3}, \text{ etc.}
\Tag{(33.1)}
\]
The momentum may be constant even when the time\hyp{}derivatives of its components
are not zero.

We must recognise then that the change of a physical entity is usually
regarded as something distinct from the change of the mathematical components
into which we resolve it. In the elementary theory a definition of the
former change is obtained by identifying it with the change of the components
in unaccelerated rectangular coordinates; but this is of no avail in the general
case because space-time may be of a kind for which no such coordinates exist.
Can we still preserve this notion of a \emph{physical} rate of change in the general
case?

Our attention is directed to the rate of change of a physical entity because
of its importance in the laws of physics, e.g.\ force is the time-rate of change
of momentum, or the space-rate of change of potential; therefore the rate of
change should be expressed by a tensor of some kind in order that it may enter
into the general physical laws. Further in order to agree with the customary
definition in elementary cases, it must reduce to the rate of change of the
rectangular components when the coordinates are Galilean. Both conditions
are fulfilled if we define the physical rate of change of the tensor by its covariant
derivative.

The covariant derivative~$A_{\mu\nu}$ consists of the term $\dd A_{\mu}/dx_{\nu}$, giving the
apparent gradient, from which is subtracted the ``spurious change'' $\{\mu\nu, \alpha\}\, A_{\alpha}$
attributable to the curvilinearity of the coordinate\hyp{}system. When Cartesian
coordinates (rectangular or oblique) are used, the $3$-index symbols vanish and
there is, as we should expect, no spurious change. For the present we shall
call~$A_{\mu\nu}$ the rate of \emph{absolute change} of the vector~$A_{\mu}$.
\index{Absolute change}%

Consider an elementary mesh in the plane of~$x_{\nu} x_{\sigma}$, the corners being at
\[
A(x_{\nu}, x_{\sigma}),\quad B(x_{\nu} + dx_{\nu}, x_{\sigma}),\quad
C(x_{\nu} + dx_{\nu}, x_{\sigma} + dx_{\sigma}),\quad D(x_{\nu}, x_{\sigma} + dx_{\sigma}).
\]
Let us calculate the whole absolute change of the vector\hyp{}field~$A_{\mu}$ as we pass
round the circuit $ABCDA$.

(1) From $A$ to $B$, the absolute change is $\PadTo[r]{-A_{\mu\sigma}\, dx_{\sigma}}{A_{\mu\nu}\, dx_{\nu}}$, calculated for~$x_{\sigma}$\footnotemark.\footnotetext
  {We suspend the summation convention since $dx_{\nu}$ and $dx_{\sigma}$ are edges of a particular mesh.
  The convention would give correct results; but it goes too fast, and we cannot keep pace with it.}%

(2) From $B$ to $C$, the absolute change is $\PadTo[r]{-A_{\mu\sigma}\, dx_{\sigma}}{A_{\mu\sigma}\, dx_{\sigma}}$, calculated for~$x_{\nu} + dx_{\nu}$.

(3) From $C$ to $D$, the absolute change is $\PadTo[r]{-A_{\mu\sigma}\, dx_{\sigma}}{-A_{\mu\nu}\, dx_{\nu}}$, calculated for~$x_{\sigma} + dx_{\sigma}$.

(4) From $D$ to $A$, the absolute change is $-A_{\mu\sigma}\, dx_{\sigma}$, calculated for~$x_{\nu}$. \\
Combining (2) and (4) the net result is the difference of the changes $A_{\mu\sigma}\, dx_{\sigma}$,
at $x_{\nu} + dx_{\nu}$ and at $x_{\nu}$~respectively. We might be tempted to set this difference
down as
\[
\frac{\dd}{\dd x_{\nu}} (A_{\mu\nu}\, dx_{\sigma})\, dx_{\nu}.
\]
But as already explained that would give only the difference of the mathematical
components and not the ``absolute difference.'' We must take the
covariant derivative instead, obtaining (since $dx_{\sigma}$~is the same for (2) and~(4))
\[
A_{\mu\sigma\nu}\, dx_{\sigma}\, dx_{\nu}.
\]
Similarly (3) and (1) give
\[
-A_{\mu\nu\sigma}\, dx_{\nu}\, dx_{\sigma},
\]
so that the total absolute change round the circuit is
\[
(A_{\mu\sigma\nu} - A_{\mu\nu\sigma})\, dx_{\nu}\, dx_{\sigma}.
\Tag{(33.2)}
\]

We should naturally expect that on returning to our starting point the
absolute change would vanish. How could there have been any absolute change
on balance, seeing that the vector is now the same~$A_{\mu}$ that we started with?
Nevertheless in general $A_{\mu\nu\sigma} \neq A_{\mu\sigma\nu}$, that is to say the order of covariant
differentiation is not permutable, and \Eq{(33.2)} does not vanish.

That this result is not unreasonable may be seen by considering a two\hyp{}dimensional
space, the surface of the ocean. If a ship's head is kept straight
on the line of its wake, the course is a great circle. Now suppose that the ship
sails round a circuit so that the final position and course are the same as at
the start. If account is kept of all the successive changes of course, and the
angles are added up, these will not give a change zero (or~$2\pi$) on balance. For
a triangular course the difference is the well-known ``spherical excess.'' Similarly
the changes of velocity do not cancel out on balance. Here we have an
illustration that the absolute changes of a vector do not cancel out on bringing
it back to its initial position.

If the present result sounds self\hyp{}contradictory, the fault lies with the name
``absolute change'' which we have tentatively applied to the thing under discussion.
The name is illuminating in some respects, because it shows the
continuity of covariant differentiation with the conceptions of elementary
physics. For instance, no one would hesitate to call~\Eq{(33.1)} the absolute rate
of change of momentum in contrast to the apparent rate of change~$dh_{1}/dt$. But
having shown the continuity, we find it better to avoid the term in the more
general case of non\hyp{}Euclidean space.

Following Levi\hyp{}Civita and Weyl we use the term \emph{parallel displacement} for
\index{Displacement!parallel}%
\index{Parallel displacement}%
what we have hitherto called displacement without ``absolute change.'' The
condition for parallel displacement is that the covariant derivative vanishes.

We have hitherto considered the absolute change necessary in order that
the vector may return to its original value, and so be a single\hyp{}valued function
of position. If we do not permit any change en route, i.e.\ if we move the vector
by parallel displacement, the same quantity will appear (with reversed sign)
as a discrepancy~$\delta A_{\mu}$ between the final and initial vectors. Since these are at
the same point the difference of the initial and final vectors can be measured
immediately. We have then by~\Eq{(33.2)}
\[
\delta A_{\mu} = (A_{\mu\nu\sigma} - A_{\mu\sigma\nu})\, dx_{\nu}\, dx_{\sigma},
\]
which may also be written
\[
\delta A_{\mu} = \tfrac{1}{2} \iint (A_{\mu\nu\sigma} - A_{\mu\sigma\nu})\, dS^{\nu\sigma},
\Tag{(33.3)}
\]
where the summation convention is now restored. We have only proved this
for an infinitesimal circuit occupying a coordinate\hyp{}mesh, for which $dS^{\mu\nu}$~has
only two non\hyp{}vanishing components $dx_{\nu}\, dx_{\sigma}$ and $-dx_{\nu}\, dx_{\sigma}$. But the equation
is seen to be a tensor\hyp{}equation, and therefore holds independently of the
coordinate\hyp{}system; thus it applies to circuits of any shape, since we can always
choose coordinates for which the circuit becomes a coordinate\hyp{}mesh. But \Eq{(33.3)}~is
still restricted to infinitesimal circuits and there is no way of extending it
to finite circuits---unlike Stokes's theorem. The reason for this restriction is as
follows---

An \emph{isolated vector~$A_{\mu}$} may be taken at the starting point and carried by
parallel displacement round the circuit, leading to a determinate value of~$\delta A_{\mu}$.
\index{Geodesic, equations of!produced by parallel displacement}%
\index{Velocity\hyp{}vector}%
In~\Eq{(33.3)} this is expressed in terms of derivatives of a \emph{vector\hyp{}field~$A_{\mu}$} extending
throughout the region of integration. For a large circuit this would involve
values of~$A_{\mu}$ remote from the initial vector, which are obviously irrelevant to
the calculation of~$\delta A_{\mu}$. It is rather remarkable that there should exist such
a formula even for an infinitesimal circuit; the fact is that although $A_{\mu\nu\sigma} - A_{\mu\sigma\nu}$
at a point formally refers to a vector\hyp{}field, its value turns out to depend solely
on the isolated vector~$A_{\mu}$ (see equation~\Eq{(34.3)}).

The contravariant vector~$dx_{\mu}/ds$ gives a direction in the four\hyp{}dimensional
world which is interpreted as a velocity from the ordinary point of view which
separates space and time. We shall usually call it a ``velocity''; its connection
with the usual three\hyp{}dimensional vector $(u, v, w)$ is given by
\[
\frac{dx_{\mu}}{ds} = \beta(u, v, w, 1),
\]
where $\beta$~is the FitzGerald factor~$dt/ds$. The length~\Eq{(26.5)} of a velocity is
always unity.

If we transfer $dx_{\mu}/ds$ continually along itself by parallel displacement we
obtain a geodesic. For by~\Eq{(29.4)} the condition for parallel displacement is
\[
\frac{\dd}{\dd x_{\nu}} \left(\frac{\dd x_{\mu}}{ds}\right) + \{\alpha\nu, \mu\}\, \frac{\dd x_{\alpha}}{ds} = 0.
\]
Hence multiplying by~$dx_{\nu}/ds$
\[
\frac{\dd^{2} x_{\mu}}{ds^{2}} + \{\alpha\nu, \mu\}\, \frac{\dd x_{\alpha}}{ds}\, \frac{dx_{\nu}}{ds} = 0,
\Tag{(33.4)}
\]
which is the condition for a geodesic~\Eq{(28.5)}. Thus in the language used at
the beginning of this section, a geodesic is a line in four dimensions whose
direction undergoes no absolute change.

\Section{34.}{The Riemann\hyp{}Christoffel tensor}

The second covariant derivative of~$A_{\mu}$ is found by inserting in~\Eq{(30.3)} the
value of~$A_{\mu\nu}$ from~\Eq{(29.3)}. This gives
%[** TN: Rebreaking]
\begin{align*}
  A_{\mu\nu\sigma}
  &= \frac{\dd}{\dd x_{\sigma}} \left(\frac{\dd A_{\mu}}{\dd x_{\nu}} - \{\mu\nu, \alpha\}\, A_{\alpha}\right) \\
  &\qquad\qquad\begin{aligned}
  &- \{\mu\sigma, \alpha\} \left(\frac{\dd A_{\alpha}}{\dd x_{\nu}} - \{\alpha\nu, \epsilon\}\, A_{\epsilon}\right) \\
  &- \{\nu\sigma, \alpha\} \left(\frac{\dd A_{\mu}}{\dd x_{\alpha}} - \{\mu\alpha, \epsilon\}\, A_{\epsilon}\right)
  \end{aligned}\displaybreak[0] \\
  &= \frac{\dd^{2} A_{\mu}}{\dd x_{\sigma}\, \dd x_{\nu}}
  \begin{aligned}[t]
    &- \{\mu\nu, \alpha\}\, \frac{\dd A_{\alpha}}{\dd x_{\sigma}}
     - \{\mu\sigma, \alpha\}\, \frac{\dd A_{\alpha}}{\dd x_{\nu}} \\
    &- \{\nu\sigma, \alpha\}\, \frac{\dd A_{\mu}}{\dd x_{\alpha}}
     + \{\nu\sigma, \alpha\}\, \{\mu\alpha, \epsilon\}\, A_{\epsilon}
  \end{aligned} \\
  &+ \{\mu\sigma, \alpha\}\, \{\alpha\nu, \epsilon\}\, A_{\epsilon}
   - A_{\alpha}\, \frac{\dd}{\dd x_{\sigma}} \{\mu\nu, \alpha\}.
  \Tag{(34.1)}
\end{align*}
The first five terms are unaltered when $\nu$ and~$\sigma$ are interchanged. The last
two terms may be written, by changing the dummy suffix~$\alpha$ to~$\epsilon$ in the last
term,
\[
A_{\epsilon}\left(\{\mu\sigma, \alpha\}\, \{\alpha\nu, \epsilon\}
- \frac{\dd}{\dd x_{\sigma}} \{\mu\nu, \epsilon\}\right).
\]
Hence
%[** TN: Re-breaking]
\begin{align*}
  A_{\mu\nu\sigma} - A_{\mu\sigma\nu}
  &= A_{\epsilon}\biggl(\{\mu\sigma, \alpha\} \{\alpha\nu, \epsilon\} - \frac{\dd}{\dd x_{\sigma}} \{\mu\nu, \epsilon\} \\
  &\qquad - \{\mu\nu, \alpha\} \{\alpha\sigma, \epsilon\} - \frac{\dd}{\dd x_{\nu}} \{\mu\sigma, \epsilon\}\biggr).
  \Tag{(34.2)}
\end{align*}
The rigorous quotient theorem shows that the co\hyp{}factor of~$A_{\epsilon}$ must be a tensor.
Accordingly we write
\[
A_{\mu\nu\sigma} - A_{\mu\sigma\nu} = A_{\epsilon} B_{\mu\nu\sigma}^{\epsilon},
\Tag{(34.3)}
\]
where
\[
  B_{\mu\nu\sigma}^{\epsilon}
  = \{\mu\sigma, \alpha\} \{\alpha\nu, \epsilon\} - \{\mu\nu, \alpha\} \{\alpha\sigma, \epsilon\}
  - \frac{\dd}{\dd x_{\nu}} \{\mu\sigma, \epsilon\}
  - \frac{\dd}{\dd x_{\sigma}} \{\mu\nu, \epsilon\}.
  \Tag{(34.4)}
\]
This is called the Riemann\hyp{}Christoffel tensor. It is only when this tensor
\index{B@$B_{\mu\nu\sigma}^{\epsilon}$ (Riemann\hyp{}Christoffel tensor)}%
\index{Riemann\hyp{}Christoffel tensor}%
vanishes that the order of covariant differentiation is permutable.

The suffix~$\epsilon$ may be lowered. Thus
\begin{align*}
  B_{\mu\nu\sigma\rho}
  &= g_{\rho\epsilon} B_{\mu\nu\sigma}^{\epsilon} \\
  &= \{\mu\sigma, \alpha\} [\alpha\nu, \rho] - \{\mu\nu, \alpha\} [\alpha\sigma, \rho]
  - \frac{\dd}{\dd x_{\nu}} [\mu\sigma, \rho]
  - \frac{\dd}{\dd x_{\sigma}} [\mu\nu, \rho] \\
  &\qquad- \{\mu\sigma, \alpha\}\, \frac{\dd g_{\rho\alpha}}{\dd x_{\nu}}
  - \{\mu\nu, \alpha\}\, \frac{\dd g_{\rho\alpha}}{\dd x_{\sigma}},
  \Tag{(34.45)}
  \intertext{where $\epsilon$~has been replaced by~$\alpha$ in the last two terms,}
  &= -\{\mu\sigma, \alpha\} [\rho\nu, \alpha] + \{\mu\nu, \alpha\} [\rho\sigma, \alpha] \\
  &\qquad+\tfrac{1}{2} \left(
  \frac{\dd^{2} g_{\rho\sigma}}{\dd x_{\mu}\, \dd x_{\nu}}
  + \frac{\dd^{2} g_{\mu\nu}}{\dd x_{\rho}\, \dd x_{\sigma}}
  - \frac{\dd^{2} g_{\mu\sigma}}{\dd x_{\rho}\, \dd x_{\nu}}
  - \frac{\dd^{2} g_{\rho\nu}}{\dd^{2} x_{\mu}\, \dd x_{\sigma}}\right),
  \Tag{(34.5)}
\end{align*}
by \Eq{(27.5)} and~\Eq{(27.1)}.

It will be seen from~\Eq{(34.5)} that $B_{\mu\nu\sigma\rho}$, besides being antisymmetrical in~$\nu$
and~$\sigma$, is also antisymmetrical in $\mu$ and~$\rho$. Also it is symmetrical for the double
interchange $\mu$ and~$\nu$, $\rho$ and~$\sigma$. It has the further cyclic property
\[
B_{\mu\nu\sigma\rho} + B_{\mu\sigma\rho\nu} + B_{\mu\rho\nu\sigma} = 0,
\Tag{(34.6)}
\]
as is easily verified from~\Eq{(34.5)}.

The general tensor of the fourth rank has $256$~different components. Here
the double antisymmetry reduces the number (apart from differences of sign)
to $6 \times 6$. $30$~of these are paired because $\mu$,~$\rho$ can be interchanged with $\nu$,~$\sigma$;
but the remaining $6$~components, in which $\mu$,~$\rho$ is the same pair of numbers as
$\nu$,~$\sigma$, are without partners. This leaves $21$~different components, between
which \Eq{(34.6)}~gives only one further relation. We conclude that the Riemann\hyp{}Christoffel
tensor has $20$~\emph{independent} components\footnotemark.\footnotetext
  {Writing the suffixes in the order $\mu\rho\sigma\nu$ the following scheme gives $21$~different components:
  \[
  \begin{array}{*{7}{c}}
    1212 & 1223 & 1313 & 1324 & 1423 & 2323 & 2424 \\
    1213 & 1224 & 1314 & 1334 & 1424 & 2324 & 2434 \\
    1214 & 1234 & 1323 & 1414 & 1434 & 2334 & 3434 \\
  \end{array}
  \]
  with the relation $1234 - 1324 + 1423 = 0$.

  If we omit those containing the suffix~$4$, we are left with $6$~components in three\hyp{}dimensional
  space. In two dimensions there is only the one component~$1212$.}

The Riemann\hyp{}Christoffel tensor is derived solely from the~$g_{\mu\nu}$ and therefore
belongs to the class of fundamental tensors. Usually we can form from
any tensor a series of tensors of continually increasing rank by covariant
differentiation. But this process is frustrated in the case of the fundamental
tensors because $g_{\mu\nu\sigma}$~vanishes identically. We have got round the gap and
reached a fundamental tensor of the fourth rank. The series can now be continued
indefinitely by covariant differentiation.

When the Riemann\hyp{}Christoffel tensor vanishes, the differential equations
\index{Integrability of parallel displacement}%
\index{Riemann\hyp{}Christoffel tensor!vanishing of}%
\[
A_{\mu\nu} = \frac{\dd A_{\mu}}{\dd x_{\nu}} - \{\mu\nu, \alpha\}\, A_{\alpha} = 0
\Tag{(34.7)}
\]
are integrable. For the integration will be possible if \Eq{(34.7)} makes~$dA_{\mu}$ or
\[
\frac{\dd A_{\mu}}{\dd x_{\nu}}\, dx_{\nu}
\]
a complete differential, i.e.\ if
\[
\{\mu\nu, \alpha\}\, A_{\alpha}\, dx_{\nu}
\]
is a complete differential. By the usual theory the condition for this is
\[
\frac{\dd}{\dd x_{\sigma}} (\{\mu\nu, \alpha\}\, A_{\alpha})
- \frac{\dd}{\dd x_{\nu}} (\{\mu\sigma, \alpha\}\, A_{\alpha}) = 0,
\]
or
\[
A_{\alpha} \left(\frac{\dd}{\dd x_{\sigma}} \{\mu\nu, \alpha\}
             - \frac{\dd}{\dd x_{\nu}} \{\mu\sigma, \alpha\}\right)
+ \{\mu\nu, \alpha\}\, \frac{\dd A_{\alpha}}{\dd x_{\sigma}}
- \{\mu\sigma, \alpha\}\, \frac{\dd A_{\alpha}}{\dd x_{\nu}} = 0.
\]
Substituting for $\dd A_{\alpha}/\dd x_{\sigma}$, $\dd A_{\alpha}/dx_{\nu}$ from~\Eq{(34.7)}
%[** TN: Not broken in the original]
\begin{multline*}
A_{\alpha} \left(\frac{\dd}{\dd x_{\sigma}} \{\mu\nu, \alpha\}
             - \frac{\dd}{\dd x_{\nu}} \{\mu\sigma, \alpha\}\right) \\
+ (\{\mu\nu, \alpha\}\, \{\alpha\sigma, \epsilon\}
- \{\mu\sigma, \alpha\}\, \{\alpha\nu, \epsilon\}) A_{\epsilon} = 0.
\end{multline*}
Changing the suffix~$\alpha$ to~$\epsilon$ in the first term, the condition becomes
\[
A_{\epsilon} B_{\mu\sigma\nu}^{\epsilon} = 0.
\]
Accordingly when $B_{\mu\sigma\nu}^{\epsilon}$~vanishes, the differential~$dA_{\mu}$ determined by~\Eq{(34.7)}
will be a complete differential, and
\[
\int dA_{\mu}
\]
between any two points will be independent of the path of integration. We
can then carry the vector~$A_{\mu}$. by parallel displacement to any point obtaining
a unique result independent of the route of transfer. If a vector is displaced
in this way all over the field, we obtain a \emph{uniform vector\hyp{}field}.
\index{Uniform!vector\hyp{}field}%

This construction of a uniform vector\hyp{}field is only possible when the
Riemann\hyp{}Christoffel tensor vanishes throughout. In other cases the equations
have no complete integral, and can only be integrated along a particular route.
E.g., we can prescribe a \emph{uniform direction} at all points of a plane, but their is
nothing analogous to a uniform direction over the surface of a sphere.

Formulae analogous to~\Eq{(34.3)} can be obtained for the second derivatives
of a tensor~$A_{...\mu..}$ instead of for a vector~$A_{\mu}$. The result is easily found to be
\[
A_{...\mu..\nu\sigma} - A_{...\mu..\sigma\nu} = \sum B_{\mu\sigma\nu}^{\epsilon} A_{...\epsilon..},
\Tag{(34.8)}
\]
the summation being taken over all the suffixes~$\mu$ of the original tensor.

The corresponding formulae for contravariant tensors follow at once, since
the $g^{\mu\nu}$ behave as constants in covariant differentiation, and suffixes may be
raised on both sides of~\Eq{(34.8)}.

\Section{35.}{Miscellaneous formulae}

The following are needed for subsequent use--- \\
Since
\begin{gather*}
g_{\mu\nu} g^{\mu\alpha} = \text{$0$ or $1$,} \\
g^{\mu\alpha} \, dg_{\mu\nu} + g_{\mu\nu}\, dg^{\mu\alpha} = 0.
\end{gather*}
Hence
\begin{align*}
  g^{\mu\alpha}\, g^{\nu\beta} \, dg_{\mu\nu}
  &= -g_{\mu\nu}\, g^{\nu\beta}\, dg^{\mu\alpha}
   = -g_{\mu}^{\beta}\, dg^{\mu\alpha} \\
  &= -dg^{\alpha\beta}.
  \Tag{(35.11)}
\end{align*}
Similarly
\[
dg_{\alpha\beta} = -g_{\mu\alpha} g_{\nu\beta}\, dg^{\mu\nu}.
\Tag{(35.12)}
\]
Multiplying by~$A^{\alpha\beta}$, we have by the rule for lowering suffixes
\begin{align*}
  A^{\alpha\beta}\, dg_{\alpha\beta}
  &= -(g_{\mu\alpha} g_{\nu\beta}A^{\alpha\beta})\, dg^{\mu\nu} \\
  &= -A_{\mu\nu}\, dg^{\mu\nu}
   = -A_{\alpha\beta}\, dg^{\alpha\beta}.
  \Tag{(35.2)}
\end{align*}
For any tensor~$B_{\alpha\beta}$ other than the fundamental tensor the corresponding
formula would be
\[
A^{\alpha\beta}\, dB_{\alpha\beta} = A_{\alpha\beta}\, dB^{\alpha\beta}
\]
by~\Eq{(26.3)}. The exception for $B_{\alpha\beta} = g_{\alpha\beta}$ arises because a change~$dg_{\alpha\beta}$ has an
additional indirect effect through altering the operation of raising and lowering
suffixes.

Again $dg$~is formed by taking the differential of each~$g_{\mu\nu}$ and multiplying
by its co\hyp{}factor~$g \cdot g^{\mu\nu}$ in the determinant. Thus
\[
\frac{dg}{g} = g^{\mu\nu}\, dg_{\mu\nu} = -g_{\mu\nu}\, dg^{\mu\nu}.
\Tag{(35.3)}
\]

The contracted $3$-index symbol
\index{Three\hyp{}index symbol!contracted}%
\begin{align*}
  \{\mu\sigma, \sigma\}
  &= \tfrac{1}{2} g^{\sigma\lambda} \left\{
    \frac{\dd g_{\mu\lambda}}{\dd x_{\sigma}}
  + \frac{\dd g_{\sigma\lambda}}{\dd x_{\mu}}
  - \frac{\dd g_{\mu\sigma}}{\dd x_{\lambda}}\right\} \\
  &= \tfrac{1}{2} g^{\sigma\lambda}\, \frac{\dd g_{\sigma\lambda}}{\dd x_{\mu}}.
\end{align*}
The other two terms cancel by interchange of the dummy suffixes $\sigma$ and~$\lambda$.
Hence by~\Eq{(35.3)}
\begin{align*}
  \{\mu\sigma, \sigma\}
  &= \frac{1}{2g}\, \frac{\dd g}{\dd x_{\mu}} \\
  &= \frac{\dd}{\dd x_{\mu}} \log \sqrt{-g}.
  \Tag{(35.4)}
\end{align*}
We use $\sqrt{-g}$ because $g$~is always negative for real coordinates.

A possible pitfall in differentiating a summed expression should be noticed.
The result of differentiating $a_{\mu\nu} x_{\mu} x_{\nu}$ with respect to~$x_{\nu}$ is not~$a_{\mu\nu} x_{\mu}$ but
$(a_{\mu\nu} + a_{\nu\mu}) x_{\mu}$. The method of performing such differentiations may be illustrated
by the following example. Let
\[
h_{\nu\tau} = a_{\mu\nu} a_{\sigma\tau} x_{\mu} x_{\sigma},
\]
where $a_{\mu\nu}$~represents constant coefficients. Then
\begin{align*}
  \frac{\dd h_{\nu\tau}}{\dd x_{\alpha}}
  &= a_{\mu\nu} a_{\sigma\tau} \left(\frac{\dd x_{\mu}}{\dd x_{\alpha}}\, x_{\sigma} + \frac{\dd x_{\sigma}}{\dd x_{\alpha}}\, x_{\mu}\right) \\
  &= a_{\mu\nu} a_{\sigma\tau} (g_{\alpha}^{\mu} x_{\sigma} + g_{\alpha}^{\sigma} x_{\mu})
  \quad\text{by \Eq{(22.3)}.}
\end{align*}
\index{Differentiation of summed expression}%
Repeating the process,
\begin{align*}
  \frac{\dd^{2} h_{\nu\tau}}{\dd x_{\alpha}\, \dd x_{\beta}}
  &= a_{\mu\nu} a_{\sigma\tau} (g_{\alpha}^{\mu} g_{\beta}^{\sigma} + g_{\alpha}^{\sigma} g_{\beta}^{\mu}) \\
  &= a_{\alpha\nu} a_{\beta\tau} + a_{\beta\nu} a_{\alpha\tau}.
\end{align*}
Hence changing dummy suffixes
\[
\frac{\dd^{2}}{\dd x_{\mu}\, \dd x_{\sigma}} (a_{\mu\nu} a_{\sigma\tau} x_{\mu} x_{\sigma})
= a_{\mu\nu} a_{\sigma\tau} + a_{\sigma\nu} a_{\mu\tau}.
\Tag{(35.5)}
\]
Similarly if $a_{\mu\nu\sigma}$~is symmetrical in its suffixes
\[
\frac{\dd^{3}}{\dd x_{\mu}\, \dd x_{\nu}\, \dd x_{\sigma}} (a_{\mu\nu\sigma} x_{\mu} x_{\nu} x_{\sigma})
= 6a_{\mu\nu\sigma}.
\Tag{(35.6)}
\]
The pitfall arises from repeating a suffix three times in one term. In these
formulae the summation applies to the repetition within the bracket, and not
to the differentiation.

\Subsection{Summary.}

Tensors are quantities obeying certain transformation laws. Their importance
lies in the fact that if a tensor equation is found to hold for one
system of coordinates, it continues to hold when any transformation of
coordinates is made. New tensors are recognised either by investigating
their transformation laws directly or by the property that the sum, difference,
product or quotient of tensors is a tensor. This is a generalisation of the
method of dimensions in physics.

The principal operations of the tensor calculus are addition, multiplication
(outer and inner), summation (\SecRef{22}), contraction (\SecRef{24}), substitution (\SecRef{25}),
raising and lowering suffixes (\SecRef{26}), covariant differentiation (\SecRefs{29}, \SecNum{30}). There
is no operation of division; but an inconvenient factor $g_{\mu\nu}$ or~$g^{\mu\nu}$ can be
removed by multiplying through by $g^{\mu\sigma}$ or~$g_{\mu\sigma}$ so as to form the substitution\hyp{}operator.
The operation of summation is practically outside our control and
always presents itself as a \Foreign{fait accompli}. The most characteristic process of
manipulation in this calculus is the free alteration of dummy suffixes (those
appearing twice in a term); it is probably this process which presents most
difficulty to the beginner.

Of special interest are the fundamental tensors or world\hyp{}tensors, of which we
have discovered two, viz.\ $g_{\mu\nu}$ and~$B_{\mu\nu\sigma\rho}$. The latter has been expressed in terms
of the former and its first and second derivatives. It is through these that the
gap between pure geometry and physics is bridged; in particular $g_{\mu\nu}$~relates
the observed quantity~$ds$ to the mathematical coordinate specification~$dx_{\mu}$.

Since in our work we generally deal with tensors, the reader may be led
to overlook the rarity of this property. The juggling tricks which we seem
to perform in our manipulations are only possible because the material used
is of quite exceptional character.

The further development of the tensor calculus will be resumed in \SecRef{48};
but a stage has now been reached at which we may begin to apply it to the
theory of gravitation.

\Chapter{III}{The Law of Gravitation}

\Section[The condition for flat space-time]{36.}{The condition for flat space-time. Natural coordinates}
\index{Flat space-time!condition for}%

\lettrine{\textcolor{lettrinecolour}{A}}{region} of the world is called \emph{flat} or \emph{homaloidal} if it is possible to
construct in it a Galilean frame of reference.

It was shown in \SecRef{4} that when the $g_{\mu\nu}$ are constants, $ds^{2}$~can be reduced
to the sum of four squares, and Galilean coordinates can be constructed. Thus
an equivalent definition of flat space-time is that it is such that coordinates
can be found for which the $g_{\mu\nu}$ are constants.

When the $g_{\mu\nu}$ are constants the $3$-index symbols all vanish; but since the
$3$-index symbols do not form a tensor, they will not in general continue to
vanish when other coordinates are substituted in the same flat region. Again,
when the $g_{\mu\nu}$ are constants, the Riemann\hyp{}Christoffel tensor, being composed
\index{Riemann\hyp{}Christoffel tensor!vanishing of}%
of products and derivatives of the $3$-index symbols, will vanish; and since it
is a tensor, it will continue to vanish when any other coordinate\hyp{}system is
substituted in the same region.

\emph{Hence the vanishing of the Riemann\hyp{}Christoffel tensor is a necessary condition
for flat space-time.}

This condition is also \emph{sufficient}---if the Riemann\hyp{}Christoffel tensor vanishes
space-time must be flat. This can be proved as follows---

We have found (\SecRef{34}) that if
\[
B_{\mu\nu\sigma}^{\epsilon} = 0,
\Tag{(36.1)}
\]
it is possible to construct a uniform vector\hyp{}field by parallel displacement of
a vector all over the region. Let $A_{(\alpha)}^{\mu}$~be four uniform vector\hyp{}fields given by
$\alpha = 1$, $2$, $3$,~$4$, so that
\[
(A_{(\alpha)}^{\mu})_{\sigma} = 0
\]
or by~\Eq{(29.4)}
\[
\frac{\dd A_{(\alpha)}^{\mu}}{\dd x_{\sigma}}
= -\{\epsilon\sigma, \mu\}\, A_{(\alpha)}^{\epsilon}.
\Tag{(36.2)}
\]
Note that $\alpha$~is not a tensor\hyp{}suffix, but merely distinguishes the four independent
vectors.

We shall use these four uniform vector\hyp{}fields to define a new coordinate\hyp{}system
distinguished by accents. Our unit mesh will be the hyperparallelopiped
contained by the four vectors at any point, and the complete mesh\hyp{}system
will be formed by successive parallel displacements of this unit mesh
until the whole region is filled. One edge of the unit mesh, given in the old
coordinates by
\[
dx_{\mu} = A_{(1)}^{\mu},
\]
has to become in the new coordinates
\[
dx_{\alpha}' = (1, 0, 0, 0).
\]
Similarly the second edge, $dx_{\mu} = A_{(2)}^{\mu}$, must become $dx_{\alpha}' = (0, 1, 0, 0)$; etc.
This requires the law of transformation
\[
dx_{\mu} = A_{(\alpha)}^{\mu}\, dx_{\alpha}'.
\Tag{(36.3)}
\]
Of course, the construction of the accented coordinate\hyp{}system depends on the
possibility of constructing uniform vector\hyp{}fields, and this depends on \Eq{(36.1)}
\index{Uniform!mesh\hyp{}system}%
being satisfied.

Since $d^{2}$~is an invariant
\begin{align*}
  g'_{\alpha\beta}\, dx_{\alpha}'\, dx_{\beta}'
  &= g_{\mu\nu}\, dx_{\mu}\, dx_{\nu} \\
  &= g_{\mu\nu} A_{(\alpha)}^{\mu} A_{(\beta)}^{\nu}\, dx_{\alpha}'\, dx_{\beta}'
  \text{ by \Eq{(36.3)}.}
  \intertext{Hence}
  g'_{\alpha\beta} &= g_{\mu\nu} A_{(\alpha)}^{\mu} A_{(\beta)}^{\nu}.
\end{align*}
Accordingly, by differentiation,
\begin{align*}
  \frac{\dd g_{\alpha\beta}'}{\dd x_{\sigma}}
  &= g_{\mu\nu} A_{(\alpha)}^{\mu}\, \frac{\dd A_{(\beta)}^{\nu}}{\dd x_{\sigma}}
  + g_{\mu\nu} A_{(\beta)}^{\nu}\, \frac{\dd A_{(\alpha)}^{\mu}}{\dd x_{\sigma}}
  + A_{(\alpha)}^{\mu} A_{(\beta)}^{\nu}\, \frac{\dd g_{\mu\nu}}{\dd x_{\sigma}} \\
  &= -g_{\mu\nu} A_{(\alpha)}^{\mu} A_{(\beta)}^{\epsilon} \{\epsilon\sigma, \nu\}
  - g_{\mu\nu} A_{(\beta)}^{\nu} A_{(\alpha)}^{\epsilon} \{\epsilon\sigma, \mu\}
  + A_{(\alpha)}^{\mu} A_{(\beta)}^{\nu}\, \frac{\dd g_{\mu\nu}}{\dd x_{\sigma}}
\end{align*}
by~\Eq{(36.2)}. By changing dummy suffixes, this becomes
\begin{align*}
  \frac{\dd g_{\alpha\beta}'}{\dd x_{\sigma}}
  &= A_{(\alpha)}^{\mu} A_{(\beta)}^{\nu} \left[-g_{\mu\epsilon} \{\nu\sigma, \epsilon\} - g_{\epsilon\nu} \{\mu\sigma, \epsilon\} + \frac{\dd g_{\mu\nu}}{\dd x_{\sigma}}\right]\displaybreak[0] \\
  &= A_{(\alpha)}^{\mu} A_{(\beta)}^{\nu} \left[-[\nu\sigma, \mu] - [\mu\sigma, \nu] + \frac{\dd g_{\mu\nu}}{\dd x_{\sigma}}\right] \\
  &= 0 \text{ by \Eq{(27.5)}.}
\end{align*}
Hence the $g_{\alpha\beta}'$ are constant throughout the region. We have thus constructed
a coordinate\hyp{}system fulfilling the condition that the $g$'s are constant, and it
follows that the space-time is flat.

It will be seen that a \emph{uniform} mesh\hyp{}system, i.e.\ one in which the unit
meshes are connected with one another by parallel displacement, is necessarily
a Cartesian system (rectangular or oblique). Uniformity in this sense
is impossible in space-time for which the Riemann\hyp{}Christoffel tensor does not
vanish, e.g.\ there can be no uniform mesh\hyp{}system on a sphere.

When space-time is not flat we can introduce coordinates which will be
approximately Galilean in a small region round a selected point, the $g_{\mu\nu}$ being
not constant but stationary there; this amounts to identifying the curved
space-time with the osculating flat space-time for a small distance round the
point. Expressing the procedure analytically, we choose coordinates such that
the $40$~derivatives $\dd g_{\mu\nu}/dx_{\sigma}$ vanish \emph{at the selected point}. It is fairly obvious
from general considerations that this will always be possible; but the following
is a formal proof. Having transferred the origin to the selected point, make
the following transformation of coordinates
\[
x_{\epsilon} = g_{\epsilon}^{\mu} x_{\mu}' - \tfrac{1}{2} \{\alpha\beta, \epsilon\}_{0}\, g_{\alpha}^{\mu}\, g_{\beta}^{\nu}\, x_{\mu}' x_{\nu}',
\Tag{(36.4)}
\]
where the value of the $3$-index symbol at the origin is to be taken. Then at
the origin
\begin{align*}
  \frac{\dd x_{\epsilon}}{\dd x_{\mu}'} &= g_{\epsilon}^{\mu},
  \Tag{(36.45)} \\
  \frac{\dd^{2} x_{\epsilon}}{\dd x_{\mu}'\, \dd x_{\nu}'}
  &= -\{\alpha\beta, \epsilon\}\, g_{\alpha}^{\mu}\, g_{\beta}^{\nu} \\
  &= -\{\alpha\beta, \epsilon\}\, \frac{\dd x_{\alpha}}{\dd x_{\mu}'}\, \frac{\dd x_{\beta}}{\dd x_{\nu}'}
  \quad\text{by \Eq{(36.45)}.}
\end{align*}
Hence by~\Eq{(31.3)}
\[
\{\mu\nu, \rho\}'\, \frac{\dd x_{\epsilon}}{\dd x_{\rho}'} = 0.
\]
But
\[
\{\mu\nu, \rho\}'\, \frac{\dd x_{\epsilon}}{\dd x_{\rho}'}
= \{\mu\nu, \rho\}'\, g_{\epsilon}^{\rho}
= \{\mu\nu, \epsilon\}'.
\]

Hence in the new coordinates the $3$-index symbols vanish at the origin;
and it follows by~\Eq{(27.4)} and~\Eq{(27.5)} that the first derivatives of the~$g_{\mu\nu}'$ vanish.
This is the preliminary transformation presupposed in \SecRef{4}.

We pass on to a somewhat more difficult transformation which is important
as contributing an insight into the significance of~$B_{\mu\nu\sigma}^{\epsilon}$.

It is not possible to make the second derivatives of the~$g_{\mu\nu}$ vanish at the
selected point (as well as the first derivatives) unless the Riemann\hyp{}Christoffel
tensor vanishes there; but a great number of other special conditions can be
imposed on the $100$~second derivatives by choosing the coordinates suitably.
Make an additional transformation of the form
\[
x_{\epsilon} = g_{\mu}^{\epsilon} x_{\mu}' + \tfrac{1}{6} a_{\mu\nu\sigma}^{\epsilon}\, x_{\mu}' x_{\nu}' x_{\sigma}',
\Tag{(36.5)}
\]
where $a_{\mu\nu\sigma}^{\epsilon}$~represents arbitrary coefficients symmetrical in $\mu$,~$\nu$,~$\rho$. This new
transformation will not affect the first derivatives of the~$g_{\mu\nu}$ at the origin,
which have already been made to vanish by the previous transformation, but
it alters the second derivatives. By differentiating~\Eq{(31.3)}, viz.\
\[
\{\mu\nu, \rho\}'\, \frac{\dd x_{\epsilon}}{\dd x_{\rho}'}
- \frac{\dd x_{\alpha}}{\dd x_{\mu}'}\, \frac{\dd x_{\beta}}{\dd x_{\nu}'}\, \{\alpha\beta, \epsilon\}
= \frac{\dd^{2} x_{\epsilon}}{\dd x_{\mu}'\, \dd x_{\nu}'},
\]
we obtain at the origin
\[
\frac{\dd}{\dd x_{\sigma}'}\{\mu\nu, \rho\}'\, \frac{\dd x_{\epsilon}}{\dd x_{\rho}'}
- \frac{\dd x_{\alpha}}{\dd x_{\mu}'}\,
  \frac{\dd x_{\beta}}{\dd x_{\nu}'}\,
  \frac{\dd x_{\gamma}}{\dd x_{\sigma}'}\,
  \frac{\dd}{\dd x_{\gamma}} \{\alpha\beta, \epsilon\}
  = \frac{\dd^{3} x_{\epsilon}}{\dd x_{\mu}'\, \dd x_{\nu}'\, \dd x_{\sigma}'},
\]
since the $3$-index symbols themselves vanish. Hence by~\Eq{(36.5)}\footnote
  {For the disappearance of the factor~$\frac{1}{6}$, see~\Eq{(35.6)}.}
\[
\frac{\dd}{\dd x_{\sigma}'} \{\mu\nu, \rho\}' \cdot g_{\rho}^{\epsilon}
- g_{\mu}^{\alpha} g_{\nu}^{\beta} g_{\sigma}^{\gamma}\, \frac{\dd}{\dd x_{\gamma}} \{\alpha\beta, \epsilon\}
= a_{\mu\nu\sigma}^{\epsilon},
\]
which reduces to
\[
\frac{\dd}{\dd x_{\sigma}'} \{\mu\nu, \epsilon\}'
- \frac{\dd}{\dd x_{\sigma}} \{\mu\nu, \epsilon\} = a_{\mu\nu\sigma}^{\epsilon}.
\Tag{(36.55)}
\]
The transformation~\Eq{(36.5)} accordingly increases $\dd \{\mu\nu, \epsilon\}/\dd x_{\sigma}$ by~$a_{\mu\nu\sigma}^{\epsilon}$.

Owing to the symmetry of~$a_{\mu\nu\sigma}^{\epsilon}$, all three quantities
\[
\frac{\dd}{\dd x_{\sigma}} \{\mu\nu, \epsilon\},\quad
\frac{\dd}{\dd x_{\nu}} \{\mu\sigma, \epsilon\},\quad
\frac{\dd}{\dd x_{\mu}} \{\nu\sigma, \epsilon\}
\]
are necessarily increased by the same amount. Now the unaltered difference
\[
\frac{\dd}{\dd x_{\nu}} \{\mu\sigma, \epsilon\}
- \frac{\dd}{\dd x_{\sigma}} \{\mu\nu, \epsilon\}
= B_{\mu\nu\sigma}^{\epsilon},
\Tag{(36.6)}
\]
since the remaining terms of~\Eq{(34.4)} vanish in the coordinates here used. We
cannot alter any of the components of the Riemann\hyp{}Christoffel tensor; but,
\index{Riemann\hyp{}Christoffel tensor!importance of}%
subject to this limitation, the alterations of the derivatives of the $3$-index
symbols are arbitrary.

The most symmetrical way of imposing further conditions is to make a
transformation such that
\[
\frac{\dd}{\dd x_{\sigma}} \{\mu\nu, \epsilon\}
+ \frac{\dd}{\dd x_{\nu}} \{\mu\sigma, \epsilon\}
+ \frac{\dd}{\dd x_{\mu}} \{\nu\sigma, \epsilon\} = 0.
\Tag{(36.7)}
\]
There are $80$~different equations of this type, each of which fixes one of the
$80$~arbitrary coefficients~$a_{\mu\nu\sigma}^{\epsilon}$. In addition there are $20$~independent equations
of type~\Eq{(36.6)} corresponding to the $20$~independent components of the
Riemann\hyp{}Christoffel tensor. Thus we have just sufficient equations to determine
uniquely the $100$~second derivatives of the~$g_{\mu\nu}$. Coordinates such that
$\dd g_{\mu\nu}/\dd x_{\sigma}$~is zero and $\dd^{2} g_{\mu\nu}/\dd x_{\sigma}\, \dd x_{\tau}$ satisfies~\Eq{(36.7)} may be called \emph{canonical
coordinates}.

By solving the $100$~equations we obtain all the $\dd^{2} g_{\mu\nu}/\dd x_{\sigma}\, \dd x_{\tau}$ for canonical
coordinates expressed as linear functions of the~$B_{\mu\nu\sigma}^{\epsilon}$.

The two successive transformations which lead to canonical coordinates
\index{Canonical coordinates}%
\index{Coordinate\hyp{}systems!canonical}%
\index{Fundamental velocity!tensors}%
are combined in the formula
\begin{multline*}
  x_{\epsilon} = g_{\mu}^{\epsilon} x_{\mu}'
  - \tfrac{1}{2}\{\mu\nu, \epsilon\}_{0}\, x_{\mu}' x_{\nu}' \\
  - \frac{1}{18}\left[\frac{\dd}{\dd x_{\mu}}\{\nu\sigma, \epsilon\}
                    + \frac{\dd}{\dd x_{\nu}} \{\mu\sigma, \epsilon\}
                    + \frac{\dd}{\dd x_{\sigma}} \{\mu\nu, \epsilon\}\right]_{0}
  x_{\mu}' x_{\nu}' x_{\sigma}'.
  \Tag{(36.8)}
\end{multline*}
At the origin $\dd x_{\epsilon}/\dd x_{\mu}' = g_{\mu}^{\epsilon}$, so that the transformation does not alter any
tensor at the origin. For example, the law of transformation of~$C_{\mu\nu\sigma}$ gives
\begin{align*}
  C_{\mu\nu\sigma}'
  = C_{\alpha\beta\gamma}\, \frac{\dd x_{\alpha}}{\dd x_{\mu}'}\,
  \frac{\dd x_{\beta}}{\dd x_{\nu}'}\,
  \frac{\dd x_{\gamma}}{\dd x_{\sigma}'}
  &= C_{\alpha\beta\gamma}\, g_{\mu}^{\alpha} g_{\nu}^{\beta} g_{\sigma}^{\gamma} \\
  &= C_{\mu\nu\sigma}.
\end{align*}
The transformation in fact alters the curvature and hypercurvature of the
axes passing through the origin, but does not alter the angles of intersection.

Consider any tensor which contains only the~$g_{\mu\nu}$ and their first and second
derivatives. In canonical coordinates the first derivatives vanish and the
second derivatives are linear functions of the~$B_{\mu\nu\sigma}^{\epsilon}$; hence the whole tensor is
a function of the~$g_{\mu\nu}$ and the~$B_{\mu\nu\sigma}^{\epsilon}$. But neither the tensor itself nor the~$g_{\mu\nu}$
and $B_{\mu\nu\sigma}^{\epsilon}$ have been altered in the reduction to canonical coordinates, hence
the same functional relation holds true in the original unrestricted coordinates.
We have thus the important result---

\emph{The only fundamental tensors which do not contain derivatives of~$g_{\mu\nu}$ beyond
the second order are functions of $g_{\mu\nu}$ and~$B_{\mu\nu\sigma}^{\epsilon}$.}

This shows that our treatment of the tensors describing the character of
space-time has been exhaustive as far as the second order. If for suitably
chosen coordinates two surfaces have the same $g_{\mu\nu}$ and $B_{\mu\nu\sigma}^{\epsilon}$ at some point,
they will be applicable to one another as far as cubes of the coordinates; the
two tensors suffice to specify the whole metric round the point to this extent.

Having made the first derivatives vanish, we can by the linear transformation
explained in \SecRef{4} give the $g_{\mu\nu}$ Galilean values at the selected point.
The coordinates so obtained are called \emph{natural coordinates} at the point and
\index{Natural coordinates}%
\index{Natural coordinates!measure}%
quantities referred to these coordinates are said to be expressed in \emph{natural
measure}. Natural coordinates are thus equivalent to Galilean coordinates
when only the $g_{\mu\nu}$ and their first derivatives are considered; the difference
appears when we study phenomena involving the second derivatives.

By making a Lorentz transformation (which leaves the coordinates still
a natural system) we can reduce to rest the material located at the point, or
an observer supposed to be stationed with his measuring appliances at the
point. The \emph{natural measure} is then further particularised as the \emph{proper\hyp{}measure}
of the material, or observer. It may be noticed that the material
will be at rest both as regards velocity and acceleration (unless it is acted on
by electromagnetic forces) because there is no field of acceleration relative to
natural coordinates.

To sum up this discussion of special systems of coordinates.---When the
Riemann\hyp{}Christoffel tensor vanishes, we can adopt Galilean coordinates
throughout the region. When it does not vanish we can adopt coordinates
which agree with Galilean coordinates at a selected point in the values of the~$g_{\mu\nu}$
and their first derivatives but not in the second derivatives; these are
called \emph{natural coordinates} at the point. Either Galilean or natural coordinates
can be subjected to Lorentz transformations, so that we can select a system
with respect to which a particular observer is at rest; this system will be the
\emph{proper\hyp{}coordinates} for that observer. Although we cannot in general make
\index{Proper\hyp{}coordinates}%
natural coordinates agree with Galilean coordinates in the second derivatives
of the~$g_{\mu\nu}$, we can impose $80$~partially arbitrary conditions on the $100$~second
derivatives; and when these conditions are selected as in~\Eq{(36.7)} the resulting
coordinates have been called \emph{canonical}.

There is another way of specialising coordinates which may be mentioned
here for completeness. It is always possible to choose coordinates such that
the determinant $g = -1$ everywhere (as in Galilean coordinates). This is
explained in \SecRef{49}.

We may also consider another class of specialised coordinates---those
which are permissible in special problems. There are certain (non\hyp{}Euclidean)
coordinates found to be most convenient in dealing with the gravitational
field of the sun, Einstein's or de~Sitter's curved world, and so on. It must be
remembered, however, that these refer to idealised problems, and coordinate\hyp{}systems
\index{Coordinate\hyp{}systems!natural}%
\index{Coordinate\hyp{}systems!proper}%
with simple properties can only be approximately realised in nature.
\index{Coordinate\hyp{}systems!statical}%
\index{Static coordinates}%
If possible a \emph{static system} of coordinates is selected, the condition for this
being that all the~$g_{\mu\nu}$ are independent of one of the coordinates~$x_{4}$ (which
must be of timelike character\footnotemark).\footnotetext
  {$dx_{4}$ will be timelike if $g_{44}$ is always positive.}
In that case the interval corresponding to
any displacement~$dx_{\mu}$ is independent of the ``time''~$x_{4}$. Such a system can,
of course, only be found if the relative configuration of the attracting masses
is maintained unaltered. If in addition it is possible to make $g_{14}$, $g_{24}$, $g_{34} = 0$
the time will be reversible, and in particular the forward velocity of light
along any track will be equal to the backward velocity; this renders the
application of the name ``time'' to~$x_{4}$ more just, since one of the alternative
conventions of \SecRef{11} is satisfied. We shall if possible employ systems which
are static and reversible in dealing with large regions of the world; problems
in which this simplification is not permissible must generally be left aside as
insoluble--e.g.\ the problem of two attracting bodies. For small regions of the
world the greatest simplification is obtained by using natural coordinates.

\Section{37.}{Einstein's law of gravitation}
\index{Einstein's law of gravitation}%
\index{G@$G_{\mu\nu}$ (Einstein tensor)}%

{\Loosen The contracted Riemann\hyp{}Christoffel tensor is formed by setting $\epsilon = \sigma$ in~$B_{\mu\nu\sigma}^{\epsilon}$.
It is denoted by~$G_{\mu\nu}$. Hence by~\Eq{(34.4)}}
\[
G_{\mu\nu} = \{\mu\sigma, \alpha\} \{\alpha\nu, \sigma\}
         - \{\mu\nu, \alpha\} \{\alpha\sigma, \sigma\}
  + \frac{\dd}{\dd x_{\nu}} \{\mu\sigma, \sigma\}
  - \frac{\dd}{\dd x_{\sigma}} \{\mu\nu, \sigma\}.
\Tag{(37.1)}
\]
The symbols containing a duplicated suffix are simplified by~\Eq{(35.4)}, viz.\
\[
\{\mu\sigma, \sigma\} = \frac{\dd}{\dd x_{\mu}} \log \sqrt{-g}.
\]
Hence, with some alterations of dummy suffixes,
%[** TN: Not broken in the original]
\begin{multline*}
G_{\mu\nu} = -\frac{\dd}{\dd x_{\alpha}} \{\mu\nu, \alpha\}
+ \{\mu\alpha, \beta\} \{\nu\beta, \alpha\} \\
+ \frac{\dd^{2}}{\dd x_{\mu}\, \dd x_{\nu}} \log \sqrt{-g}
- \{\mu\nu, \alpha\}\, \frac{\dd}{\dd x_{\alpha}} \log \sqrt{-g}.
\Tag{(37.2)}
\end{multline*}

Contraction by setting $\epsilon = \mu$ does not provide an alternative tensor, because
\[
B_{\mu\nu\sigma}^{\mu} = g^{\mu\rho} B_{\mu\nu\sigma\rho} = 0,
\]
owing to the antisymmetry of~$B_{\mu\nu\sigma\rho}$ in $\mu$ and~$\rho$.

The law
\[
G_{\mu\nu} = 0,
\Tag{(37.3)}
\]
in empty space, is chosen by Einstein for his law of gravitation.

We see from~\Eq{(37.2)} that $G_{\mu\nu}$~is a symmetrical tensor; consequently the law
provides $10$~partial differential equations to determine the~$g_{\mu\nu}$. It will be found
later (\SecRef{52}) that there are $4$~identical relations between them, so that the
number of equations is effectively reduced to~$6$. The equations are of the
second order and involve the second differential coefficients of~$g_{\mu\nu}$ linearly. We
proved in \SecRef{36} that tensors not containing derivatives beyond the second must
necessarily be compounded from $g_{\mu\nu}$ and~$B_{\mu\nu\sigma}^{\epsilon}$; so that, unless we are prepared
to go beyond the second order, the choice of a law of gravitation is very limited,
and we can scarcely avoid relying on the tensor~$G_{\mu\nu}$\footnotemark.\footnotetext
  {The law $B_{\mu\nu\sigma\rho} = 0$ (giving flat space-time throughout all empty regions) would obviously be
  too stringent, since it does not admit of the existence of irreducible fields of force.}

Without introducing higher derivatives, which would seem out of place in
this problem, we can suggest as an alternative to~\Eq{(37.3)} the law
\[
G_{\mu\nu} = \lambda g_{\mu\nu},
\Tag{(37.4)}
\]
where $\lambda$~is a universal constant. There are theoretical grounds for believing
that this is actually the correct form; but it is certain that $\lambda$~must be an
extremely small constant, so that in practical applications we still take \Eq{(37.3)}
as sufficiently approximate. The introduction of the small constant~$\lambda$ leads to
the spherical world of Einstein or de~Sitter to which we shall return in
Chapter~\ChapNum{V}\@.

The spur
\[
G = g^{\mu\nu} G_{\mu\nu}
\Tag{(37.5)}
\]
is called the Gaussian curvature, or simply the \emph{curvature,} of space-time. It
\index{Curvature!Gaussian}%
\index{Gaussian curvature}%
must be remembered, however, that the deviation from flatness is described
in greater detail by the tensors $G_{\mu\nu}$ and $B_{\mu\nu\sigma\rho}$ (sometimes called \emph{components of
curvature}) and the vanishing of~$G$ is by no means a sufficient condition for fiat
space-time.

Einstein's law of gravitation expresses the fact that the geometry of an
empty region of the world is not of the most general Riemannian type, but is
limited. General Riemannian geometry corresponds to the quadratic form~\Eq{(2.1)}
with the $g$'s entirely unrestricted functions of the coordinates; Einstein
asserts that the natural geometry of an empty region is not of so unlimited a
kind, and the possible values of the $g$'s are restricted to those which satisfy
the differential equations~\Eq{(37.3)}. It will be remembered that a field of force
arises from the discrepancy between the natural geometry of a coordinate\hyp{}system
and the abstract Galilean geometry attributed to it; thus any law
governing a field of force must be a law governing the natural geometry.
That is why the law of gravitation must appear as a restriction on the possible
natural geometry of the world. The inverse\hyp{}square law, which is a
plausible law of weakening of a supposed absolute force, becomes quite unintelligible
(and indeed impossible) when expressed as a restriction on the
intrinsic geometry of space-time; we have to substitute some law obeyed
by the tensors which describe the world\hyp{}conditions determining the natural
geometry.

%[** TN: Shortened running head]
\Section[The field of an isolated particle]{38.}{The gravitational field of an isolated particle}
\index{Gravitational field of a particle}%
\index{Particle!gravitational field of}%

We have now to determine a particular solution of the equations~\Eq{(37.3)}.
The solution which we shall obtain will ultimately be shown to correspond to
the field of an isolated particle continually at rest at the origin; and in seeking
a solution we shall be guided by our general idea of the type of solution to be
expected for such a particle. This preliminary argument need not be rigorous;
the final test is whether the formulae suggested by it satisfy the equations
to be solved.

In flat space-time the interval, referred to spherical polar coordinates and
time, is
\[
ds^{2} = -dr^{2} - r^{2}\, d\theta^2 - r^{2} \sin^{2}\theta\, d\phi^{2} + dt^{2}.
\Tag{(38.11)}
\]
If we consider what modifications of this can be made without destroying the
spherical symmetry in space, the symmetry as regards past and future time,
or the static condition, the most general possible form appears to be
\[
ds^{2} = -U(r)\, dr^{2} - V(r)\, (r^{2}\, d\theta^2 + r^{2} \sin^{2}\theta\, d\phi^{2}) + W(r)\, dt^{2},
\Tag{(38.12)}
\]
where $U$, $V$, $W$ are arbitrary functions of~$r$. Let
\[
r_{1}^{2} = r^{2} V(r).
\]
Then \Eq{(38.12)} becomes of the form
\[
ds^{2} = - U_{1}(r_{1})\, dr_{1}^{2} - r_{1}^{2}\, d\theta^2 - r_{1}^{2} \sin^{2}\theta\, d\phi^{2} + W_{1}(r_{1})\, dt^{2},
\Tag{(38.13)}
\]
where $U_{1}$ and $W_{1}$ are arbitrary functions of~$r_{1}$. There is no reason to regard
$r$ in~\Eq{(38.12)} as more immediately the counterpart of~$r$ in~\Eq{(38.11)} than $r_{1}$~is. If
the functions $U$, $V$, $W$ differ only slightly from unity, both $r$ and $r_{1}$ will have
approximately the properties of the radius\hyp{}vector in Euclidean geometry; but
no length in non\hyp{}Euclidean space can have exactly the properties of a Euclidean
radius\hyp{}vector, and it is arbitrary whether we choose $r$ or~$r_{1}$ as its closest representative.
We shall here choose~$r_{1}$, and accordingly drop the suffix, writing
\Eq{(38.13)} in the form
\[
ds^{2} = -e^{\lambda}\, dr^{2} - r^{2}\, d\theta^2 - r^{2} \sin^{2}\theta\, d\phi^{2} + e^{\nu}\, dt^{2},
\Tag{(38.2)}
\]
where $\lambda$ and~$\nu$ are functions of $r$~only.

Moreover since the gravitational field (or disturbance of flat space-time)
due to a particle diminishes indefinitely as we go to an infinite distance, we
must have $\lambda$ and~$\nu$ tend to zero as $r$~tends to infinity. Formula~\Eq{(38.2)} will
then reduce to~\Eq{(38.11)} at an infinite distance from the particle.

Our coordinates are
\[
x_{1} = r,\quad
x_{2} = \theta,\quad
x_{3} = \phi,\quad
x_{4} = t,
\]
and the fundamental tensor is by~\Eq{(38.2)}
\[
g_{11} = -e^{\lambda},\
g_{22} = -r^{2},\
g_{33} = -r^{2} \sin^{2}\theta,\
g_{44} = e^{\nu},
\Tag{(38.31)}
\]
and
\[
g_{\mu\nu} = 0\quad\text{if $\mu \neq \nu$.}
\]

The determinant~$g$ reduces to its leading diagonal $g_{11} g_{22} g_{33} g_{44}$. Hence
\[
-g = e^{\lambda + \nu} r^{4} \sin^{2}\theta,
\Tag{(38.32)}
\]
and $g^{11} = 1/g_{11}$, etc., so that
\[
g^{11} = -e^{-\lambda},\ \
g^{22} = -1/r^{2},\ \
g^{33} = -1/r^{2} \sin^{2}\theta,\ \
g^{44} = e^{-\nu}.
\Tag{(38.33)}
\]

Since all the $g^{\mu\nu}$ vanish except when the two suffixes are the same, the
summation disappears in the formula for the $3$-index symbols~\Eq{(27.2)}, and
\[
\{\mu\nu, \sigma\}
= \tfrac{1}{2} g^{\tau\sigma}\left(
  \frac{\dd g_{\mu\sigma}}{\dd x_{\nu}}
+ \frac{\dd g_{\nu\sigma}}{\dd x_{\mu}}
+ \frac{\dd g_{\mu\nu}}{\dd x_{\sigma}}
\right)\quad\text{not summed.}
\]

If $\mu$, $\nu$, $\sigma$ denote \emph{different} suffixes we get the following possible cases (the
summation convention being suspended):
%[** TN: Punctuation moved in-line]
\[
\left.
\begin{aligned}
  \{\mu\mu, \mu\}
  &= \Neg\tfrac{1}{2} g^{\mu\mu}\, \frac{\dd g_{\mu\mu}}{\dd x_{\mu}}
  = \tfrac{1}{2} \frac{\dd}{\dd x_{\mu}} (\log g_{\mu\mu}), \\
  \{\mu\mu, \nu\}
  &= -\tfrac{1}{2} g^{\mu\nu}\, \frac{\dd g_{\mu\mu}}{\dd x_{\nu}}, \\
  \{\mu\nu, \nu\}
  &= \Neg\tfrac{1}{2} g^{\nu\nu}\, \frac{\dd g_{\nu\nu}}{\dd x_{\mu}}
  = \tfrac{1}{2} \frac{\dd}{\dd x_{\mu}} (\log g_{\nu\nu}), \\
  \{\mu\nu, \sigma\}
  &= \Neg0.
\end{aligned}\right\}
\Tag{(38.4)}
\]

It is now easy to go systematically through the $40$ $3$-index symbols calculating
the values of those which do not vanish. We obtain the following
results, the accent denoting differentiation with respect to~$r$:
%[** TN: Punctuation moved in-line]
\[
\left.
\begin{aligned}
\{11, 1\} &= \tfrac{1}{2} \lambda', \\
\{12, 2\} &= 1/r, \\
\{13, 3\} &= 1/r, \\
\{14, 4\} &= \tfrac{1}{2} \nu', \\
\{22, 1\} &= -re^{-\lambda}, \\
\{23, 3\} &= \cot\theta, \\
\{33, 1\} &= -r\sin^{2}\theta\, e^{-\lambda}, \\
\{33, 2\} &= -\sin\theta \cos\theta, \\
\{44, 1\} &= \tfrac{1}{2} e^{\nu-\lambda} \nu'.
\end{aligned}
\right\}
\Tag{(38.5)}
\]
The remaining $31$~symbols vanish. Note that $\{21, 2\}$ is the same as $\{12, 2\}$, etc.

These values must be substituted in~\Eq{(37.2)}. As there may be some pitfalls
in carrying this out, we shall first write out the equations~\Eq{(37.2)} in full, omitting
the terms ($223$~in number) which now obviously vanish.
%[** TN: Re-broken]
\begin{align*}
  G_{11} &= -\frac{\dd}{\dd r} \{11, 1\} \begin{aligned}[t]
    &+ \{11, 1\} \{11, 1\} + \{12, 2\} \{12, 2\} \\
    &+ \{13, 3\} \{13, 3\} + \{14, 4\} \{14, 4\}
  \end{aligned} \\
&\qquad + \frac{\dd^{2}}{\dd r^{2}} \log\sqrt{-g} - \{11, 1\}\, \frac{\dd}{\dd r} \log\sqrt{-g},\displaybreak[0] \\
G_{22} &= -\frac{\dd}{\dd r} \{22, 1\} + 2\{22, 1\} \{21, 2\} + \{23, 3\} \{23, 3\} \\
&\qquad + \frac{\dd^{2}}{\dd\theta^2} \log\sqrt{-g} - \{22, 1\}\, \frac{\dd}{\dd r} \log\sqrt{-g},\displaybreak[0] \\
G_{33} &= -\frac{\dd}{\dd r} \{33, 1\} - \frac{\dd}{\dd\theta}\, \{33, 2\} \\
&\qquad + 2\{33, 1\} \{31, 3\} + 2\{33, 2\} \{32, 3\} \\
&\qquad - \{33, 1\}\, \frac{\dd}{\dd r} \log\sqrt{-g} - \{33, 2\}\, \frac{\dd}{\dd\theta} \log\sqrt{-g},\displaybreak[0] \\
G_{44} &= -\frac{\dd}{\dd r} \{44, 1\} + 2\{44, 1\} \{41, 4\} - \{44, 1\}\, \frac{\dd}{\dd r} \log\sqrt{-g},\displaybreak[0] \\
G_{12} &= \Neg \{13, 3\} \{23, 3\} - \{12, 2\}\, \frac{\dd}{\dd\theta} \log\sqrt{-g}.
\end{align*}
The remaining components contain no surviving terms.

Substitute from \Eq{(38.5)} and \Eq{(38.32)} in these, and collect the terms. The
equations to be satisfied become\label{eqn:(38.6)}
\begin{align*}
  G_{11} &= \tfrac{1}{2}\nu'' - \tfrac{1}{4} \lambda' \nu' + \tfrac{1}{4} \nu'^{2} - \lambda'/r = 0,
  \Tag{(38.61)}\displaybreak[0] \\
  G_{22} &= e^{-\lambda} \bigl(1 + \tfrac{1}{2}r(\nu' - \lambda')\bigr) - 1 = 0,
  \Tag{(38.62)}\displaybreak[0] \\
  G_{33} &= \sin^{2}\theta \cdot e^{-\lambda} \bigl(1 + \tfrac{1}{2}r(\nu' - \lambda')\bigr) - \sin^{2}\theta = 0,
  \Tag{(38.63)}\displaybreak[0] \\
  G_{44} &= e^{\nu-\lambda} (-\tfrac{1}{2} \nu'' + \tfrac{1}{4} \lambda' \nu' - \tfrac{1}{4} \nu'^{2} - \nu'/r) = 0,
  \Tag{(38.64)} \\
%[** TN: Omitted extra "= 0"]
  G_{12} &= 0.
  \Tag{(38.65)}
\end{align*}
We may leave aside \Eq{(38.63)} which is a mere repetition of~\Eq{(38.62)}; then there
are left three equations to be satisfied by $\lambda$ and~$\nu$. From \Eq{(38.61)} and \Eq{(38.64)}
we have $\lambda' = -\nu'$. Since $\lambda$ and~$\nu$ are to vanish together at $r = \infty$, this requires
that
\[
\lambda = -\nu.
\]
Then \Eq{(38.62)} becomes
\[
e^{\nu} (1 + r\nu') = 1.
\]
Set $e^{\nu} = \gamma$, then
\[
\gamma + r\gamma' = 1.
\]
Hence, integrating,
\[
\gamma = 1 - \frac{2m}{r},
\Tag{(38.7)}
\]
where $2m$~is a constant of integration.

It will be found that all three equations are satisfied by this solution.
Accordingly, substituting $e^{-\lambda} = e^{\nu} = \gamma$ in~\Eq{(38.2)},
\[
ds^{2} = -\gamma^{-1}\, dr^{2} - r^{2}\, d\theta^2 - r^{2} \sin^{2}\theta\, d\phi^{2} + \gamma\, dt^{2},
\Tag{(38.8)}
\]
where $\gamma = 1 - 2m/r$, is a particular solution of Einstein's gravitational equations
$G_{\mu\nu} = 0$. The solution in this form was first obtained by Schwarzschild.

\Section{39.}{Planetary orbits}
\index{Orbits of planets}%
\index{Planetary orbits}%

According to~\Eq{(15.7)} the track of a particle moving freely in the space-time
given by~\Eq{(38.8)} is determined by the equations of a geodesic~\Eq{(28.5)}, viz.\
\[
\frac{d^{2}x_{\alpha}}{ds^{2}} + \{\mu\nu, \alpha\}\, \frac{dx_{\mu}}{ds}\, \frac{dx_{\nu}}{ds} = 0.
\Tag{(39.1)}
\]

Taking first $\alpha = 2$, the surviving terms are
\[
\frac{d^{2}x_{2}}{ds^{2}}
  + \{12, 2\}\, \frac{dx_{1}}{ds}\, \frac{dx_{2}}{ds}
  + \{21, 2\}\, \frac{dx_{2}}{ds}\, \frac{dx_{1}}{ds}
  + \{33, 2\}\, \frac{dx_{3}}{ds}\, \frac{dx_{3}}{ds} = 0,
\]
or using~\Eq{(38.5)}
\[
\frac{d^{2}\theta}{ds^{2}} + \frac{2}{r}\, \frac{dr}{ds}\, \frac{d\theta}{ds} - \cos\theta \sin\theta \left(\frac{d\phi}{ds}\right)^{2} = 0.
\Tag{(39.2)}
\]
Choose coordinates so that the particle moves initially in the plane $\theta = \frac{1}{2}\pi$.
Then $d\theta/ds = 0$ and $\cos\theta = 0$ initially, so that $d^{2}\theta/ds^{2} = 0$. The particle therefore
continues to move in this plane, and we may simplify the remaining
equations by putting $\theta = \frac{1}{2}\pi$ throughout. The equations for $\alpha = 1$, $3$,~$4$ are
found in like manner, viz.\
\begin{align*}
  \frac{d^{2}r}{ds^{2}} + \tfrac{1}{2}\lambda' \left(\frac{dr}{ds}\right)^{2}
  - re^{-\lambda} \left(\frac{d\phi}{ds}\right)^{2}
  + \tfrac{1}{2} e^{\nu-\lambda} \nu' \left(\frac{dt}{ds}\right)^{2}
  &= 0,
  \Tag{(39.31)}\displaybreak[0] \\
  \frac{d^{2}\phi}{ds^{2}} + \frac{2}{r}\, \frac{dr}{ds}\, \frac{d\phi}{ds} &= 0,
  \Tag{(39.32)}\displaybreak[0] \\
  \frac{d^{2}t}{ds^{2}} + \nu'\, \frac{dr}{ds}\, \frac{dt}{ds} &= 0.
  \Tag{(39.33)}
\end{align*}
The last two equations may be integrated immediately, giving
\begin{align*}
  r^{2}\, \frac{d\phi}{ds} &= h,
  \Tag{(39.41)}\displaybreak[0] \\
  \frac{dt}{ds} &= ce^{-\nu} = c/\gamma,
  \Tag{(39.42)}
\end{align*}
where $h$ and $c$ are constants of integration.

Instead of troubling to integrate~\Eq{(39.31)} we can use in place of it \Eq{(38.8)}
which plays here the part of an integral of energy. It gives
\[
\gamma^{-1} \left(\frac{dr}{ds}\right)^{2}
  + r^{2} \left(\frac{d\phi}{ds}\right)^{2}
  - \gamma \left(\frac{dt}{ds}\right)^{2} = -1.
\Tag{(39.43)}
\]

Eliminating $dt$ and~$ds$ by means of \Eq{(39.41)} and~\Eq{(39.42)}
\[
\frac{1}{\gamma} \left(\frac{h}{r^{2}}\, \frac{dr}{d\phi}\right)^{2}
  + \frac{h^{2}}{r^{2}} - \frac{c^2}{\gamma} = -1,
\Tag{(39.44)}
\]
whence, multiplying through by~$\gamma$ or~$(1 - 2m/r)$,
\[
\left(\frac{h}{r^{2}}\, \frac{dr}{d\phi}\right)^{2}
  + \frac{h^{2}}{r^{2}} = c^2 - 1 + \frac{2m}{r} + \frac{2m}{r} \cdot \frac{h^{2}}{r^{2}},
\]
or writing $1/r = u$,
\[
\left(\frac{du}{d\phi}\right)^{2} + u^2 = \frac{c^2 - 1}{h^{2}} + \frac{2m}{h^{2}} u + 2mu^{3}.
\Tag{(39.5)}
\]

Differentiating with respect to~$\phi$, and removing the factor~$\dfrac{du}{d\phi}$,
\[
\frac{d^{2}u}{d\phi^{2}} + u = \frac{m}{h^{2}} + 3mu^2,
\Tag{(39.61)}
\]
with
\[
r^{2}\, \frac{d\phi}{ds} = h.
\Tag{(39.62)}
\]

Compare these with the equations of a Newtonian orbit
\[
\frac{d^{2}u}{d\phi^{2}} + u = \frac{m}{h^{2}}
\Tag{(39.71)}
\]
with
\[
r^{2}\, \frac{d\phi}{dt} = h.
\Tag{(39.72)}
\]

In \Eq{(39.61)} the ratio of $3mu^2$ to~$m/h^{2}$ is $3h^{2}u^2$, or by \Eq{(39.62)}
\[
3\left(r\, \frac{d\phi}{ds}\right)^{2}.
\]
For ordinary speeds this is an extremely small quantity---practically three
times the square of the transverse velocity in terms of the velocity of light.
For example, this ratio for the earth is 0.00000003. In practical cases the extra
term in~\Eq{(39.61)} will represent an almost inappreciable correction to the Newtonian
orbit~\Eq{(39.71)}.

Again in \Eq{(39.62)} and \Eq{(39.72)} the difference between $ds$ and~$dt$ is equally
insignificant, even if we were sure of what is meant by~$dt$ in the Newtonian
theory. The \emph{proper\hyp{}time} for the body is~$ds$, and it might perhaps be urged
\index{Proper\hyp{}time}%
that $dt$~in equation~\Eq{(39.72)} is intended to refer to this; but on the other hand
$s$~cannot be used as a coordinate since $ds$~is not a complete differential, and
Newton's ``time'' is always assumed to be a coordinate.

Thus it appears that a particle moving in the field here discussed will
behave as though it were under the influence of the Newtonian force exerted
by a particle of gravitational mass~$m$ at the origin, the motion agreeing with
the Newtonian theory to the order of accuracy for which that theory has been
confirmed by observation.

By showing that our solution satisfies $G_{\mu\nu} = 0$, we have proved that it
describes a possible state of the world which might be met with in nature
under suitable conditions. By deducing the orbit of a particle, we have discovered
how that state of the world would be recognised observationally if it
did exist. In this way we conclude that the space-time field represented by~\Eq{(38.8)}
is the one which accompanies (or ``is due to'') a particle of mass~$m$ at
the origin.

The gravitational mass~$m$ is the measure adopted in the Newtonian theory
of the power of the particle in causing a field of acceleration around it, the
units being here chosen so that the velocity of light and the constant of gravitation
are both unity. It should be noticed that we have as yet given no
reason to expect that $m$~in the present chapter has anything to do with the
$m$ introduced in \SecRef{12} to measure the inertial properties of the particle.

For a circular orbit the Newtonian theory gives
\[
m = \omega^2 r^{3} = v^2 r,
\]
the constant of gravitation being unity. Applying this to the earth, $v = 30$ km.\
per~sec.\ $= 10^{-4}$ in terms of the velocity of light, and $r = 1.5 \cdot 10^{8}$~km. Hence
the mass~$m$ of the sun is approximately 1.5 kilometres. The mass of the earth
is 1/300,000th of this, or about 5 millimetres\footnotemark.\footnotetext
  {Objection is sometimes taken to the use of a centimetre as a unit of gravitational (i.e.\
  gravitation\hyp{}exerting) mass; but the same objection would apply to the use of a gram, since the
  gram is properly a measure of a different property of the particle, viz.\ its \emph{inertia}. Our constant
  of integration~$m$ is clearly a length and the reader may, if he wishes to make this clear, call it
  the gravitational radius instead of the gravitational mass. But when it is realised that the gravitational
  radius in centimetres, the inertia in grams, and the energy in ergs, are merely measure\hyp{}numbers
  in different codes of the \emph{same} intrinsic quality of the particle, it seems unduly pedantic
  to insist on the older discrimination of these units which grew up on the assumption that they
  measured qualities which were radically different.}

  More accurately, the mass of the sun, $1.99 \cdot 10^{33}$ grams, becomes in gravitational
\index{Gravitational mass of sun}%
\index{Sun, gravitational mass of}%
  units 1.47 kilometres; and other masses are converted in a like proportion.

\Section{40.}{The advance of perihelion}
\index{Perihelion!advance of}%

The equation \Eq{(39.5)} for the orbit of a planet can be integrated in terms
of elliptic functions; but we obtain the astronomical results more directly by
a method of successive approximation. We proceed from equation~\Eq{(39.61)}
\[
\frac{d^{2}u}{d\phi^{2}} + u = \frac{m}{h^{2}} + 3mu^2.
\Tag{(40.1)}
\]
Neglecting the small term~$3mu^2$, the solution is
\[
u = \frac{m}{h^{2}} \bigl(1 + e\cos(\phi - \varpi)\bigr),
\Tag{(40.2)}
\]
as in Newtonian dynamics. The constants of integration, $e$~and $\varpi$, are the
eccentricity and longitude of perihelion.

Substitute this first approximation in the small term~$3mu^2$, then \Eq{(40.1)}
becomes
\[
\frac{d^{2}u}{d\phi^{2}} + u = \frac{m}{h^{2}} + 3\frac{m^{3}}{h^{4}}
+ 6\frac{m^{3}}{h^{4}} e\cos(\phi - \varpi)
+ \frac{3}{2}\, \frac{m^{3}}{h^{4}} e^{2} \bigl(1 + 2\cos(\phi - \varpi)\bigr).
%[** TN: Handwritten correction \cos 2(\phi - \varpi); unable to verify]
\Tag{(40.3)}
\]
Of the additional terms the only one which can produce an effect within the
range of observation is the term in $\cos(\phi - \varpi)$; this is of the right period to
produce a continually increasing effect by resonance. Remembering that the
particular integral of
\[
\frac{d^{2} u}{d\phi^{2}} + u = A\cos\phi
\]
is
\[
u = \tfrac{1}{2}A \phi \sin\phi,
\]
this term gives a part of~$u$
\[
u_{1} = 3\frac{m^{3}}{h^{4}} e\phi \sin(\phi - \varpi),
\Tag{(40.4)}
\]
which must be added to the complementary integral~\Eq{(40.2)}. Thus the second
approximation is
\begin{align*}
  u &= \frac{m}{h^{2}} \Bigl(1 + e\cos(\phi - \varpi) + 3 \frac{m^{2}}{h^{2}} e\phi \sin(\phi - \varpi)\Bigr) \\
  &= \frac{m}{h^{2}} \bigl(1 + e\cos(\phi - \varpi - \delta\varpi)\bigr),
\end{align*}
where
\[
\delta\varpi = 3\frac{m^{2}}{h^{2}} \phi,
\Tag{(40.5)}
\]
and $(\delta\varpi)^{2}$~is neglected.

Whilst the planet moves through $1$~revolution, the perihelion~$\varpi$ advances
a fraction of a revolution equal to
\[
\frac{\delta\varpi}{\phi} = \frac{3m^{2}}{h^{2}} = \frac{3m}{a(1 - e^{2})},
\Tag{(40.6)}
\]
using the well-known equation of areas $h^{2} = ml = ma(1 - e^{2})$.%

Another form is obtained by using Kepler's third law,
\index{Kepler's third law}%
\[
m = \left(\frac{2\pi}{T}\right)^{2} a^{3},
\]
giving
\[
\frac{\delta\varpi}{\phi} = \frac{12\pi^{2} a^2}{c^2 T^{2}(1 - e^{2})},
\Tag{(40.7)}
\]
where $T$~is the period, and the velocity of light~$c$ has been reinstated.

This advance of the perihelion is appreciable in the case of the planet
Mercury, and the predicted value is confirmed by observation.

For a circular orbit we put $dr/ds$, $d^{2}r/ds^{2} = 0$, so that \Eq{(39.31)}~becomes
\[
-re^{-\lambda} \left(\frac{d\phi}{ds}\right)^{2} + \tfrac{1}{2} e^{\nu-\lambda} \nu' \left(\frac{dt}{ds}\right)^{2} = 0.
\]
Whence
\begin{align*}
  \left(\frac{d\phi}{dt}\right)^{2}
  &= \tfrac{1}{2} e^{\nu} \nu'/r = \tfrac{1}{2} \gamma'/r \\
  &= m/r^{3},
\end{align*}
so that Kepler's third law is \emph{accurately} fulfilled. This result has no observational
significance, being merely a property of the particular definition of~$r$
here adopted. Slightly different coordinate\hyp{}systems exist which might with
equal right claim to correspond to polar coordinates in flat space-time; and
for these Kepler's third law would no longer be exact.

We have to be on our guard against results of this latter kind which would
only be of interest if the radius\hyp{}vector were a directly measured quantity instead
of a conventional coordinate. The advance of perihelion is a phenomenon
of a different category. Clearly the number of years required for an eccentric
orbit to make a complete revolution returning to its original position is capable
of observational test, unaffected by any convention used in defining the exact
length of the radius\hyp{}vector.

For the four inner planets the following table gives the corrections to the
\index{Elements of inner planets}%
\index{Mercury, perihelion of}%
centennial motion of perihelion predicted by Einstein's theory:
\[
\begin{array}{l@{\qquad}l@{\qquad}l}
  & \multicolumn{1}{c}{\delta\varpi} & \multicolumn{1}{c}{e\, \delta\varpi} \\
\text{Mercury} & +42''.9 & +8''.82 \\
\text{Venus} & +\PadTo[r]{42''}{8}.6 & +\PadTo[r]{8''}{0}.05 \\
\text{Earth} & +\PadTo[r]{42''}{3}.8 & +\PadTo[r]{8''}{0}.07 \\
\text{Mars}  & +\PadTo[r]{42''}{1}.35& +\PadTo[r]{8''}{0}.13 \\
\end{array}
\]
The product $e\, \delta\varpi$ is a better measure of the observable effect to be looked for,
and the correction is only appreciable in the case of Mercury. After applying
these corrections to~$e\, \delta\varpi$, the following discrepancies between theory and observation
remain in the secular changes of the elements of the inner planets,
$i$~and $\Omega$ being the inclination and the longitude of the node:
\[
\small
\begin{array}{lrrrr}
  & \multicolumn{1}{c}{e\, \delta\varpi} & \multicolumn{1}{c}{\delta e} & \multicolumn{1}{c}{\sin i\, \delta\Omega} & \multicolumn{1}{c}{\delta i} \\
  \text{Mercury} &
  -0''.58 \pm 0''.29 & -0''.88 \pm 0''.33 & +0''.46 \pm 0''.34 & +0''.38 \pm 0''.54 \\
  \text{Venus}   &
  -\PadTo[r]{0''}{0}.11 \pm \PadTo[r]{0''}{0}.17 & +\PadTo[r]{0''}{0}.21 \pm \PadTo[r]{0''}{0}.21 & +\PadTo[r]{0''}{0}.53 \pm \PadTo[r]{0''}{0}.12 & + \PadTo[r]{0''}{0}.38 \pm \PadTo[r]{0''}{0}.22 \\
  \text{Earth}   & \PadTo[r]{0''}{0}.00 \pm \PadTo[r]{0''}{0}.09 & +\PadTo[r]{0''}{0}.02 \pm \PadTo[r]{0''}{0}.07 & \PadTo{+0''.46}{\cdots} \quad \PadTo{0''.34}{\cdots} & -\PadTo[r]{0''}{0}.22 + \PadTo[r]{0''}{0}.18 \\
  \text{Mars}    & +\PadTo[r]{0''}{0}.51 \pm \PadTo[r]{0''}{0}.23 & +\PadTo[r]{0''}{0}.29 \pm \PadTo[r]{0''}{0}.18 & -\PadTo[r]{0''}{0}.11 \pm \PadTo[r]{0''}{0}.15 & -\PadTo[r]{0''}{0}.01 \pm \PadTo[r]{0''}{0}.13 \\
\end{array}
\]
The probable errors here given include errors of observation, and also errors
in the theory due to uncertainty of the masses of the planets. The positive
sign indicates excess of observed motion over theoretical motion\footnotemark.\footnotetext
  {Newcomb, \Title{Astronomical Constants}. His results have been slightly corrected by using a
  modern value of the constant of precession in the above table; see de~Sitter, \Title{Monthly Notices,}
  vol.~76, p.~728.}

Einstein's correction to the perihelion of Mercury has removed the principal
discordance in the table, which on the Newtonian theory was nearly $30$~times
the probable error. Of the $15$~residuals $8$~exceed the probable error,
and $3$~exceed twice the probable error---as nearly as possible the proper proportion.
But whereas we should expect the greatest residual to be about $3$~times
the probable error, the residual of the node of Venus is rather excessive
at $4\frac{1}{2}$~times the probable error, and may perhaps be a genuine discordance.
Einstein's theory throws no light on the cause of this discordance.

\Section{41.}{The deflection of light}
\index{Deflection of light}%
\index{Light!deflection in gravitational field}%

For motion with the speed of light $ds = 0$, so that by~\Eq{(39.62)} $h = \infty$, and
the orbit~\Eq{(39.61)} reduces to
\[
\frac{d^{2}u}{d\phi^{2}} + u = 3mu^2.
\Tag{(41.1)}
\]
The track of a light-pulse is also given by a geodesic with $ds = 0$ according to~\Eq{(15.8)}.
Accordingly the orbit~\Eq{(41.1)} gives the path of a ray of light.

We integrate by successive approximation. Neglecting $3mu^2$ the solution
of the approximate equation
\[
\frac{d^{2}u}{d\phi^{2}} + u = 0
\]
is the straight line
\[
u = \frac{\cos\phi}{R}.
\Tag{(41.2)}
\]
Substituting this in the small term~$3mu^2$, we have
\[
\frac{d^{2}u}{d\phi^{2}} + u = \frac{3m}{R^{2}} \cos^{2}\phi.
\]
A particular integral of this equation is
\[
u_{1} = \frac{m}{R^{2}} (\cos^{2}\phi + 2\sin^{2}\phi),
\]
so that the complete second approximation is
\[
u = \frac{\cos\phi}{R} + \frac{m}{R^{2}} (\cos^{2}\phi + 2\sin^{2}\phi).
\Tag{(41.3)}
\]

Multiply through by~$rR$,
\[
R = r\cos\phi + \frac{m}{R} (r\cos^{2}\phi + 2r\sin^{2}\phi),
\]
or in rectangular coordinates, $x = r\cos\phi$, $y = r\sin\phi$,
\[
x = R - \frac{m}{R}\, \frac{x^{2} + 2y^{2}}{\sqrt{x^{2} + y^{2}}}.
\Tag{(41.4)}
\]
The second term measures the very slight deviation from the straight line
$x = R$. The asymptotes are found by taking $y$ very large compared with~$x$.
The equation then becomes
\[
x = R - \frac{m}{R}(\pm2y),
\]
and the small angle between the asymptotes is (in circular measure)
\[
\frac{4m}{R}.
\]
For a ray grazing the sun's limb, $m = 1.47$ km., $R = 697000$ km., so that the
\index{Eclipse results}%
deflection should be~$1''.75$. The observed values obtained by the British
eclipse expeditions in 1919 were
\[
\begin{array}{l@{\qquad}c}
\text{Sobral expedition} & 1''.98 \pm 0''.12 \\
\text{Principe expedition} & 1''.61 \pm 0''.30 \\
\end{array}
\]

It has been explained in \Title{Space, Time and Gravitation} that this deflection
is double that which might have been predicted on the Newtonian theory.
In this connection the following paradox has been remarked. Since the curvature
of the light-track is doubled, the acceleration of the light at each point
is double the Newtonian acceleration; whereas for a slowly moving object the
acceleration is practically the same as the Newtonian acceleration. To a man
in a lift descending with acceleration~$m/r^{2}$ the tracks of ordinary particles will
appear to be straight lines; but it looks as though it would require an acceleration~$2m/r^{2}$
to straighten out the light-tracks. Does not this contradict the
principle of equivalence?

The fallacy lies in a confusion between two meanings of the word ``curvature.''
\index{Acceleration of light-pulse}%
\index{Curvature of light-tracks}%
\index{Geodesic curvature}%
The \emph{coordinate} curvature obtained from the equation of the track~\Eq{(41.4)}
is not the \emph{geodesic} curvature. The latter is the curvature with which the local
observer---the man in the lift---is concerned. Consider the curved light-track
traversing the hummock corresponding to the sun's field; its curvature can be
reckoned by projecting it either on the base of the hummock or on the tangent
plane at any point. The curvatures of the two projections will generally be
different. The projection into Euclidean coordinates $(x, y)$ used in~\Eq{(41.4)} is the
projection on the base of the hummock; in applying the principle of equivalence
the projection is on the tangent plane, since we consider a region of the
curved world so small that it cannot be discriminated from its tangent plane.

\Section{42.}{Displacement of the Fraunhofer lines}
\index{Displacement of spectral lines to red, in sun}%
\index{Fraunhofer lines, displacement of}%
\index{Red-shift!of spectral lines in sun}%
\index{Spectral lines, displacement!in sun}%

Consider a number of similar atoms vibrating at different points in the
region. Let the atoms be momentarily at rest in our coordinate\hyp{}system
$(r, \theta, \phi, t)$. The test of similarity of the atoms is that corresponding intervals
should be equal, and accordingly the \emph{interval} of vibration of all the atoms will
\index{Atom, time of vibration of}%
be the same.

Since the atoms are at rest we set $dr$, $d\theta$, $d\phi = 0$ in~\Eq{(38.8)}, so that
\[
ds^{2} = \gamma\, dt^{2}.
\Tag{(42.1)}
\]
Accordingly the \emph{times} of vibration, of the differently placed atoms will be
inversely proportional to~$\sqrt{\gamma}$.

Our system of coordinates is a \emph{static system,} that is to say the $g_{\mu\nu}$ do not
change with the time. (An arbitrary coordinate\hyp{}system has not generally this
property; and further when we have to take account of two or more attracting
bodies, it is in most cases impossible to find a strictly static system of coordinates.)
Taking an observer at rest in the system $(r, \theta, \phi, t)$ a wave emitted
by one of the atoms will reach him at a certain time~$\delta t$ after it leaves the
atom; and owing to the static condition this time-lag remains constant for
subsequent waves. Consequently the waves are received at the same time\hyp{}periods
as they are emitted. We are therefore able to compare the time\hyp{}periods
$dt$ of the different atoms, by comparing the periods of the waves received from
them, and can verify experimentally their dependence on the value of~$\sqrt{\gamma}$ at
the place where they were emitted. Naturally the most hopeful test is the
comparison of the waves received from a solar and a terrestrial atom whose
periods should be in the ratio $1.00000212 : 1$. For wave-length $4000$~Å, this
amounts to a relative displacement of $0.0082$~Å of the respective spectral
lines. The verdict of experiment is not yet such as to secure universal assent;
but it is now distinctly more favourable to Einstein's theory than when \Title{Space,
Time and Gravitation} was written.

The quantity~$dt$ is merely an auxiliary quantity introduced through the
equation~\Eq{(38.8)} which defines it. The fact that it is carried to us unchanged
by light-waves is not of any physical interest, since $dt$~was \emph{defined} in such a
way that this must happen. The absolute quantity~$ds$, the interval of the
vibration, is not carried to us unchanged, but becomes gradually modified as
the waves take their course through the non\hyp{}Euclidean space-time. It is in
transmission through the solar system that the absolute difference is introduced
into the waves, which the experiment hopes to detect.

The argument refers to \emph{similar} atoms and the question remains whether,
for example, a hydrogen atom on the sun is truly similar to a hydrogen atom
on the earth. Strictly speaking it cannot be exactly similar because it is in a
different kind of space-time, in which it would be impossible to make a finite
structure exactly similar to one existing in the space-time near the earth. But
if the interval of vibration of the hydrogen atom is modified by the kind of
space-time in which it lies, the difference must be dependent on some invariant
of the space-time. The simplest invariant which differs at the sun and the
earth is the square of the length of the Riemann\hyp{}Christoffel tensor, viz.\
\[
B_{\mu\nu\sigma}^{\epsilon} B_{\epsilon}^{\mu\nu\sigma}.
\]
The value of this can be calculated from~\Eq{(38.8)} by the method used in that
section for calculating the~$G_{\mu\nu}$. The result is
\[
48\, \frac{m^{2}}{r^{6}}.
\]
By consideration of dimensions it seems clear that the proportionate change
of~$ds$ would be of the order
\[
\frac{\sigma^{4} m^{2}}{r^{6}},
\]
where $\sigma$~is the radius of the atom; there does not seem to be any other length
concerned. For a comparison of solar and terrestrial atoms this would be about
$10^{-100}$. In any case it seems impossible to construct from the invariants of
space-time a term which would compensate the predicted shift of the spectral
lines, which is proportional to~$m/r$.

\Section{43.}{Isotropic coordinates}
\index{Coordinate\hyp{}systems!isotropic}%
\index{Isotropic coordinates}%

We can transform the expression for the interval~\Eq{(38.8)} by making the
substitution
\[
r = \left(1 + \frac{m}{2r_{1}}\right)^{2} r_{1},
\Tag{(43.1)}
\]
so that
\begin{align*}
dr &= \left(1 - \frac{m^{2}}{4r_{1}^{2}}\right) dr_{1}, \\
\gamma &= \left(1 - \frac{m}{2r_{1}}\right)^{2}\bigg/\left(1 + \frac{m}{2r_{1}}\right)^{2}.
\end{align*}

Then \Eq{(38.8)}~becomes
\[
  ds^{2} = -(1 + m/2r_{1})^{4} (dr_{1} + r_{1}^{2}\, d\theta^2 + r_{1}^{2}\sin^{2}\theta\, d\phi^{2})
  + \frac{(1 - m/2r_{1})^{2}}{(1 + m/2r_{1})^{2}}\, dt^{2}.
  \Tag{(43.2)}
\]

The coordinates $(r_{1},\theta, \phi)$ are called \emph{isotropic} polar coordinates. The corresponding
isotropic rectangular coordinates are obtained by putting
\[
x = r_{1} \sin\theta \cos\phi,\quad
y = r_{1} \sin\theta \sin\phi,\quad
z = r_{1} \cos\theta,
\]
giving
\[
ds^{2} = -(1 + m/2r_{1})^{4} (dx^{2} + dy^{2} + dz^{2})
+ \frac{(1 - m/2r_{1})^{2}}{(1 + m/2r_{1})^{2}}\, dt^{2},
\Tag{(43.3)}
\]
with
\[
r_{1} = \sqrt{x^{2} + y^{2} + z^{2}}.
\]

This system has some advantages. For example, to obtain the motion of
a light-pulse we set $ds = 0$ in~\Eq{(43.3)}. This gives
\[
\left(\frac{dx}{dt}\right)^{2}
+ \left(\frac{dy}{dt}\right)^{2}
+ \left(\frac{dz}{dt}\right)^{2}
= \frac{(1 - m/2r_{1})^{2}}{(1 + m/2r_{1})^{6}}.
\]
At a distance~$r_{1}$ from the origin the velocity of light is accordingly
\index{Velocity of light!in sun's gravitational field}%
\[
\frac{(1 - m/2r_{1})}{(1 + m/2r_{1})^{3}}
\Tag{(43.4)}
\]
in all directions. For the original coordinates of~\Eq{(38.8)} the velocity of light is
not the same for the radial and transverse directions.

Again in the isotropic system the coordinate length ($\sqrt{x^{2} + y^{2} + z^{2}}$) of
a small rod which is rigid ($ds = \text{constant}$) does not alter when the orientation
of the rod is altered. This system of coordinates is naturally arrived at when
we partition space by rigid scales or by light\hyp{}triangulations in a small region,
e.g.\ in terrestrial measurements. Since the ultimate measurements involved
in any observation are carried out in a terrestrial laboratory we ought, strictly
speaking, always to employ the isotropic system which conforms to assumptions
made in those measurements\footnotemark.\footnotetext
  {But the terrestrial laboratory is falling freely towards the sun, and is therefore accelerated
  relatively to the coordinates $(x, y, z, t)$.}
But on the earth the quantity~$m/r$ is negligibly
small, so that the two systems coalesce with one another and with Euclidean
coordinates. Non\hyp{}Euclidean geometry is only required in the theoretical part
of the investigation---the laws of planetary motion and propagation of light
through regions where $m/r$~is not negligible; as soon as the light-waves have
been safely steered into the terrestrial observatory, the need for non\hyp{}Euclidean
geometry is at an end, and the difference between the isotropic and non\hyp{}isotropic
systems practically disappears.

In either system the forward velocity of light along any line is equal to
the backward velocity. Consequently the coordinate~$t$ conforms to the convention
(\SecRef{11}) that simultaneity may be determined by means of light\hyp{}signals.
If we have a clock at~$A$ and send a light\hyp{}signal at time~$t_{A}$ which reaches~$B$
and is immediately reflected so as to return to~$A$ at time~$t_{A}'$, the time of arrival
at~$B$ will be $\frac{1}{2}(t_{A} + t_{A}')$ just as in the special relativity theory. But the alternative
convention, that simultaneity can be determined by slow transport of
chronometers, breaks down when there is a gravitational field. This is evident
from \SecRef{42}, since the time-rate of a clock will depend on its position in the field.
In any case slow transport of a clock is unrealisable because of the acceleration
which all objects must submit to.

The isotropic system could have been found directly by seeking particular
solutions of Einstein's equations having the form~\Eq{(38.12)}, or
\[
ds^{2} = -e^{\lambda}\, dr^{2} - e^{\mu}(r^{2}\, d\theta^2 + r^{2}\sin^{2}\theta\, d\phi^{2}) + e^{\nu}\, dt^{2},
\]
where $\lambda$, $\mu$, $\nu$ are functions of~$r$. By the method of \SecRef{38}, we find
\[
\left.
\begin{aligned}
G_{11} &= \mu'' + \tfrac{1}{2}\nu'' + \frac{2}{r}\mu' - \frac{1}{r}\lambda' + \tfrac{1}{2}\mu'^{2} - \tfrac{1}{2}\lambda'\mu' - \tfrac{1}{4}\lambda'\nu' + \tfrac{1}{4}\nu'^{2}\\
G_{22} &= e^{\mu-\lambda} \bigl[1 + 2r\mu' + \tfrac{1}{2}r(\nu' - \lambda') + \tfrac{1}{2}r^{2}\mu''\\
  &\qquad\qquad+ \tfrac{1}{2}r^{2}\mu'(\mu' + \tfrac{1}{2}\nu' - \tfrac{1}{2}\lambda')\bigr] - 1\\
G_{33} &= G_{22} \sin^{2}\theta\\
G_{44} &= -e^{\nu-\lambda}\left[\tfrac{1}{2}\nu'' + \frac{1}{r}\nu' + \tfrac{1}{2}\nu'\mu' - \tfrac{1}{4}\lambda'\nu' + \tfrac{1}{4}\nu'^{2}\right]
\end{aligned}
\right\}
\Tag{(43.5)}
\]
The others are zero.

Owing to an identical relation between $G_{11}$, $G_{22}$ and~$G_{44}$, the vanishing of
this tensor gives only two equations to determine the three unknowns $\lambda$, $\mu$,~$\nu$.
There exists therefore an infinite series of particular solutions, differing
according to the third equation between $\lambda$, $\mu$, $\nu$ which is at our disposal. The
two solutions hitherto considered are obtained by taking $\mu = 0$, and $\lambda = \mu$,
respectively. The same series of solutions is obtained in a simpler way by
substituting arbitrary functions of~$r$ instead of~$r$ in~\Eq{(38.8)}.

The possibility of substituting any function of~$r$ for~$r$ without destroying
the spherical symmetry is obvious from the fact that a coordinate is merely
an identification\hyp{}number; but analytically this possibility is bound up with
the existence of an identical relation between $G_{11}$, $G_{22}$ and~$G_{44}$, which makes
the equations too few to determine a unique solution.

This introduces us to a theorem of great consequence in our later work.
\index{Identities satisfied by $G_{\mu\nu}$}%
If Einstein's ten equations $G_{\mu\nu} = 0$ were all independent, the ten~$g_{\mu\nu}$ would be
uniquely determined by them (the boundary conditions being specified). The
expression for~$ds^{2}$ would be unique and no transformation of coordinates would
be possible. Since we know that we can transform coordinates as we please,
there must exist identical relations between the ten~$G_{\mu\nu}$; and these will be
found in \SecRef{52}.

\Section{44.}{Problem of two bodies---Motion of the moon}
\index{Moon, motion of}%
\index{Problem!of two bodies}%
\index{Two bodies, problem of}%

The field described by the~$g_{\mu\nu}$ may be (artificially) divided into a \emph{field of
pure inertia} represented by the Galilean values, and a \emph{field of force} represented
by the deviations of the~$g_{\mu\nu}$ from the Galilean values. It is not possible
to superpose the fields of force due to two attracting particles; because the
sum of the two solutions will not satisfy $G_{\mu\nu} = 0$, these equations being non\hyp{}linear
in the~$g_{\mu\nu}$.

No solution of Einstein's equations has yet been found for a field with two
singularities or particles. The simplest case to be examined would be that of
two equal particles revolving in circular orbits round their centre of mass.
Apparently there should exist a statical solution with two equal singularities;
but the conditions at infinity would differ from those adopted for a single
particle since the axes corresponding to the static solution constitute what is
called a rotating system. The solution has not been found, and it is even
possible that no such statical solution exists. I do not think it has yet been
proved that two bodies can revolve without radiation of energy by gravitational
waves. In discussions of this radiation problem there is a tendency to beg the
question; it is not sufficient to constrain the particles to revolve uniformly,
then calculate the resulting gravitational waves, and verify that the radiation
of gravitational energy across an infinite sphere is zero. That shows that a
statical solution is not obviously inconsistent with itself, but does not demonstrate
its possibility.

The problem of two bodies on Einstein's theory remains an outstanding
challenge to mathematicians---like the problem of three bodies on Newton's
theory.

For practical purposes methods of approximation will suffice. We shall
consider the problem of the field due to the combined attractions of the earth
and sun, and apply it to find the modifications of the moon's orbit required by
the new law of gravitation. The problem has been treated in considerable
detail by de~Sitter\footnotemark.\footnotetext
  {\Title{Monthly Notices,} vol.~77, p.~155.}
We shall not here attempt a complete survey of the
problem; but we shall seek out the largest effects to be looked for in refined
observations. There are three sources of fresh perturbations:

(1) The sun's attraction is not accurately given by Newton's law, and the
solar perturbations of the moon's orbit will require corrections.

(2) Cross-terms between the sun's and the earth's fields of force will arise,
since these are not additive.

(3) The earth's field is altered and would \Foreign{inter alia} give rise to a motion
of the lunar perigee analogous to the motion of Mercury's perihelion. It is
easily calculated that this is far too small to be detected.

If $\Omega_{S}$, $\Omega_{E}$ are the Newtonian potentials of the sun and earth, the leading
terms of (1), (2), (3) will be relatively of order of magnitude
\[
\Omega_{S}^{2},\quad
\Omega_{S}\Omega_{E},\quad
\Omega_{E}.
\]
For the moon $\Omega_{S} = 750\, \Omega_{E}$. We may therefore confine attention to terms of
type~(1). If these prove to be too small to be detected, the others will presumably
be not worth pursuing.

We were able to work out the planetary orbits from Einstein's law independently
of the Newtonian theory; but in the problem of the moon's motion
we must concentrate attention on the difference between Einstein's and Newton's
formulae if we are to avoid repeating the whole labour of the classical
lunar theory. In order to make this comparison we transform \Eq{(39.31)} and
\Eq{(39.32)} so that $t$~is used as the independent variable.
\begin{align*}
  \frac{d^{2}}{ds^{2}}
  &= \left(\frac{dt}{ds}\right)^{2} \frac{d^{2}}{dt^{2}} + \frac{dt}{ds}\, \frac{d}{dt} \left(\frac{dt}{ds}\right) \frac{d}{dt} \\
  &= \left(\frac{dt}{ds}\right)^{2} \left(\frac{d^{2}}{dt^{2}} + \lambda'\, \frac{dr}{dt}\, \frac{d}{dt}\right)
  \qquad\text{by \Eq{(39.42)}.}
\end{align*}
Hence the equations \Eq{(39.31)} and \Eq{(39.32)} become
\begin{gather*}
  \frac{d^{2}r}{dt^{2}}
  + \tfrac{3}{2}\lambda'\left(\frac{dr}{dt}\right)^{2}
  - re^{-\lambda} \left(\frac{d\phi}{dt}\right)^{2}
  + \tfrac{1}{2} e^{2\nu} \nu' = 0,\displaybreak[0] \\
  \frac{d^{2}\phi}{dt^{2}}
  + \lambda'\, \frac{dr}{dt}\, \frac{d\phi}{dt}
  + \frac{2}{r}\, \frac{dr}{dt}\, \frac{d\phi}{dt} = 0.
\end{gather*}
Whence
\[
\left.
\begin{aligned}
  \frac{d^{2}r}{dt^{2}} - r\left(\frac{d\phi}{dt}\right)^{2} + \frac{m}{r^{2}} &= R\\
  r\left(\frac{d^{2}\phi}{dt^{2}} + \frac{2}{r}\, \frac{dr}{dt}\, \frac{d\phi}{dt}\right) &= \Phi
\end{aligned}
\right\}
\Tag{(44.1)}
\]
where
\[
\left.
\begin{aligned}
  R &= -\tfrac{3}{2}\lambda' u^2 - \frac{2m}{r^{2}}\, v^2 + \frac{2m^{2}}{r^{3}}\\
  \Phi &= -\lambda' u v
\end{aligned}
\right\}
\Tag{(44.21)}
\]
and
\[
u = dr/dt,\quad v = r\, d\phi/dt.
\]

Equations \Eq{(44.1)} show that $R$ and~$\Phi$ are the radial and transverse perturbing
forces which Einstein's theory adds to the classical dynamics. To a
sufficient approximation $\lambda' = -2m/r^{2}$, so that
\[
\left.
\begin{aligned}
  R &= \frac{m}{r^{2}} (3u^2 - 2v^2) + \frac{2m^{2}}{r^{3}}\\
  \Phi &= \frac{m}{r^{2}} \cdot 2 u v
\end{aligned}
\right\}
\Tag{(44.22)}
\]

In three\hyp{}dimensional problems the perturbing forces become
\[
\left.
\begin{aligned}
  R &= \frac{m}{r^{2}} (3u^2 - 2v^2 - 2w^{2}) + \frac{2m^{2}}{r^{3}}\\
  \Phi &= \frac{m}{r^{2}} \cdot 2 u v\\
  Z &= \frac{m}{r^{2}} \cdot 2 u w
\end{aligned}
\right\}
\Tag{(44.23)}
\]

It must be pointed out that these perturbing forces are Einstein's corrections
to the law of central force~$m/r^{2}$, where $r$~is \emph{the coordinate used in our
previous work}. Whether these forces represent the actual differences between
Einstein's and Newton's laws depends on what Newton's~$r$ is supposed to
signify. De~Sitter, making a slightly different choice of~$r$, obtains different
expressions for $R$,~$\Phi$\footnotemark.\footnotetext
  {\Title{Monthly Notices,} vol.~76, p.~723, equations~(53).}
One cannot say that one set of perturbing forces
rather than the other represents the difference from the older theory, because
the older theory was not sufficiently explicit. The classical lunar theory
has been worked out on the basis of the law~$m/r^{2}$; the ambiguous quantity~$r$
occurs in the results, and according as we have assigned to it one meaning or
another, so we shall have to apply different corrections to those results. But
the final comparison with observation does not depend on the choice of the
intermediary quantity~$r$.

Take fixed rectangular axes referred to the ecliptic with the sun as origin,
and let \\
\Indent $(a, 0, 0)$ be the coordinates of the earth at the instant considered, \\
\Indent $(x, y, z)$ the coordinates of the moon relative to the earth.

Taking the earth's orbit to be circular and treating the mass of the moon
as infinitesimal, the earth's velocity will be $(0, v, 0)$, where $v^2 = m/a$.

To find the difference of the forces $R$, $\Phi$, $Z$ on the moon and on the earth,
we differentiate~\Eq{(44.23)} and set
\[
\delta r = x,\quad
\delta(u, v, w) = (dx/dt, dy/dt, dz/dt),
\]
and, after the differentiation,
\[
r = a,\quad
(u, v, w) = (0, v, 0).
\]
The result will give the perturbing forces on the moon's motion relative to
the earth, viz.\
\[
\left.
\begin{aligned}
  \delta R
  &= \begin{aligned}[t]
    X &= \frac{4mx}{a^{3}}\, v^2 - \frac{6m^{2}x}{a^{4}} - \frac{4m}{a^2}\, v\, \frac{dy}{dt} \\
    &= -\frac{2m^{2}x}{a^{4}} - \frac{4m}{a^2}\, v\, \frac{dy}{dt}
  \end{aligned} \\
  \delta\Phi &= Y = \frac{2m}{a^2}\, v\, \frac{dx}{dt}\\
  Z &= 0
\end{aligned}
\right\}
\Tag{(44.3)}
\]

We shall omit the term $-2m^{2}x/a^{4}$ in~$X$. It can be verified that it gives
no important observable effects. It produces only an apparent distortion of
the orbit attributable to our use of non\hyp{}isotropic coordinates (\SecRef{43}). Transforming
to new axes $(\xi, \eta)$ rotated through an angle~$\theta$ with respect to $(x, y)$
the remaining forces become
\[
\left.
\begin{aligned}
  \Xi &= \frac{m}{a^2} v \left(-2\cos\theta \sin\theta\, \frac{d\xi}{dt} - (4\cos^{2}\theta + 2\sin^{2}\theta)\, \frac{d\eta}{dt}\right)\\
  \Eta &= \frac{m}{a^2} v \left(\Neg2\cos\theta \sin\theta\, \frac{d\eta}{dt} + (4\sin^{2}\theta + 2\cos^{2}\theta)\, \frac{d\xi}{dt}\right)
\end{aligned}
\right\}
\Tag{(44.4)}
\]

We keep the axes $(\xi, \eta)$ permanently fixed; the angle~$\theta$ which gives the
direction of the sun (the old axis of~$x$) will change uniformly, and in the long
run take all values with equal frequency independently of the moon's position
in its orbit. We can only hope to observe the secular effects of the small forces
$\Xi$,~$\Eta$, accumulated through a long period of time. Accordingly, averaging the
trigonometrical functions, the secular terms are
\[
\left.
\begin{aligned}
  \Xi &= -3\frac{m}{a^2}\, v\, \frac{d\eta}{dt}
  = -2\omega\, \frac{d\eta}{dt}\\
  \Eta &= \Neg 3\frac{m}{a^2}\, v\, \frac{d\xi}{dt}
  = \Neg2\omega\, \frac{d\xi}{dt}
\end{aligned}
\right\}
\Tag{(44.5)}
\]
where
\[
\omega = \tfrac{3}{2}mv/a^2.
\Tag{(44.6)}
\]

If $(F_{\xi}, F_{\eta})$ is the Newtonian force, the equations of motion including these
secular perturbing forces will be
\[
\frac{d^{2}\xi}{dt^{2}} + 2\omega\, \frac{d\eta}{dt} = F_{\xi},\quad
\frac{d^{2}\eta}{dt^{2}} - 2\omega\, \frac{d\xi}{dt} = F_{\eta}.
\Tag{(44.7)}
\]

It is easily seen that $\omega$~is a very small quantity, so that $\omega^2$~is negligible.
The equations~\Eq{(44.7)} are then recognised as the Newtonian equations referred
to axes rotating with angular velocity~$-\omega$. Thus if we take the Newtonian
orbit and give it an angular velocity~$+\omega$, the result will be the solution of~\Eq{(44.7)}.
The leading correction to the lunar theory obtained from Einstein's
equations is a precessional effect, indicating that the classical results refer to
a frame of reference advancing with angular velocity~$\omega$ compared with the
general inertial frame of the solar system.

From this cause the moon's node and perigee will advance with velocity~$\omega$.
If $\Omega$~is the earth's angular velocity
\[
\frac{\omega}{\Omega} = \frac{3}{2}\, \frac{m}{a} = \tfrac{3}{2} \cdot 10^{-8}.
\]
Hence the advance of perigee and node in a century is
\[
3\pi \cdot 10^{-6} \text{radians} = 1''.94.
\]

We may notice the very simple theoretical relation that Einstein's corrections
cause an advance of the moon's perigee which is \emph{one half} the advance
of the earth's perihelion.

Neither the lunar theory nor the observations are as yet carried quite far
enough to take account of this small effect; but it is only a little below the
limit of detection. The result agrees with de~Sitter's value except in the second
decimal place which is only approximate.

There are well-known irregular fluctuations in the moon's longitude which
attain rather large values; but it is generally considered that these are not
of a type which can be explained by any amendment of gravitational theory
and their origin must be looked for in other directions. At any rate Einstein's
theory throws no light on them.

The advance of $1''.94$ per century has not exclusive reference to the
\index{Absolute change!rotation}%
\index{Foucault's pendulum}%
\index{Inertial frame, precession of}%
\index{Perigee, advance of}%
\index{Precession of inertial frame}%
\index{Rotation, absolute}%
moon; in fact the elements of the moon's orbit do not appear in~\Eq{(44.6)}. It
represents a property of the space surrounding the earth---a precession of the
inertial frame in this region relative to the general inertial frame of the sidereal
system. If the earth's rotation could be accurately measured by Foucault's
pendulum or by gyrostatic experiments, the result would differ from the
rotation relative to the fixed stars by this amount. This result seems to have
been first pointed out by J.~A. Schouten. One of the difficulties most often
urged against the relativity theory is that the earth's rotation relative to the
mean of the fixed stars appears to be an absolute quantity determinable by
dynamical experiments on the earth\footnotemark;\footnotetext
  {\Title{Space, Time and Gravitation,} p.~152.}
it is therefore of interest to find that
these two rotations are not exactly the same, and the earth's rotation relative
to the stellar system (supposed to agree with the general inertial frame of the
universe) \emph{cannot be determined except by astronomical observations}.

The argument of the relativist is that the observed effect on Foucault's
pendulum can be accounted for indifferently by a field of force or by rotation.
The anti\hyp{}relativist replies that the field of force is clearly a mathematical
fiction, and the only possible \emph{physical cause} must be absolute rotation. It is
pointed out to him that nothing essential is gained by choosing the so\hyp{}called
non\hyp{}rotating axes, because in any case the main part of the field of force
remains, viz.\ terrestrial gravitation. He retorts that with his non\hyp{}rotating
axes he has succeeded in making the field of force vanish at infinity, so that
the residuum is accounted for as a local disturbance by the earth; whereas,
if axes fixed in the earth are admitted, the corresponding field of force becomes
larger and larger as we recede from the earth, so that the relativist demands
enormous forces in distant parts for which no physical cause can be assigned.
Suppose, however, that the earth's rotation were much slower than it is now,
and that Foucault's experiment had indicated a rotation of only $-1''.94$ per
century. Our two disputants on the cloud-bound planet would no doubt carry
on a long argument as to whether this was essentially an absolute rotation of
the earth in space, the irony of the situation being that the earth all the while
was non\hyp{}rotating in the anti\hyp{}relativist's sense, and the proposed transformation
to allow for the Foucault rotation would actually have the effect of introducing
the enormous field of force in distant parts of space which was so much objected
to. When the origin of the $1''.94$ has been traced as in the foregoing investigation,
the anti\hyp{}relativist who has been arguing that the observed effect is
definitely caused by rotation, must change his position and maintain that it
is definitely due to a gravitational perturbation exerted by the sun on Foucault's
pendulum; the relativist holds to his view that the two causes are not
distinguishable.

\Section{45.}{Solution for a particle in a curved world}
\index{Particle!gravitational field of}%

In later work Einstein has adopted the more general equations~\Eq{(37.4)}
\[
G_{\mu\nu} = \alpha g_{\mu\nu}.
\Tag{(45.1)}
\]
In this case we must modify~\Eq{(38.61)}, etc.\ by inserting $\alpha g_{\mu\nu}$  on the right. We
then obtain
\begin{gather*}
  \tfrac{1}{2}\nu'' - \tfrac{1}{4}\lambda'\nu' + \tfrac{1}{4}\nu'^{2} - \lambda'/r = -\alpha e^{\lambda},
  \Tag{(45.21)}\displaybreak[0] \\
  e^{-\lambda}\bigl(1 + \tfrac{1}{2}r(\nu' - \lambda')\bigr) - 1 = -\alpha r^{2},
  \Tag{(45.22)}\displaybreak[0] \\
  e^{\nu-\lambda}(-\tfrac{1}{2}\nu'' + \tfrac{1}{4}\lambda'\nu' - \tfrac{1}{4}\nu'^{2} - \nu'/r) = \alpha e^{\nu}.
  \Tag{(45.23)}
\end{gather*}
From \Eq{(45.21)} and \Eq{(45.23)}, $\lambda' = -\nu'$, so that we may take $\lambda = -\nu$. An additive
constant would merely amount to an alteration of the unit of time. Equation~\Eq{(45.22)}
then becomes
\[
e^{\nu} (1 + r\nu') = 1 - \alpha r^{2}.
\]
Let $e^{\nu} = \gamma$; then
\[
\gamma + r\gamma' = 1 - \alpha r^{2}
\]
which on integration gives
\[
\gamma = 1 - \frac{2m}{r} - \tfrac{1}{3}\alpha r^{2}.
\Tag{(45.3)}
\]
The only change is the substitution of this new value of~$\gamma$ in~\Eq{(38.8)}.

By recalculating the few steps from \Eq{(39.44)} to \Eq{(39.61)} we obtain the
equation of the orbit
\[
\frac{d^{2}u}{d\phi^{2}} + u = \frac{m}{h^{2}} + 3mu^2 - \frac{1}{3}\, \frac{\alpha}{h^{2}}\, u^{-3}.
\Tag{(45.4)}
\]
The effect of the new term in~$\alpha$ is to give an additional motion of perihelion
\index{Perihelion!in curved world}%
\[
\frac{\delta\varpi}{\phi} = \frac{1}{2}\, \frac{\alpha h^{6}}{m^{4}}
= \frac{1}{2}\, \frac{\alpha a^{3}}{m}(1 - e^{2})^{3}.
\Tag{(45.5)}
\]

At a place where $\gamma$~vanishes there is an impassable barrier, since any change~$dr$
corresponds to an infinite distance~$i\, ds$ surveyed by measuring\hyp{}rods. The
two roots of the quadratic~\Eq{(45.3)} are approximately
\[
r = 2m\quad\text{and}\quad
r = \sqrt{3/\alpha}.
\]
The first root would represent the boundary of the particle---if a genuine particle
\index{Horizon of world}%
could exist---and give it the appearance of impenetrability. The second
barrier is at a very great distance and may be described as the horizon of the
world.

It is clear that the latter barrier (or illusion of a barrier) cannot be at a
less distance than the most remote celestial objects observed, say $10^{25}~\text{cm}$.
This makes $\alpha$ less than $10^{-50}~(\text{cm.})^{-2}$. Inserting this value in~\Eq{(45.5)} we find
that the additional motion of perihelion will be well below the limit of observational
detection for all planets in the solar system\footnotemark.\footnotetext
  {This could scarcely have been asserted a few years ago, when it was not known that the
  stars extended much beyond $1000$ parsecs distance. A horizon distant $700$ parsecs corresponds to
  a centennial motion of about~$1''$ in the earth's perihelion, and greater motion for the more
  distant planets in direct proportion to their periods.}

If in~\Eq{(45.3)} we set $m = 0$, we abolish the particle at the origin and obtain
the solution for an entirely empty world
\[
ds^{2} = -(1 - \tfrac{1}{3}\alpha r^{2})^{-1}\, dr^{2}
- r^{2} d\theta^2 - r^{2} \sin^{2}\theta\, d\phi^{2}
+ (1 - \tfrac{1}{3}\alpha r^{2})\, dt^{2}.
\Tag{(45.6)}
\]
This will be further discussed in Chapter~\ChapNum{V}\@.

\Section{46.}{Transition to continuous matter}
\index{Continuous matter, gravitation in}%
\index{Einstein's law of gravitation!in continuous matter}%

In the Newtonian theory of attractions the potential~$\Omega$ in empty space
satisfies the equation
\[
\nabla^2\Omega = 0,
\]
of which the elementary solution is $\Omega = m/r$; then by a well-known procedure
we are able to deduce that in continuous matter
\[
\nabla^2\Omega = -4\pi\rho.
\Tag{(46.1)}
\]

We can apply the same principle to Einstein's potentials~$g_{\mu\nu}$, which in
empty space satisfy the equations $G_{\mu\nu} = 0$. The elementary solution has been
found, and it remains to deduce the modification of the equations in continuous
matter. The logical aspects of the transition from discrete particles to continuous
density need not be discussed here, since they are the same for both
theories.

When the square of~$m/r$ is neglected, the isotropic solution~\Eq{(43.3)} for a
particle continually at rest becomes\footnote
  {This approximation though sufficient for the present purpose is not good enough for a
  discussion of the perihelion of Mercury. The term in~$m^{2}/r^{2}$ in the coefficient of~$dt^{2}$ would have to
  be retained.}
\[
ds^{2} = -\left(1 + \frac{2m}{r}\right) (dx^{2} + dy^{2} + dz^{2}) + \left(1 - \frac{2m}{r}\right) dt^{2}.
\Tag{(46.15)}
\]
The particle need not be at the origin provided that $r$~is the distance from
the particle to the point considered.

Summing the fields of force of a number of particles, we obtain
\[
ds^{2} = -(1 + 2\Omega) (dx^{2} + dy^{2} + dz^{2}) + (1 - 2\Omega) dt^{2},
\Tag{(46.2)}
\]
where
\[
\Omega = \sum \frac{m}{r}
= \text{Newtonian potential at the point considered.}
\]
The inaccuracy in neglecting the interference of the fields of the particles is
of the same order as that due to the neglect of~$m^{2}/r^{2}$, if the number of particles
is not unduly large.

Now calculate the~$G_{\mu\nu}$ for the expression~\Eq{(46.2)}. We have
%[** TN: Not broken in the original]
\begin{align*}
  G_{\mu\nu} &= g^{\sigma\rho} B_{\mu\nu\sigma\rho} \\
  &= \tfrac{1}{2}g^{\sigma\rho} \left(
  \frac{\dd^{2} g_{\mu\nu}}{\dd x_{\rho}\, \dd x_{\sigma}}
  + \frac{\dd^{2} g_{\rho\sigma}}{\dd x_{\mu}\, \dd x_{\nu}}
  - \frac{\dd^{2} g_{\mu\sigma}}{\dd x_{\rho}\, \dd x_{\nu}}
  - \frac{\dd^{2} g_{\rho\nu}}{\dd x_{\mu}\, \dd x_{\sigma}}\right)
  \Tag{(46.3)}
\end{align*}
by~\Eq{(34.5)}. The non\hyp{}linear terms are left out because they would involve~$\Omega^2$
which is of the order~$(m/r)^{2}$ already neglected.

The only terms which survive are those in which the $g$'s have like suffixes.
Consider the last three terms in the bracket; for $G_{11}$ they become
\[
\frac{1}{2}\biggl(g^{11}\, \frac{\dd^{2} g_{11}}{\dd x_{1}^{2}}
+ g^{22}\, \frac{\dd^{2} g_{22}}{\dd x_{1}^{2}}
+ g^{33}\, \frac{\dd^{2} g_{33}}{\dd x_{1}^{2}}
+ g^{44}\, \frac{\dd^{2} g_{44}}{\dd x_{1}^{2}}
- g^{11}\, \frac{\dd^{2} g_{11}}{\dd x_{1}^{2}}
- g^{11}\, \frac{\dd^{2} g_{11}}{\dd x_{1}^{2}}\biggr).
\]
Substituting for the $g$'s from~\Eq{(46.2)} we find that the result vanishes (neglecting~$\Omega^2$).
For $G_{44}$ the result vanishes for a different reason, viz.\ because $\Omega$~does not
contain $x_{4}$ ($= t$). Hence
\[
G_{\mu\nu} = \tfrac{1}{2} g^{\sigma\rho}\, \frac{\dd^{2} g_{\mu\nu}}{\dd x_{\sigma}\, \dd x_{\rho}}
= \tfrac{1}{2} \Wave g_{\mu\nu}\quad\text{as in \Eq{(30.65)}.}
\Tag{(46.4)}
\]
Since time is not involved
\begin{gather*}
  \Wave = - \nabla^2, \\
  \begin{aligned}
    G_{11},\ G_{22},\ G_{33},\ G_{44}
    &= -\tfrac{1}{2} \nabla^2(g_{11}, g_{22}, g_{33}, g_{44}) \\
    &= \nabla^2\Omega \qquad\text{by \Eq{(46.2)}.}
  \end{aligned}
\end{gather*}
Hence, making at this point the transition to continuous matter,
\[
G_{11},\ G_{22},\ G_{33},\ G_{44} = -4\pi\rho
\qquad\text{by \Eq{(46.1)}.}
\Tag{(46.5)}
\]

Also
\begin{align*}
  G = g^{\mu\nu} G_{\mu\nu}
  &= -G_{11} - G_{22} - G_{33} + G_{44} \\
  &= 8\pi\rho
  \end{align*}
to the same approximation.

Consider the tensor defined by
\index{T@$T_{\mu\nu}$ (energy\hyp{}tensor)}%
\[
-8\pi T_{\mu\nu} = G_{\mu\nu} - \tfrac{1}{2} g_{\mu\nu} G.
\Tag{(46.6)}
\]
We readily find
\[
T_{\mu\nu}  = 0,\quad\text{except $T_{44} = \rho$,}
\]
and raising the suffixes
\[
T^{\mu\nu} = 0,\quad\text{except $T^{44} = \rho$,}
\Tag{(46.7)}
\]
since the $g^{\mu\nu}$~are Galilean to the order of approximation required.

Consider the expression
\[
\rho_{0}\, \frac{dx_{\mu}}{ds}\,\frac{dx_{\nu}}{ds},
\]
where $dx_{\mu}/ds$~refers to the motion of the matter, and $\rho_{0}$~is the proper\hyp{}density
(an invariant). The matter is at rest in the coordinates hitherto used, and
consequently
\[
\frac{dx_{1}}{ds},\
\frac{dx_{2}}{ds},\
\frac{dx_{3}}{ds} = 0,\quad
\frac{dx_{4}}{ds} = 1,
\]
so that all components of the expression vanish, except the component $\mu$,~$\nu = 4$
which is equal to~$\rho_{0}$. Accordingly in these coordinates
\[
T^{\mu\nu} = \rho_{0}\, \frac{dx_{\mu}}{ds}\,\frac{dx_{\nu}}{ds},
\Tag{(46.8)}
\]
since the density~$\rho$ in~\Eq{(46.7)} is clearly the proper\hyp{}density.

Now \Eq{(46.8)}~is a tensor equation\footnotemark,\footnotetext
  {When an equation is stated to be a tensor equation, the reader is expected to verify that the
  covariant dimensions of both sides are the same.}
and since it has been verified for one set
of coordinates it is true for all coordinate\hyp{}systems. Equations \Eq{(46.6)} and \Eq{(46.8)}
together give the extension of Einstein's law of gravitation for a region containing
continuous matter of proper\hyp{}density~$\rho_{0}$ and velocity~$dx_{\mu}/ds$.

The question remains whether the neglect of~$m^{2}$ causes any inaccuracy in
these equations. In passing to continuous matter we diminish~$m$ for each
particle indefinitely, but increase the number of particles in a given volume.
To avoid increasing the number of particles we may diminish the volume, so
that the formulae~\Eq{(46.5)} will be true for the limiting case of a point inside a
very small portion of continuous matter. Will the addition of surrounding
matter in large quantities make any difference? This can contribute nothing
directly to the tensor~$G_{\mu\nu}$, since so far as this surrounding matter is concerned
the point is in empty space; but Einstein's equations are non\hyp{}linear and we
must consider the possible cross-terms.

Draw a small sphere surrounding the point~$P$ which is being considered.
Let $g_{\mu\nu} = \delta_{\mu\nu} + h_{\mu\nu} + h_{\mu\nu}'$, where $\delta_{\mu\nu}$~represents the Galilean values, and $h_{\mu\nu}$~and
$h_{\mu\nu}'$ represent the fields of force contributed independently by the matter internal
to and external to the sphere. By \SecRef{36} we can choose coordinates such
that at~$P$ $h_{\mu\nu}'$~and its first derivatives vanish; and by the symmetry of the
sphere the first derivatives of~$h_{\mu\nu}$ vanish, whilst $h_{\mu\nu}$~itself tends to zero for an
infinitely small sphere. Hence the cross-terms which are of the form
\[
h_{\sigma\tau}'\, \frac{\dd h_{\mu\nu}}{\dd x_{\lambda}\, \dd x_{\rho}},\
\frac{\dd h_{\sigma\tau}'}{\dd x_{\lambda}}\, \frac{\dd h_{\mu\nu}}{\dd x_{\rho}},\
\text{and }
h_{\sigma\tau}\, \frac{\dd h_{\mu\nu}'}{\dd x_{\lambda}\, \dd x_{\rho}}
\]
will all vanish at~$P$. Accordingly with these limitations there are no cross-terms,
and the sum of the two solutions $h_{\mu\nu}$ and $h_{\mu\nu}'$ is also a solution of the
accurate equations. Hence the values~\Eq{(46.5)} remain true. It will be seen that
the limitation is that the coordinates must be ``natural coordinates'' at the
point~$P$. We have already paid heed to this in taking~$\rho$ to be the proper\hyp{}density.

We have assumed that the matter at~$P$ is not accelerated with respect to
these natural axes at~$P$. (The original particles had to be \emph{continually} at rest,
otherwise the solution~\Eq{(46.15)} does not apply.) If it were accelerated there
would have to be a stress causing the acceleration. We shall find later that
a stress contributes additional terms to the~$G_{\mu\nu}$. The formulae~\Eq{(46.5)} apply
only strictly when there is no stress and the continuous medium is specified
by one variable only, viz.\ the density.

The reader may feel that there is still some doubt as to the rigour of this
justification of the neglect of~$m^{2}$\footnotemark.\footnotetext
  {To illustrate the difficulty, what exactly does $\rho_{0}$ mean, assuming that it is not defined by
  \Eq{(46.6)} and~\Eq{(46.7)}? If the particles do not interfere with each other's fields, $\rho_{0}$~is $\sum m$~per unit
  volume; but if we take account of the interference, $m$~is undefined---it is the constant of integration
  of an equation which does not apply. Mathematically, we cannot say what $m$~would have
  been if the other particles had been removed; the question is nonsensical. Physically we could
  no doubt say what would have been the masses of the atoms if widely separated from one another,
  and compare them with the gravitational power of the atoms under actual conditions; but that
  involves laws of atomic structure which are quite outside the scope of the argument.}
Lest he attach too great importance to the
matter, we may state at once that the subsequent developments will not be
based on this investigation. In the next chapter we shall arrive at the same
formulae by a different line of argument, and proceed in the reverse direction
from the laws of continuous matter to the particular case of an isolated
particle.

The equation~\Eq{(46.2)} is a useful expression for the gravitational field due
\index{Stress\hyp{}system!gravitational field due to}%
to a static distribution of mass. It is only a first approximation correct to the
order~$m/r$, but \emph{no second approximation exists} except in the case of a solitary
particle. This is because when more than one particle is present accelerations
necessarily occur, so that there cannot be an exact solution of Einstein's
equations corresponding to a number of particles continually at rest. It follows
that any constraint which could keep them at rest must necessarily be of such
a nature as to contribute a gravitational field on its own account.

It will be useful to give the values of $G_{\mu\nu} - \frac{1}{2} g_{\mu\nu} G$ corresponding to the
symmetrical formula for the interval~\Eq{(38.2)}. By varying $\lambda$~and $\nu$ this can represent
any distribution of continuous matter with spherical symmetry. We have
\[
G = -e^{-\lambda} \bigl(\nu'' - \tfrac{1}{2}\lambda'\nu' + \tfrac{1}{2}\nu'^{2} + 2(\nu' - \lambda')/r + 2(1 - e^{\lambda})/r^{2}\bigr)
\]
\[\left.
\begin{aligned}
  &G_{11} - \tfrac{1}{2} g_{11} G
  = -\nu'/r - (1 - e^{\lambda})/r^{2}\\
  &G_{22} - \tfrac{1}{2} g_{22} G
 = -r^{2} e^{-\lambda} \bigl(\tfrac{1}{2}\nu'' - \tfrac{1}{4}\nu'\lambda' + \tfrac{1}{4}\nu'^{2} + \tfrac{1}{2}(\nu' - \lambda')/r\bigr)\\
  &G_{33} - \tfrac{1}{2} g_{33} G
 = -r^{2}\sin^{2}\theta\, e^{-\lambda} \bigl(\tfrac{1}{2}\nu'' - \tfrac{1}{4}\nu'\lambda' + \tfrac{1}{4}\nu'^{2} + \tfrac{1}{2}(\nu' - \lambda')/r\bigr)\\
  &G_{44} - \tfrac{1}{2} g_{44} G
  = \Neg e^{\nu-\lambda} \bigl(-\lambda'/r + (1 - e^{\lambda})/r^{2}\bigr)
\end{aligned}
\right\}
\Tag{(46.9)}
\]

\Section{47.}{Experiment and deductive theory}
\index{Postulates, list of}%

So far as I am aware, the following is a complete list of the postulates
which have been introduced into our mathematical theory up to the present
stage:

1. The fundamental hypothesis of \SecRef{1}.

2. The interval depends on a quadratic function of four coordinate\hyp{}differences
(\SecRef{2}).

3. The path of a freely moving particle is in all circumstances a geodesic
(\SecRef{15}).

4. The track of a light-wave is a geodesic with $ds = 0$ (\SecRef{15}).

5. The law of gravitation for empty space is $G_{\mu\nu} = 0$, or more probably
$G_{\mu\nu} = \lambda g_{\mu\nu}$, where $\lambda$~is a very small constant (\SecRef{37}).

No.~4 includes the identification of the velocity of light with the fundamental
velocity, which was originally introduced as a separate postulate in \SecRef{6}.

In the mathematical theory we have two objects before us---to examine
how we may test the truth of these postulates, and to discover how the laws
which they express originate in the structure of the world. We cannot neglect
either of these aims; and perhaps an ideal logical discussion would be divided
into two parts, the one showing the gradual ascent from experimental evidence
to the finally adopted specification of the structure of the world, the other
starting with this specification and deducing all observational phenomena.
The latter part is specially attractive to the mathematician for the proof may
be made rigorous; whereas at each stage in the ascent some new inference or
generalisation is introduced which, however plausible, can scarcely be considered
incontrovertible. We can show that a certain structure will explain
all the phenomena; we cannot show that nothing else will.

We may put to the experiments three questions in \Foreign{crescendo}. Do they
\index{Deductive theory and experiment}%
\index{Experiment and deductive theory}%
\index{Inductive theory}%
verify? Do they suggest? Do they (within certain limitations) compel the
laws we adopt? It is when the last question is put that the difficulty arises
for there are always limitations which will embarrass the mathematician who
wishes to keep strictly to rigorous inference. What, for example, does experiment
enable us to assert with regard to the gravitational field of a particle
(the other four postulates being granted)? Firstly, we are probably justified
in assuming that the interval can be expressed in the form~\Eq{(38.2)}, and experiment
shows that $\lambda$~and $\nu$ tend to zero at great distances. Provided that $e^{\lambda}$ and
$e^{\nu}$ are simple functions it will be possible to expand the coefficients in the form
%[** TN: Not broken in the original]
\begin{multline*}
  ds^{2} = -\biggl(1 + \frac{a_{1}}{r} + \frac{a_{2}}{r^{2}} + \cdots\biggr)^{-1} dr^{2}
  - r^{2}\, d\theta^2 - r^{2}\sin^{2}\theta\, d\phi^{2} \\
  + \biggl(1 + \frac{b_{1}}{r} + \frac{b_{2}}{r^{2}} + \frac{b_{3}}{r^{3}} + \cdots\biggr)^{-1} dt^{2}.
\end{multline*}
Now reference to \SecRefs{39}, \SecNum{40}, \SecNum{41} enables us to decide the following points: \\
\Indent (1) The Newtonian law of gravitation shows that $b_{1} = -2m$. \\
\Indent (2) The observed deflection of light then shows that $a_{1} = -2m$. \\
\Indent (3) The motion of perihelion of Mercury then shows that $b_{2} = 0$. \\
The last two coefficients are not determined experimentally with any high
accuracy; and we have no experimental knowledge of the higher coefficients.
If the higher coefficients are zero we can proceed to deduce that this field
satisfies $G_{\mu\nu} = 0$.

If small concessions are made, the case for the law $G_{\mu\nu} = 0$ can be
strengthened. Thus if only one linear constant~$m$ is involved in the specification
of the field, $b_{2}$~must contain~$m^{3}$, and the corresponding term is of order~$(m/r)^{3}$,
an extremely small quantity. Whatever the higher coefficients may
be, $G_{\mu\nu}$~will then vanish to a very high order of approximation.

Turning to the other object of our inquiry, we have yet to explain how
these five laws originate in the structure of the world. In the next chapter
we shall be concerned mainly with Nos.~3 and~5, which are not independent
of one another. They will be replaced by a broader principle which contains
them both and is of a more axiomatic character. No.~4 will be traced to its
origin in the electromagnetic theory of Chapter~\ChapNum{VI}\@. Finally a synthesis of
these together with Nos.~1 and~2 will be attempted in the closing chapter.

The following forward references will enable the reader to trace exactly
what becomes of these postulates in the subsequent advance towards more
primitive conceptions:

Nos.~1 and~2 are not further considered until \SecRef{97}.

No.~3 is obtained directly from the law of gravitation in \SecRef{56}.

No.~4 is obtained from the electromagnetic equations in \SecRef{74}. These are
traced to their origin in \SecRef{96}.

No.~5 is obtained from the principle of identification in \SecRef{54}, and more
completely from the principle of measurement in \SecRef{66}. The possibility of
alternative laws is discussed in \SecRef{62}.

In the last century the ideal explanation of the phenomena of nature consisted
\index{Explanation of phenomena, ideal}%
in the construction of a mechanical model, which would act in the way
observed. Whatever may be the practical helpfulness of a model, it is no
longer recognised as contributing in any way to an ultimate explanation. A
little later, the standpoint was reached that on carrying the analysis as far as
possible we must ultimately come to a set of differential equations of which
further explanation is impossible. We can then trace the \Foreign{modus operandi,} but
as regards ultimate causes we have to confess that ``things happen so, because
the world was made in that way.'' But in the kinetic theory of gases and in
thermodynamics we have laws which can be explained much more satisfactorily.
The principal laws of gases hold, not because a gas is made ``that way,'' but
because it is made ``just anyhow.'' This is perhaps not to be taken quite
literally; but if we could see that there was the same inevitability in Maxwell's
laws and in the law of gravitation that there is in the laws of gases, we
should have reached an explanation far more complete than an ultimate arbitrary
differential equation. This suggests striving for an ideal---to show, not
that the laws of nature come from a special construction of the ultimate basis
of everything, but that the same laws of nature would prevail for the widest
possible variety of structure of that basis. The complete ideal is probably
unattainable and certainly unattained; nevertheless we shall be influenced
by it in our discussion, and it appears that considerable progress in this
direction is possible.

\Chapter{IV}{Relativity Mechanics}

\Section{48.}{The antisymmetrical tensor of the fourth rank}
\index{Antisymmetrical tensors!of fourth rank}%

\lettrine{\textcolor{lettrinecolour}{A}}{tensor} $A_{\mu\nu}$ is said to be antisymmetrical if $A_{\nu\mu} = -A_{\mu\nu}$.
It follows that $A_{11} = -A_{11}$, so that $A_{11}$, $A_{22}$, $A_{33}$, $A_{44}$ must all be zero.

Consider a tensor of the fourth rank $E^{\alpha\beta\gamma\delta}$ which is antisymmetrical for
all pairs of suffixes. Any component with two suffixes alike must be zero,
since by the rule of antisymmetry $E^{\alpha\beta11} = -E^{\alpha\beta11}$. In the surviving components,
$\alpha$, $\beta$, $\gamma$, $\delta$, being all different, must stand for the numbers $1$, $2$, $3$, $4$
in arbitrary order. We can pass from any of these components to~$E^{1234}$ by a
series of interchanges of the suffixes in pairs, and each interchange merely
reverses the sign. Writing $E$ for~$E^{1234}$, all the $256$~components have one or
other of the values
\[
+E,\quad 0,\quad -E.
\]

We shall write
\[
E^{\alpha\beta\gamma\delta} = E \cdot \epsilon_{\alpha\beta\gamma\delta},
\Tag{(48.1)}
\]
where
\begin{align*}
  \epsilon_{\alpha\beta\gamma\delta}
  &= \Neg0, \text{ when the suffixes are not all different,} \\
  &= +1, \parbox[t]{0.8\textwidth}{ when they can be brought to the order $1$, $2$, $3$, $4$ by an even
    number of interchanges,} \\
  &= -1, \text{ when an odd number of interchanges is needed.}
\end{align*}

It will appear later that $E$~is not an invariant; consequently $\epsilon_{\alpha\beta\gamma\delta}$ is not
a tensor.

The coefficient $\epsilon_{\alpha\beta\gamma\delta}$ is particularly useful for dealing with determinants.
If $|k_{\mu\nu}|$ denotes the determinant formed with the elements~$k_{\mu\nu}$ (which need
\index{Determinants, manipulation of}%
not form a tensor), we have
\[
4! \times |k_{\mu\nu}| = \epsilon_{\alpha\beta\gamma\delta} \epsilon_{\epsilon\zeta\eta\theta}\, k_{\alpha\epsilon} k_{\beta\zeta} k_{\gamma\eta} k_{\delta\theta},
\Tag{(48.2)}
\]
because the terms of the determinant are obtained by selecting four elements,
one from each row ($\alpha$, $\beta$, $\gamma$, $\delta$, all different) and also from each column ($\epsilon$, $\zeta$, $\eta$, $\theta$,
all different) and affixing the $+$~or $-$ sign to the product according as the
order of the columns is brought into the order of the rows by an even or odd
number of interchanges. The factor~$4!$ appears because every possible permutation
of the same four elements is included separately in the summation
on the right.

It is possible by corresponding formulae to define and manipulate determinants
in three dimensions (with $64$~elements arranged in a cube) or in
four dimensions.

Note that
\[
\epsilon_{\alpha\beta\gamma\delta} \epsilon_{\epsilon\zeta\eta\theta} = 4!.
\Tag{(48.31)}
\]

The determinants with which we are most concerned are the fundamental
determinant~$g$ and the Jacobian of a transformation
\index{Jacobian}%
\[
J = \frac{\dd(x_{1}', x_{2}', x_{3}', x_{4}')}{\dd(x_{1}, x_{2}, x_{3}, x_{4})}.
\]
By~\Eq{(48.2)}
\begin{align*}
  4!\, g &= \epsilon_{\alpha\beta\gamma\delta}\, \epsilon_{\epsilon\zeta\eta\theta}\,
  g_{\alpha\epsilon}\, g_{\beta\zeta}\, g_{\gamma\eta}\, g_{\delta\theta},
\Tag{(48.32)} \\
  4!\, J &= \epsilon_{\alpha\beta\gamma\delta}\, \epsilon_{\epsilon\zeta\eta\theta}\,
  \frac{\dd x_{\epsilon}'}{\dd x_{\alpha}}\,
  \frac{\dd x_{\zeta}'}{\dd x_{\beta}}\,
  \frac{\dd x_{\eta}'}{\dd x_{\gamma}}\,
  \frac{\dd x_{\theta}'}{\dd x_{\delta}}.
\Tag{(48.33)}
\end{align*}
To illustrate the manipulations we shall prove that\footnote
  {A shorter proof is given at the end of this section.}
\[
g = J^{2} g'.
\]
By \Eq{(48.32)} and~\Eq{(48.33)}
\begin{multline*}
  (4!)^{3} J^{2} g'
  = \epsilon_{\alpha\beta\gamma\delta}\, \epsilon_{\epsilon\zeta\eta\theta}\,
  g_{\alpha\epsilon}'\, g_{\beta\zeta}'\, g_{\gamma\eta}'\, g_{\delta\theta}'
  \cdot \epsilon_{\iota\kappa\lambda\mu}\, \epsilon_{\nu\xi\omicron\varpi}\,
  \frac{\dd x_{\nu}'}{\dd x_{\iota}}\,
  \frac{\dd x_{\xi}'}{\dd x_{\kappa}}\,
  \frac{\dd x_{\omicron}'}{\dd x_{\lambda}}\,
  \frac{\dd x_{\varpi}'}{\dd x_{\mu}} \\
  \cdot \epsilon_{\rho\sigma\tau\upsilon}\, \epsilon_{\phi\chi\psi\omega}\,
  \frac{\dd x_{\phi}'}{\dd x_{\rho}}\,
  \frac{\dd x_{\chi}'}{\dd x_{\sigma}}\,
  \frac{\dd x_{\psi}'}{\dd x_{\tau}}\,
  \frac{\dd x_{\omega}'}{\dd x_{\upsilon}}.
  \Tag{(48.41)}
\end{multline*}
There are about $280$~billion terms on the right, and we proceed to rearrange
those which do not vanish.

For non\hyp{}vanishing terms the letters $\nu$, $\xi$, $\omicron$, $\varpi$ denote the same suffixes as
$\alpha$, $\beta$, $\gamma$, $\delta$, but (usually) in a different order. Permute the four factors in which
they occur so that they come into the same order; the suffixes of the denominators
will then come into a new order, say, $i$, $k$, $l$,~$m$. Thus
\[
  \frac{\dd x_{\nu}'}{\dd x_{\iota}}\,
  \frac{\dd x_{\xi}'}{\dd x_{\kappa}}\,
  \frac{\dd x_{\omicron}'}{\dd x_{\lambda}}\,
  \frac{\dd x_{\varpi}'}{\dd x_{\mu}}
  = \frac{\dd x_{\alpha}'}{\dd x_{i}}\,
  \frac{\dd x_{\beta}'}{\dd x_{k}}\,
  \frac{\dd x_{\gamma}'}{\dd x_{l}}\,
  \frac{\dd x_{\delta}'}{\dd x_{m}}.
\Tag{(48.42)}
\]

Since the number of interchanges of the denominators is the same as the
number of interchanges of the numerators
\[
\frac{\epsilon_{\nu\xi\omicron\varpi}}{\epsilon_{\alpha\beta\gamma\delta}}
= \pm1 = \frac{\epsilon_{\iota\kappa\lambda\mu}}{\epsilon_{iklm}},
\Tag{(48.43)}
\]
so that the result of the transposition is
\[
  \epsilon_{\alpha\beta\gamma\delta}\, \epsilon_{\iota\kappa\lambda\mu}\,
  \frac{\dd x_{\nu}'}{\dd x_{\iota}}\,
  \frac{\dd x_{\xi}'}{\dd x_{\kappa}}\,
  \frac{\dd x_{\omicron}'}{\dd x_{\lambda}}\,
  \frac{\dd x_{\varpi}'}{\dd x_{\mu}}
  = \epsilon_{\nu\xi\omicron\varpi}\, \epsilon_{iklm}\,
  \frac{\dd x_{\alpha}'}{\dd x_{i}}\,
  \frac{\dd x_{\beta}'}{\dd x_{k}}\,
  \frac{\dd x_{\gamma}'}{\dd x_{l}}\,
  \frac{\dd x_{\delta}'}{\dd x_{m}}.
  \Tag{(48.5)}
\]
Making a similar transposition of the last four terms, \Eq{(48.41)}~becomes
\begin{multline*}
  (4!)^{3} J^{2} g'
  = g_{\alpha\epsilon}'\, g_{\beta\zeta}'\, g_{\gamma\eta}'\, g_{\delta\theta}'
  \cdot \frac{\dd x_{\alpha}'}{\dd x_{i}}\,
  \frac{\dd x_{\beta}'}{\dd x_{k}}\,
  \frac{\dd x_{\gamma}'}{\dd x_{l}}\,
  \frac{\dd x_{\delta}'}{\dd x_{m}}
  \cdot \frac{\dd x_{\epsilon}'}{\dd x_{r}}\,
  \frac{\dd x_{\zeta}'}{\dd x_{s}}\,
  \frac{\dd x_{\eta}'}{\dd x_{t}}\,
  \frac{\dd x_{\theta}'}{\dd x_{u}} \\
  \cdot \epsilon_{iklm}\, \epsilon_{\nu\xi\omicron\varpi}\, \epsilon_{\nu\xi\omicron\varpi}\,
  \epsilon_{rstu}\, \epsilon_{\phi\chi\psi\omega}\, \epsilon_{\phi\chi\psi\omega}.
\end{multline*}
But by~\Eq{(23.22)}
\[
g_{\alpha\epsilon}'\, \frac{\dd x_{\alpha}'}{\dd x_{i}}\, \frac{\dd x_{\epsilon}'}{\dd x_{r}}
= g_{ir}.
\]
Hence
\begin{align*}
  (4!)^{3} J^{2} g'
  &= (4!)^{2} \epsilon_{iklm}\, \epsilon_{rstu}\, g_{ir}\, g_{ks}\, g_{lt}\, g_{mu} \\
  &= (4!)^{3} g,
\end{align*}
which proves the theorem.

Returning to $E^{\alpha\beta\gamma\delta}$, its tensor\hyp{}transformation law is
\[
E'^{\mu\nu\sigma\tau} = E^{\alpha\beta\gamma\delta}\,
\frac{\dd x_{\mu}'}{\dd x_{\alpha}}\,
\frac{\dd x_{\nu}'}{\dd x_{\beta}}\,
\frac{\dd x_{\sigma}'}{\dd x_{\gamma}}\,
\frac{\dd x_{\tau}'}{\dd x_{\delta}}.
\]
Whence multiplying by~$\epsilon_{\mu\nu\sigma\tau}$ and using~\Eq{(48.1)}
\[
E' \cdot \epsilon_{\mu\nu\sigma\tau}\, \epsilon_{\mu\nu\sigma\tau}
= E \cdot \epsilon_{\alpha\beta\gamma\delta}\, \epsilon_{\mu\nu\sigma\tau}\,
\frac{\dd x_{\mu}'}{\dd x_{\alpha}}\,
\frac{\dd x_{\nu}'}{\dd x_{\beta}}\,
\frac{\dd x_{\sigma}'}{\dd x_{\gamma}}\,
\frac{\dd x_{\tau}'}{\dd x_{\delta}},
\]
so that by \Eq{(48.31)} and~\Eq{(48.33)}
\[
E' = JE.
\Tag{(48.6)}
\]
Thus $E$~is not an invariant for transformations of coordinates.

Again
\[
E^{\alpha\beta\gamma\delta}\, E^{\epsilon\zeta\eta\theta}\,
g_{\alpha\epsilon}\, g_{\beta\zeta}\, g_{\gamma\eta}\, g_{\delta\theta}
\]
is seen by inspection to be an invariant. But this is equal to
\begin{align*}
  &E^{2} \epsilon_{\alpha\beta\gamma\delta}\, \epsilon_{\mu\nu\sigma\tau}\,
  g_{\alpha\epsilon}\, g_{\beta\zeta}\, g_{\gamma\eta}\, g_{\delta\theta} \\
= {} &4!\, E^{2} g.
\end{align*}
Hence
\[
\text{$E^{2} g$ is an invariant.}
\Tag{(48.65)}
\]
Accordingly
\[
E^{2} g = E'^{2} g' = (EJ)^{2} g',\quad\text{ by \Eq{(48.6)}}
\]
giving another proof that
\[
g = J^{2} g'.
\Tag{(48.7)}
\]

\emph{Corollary.} If $a$~is the determinant formed from the components~$a_{\mu\nu}$ of
\emph{any} covariant tensor, $E^{2} a$~is an invariant and
\[
a = J^{2} a'.
\Tag{(48.8)}
\]

\Section{49.}{Element of volume. Tensor\hyp{}density}

In \SecRef{32} we found that the surface\hyp{}element corresponding to the parallelogram
contained by two displacements, $\delta_{1} x_{\mu}$, $\delta_{2} x_{\mu}$, is the antisymmetrical tensor
\[
dS^{\mu\nu} = \left|\begin{array}{@{}cc@{}}
\delta_{1} x_{\mu} & \delta_{1} x_{\nu} \\
\delta_{2} x_{\mu} & \delta_{2} x_{\nu} \\
\end{array}\right|.
\]
Similarly we define the volume\hyp{}element (four\hyp{}dimensional) corresponding to
\index{Volume\hyp{}element}%
the hyperparallelopiped contained by four displacements, $\delta_{1} x_{\mu}$, $\delta_{2} x_{\mu}$, $\delta_{3} x_{\mu}$, $\delta_{4} x_{\mu}$,
as the tensor
\[
dV^{\mu\nu\sigma\tau} = \left|\begin{array}{@{}cccc@{}}
\delta_{1} x_{\mu} & \delta_{1} x_{\nu} & \delta_{1} x_{\sigma} & \delta_{1} x_{\tau} \\
\delta_{2} x_{\mu} & \delta_{2} x_{\nu} & \delta_{2} x_{\sigma} & \delta_{2} x_{\tau} \\
\delta_{3} x_{\mu} & \delta_{3} x_{\nu} & \delta_{3} x_{\sigma} & \delta_{3} x_{\tau} \\
\delta_{4} x_{\mu} & \delta_{4} x_{\nu} & \delta_{4} x_{\sigma} & \delta_{4} x_{\tau} \\
\end{array}\right|.
\Tag{(49.1)}
\]
It will be seen that the determinant is an antisymmetrical tensor of the fourth
rank, and its $256$~components accordingly have one or other of the three values
\[
+dV,\quad 0,\quad -dV,
\]
where $dV = \pm dV^{1234}$. It follows from \Eq{(48.65)} that $(dV)^{2} g$~is an invariant, so
that
\[
\text{$\sqrt{-g} \cdot dV$ is an invariant.}
\Tag{(49.2)}
\]

Since the sign of $dV^{1234}$ is associated with some particular cycle of
enumeration of the edges of the parallelopiped, which is not usually of any
importance, the single positive quantity~$dV$ is usually taken to represent the
volume\hyp{}element fully. Summing a number of infinitesimal volume\hyp{}elements,
\index{Volume!physical and geometrical}%
we have
\[
\iiiint \sqrt{-g} \cdot dV\quad\text{is an invariant,}
\Tag{(49.3)}
\]
the integral being taken over any region defined independently of the
coordinates.

When the quadruple integral is regarded as the limit of a sum, the infinitesimal
parallelopipeds may be taken of any shape and orientation; but for
analytical integration we choose them to be coincident with meshes of the
coordinate\hyp{}system that is being used, viz.\
\[
\delta_{1} x_{\mu} = (dx_{1}, 0, 0, 0);\
\delta_{2} x_{\mu} = (0, dx_{2} , 0, 0);\quad\text{etc.}
\]
Then \Eq{(49.1)}~reduces to a single diagonal
\[
dV = dx_{1}\, dx_{2}\, dx_{3}\, dx_{4}.
\]
We write $d\tau$ for the volume\hyp{}element when chosen in this way, so that
\[
d\tau = dx_{1}\, dx_{2}\, dx_{3}\, dx_{4}.
\]

It is not usually necessary to discriminate between~$d\tau$ and the more
general expression~$dV$; and we shall usually regard $\sqrt{-g} \cdot d\tau$ as an invariant.
Strictly speaking we mean that $\sqrt{-g} \cdot d\tau$ behaves as an invariant in volume\hyp{}integration;
whereas $\sqrt{-g} \cdot dV$ is intrinsically invariant.

For Galilean coordinates $x$, $y$, $z$, $t$, we have $\sqrt{-g} = 1$, so that
\[
\sqrt{-g}\, d\tau = dx\, dy\, dz\, dt.
\Tag{(49.41)}
\]
Further if we take an observer at rest in this Galilean system, $dx\, dy\,dz$~is his
element of proper\hyp{}volume (three\hyp{}dimensional)~$dW$, and $dt$~is his proper\hyp{}time~$ds$.
\index{Proper\hyp{}volume}%
Hence
\[
\sqrt{-g}\, d\tau = dW\, ds.
\Tag{(49.42)}
\]

By \Eq{(49.41)} we see that $\sqrt{-g}\, d\tau$ is the volume in natural measure of the
four\hyp{}dimensional element. This natural or invariant volume is a physical
conception---the result of physical measures made with unconstrained scales;
it may be contrasted with the geometrical volume~$dV$ or~$d\tau$, which expresses
the number of unit meshes contained in the region.

Let $T$~be a scalar, i.e.\ an invariant function of position; then, since
$T\sqrt{-g}\, dV$~is an invariant,
\[
\int T\sqrt{-g}\, d\tau\quad\text{is an invariant}
\]
for any absolutely defined four\hyp{}dimensional region. Each unit mesh (whose
edges $dx_{1}$, $dx_{2}$, $dx_{3}$, $dx_{4}$ are unity) contributes the amount $T\sqrt{-g}$ to this
invariant. Accordingly we call $T\sqrt{-g}$ the \emph{scalar\hyp{}density}\footnote
  {I have usually avoided the superfluous word ``scalar,'' which is less expressive than its
  synonym ``invariant.'' But it is convenient here in order to avoid confusion between the density
  \emph{of an invariant} and a density \emph{which is invariant}. The latter,~$\rho_{0}$, has hitherto been called the
  invariant density (without the hyphen).}
or \emph{invariant\hyp{}density}.
\index{Density!scalar\hyp{} and tensor\hyp{}}%
\index{Scalar\hyp{}density}%
\index{Invariant\hyp{}density (scalar\hyp{}density)}%

A nearly similar result is obtained for tensors. The integral
\[
\int T^{\mu\nu} \sqrt{-g}\, d\tau
\]
over an absolutely defined region is not a tensor; because, although it is the
sum of a number of tensors, these tensors are not located at the same point
and cannot be combined (\SecRef{33}). But in the limit as the region is made
infinitely small its transformation law approaches more and more nearly that
of a single tensor. Thus $T^{\mu\nu} \sqrt{-g}$~is a \emph{tensor\hyp{}density}, representing the amount
\index{Tensor\hyp{}density}%
per unit mesh of a tensor in the infinitesimal region round the point.

It is usual to represent the tensor\hyp{}density corresponding to any tensor by
\index{German letters, denoting tensor\hyp{}densities}%
the corresponding German letter; thus
\[
\mf{T}^{\mu\nu} \equiv T^{\mu\nu} \sqrt{-g};\quad
\mf{T} \equiv T \sqrt{-g}.
\Tag{(49.5)}
\]
By \Eq{(48.1)}
\[
\mf{E}^{\alpha\beta\gamma\delta} \equiv E^{\alpha\beta\gamma\delta} \sqrt{-g}
= E \sqrt{-g} \cdot \epsilon_{\alpha\beta\gamma\delta},
\Tag{(49.51)}
\]
and since $E \sqrt{-g}$ is an invariant it follows that $\epsilon_{\alpha\beta\gamma\delta}$~is a tensor\hyp{}density.

Physical quantities are of two main kinds, e.g.\
\index{Intensity and quantity}%
\index{Quantity and intensity}%
\begin{alignat*}{2}
  &\text{Field of acceleration}
  &&= \text{\emph{intensity} of some condition at a point,} \\
    &\text{Momentum}
    &&= \text{\emph{quantity} of something in a volume.}
\end{alignat*}
The latter kind are naturally expressed as ``so much per unit mesh.'' Hence
\emph{intensity} is naturally described by a tensor, and \emph{quantity} by a tensor\hyp{}density.
We shall find $\sqrt{-g}$~continually appearing in our formulae; that is an indication
that the physical quantities concerned are strictly tensor\hyp{}densities rather
than tensors. In the general theory tensor\hyp{}densities are at least as important
as tensors.

We can only speak of the amount of momentum in a large volume when
a definite system of coordinates has been fixed. The total momentum is the
sum of the momenta in different elements of volume; and for each element
there will be different coefficients of transformation, when a change of coordinates
is made. The only case in which we can state the amount of something
in a large region without fixing a special system of coordinates is when we
are dealing with an invariant; e.g.\ the amount of ``Action'' in a large region
is independent of the coordinates. In short, tensor\hyp{}analysis (except in the
degenerate case of invariants) deals with things located at a point and not
spread over a large region; that is why we usually have to use densities
instead of quantities.

Alternatively we can express a physical quantity of the second kind as
``so much per unit natural volume ($\sqrt{-g}\, d\tau$)''; it is then represented by a
tensor. From the physical point of view it is perhaps as rational to express
it in this way, as to express it by a tensor\hyp{}density ``so much per unit mesh~($d\tau$).''
But analytically this is a somewhat hybrid procedure, because we seem
to be employing simultaneously two systems of coordinates, the one openly
for measuring the physical quantity, the other (a natural system) implicitly
for measuring the volume containing it. It cannot be considered wrong in a
physical sense to represent quantities of the second kind by tensors; but the
analysis exposes our sub\hyp{}conscious reference to~$\sqrt{-g}\, d\tau$, by the repeated
appearance of~$\sqrt{-g}$ in the formulae.

In any kind of space-time it is possible to choose coordinates such that
$\sqrt{-g} = 1$ everywhere; for if three of the systems of partitions have been
drawn arbitrarily, the fourth can be drawn so as to intercept meshes all of
equal natural volume. In such coordinates tensors and tensor\hyp{}densities become
equivalent, and the algebra may be simplified; but although this simplification
does not involve any loss of generality, it is liable to obscure the deeper
significance of the theory, and it is not usually desirable to adopt it.

The quantity obtained by dividing a tensor by~$\sqrt{-g}$ may be called a \emph{tensor\hyp{}volume}.
We shall indicate tensor\hyp{}volumes by calligraphic type, so that
\[
\mf{T}^{\mu\nu} = T^{\mu\nu}\sqrt{-g} \quad \mc{T}^{\mu\nu} = T^{\mu\nu}/\sqrt{-g}
\Tag{(49.6)}
\]
Evidently~$\mf{T}_{\mu\nu}\mc{T}^{\mu\nu}$ is an invariant, German character cancelling calligraphic.

By~\Eq{(49.2)} $dV$ an invariant\hyp{}volume and should be denoted by $d\mc{V}$.

The coefficient~$\epsilon_{\alpha\beta\gamma\delta}$ is at the same time a contravariant tensor\hyp{}density
and a covariant tensor\hyp{}volume.
We may thus write
\[
\epsilon_{\alpha\beta\gamma\delta} = \mc{E}_{\alpha\beta\gamma\delta} = \mf{E}_{\alpha\beta\gamma\delta}
\Tag{(49.7)}
\]
The product~$\mc{E}_{\alpha\beta\gamma\delta}\mf{E}^{\alpha\beta\gamma\delta}$ should evidently be invariant;
this is satisfied because by~\Eq{(48.31)} it has the constant value $4!$.

By means of this coefficient we can associate a covariant tensor\hyp{}volume with any antisymmetrical contravariant tensor.
This process is especially important in connection with space\hyp{}elements of $1,2,3$ or $4$ dimensions, which are
antisymmetrical contravariant tensors.
For example, the four\hyp{}dimensional element of volume is measured by either the tensor~$dV^{\alpha\beta\gamma\delta}$
or the invariant volume $d\mc{V}$ connected by the relation
\[
(4!)d\mc{V} = \mc{E}_{\alpha\beta\gamma\delta} dV^{\alpha\beta\gamma\delta}
\]
Similarly the surface\hyp{}element is represented by~$dS^{\alpha\beta}$ or~$d\mc{S}'_{\alpha\beta}$, where
\[
d\mc{S}'_{\alpha\beta} = \mc{E}_{\alpha\beta\gamma\delta} dS^{\gamma\delta}
\Tag{(49.8)}
\]
The necessity for inserting the accent should be noticed; the result of this operation does not
give~$d\mc{S}_{\alpha\beta}$ which according to previous definitions is derived from~$dS^{\alpha\beta}$ by
lowering the two suffixes and dividing by~$\sqrt{-g}$.

The representation of surface\hyp{}elements by an adjoint vector in elementary three\hyp{}dimensional theory arises in this way.
If
\[
d\mc{S}'_\alpha = \mc{E}_{\alpha\beta\gamma} dS^{\beta\gamma},
\Tag{(49.9)}
\]
the vector\hyp{}volume~$d\mc{S}'_\alpha$ can be used as a measure of the surface\hyp{}element.
The elementary theory (restricted to rectangular coordinates) does not discriminate between vectors and vector\hyp{}volumes.

From a covariant antisymmetrical tensor~$F_{\alpha\beta}$ we can derive two different
tensor\hyp{}densities~$\mf{F}^{\alpha\beta}$ and~$\mf{F}'^{\alpha\beta}$, thus
\[
\mf{F}^{\alpha\beta} = g^{\alpha\gamma} g^{\beta\delta} F_{\gamma\delta} \sqrt{-g} \quad \mf{F}'^{\alpha\beta} = \mf{E}^{\alpha\beta\gamma\delta} F_{\gamma\delta};
\Tag{(49.10)}
\]
the latter is obtained merely by rearranging the components of~$F_{\alpha\beta}$.

As an illustration we can prove that
\[
|\mf{g}^{\mu\nu}| = |g_{\mu\nu}|
\Tag{(49.11)}
\]

For this equation is equivalent to
\[
\mc{E}_{\alpha\beta\gamma\delta} \mc{E}_{\epsilon\zeta\eta\theta}
  \mf{g}^{\alpha\epsilon} \mf{g}^{\beta\zeta} \mf{g}^{\gamma\eta} \mf{g}^{\delta\theta} =
\mf{E}^{\alpha\beta\gamma\delta} \mf{E}^{\epsilon\zeta\eta\theta}
  g_{\alpha\epsilon} g_{\beta\zeta} g_{\gamma\eta} g_{\delta\theta}
\]
Both sides are seen to have the same dimensions, viz. those of the square of an invariant\hyp{}density, and so transform
by the same law.
In natural coordinates the two determinants are identical; hence their values are equal in all coordinate\hyp{}systems.

\Section{50.}{The problem of the rotating disc}
\index{Problem!of rotating disc}%
\index{Rotating disc}%

We may consider at this point a problem of some historic interest---

A disc made of homogeneous incompressible material is caused to rotate
with angular velocity~$\omega$; to find the alteration in length of the radius.

The old paradox associated with this problem---that the circumference
moving longitudinally might be expected to contract, whilst the radius moving
transversely is unaltered---no longer troubles us\footnotemark.\footnotetext
  {\Title{Space, Time and Gravitation,} p.~75.}
But the general theory of
relativity gives a quantitative answer to the problem, which was first obtained
by Lorentz by a method different from that given here\footnotemark.\footnotetext
  {\Title{Nature,} vol.~106, p.~795.}

We must first have a clear understanding of what is meant by the word
\index{Incompressibility}%
``incompressible''. Let us isolate an element of the rotating disc, and refer it to
axes with respect to which it has no velocity or acceleration (proper\hyp{}measure);
then except for the fact that it is under stress due to the cohesive forces of
surrounding matter, it is relatively in the same state as an element of the
non\hyp{}rotating disc referred to fixed axes. Now the meaning of \emph{incompressible}
is that no stress\hyp{}system can make any difference in the closeness of packing
of the molecules; hence the particle\hyp{}density a (referred to proper\hyp{}measure)
is the same as for an element of the non\hyp{}rotating disc. But the particle\hyp{}density~$\sigma'$
referred to axes fixed in space may be different.

We might write down at once by~\Eq{(14.1)}
\[
\sigma' = \sigma (1 - \omega^2 r^{2})^{-\frac{1}{2}},
\]
since $\omega r$~is the velocity of the element. This would in fact give the right
result. But in \SecRef{14} acceleration was not taken into account and we ought to
proceed more rigorously. We use the accented coordinates of \SecRef{15} for our
rotating system, and easily calculate from~\Eq{(15.4)} that
\[
\sqrt{-g'} = 1,
\]
and since $x_{1}'$, $x_{2}'$, $x_{3}'$ are constant for an element of the disc, the proper\hyp{}time
\[
ds = \sqrt{1 - \omega^2(x_{1}'^{2} + x_{2}'^{2})}\, dx_{4}'.
\]

If $dW$~is the proper\hyp{}volume of the element, by~\Eq{(49.42)}
\[
dW\, ds = \sqrt{-g'} \cdot dx_{1}'\, dx_{2}'\, dx_{3}'\, dx_{4}'.
\]
Hence
\begin{align*}
  dW &= \bigl(1 - \omega^2(x_{1}'^{2} + x_{2}'^{2})\bigr)^{-\frac{1}{2}}\, dx_{1}'\, dx_{2}'\, dx_{3}' \\
  &= (1 - \omega^2 r^{2})^{-\frac{1}{2}}\, r'\, dr'\, d\theta'\, dx_{3}'.
\end{align*}
If the thickness of the disc is $\delta x_{3}' = b$, and its boundary is given by $r' = a'$,
the total number of particles in the disc will be
\[
N = \int \sigma\, dW
= 2\pi\sigma b \int_{0}^{a'} (1 - \omega^2 r'^{2})^{-\frac{1}{2}}\, r'\, dr'.
\]
Since this number is unaltered by the rotation, $a'$~must be a function of~$\omega$
such that
\[
\int_{0}^{a'} (1 - \omega^2 r^{2})^{-\frac{1}{2}}\, r'\, dr' = \text{const.},
\]
or
\[
\frac{1}{\omega^2}\bigl(1 - \sqrt{1 - \omega^2 a'^{2}}\bigr) = \text{const.}
\]
Expanding the square\hyp{}root, this gives approximately
\[
\tfrac{1}{2} a'^{2} (1 + \tfrac{1}{4} \omega^2 a'^{2}) = \text{const.},
\]
so that if $a$~is the radius of the disc at rest
\[
a' = (1 + \tfrac{1}{8} \omega^2 a'^{2}) = a.
\]
Hence to the same approximation
\[
a' = a (1 - \tfrac{1}{8} \omega^2 a^2).
\]
Note that $a'$~is the radius of the rotating disc according to measurement with
fixed scales, since the rotating and non\hyp{}rotating coordinates have been connected
by the elementary transformation~\Eq{(15.3)}.

We see that the contraction is one quarter of that predicted by a crude
application of the FitzGerald formula to the circumference.

This proof has been criticised as defective in that no explanation is given as to why the thickness~$b$ of the disc
is assumed to be unaltered by rotation.
We may examine this question by the same general method.
The essential point is that in an incompressible and (it should be added) perfectly rigid disc there is no response
of any kind to applied stress, so that stresses may be ignored;
any difference between rotating and non\hyp{}rotating elements must be a difference of \emph{description,} not of
intrinsic structure.
Thus the configuration of molecules when referred to proper\hyp{}coordinates will be the same in rotating as in
non\hyp{}rotating elements of the material.
But the transformation to proper\hyp{}coordinates does not affect~$x_3$, so that the spacing of the molecules along this
coordinate is unaltered by the rotation.
The thickness of the disc---or length of the chain of molecules extending from the lower to the
upper surface---is accordingly unaltered.

\Section{51.}{The divergence of a tensor}
\index{Contracted derivative (divergence)}%
\index{Divergence!tensor@of a tensor}%

In the elementary theory of vectors the divergence
\[
\frac{\dd X}{\dd x} + \frac{\dd Y}{\dd y} + \frac{\dd Z}{\dd z}
\]
is important; we can to some extent grasp its geometrical significance. In
our general notation, this expression becomes
\[
\frac{\dd A^{\mu}}{\dd x_{\mu}}.
\]
But evidently a more fundamental operation is to take the covariant derivatives
which will give an invariant
\[
(A^{\mu})_{\mu}.
\]
We therefore define the \emph{divergence} of a tensor as its contracted covariant
derivative.

By \Eq{(29.4)}
\begin{align*}
  (A^{\mu})_{\mu}
  &= \frac{\dd A^{\mu}}{\dd x_{\mu}} + \{\epsilon\mu, \mu\} A^{\epsilon} \\
  &= \frac{\dd A^{\mu}}{\dd x_{\mu}}
  + A^{\epsilon} \cdot \frac{1}{\sqrt{-g}}\, \frac{\dd}{\dd x_{\epsilon}} \sqrt{-g}
  \quad\text{by~\Eq{(35.4)}} \\
  &= \frac{1}{\sqrt{-g}}\, \frac{\dd}{\dd x_{\mu}} (A^{\mu} \sqrt{-g}),
  \Tag{(51.11)}
\end{align*}
since $\epsilon$~may be replaced by~$\mu$. In terms of tensor\hyp{}density this may be written
\[
A_{\mu}^{\mu} \sqrt{-g}
= \mf{A}_{\mu}^{\mu}
= \frac{\dd}{\dd x_{\mu}} \mf{A}^{\mu}.
\Tag{(51.12)}
\]

The divergence of~$A_{\mu}^{\nu}$ is by~\Eq{(30.2)}
\begin{align*}
  (A_{\mu}^{\nu})_{\nu}
  &= \frac{\dd}{\dd x_{\nu}} A_{\mu}^{\nu} + \{\alpha\nu,\nu\} A_{\mu}^{\alpha} - \{\mu\nu, \alpha\} A_{\alpha}^{\nu} \\
  &= \frac{1}{\sqrt{-g}}\, \frac{\dd}{\dd x_{\nu}} (A_{\mu}^{\nu} \sqrt{-g}) - \{\mu\nu, \alpha\} A_{\alpha}^{\nu},
  \Tag{(51.2)}
\end{align*}
by the same reduction as before. The last term gives
\[
-\frac{1}{2} \left(\frac{\dd g_{\mu\beta}}{\dd x_{\nu}}
+ \frac{\dd g_{\nu\beta}}{\dd x_{\mu}}
- \frac{\dd g_{\mu\nu}}{\dd x_{\beta}}\right) A^{\beta\nu}.
\]

When $A^{\beta\nu}$~is a symmetrical tensor, two of the terms in the bracket cancel
by interchange of $\beta$ and~$\nu$, and we are left with $-\dfrac{1}{2}\, \dfrac{\dd g_{\beta\nu}}{\dd x_{\mu}}\, A^{\beta\nu}$.

Hence for \emph{symmetrical tensors}
\[
(A_{\mu}^{\nu})_{\nu}
= \frac{1}{\sqrt{-g}}\, \frac{\dd}{\dd x_{\nu}} (A_{\mu}^{\nu} \sqrt{-g})
- \frac{1}{2}\, \frac{\dd g_{\alpha\beta}}{\dd x_{\mu}}\, A^{\alpha\beta},
\Tag{(51.31)}
\]
or, by~\Eq{(35.2)},
\[
(A_{\mu}^{\nu})_{\nu}
= \frac{1}{\sqrt{-g}}\, \frac{\dd}{\dd x_{\nu}} (A_{\mu}^{\nu} \sqrt{-g})
+ \frac{1}{2}\, \frac{\dd g^{\alpha\beta}}{\dd x_{\mu}}\, A_{\alpha\beta}.
\Tag{(51.32)}
\]

For \emph{antisymmetrical tensors,} it is easier to use the contravariant associate,
\[
(A^{\mu\nu})_{\nu}
= \frac{\dd}{\dd x_{\nu}} A^{\mu\nu} + \{\alpha\nu,\nu\} A^{\mu\alpha} + \{\alpha\nu, \mu\} A^{\alpha\nu}.
\Tag{(51.41)}
\]
The last term vanishes owing to the antisymmetry. Hence
\[
(A^{\mu\nu})_{\nu}
= \frac{1}{\sqrt{-g}}\, \frac{\dd}{\dd x_{\nu}} (A^{\mu\nu} \sqrt{-g}).
\Tag{(51.42)}
\]
Introducing tensor\hyp{}densities our results become
\begin{align*}
  \mf{A}_{\mu\nu}^{\nu}
  &= \frac{\dd}{\dd x_{\nu}} \mf{A}_{\mu}^{\nu}
  - \tfrac{1}{2} \mf{A}^{\alpha\beta}\, \frac{\dd g_{\alpha\beta}}{\dd x_{\mu}}
  & \text{(symmetrical tensors),}&
  \Tag{(51.51)} \\
  \mf{A}_{\nu}^{\mu\nu}
  &= \frac{\dd}{\dd x_{\nu}} \mf{A}^{\mu\nu}
  & \text{(antisymmetrical tensors).} &
  \Tag{(51.52)}
\end{align*}

\Section{52.}{The four identities}
\index{Identities satisfied by $G_{\mu\nu}$}%

We shall now prove the fundamental theorem of me\-cha\-nics---
\index{Fundamental theorem of mechanics}%
\[
\text{\emph{The divergence of $G_{\mu}^{\nu} - \tfrac{1}{2}g_{\mu}^{\nu} G$ is identically zero.}}
\index{Divergence!energy@of energy\hyp{}tensor}%
\Tag{(52)}
\]

In three dimensions the vanishing of the divergence is the condition of
continuity of flux, e.g.\ in hydrodynamics $\dd u/\dd x + \dd v/\dd y + \dd w/\dd z = 0$. Adding a
time\hyp{}coordinate, this becomes the condition of \emph{conservation} or \emph{permanence,} as
\index{Permanence}%
will be shown in detail later. \emph{It will be realised how important for a theory
of the material world is the discovery of a world\hyp{}tensor which is inherently
permanent.}

I think it should be possible to prove~\Eq{(52)} by geometrical reasoning in
continuation of the ideas of \SecRef{33}. But I have not been able to construct a
geometrical proof and must content myself with a clumsy analytical verification.

By the rules of covariant differentiation
\[
(g_{\mu}^{\nu} G)_{\nu} = g_{\mu}^{\nu}\, \frac{\dd G}{\dd x_{\nu}} = \frac{\dd G}{\dd x_{\mu}}.
\]
Thus the theorem reduces to
\[
G_{\mu\nu}^{\nu} = \frac{1}{2}\, \frac{\dd G}{\dd x_{\mu}}.
\Tag{(52.1)}
\]
For $\mu = 1$, $2$, $3$, $4$, these are the four identities referred to in \SecRef{37}. By~\Eq{(51.32)}
\[
G_{\mu\nu}^{\nu}
= \frac{1}{\sqrt{-g}}\, \frac{\dd}{\dd x_{\nu}} (G_{\mu}^{\nu} \sqrt{-g})
+ \tfrac{1}{2} G_{\alpha\beta}\, \frac{\dd g^{\alpha\beta}}{\dd x_{\mu}},
\]
and since $G = g^{\alpha\beta} G_{\alpha\beta}$
\[
\frac{1}{2}\, \frac{\dd G}{\dd x_{\mu}}
= \tfrac{1}{2} g^{\alpha\beta}\, \frac{\dd G_{\alpha\beta}}{\dd x_{\mu}}
+ \tfrac{1}{2} G_{\alpha\beta}\, \frac{\dd g^{\alpha\beta}}{\dd x_{\mu}}.
\]
Hence, subtracting, we have to prove that
\[
\frac{1}{\sqrt{-g}}\, \frac{\dd}{\dd x_{\nu}} (G_{\mu}^{\nu} \sqrt{-g})
= \tfrac{1}{2} g^{\alpha\beta}\, \frac{\dd G_{\alpha\beta}}{\dd x_{\mu}}.
\Tag{(52.2)}
\]
Since \Eq{(52)}~is a tensor relation it is sufficient to show that it holds for a special
coordinate\hyp{}system; only we must be careful that our special choice of coordinate\hyp{}system
does not limit the kind of space-time and so spoil the generality
of the proof. It has been shown in \SecRef{36} that in any kind of space-time, coordinates
can be chosen so that all the first derivatives~$\dd g_{\mu\nu}/\dd x_{\sigma}$ vanish at a
particular point; we shall therefore lighten the algebra by taking coordinates
such that at the point considered
\[
\frac{\dd g_{\mu\nu}}{\dd x_{\sigma}} = 0.
\Tag{(52.3)}
\]
This condition can, of course, only be applied after all differentiations have
been performed. Then
\[
\frac{1}{\sqrt{-g}}\, \frac{\dd}{\dd x_{\nu}} (G_{\mu}^{\nu} \sqrt{-g})
= \frac{1}{\sqrt{-g}}\, \frac{\dd}{\dd x_{\nu}} (g^{\nu\tau} g^{\sigma\rho} \sqrt{-g}\cdot B_{\mu\tau\sigma\rho}).
\]
Owing to~\Eq{(52.3)} $g^{\nu\tau} g^{\sigma\rho} \sqrt{-g}$ can be taken outside the differential operator,
giving
\[
g^{\nu\tau} g^{\sigma\rho}\, \frac{\dd}{\dd x_{\nu}} B_{\mu\tau\sigma\rho},
\]
which by~\Eq{(34.5)} is equal to
\[
\tfrac{1}{2} g^{\nu\tau} g^{\sigma\rho} \left(
  \frac{\dd^{2} g_{\rho\sigma}}{\dd x_{\mu}\, \dd x_{\tau}}
  + \frac{\dd^{2} g_{\mu\tau}}{\dd x_{\rho}\, \dd x_{\sigma}}
  - \frac{\dd^{2} g_{\mu\sigma}}{\dd x_{\rho}\, \dd x_{\tau}}
  - \frac{\dd^{2} g_{\rho\tau}}{\dd x_{\mu}\, \dd x_{\sigma}}\right).
\Tag{(52.4)}
\]
The rest of~$B_{\mu\tau\sigma\rho}$ is omitted because it consists of products of two vanishing
factors ($3$-index symbols), so that after differentiation by~$\dd x_{\nu}$ one vanishing
factor always remains.

By the double interchange $\sigma$ for~$\tau$, $\rho$~for $\nu$, two terms in~\Eq{(52.4)} cancel out,
leaving
\[
\frac{1}{\sqrt{-g}}\, \frac{\dd}{\dd x_{\nu}} (G_{\mu}^{\nu} \sqrt{-g})
= \tfrac{1}{2} g^{\nu\tau} g^{\sigma\rho}\, \frac{\dd}{\dd x_{\nu}} \left(
    \frac{\dd^{2} g_{\rho\sigma}}{\dd x_{\mu}\, \dd x_{\tau}}
  - \frac{\dd^{2} g_{\rho\tau}}{\dd x_{\mu}\, \dd x_{\sigma}}\right).
\Tag{(52.51)}
\]
Similarly
\begin{align*}
  \tfrac{1}{2} g^{\alpha\beta}\, \frac{\dd G_{\alpha\beta}}{\dd x_{\mu}}
  &= \tfrac{1}{2} g^{\nu\tau}\, \frac{\dd G_{\nu\tau}}{\dd x_{\mu}}
  = \tfrac{1}{2} g^{\nu\tau}\, \frac{\dd}{\dd x_{\mu}} (g^{\sigma\rho} B_{\nu\tau\sigma\rho}) \\
  &= \tfrac{1}{4} g^{\nu\tau} g^{\sigma\rho}\, \frac{\dd}{\dd x_{\mu}}
  \left(\frac{\dd^{2} g_{\rho\sigma}}{\dd x_{\nu}\, \dd x_{\tau}}
  + \frac{\dd^{2} g_{\nu\tau}}{\dd x_{\rho}\, \dd x_{\sigma}}
  - \frac{\dd^{2} g_{\nu\sigma}}{\dd x_{\rho}\, \dd x_{\tau}}
  - \frac{\dd^{2} g_{\rho\tau}}{\dd x_{\nu}\, \dd x_{\sigma}}\right) \\
  &= \tfrac{1}{2} g^{\nu\tau} g^{\sigma\rho}\, \frac{\dd}{\dd x_{\mu}} \left(
    \frac{\dd^{2} g_{\rho\sigma}}{\dd x_{\nu}\, \dd x_{\tau}}
  - \frac{\dd^{2} g_{\rho\tau}}{\dd x_{\nu}\, \dd x_{\sigma}}\right),
  \Tag{(52.52)}
\end{align*}
since the double interchange $\sigma$ for~$\tau$, $\rho$~for $\nu$, causes two terms to become
equal to the other two.

Comparing \Eq{(52.51)} and \Eq{(52.52)} we see that the required result is established
for coordinates chosen so as to have the property~\Eq{(52.3)} at the point
considered; and since it is a tensor equation it must hold true for all systems
of coordinates.

The four identities can be obtained in a more elegant way as follows.
We must first establish the identity
\[
(B^{\epsilon}_{\mu\nu\sigma})_\tau + (B^{\epsilon}_{\mu\sigma\tau})_\nu + (B^{\epsilon}_{\mu\tau\nu})_\sigma = 0,
\Tag{(52.6)}
\]
where the final suffix denotes covariant differentiation.
To prove this we evaluate the left-hand side in terms of $3$-index symbols by substituting~\Eq{(34.4)}
in~\Eq{(30.4)};
but it is only necessary to proceed far enough to see that the second derivatives of the $3$-index symbols
cancel out cyclically, and that the first derivatives occur only in combination with a $3$-index symbol itself as
co\hyp{}factor.
Hence the whole expression will vanish when the $3$-index symbols (but not their derivatives) vanish,
i.e.\ in natural coordinates.
The result is thus proved for natural coordinates, and since it is a tensor equation it will be true for all
coordinate\hyp{}systems.

Lowering the suffix~$\epsilon$, and using the antisymmetrical properties of~$B_{\mu\nu\sigma\epsilon}$ we have
\[
(B_{\mu\nu\sigma\epsilon})_\tau - (B_{\mu\tau\sigma\epsilon})_\nu + (B_{\epsilon\nu\tau\mu})_\sigma = 0.
\]
Hence, multiplying by~$g^{\sigma\epsilon}g^{\mu\tau}$,
\[
(G^\tau_\nu)_\tau - (G)_\nu + (G^\sigma_\nu)_\sigma = 0,
\]
which is equivalent to~\Eq{(52.1)}.

The crude statement in~\SecRef{37}, that owing to the existence of these 4 identical relations the number of
gravitational equations is effectively reduced to 6, requires some amplification.
A relation between the first derivatives of the~$G_{\mu\nu}$ is not so restrictive as a relation between
the~$G_{\mu\nu}$ themselves, and it is not true that if 6 of the~$G_{\mu\nu}$ are made to vanish the remaining 4 will
identically vanish.
If we consider the 40 covariant derivatives~$(G^\nu_\mu)_\sigma$, 4 of these depend on the others, so that the
vanishing of 36 of the derivatives ensures that all 40 will vanish.
The effect is that the scheme of equations for determining the~$g_{\mu\nu}$ is incomplete by 4,
so that there remains a four-fold arbitrariness in the values of the~$g_{\mu\nu}$ and therefore of
the coordinate\hyp{}system.

\Section{53.}{The material energy\hyp{}tensor}
\index{Energy\hyp{}tensor of matter}%
\index{T@$T_{\mu\nu}$ (energy\hyp{}tensor)}%

Let $\rho_{0}$~be the proper\hyp{}density of matter, and let $dx_{\mu}/ds$ refer to the motion
of the matter; we write, as in~\Eq{(46.8)},
\[
T^{\mu\nu} = \rho_{0}\, \frac{dx_{\mu}}{ds}\, \frac{dx_{\nu}}{ds}.
\Tag{(53.1)}
\]
Then $T^{\mu\nu}$ (with the associated mixed and covariant tensors) is called the
\emph{energy\hyp{}tensor} of the matter.

For matter moving with any velocity relative to Galilean coordinates, the
coordinate\hyp{}density~$\rho$ is given by
\[
\rho = \rho_{0} \left(\frac{dt}{ds}\right)^{2},
\Tag{(53.2)}
\]
for, as explained in~\Eq{(14.2)}, the FitzGerald factor $\beta = dt/ds$ appears twice, once
for the increase of mass with velocity and once for the contraction of volume.

Hence in Galilean coordinates
\[
T^{\mu\nu} = \rho\, \frac{dx_{\mu}}{dt}\, \frac{dx_{\nu}}{dt},
\Tag{(53.3)}
\]
so that if $u$, $v$, $w$ are the components of velocity
\[
T^{\mu\nu} =
\begin{array}[t]{@{}c@{,\quad}c@{,\quad}c@{,\quad}c@{}}
  \rho u^2 & \rho v u & \rho w u & \rho u\\
  \rho u v & \rho v^2 & \rho w v & \rho v\\
  \rho u w & \rho v w & \rho w^{2} & \rho w\\
  \rho u & \rho v & \rho w & \rho\\
\end{array}
\Tag{(53.4)}
\]

In matter atomically constituted, a volume which is regarded as small for
macroscopic treatment contains particles with widely divergent motions. Thus
the terms in~\Eq{(53.4)} should be summed for varying motions of the particles.
For macroscopic treatment we express the summation in the following way.---

%[** TN: No indent in the original]
{\Loosen Let $(u, v, w)$ refer to the motion of the centre of mass of the element, and
\index{Stress\hyp{}system}%
$(u_{1}, v_{1}, w_{1})$ be the internal motion of the particles relative\footnote
  {In the sense of elementary mechanics, i.e.\ the simple difference of the velocities.}
to the centre of
mass. Then in a term of our tensor such as $\sum \rho(u + u_{1})(v + v_{1})$, the cross\hyp{}products
will vanish, leaving $\sum \rho u v + \sum \rho u_{1}v_{1}$. Now $\sum \rho u_{1}v_{1}$ represents the rate of
transfer of $u$-momentum by particles crossing a plane perpendicular to the
$y$-axis, and is therefore equal to the internal stress usually denoted by~$p_{xy}$.
We have therefore to add to~\Eq{(53.4)} the tensor formed by the internal stresses,
bordered by zeroes. The summation can now be omitted, $\rho$~referring to the
whole density, and $u$, $v$, $w$ to the average or mass\hyp{}motion of macroscopic
elements. Accordingly}
\[
T^{\mu\nu} =
\begin{array}[t]{@{}c@{,\quad}c@{,\quad}c@{,\quad}c@{}}
  p_{xx} + \rho u^2 & p_{yx} + \rho v u & p_{zx} + \rho w u & \rho u\\
  p_{xy} + \rho u v & p_{yy} + \rho v^2 & p_{yz} + \rho w v & \rho v\\
  p_{xz} + \rho u w & p_{yz} + \rho v w & p_{zz} + \rho w^{2} & \rho w\\
  \rho u & \rho v & \rho w & \rho\\
\end{array}
\Tag{(53.5)}
\]
Consider the equations
\[
\frac{\dd T^{\mu\nu}}{\dd x_{\nu}} = 0.
\Tag{(53.6)}
\]
Taking first $\mu = 4$, this gives by~\Eq{(53.5)}
\[
\frac{\dd(\rho u)}{\dd x} + \frac{\dd(\rho v)}{\dd y} + \frac{\dd(\rho w)}{\dd z} + \frac{\dd\rho}{\dd t} = 0,
\Tag{(53.71)}
\]
which is the usual ``equation of continuity'' in hydrodynamics.
\index{Continuity, equation of}%
\index{Hydrodynamics, equations of}%

For $\mu = 1$, we have
\begin{align*}
  \frac{\dd p_{xx}}{\dd x} + \frac{\dd p_{xy}}{\dd y} + \frac{\dd p_{xz}}{\dd z}
  &= -\left(\frac{\dd(\rho u^2)}{\dd x} + \frac{\dd(\rho u v)}{\dd y}
          + \frac{\dd(\rho u w)}{\dd z} + \frac{\dd(\rho u)}{\dd t}\right) \\
  &= -u \left(\frac{\dd(\rho u)}{\dd x} + \frac{\dd(\rho v)}{\dd y}
          + \frac{\dd(\rho w)}{\dd z} + \frac{\dd\rho}{\dd t}\right) \\
  &\quad -\rho \left(u\, \frac{\dd u}{\dd x} + v\, \frac{\dd u}{\dd y}
          + w\, \frac{\dd u}{\dd z} + \frac{\dd u}{\dd t}\right) \\
  &= -\rho\, \frac{Du}{Dt}
\Tag{(53.72)}
\end{align*}
by~\Eq{(53.71)}. $Du/Dt$~is the acceleration of the element of the fluid.

This is the well-known equation of hydrodynamics when no body\hyp{}force is
\index{Hydrodynamics, equations of}%
acting. (By adopting Galilean coordinates any field of force acting on the
mass of the fluid has been removed.)

Equations \Eq{(53.71)} and \Eq{(53.72)} express directly the conservation of mass
and momentum, so that for Galilean coordinates these principles are contained
\index{Momentum!conservation of}%
in
\[
\dd T^{\mu\nu}/\dd x_{\nu} = 0.
\]
In fact $\dd T^{\mu\nu}/\dd x_{\nu}$ represents the rate of creation of momentum and mass in
unit volume. In classical hydrodynamics momentum may be created in the
volume (i.e.\ may appear in the volume without having crossed the boundary)
by the action of a body\hyp{}force $\rho X$, $\rho Y$, $\rho Z$; and these terms are added on the
right-hand side of~\Eq{(53.72)}. The creation of mass is considered impossible.
Accordingly the more general equations of \emph{classical hydrodynamics} are
\[
\frac{\dd T^{\mu\nu}}{\dd x_{\nu}} = (\rho X, \rho Y, \rho Z, 0).
\Tag{(53.81)}
\]
In the \emph{special relativity theory} mass is equivalent to energy, and the body\hyp{}forces
by doing work on the particles will also create mass, so that
\[
\frac{\dd T^{\mu\nu}}{\dd x_{\nu}} = (\rho X, \rho Y, \rho Z, \rho S),
\Tag{(53.82)}
\]
where $\rho S$~is the work done by the forces $\rho X$, $\rho Y$, $\rho Z$. These older formulae
are likely to be only approximate; and the exact formulae must be deduced
by extending the general relativity theory to the case when fields of force are
present, viz.\ to non\hyp{}Galilean coordinates.

It is often convenient to use the mixed tensor~$T_{\mu}^{\nu}$ in place of~$T^{\mu\nu}$. For
Galilean coordinates we obtain from~\Eq{(53.5)}\footnote
  {E.g.\ $T_{2}^{1} = \rho_{\sigma 2} T^{\sigma 1} = 0 - T^{21} + 0 + 0$.}
\[
\begin{array}[t]{r}
  T_{\mu}^{\nu} \\
  \MuNuarrow
\end{array}
=
\begin{array}[t]{@{}c@{,\quad}c@{,\quad}c@{,\quad}c@{}}
  -p_{xx} - \rho u^2 & -p_{yx} - \rho v u & -p_{zx} - \rho w u & \rho u\\
  -p_{xy} - \rho u v & -p_{yy} - \rho v^2 & -p_{yz} - \rho w v & \rho v\\
  -p_{xz} - \rho u w & -p_{yz} - \rho v w & -p_{zz} - \rho w^{2} & \rho w\\
  -\rho u & -\rho v & -\rho w & \rho\\
\end{array}
\Tag{(53.91)}
\]
The equation equivalent to~\Eq{(53.82)} is then
\[
\frac{\dd T_{\mu}^{\nu}}{\dd x_{\nu}} = (-\rho X,-\rho Y,-\rho Z, \rho S).
\Tag{(53.92)}
\]
That is to say $\dd T_{\mu}^{\nu}/\dd x_{\nu}$ is the rate of creation of negative momentum and of
positive mass or energy in unit volume.

\Section{54.}{New derivation of Einstein's law of gravitation}
\index{Continuous matter, gravitation in}%
\index{Einstein's law of gravitation!in continuous matter}%

We have found that for Galilean coordinates
\[
\frac{\dd T^{\mu\nu}}{\dd x_{\nu}} = 0.
\Tag{(54.1)}
\]
This is evidently a particular case of the tensor equation
\[
(T^{\mu\nu})_{\nu} = 0.
\Tag{(54.21)}
\]
Or we may use the equivalent equation
\[
(T_{\mu}^{\nu})_{\nu} = 0,
\Tag{(54.22)}
\]
which results from lowering the suffix~$\mu$. In other words the divergence of
\index{Divergence!energy@of energy\hyp{}tensor}%
the energy\hyp{}tensor vanishes.

Taking the view that energy, stress, and momentum belong to the world
(space-time) and not to some extraneous substance in the world, we must
identify the energy\hyp{}tensor with some fundamental tensor, i.e.\ a tensor belonging
to the fundamental series derived from~$g_{\mu\nu}$.

The fact that the divergence of~$T_{\mu}^{\nu}$ vanishes points to an identification
with $(G_{\mu}^{\nu} - \frac{1}{2}g_{\mu}^{\nu} G)$ whose divergence vanishes \emph{identically} (\SecRef{52}). Accordingly
we set
\[
G_{\mu}^{\nu} - \tfrac{1}{2}g_{\mu}^{\nu} G - 8\pi T_{\mu}^{\nu},
\Tag{(54.3)}
\]
the factor~$8\pi$ being introduced for later convenience in coordinating the units.

To pass from \Eq{(54.1)} to \Eq{(54.21)} involves an appeal to the hypothetical
Principle of Equivalence; but by taking \Eq{(54.3)} as our fundamental equation
\index{Identification, Principle of}%
\index{Principle!of identification}%
of gravitation \Eq{(54.21)}~becomes an identity requiring no hypothetical assumption.

We thus arrive at the law of gravitation for continuous matter~\Eq{(46.6)}
\index{Matter!identification of}%
but with a different justification. Appeal is now made to a Principle of
Identification. Our deductive theory starts with the interval (introduced by
the fundamental axiom of \SecRef{1}), from which the tensor~$g_{\mu\nu}$ is immediately
obtained. By pure mathematics we derive other tensors $G_{\mu\nu}$, $B_{\mu\nu\sigma\rho}$, and if
necessary more complicated tensors. These constitute our world\hyp{}building
material; and the aim of the deductive theory is to construct from this a
world which functions in the same way as the known physical world. If we
succeed, mass, momentum, stress, etc.\ must be the vulgar names for certain
analytical quantities in the deductive theory; and it is this stage of naming
the analytical tensors which is reached in~\Eq{(54.3)}. If the theory provides a
tensor $G_{\mu}^{\nu} - \frac{1}{2} g_{\mu}^{\nu} G$ which behaves in exactly the same way as the tensor
summarising the mass, momentum and stress of matter is observed to behave,
it is difficult to see how anything more could be required of it\footnotemark.\footnotetext
  {For a complete theory it would be necessary to show that matter as now defined has a
\index{Atomicity@Atomicity|indexfn}%
  tendency to aggregate into atoms leaving large tracts of the world vacant. The relativity theory
  has not yet succeeded in finding any clue to the phenomenon of atomicity.}

By means of \Eq{(53.91)} and~\Eq{(54.3)} the physical quantities $\rho$, $u$, $v$, $w$, $p_{xx}$, \dots, $p_{zz}$
are identified in terms of the fundamental tensors of space-time. There are $10$~of
these physical quantities and $10$~different components of $G_{\mu}^{\nu} - \frac{1}{2}g_{\mu}^{\nu} G$, so that
the identification is just sufficient. It will be noticed that this identification
gives a dynamical, not a kinematical definition of the velocity of matter
$u$, $v$, $w$; it is appropriate, for example, to the case of a rotating homogeneous
and continuous fly-wheel, in which there is no velocity of matter in the kinematical
sense, although a dynamical velocity is indicated by its gyrostatic
\index{Dynamical velocity}%
properties\footnotemark.\footnotetext
  {\Title{Space, Time and Gravitation,} p.~194.}
The connection with the ordinary kinematical velocity, which
\index{Kinematical velocity}%
determines the direction of the world-line of a particle in four dimensions, is
followed out in \SecRef{56}.

Contracting \Eq{(54.3)} by setting $\nu = \mu$, and remembering that $g_{\mu}^{\mu} = 4$, we have
\[
G = 8\pi T,
\Tag{(54.4)}
\]
so that an equivalent form of~\Eq{(54.3)} is
\[
G_{\mu}^{\nu} = -8\pi(T_{\mu}^{\nu} - \tfrac{1}{2} g_{\mu}^{\nu}T).
\Tag{(54.5)}
\]
When there is no material energy\hyp{}tensor this gives
\[
G_{\mu}^{\nu} = 0,
\]
which is equivalent to Einstein's law $G_{\mu\nu} = 0$ for empty space.

According to the new point of view Einstein's law of gravitation does not
impose any limitation on the basal structure of the world. $G_{\mu\nu}$~may vanish or
it may not. If it vanishes we say that space is empty; if it does not vanish
we say that momentum or energy is present; and our practical test whether
space is occupied or not---whether momentum and energy exist there---is the
test whether $G_{\mu\nu}$~exists or not\footnotemark.\footnotetext
  {We are dealing at present with mechanics only, so that we can scarcely discuss the part
  played by electromagnetic fields (light) in conveying to us the impression that space is occupied
  by something. But it may be noticed that the \emph{crucial} test is mechanical. A real image has the
  optical properties but not the mechanical properties of a solid body.}

Moreover it is not an accident that it should be this particular tensor
which is capable of being recognised by us. It is because its divergence
vanishes---because it satisfies the law of conservation---that it fulfils the
primary condition for being recognised as substantial. If we are to surround
ourselves with a perceptual world at all, we must recognise as substance that
which has some element of permanence. We may not be able to explain how
the mind recognises as substantial the world\hyp{}tensor $G_{\mu}^{\nu} - \frac{1}{2}g_{\mu}^{\nu} G$, but we can
see that it could not well recognise anything simpler. There are no doubt
minds which have not this predisposition to regard as substantial the things
which are permanent; but we shut them up in lunatic asylums.

The invariant
\begin{align*}
  T &= g_{\mu\nu} T^{\mu\nu} \\
  &= g_{\mu\nu} \cdot \rho_{0}\, \frac{dx_{\mu}}{ds}\, \frac{dx_{\nu}}{ds} \\
  &= \rho_{0},
\end{align*}
since
\[
g_{\mu\nu}\, dx_{\mu}\, dx_{\nu} = ds^{2}.
\]

Thus
\[
G = 8\pi T = 8\pi \rho_{0}.
\Tag{(54.6)}
\]

Einstein and de~Sitter obtain a naturally curved world by taking instead
of~\Eq{(54.3)}
\[
G_{\mu}^{\nu} - \tfrac{1}{2}g_{\mu}^{\nu}(G - 2\lambda) = -8\pi T_{\mu}^{\nu},
\Tag{(54.71)}
\]
where $\lambda$~is a constant. Since the divergence of~$g_{\mu}^{\nu}$ or of~$g^{\mu\nu}$ vanishes, the
divergence of this more general form will also vanish, and the laws of conservation
of mass and momentum are still satisfied identically. Contracting
\Eq{(54.71)}, we have
\[
G - 4\lambda = 8\pi T = 8\pi \rho_{0}.
\Tag{(54.72)}
\]
For empty space $G = 4\lambda$, and $T_{\mu}^{\nu} = 0$; and thus the equation reduces to
\[
G_{\mu}^{\nu} = \lambda g_{\mu}^{\nu},
\]
or
\[
G_{\mu\nu} = \lambda g_{\mu\nu},
\]
as in~\Eq{(37.4)}.

When account is taken of the stresses in continuous matter, or of the
\index{Invariant density (proper\hyp{}density)}%
molecular motions in discontinuous matter, the proper\hyp{}density of the matter
\index{Density!definitions of proper-}%
requires rather careful definition. There are at least three possible definitions
which can be justified; and we shall denote the corresponding quantities by
$\rho_{0}$, $\rho_{00}$,~$\rho_{000}$.

(1) We define
\[
\rho_{0} = T.
\]
By reference to~\Eq{(54.6)} it will be seen that this represents the sum of the
densities of the particles with different motions, \emph{each particle being referred
to axes with respect to which it is itself at rest}.

(2) We can sum the densities for the different particles referring them
all to axes which are at rest in the matter as a whole. The result is denoted
by~$\rho_{00}$. Accordingly
\[
\text{$\rho_{00} = T_{44}$ referred to axes at rest in the matter as a whole.}
\]

(3) If a perfect fluid is referred to axes with respect to which it is at rest,
the stresses $p_{xx}$, $p_{yy}$, $p_{zz}$ are each equal to the hydrostatic pressure~$p$. The
\index{Hydrostatic pressure}%
\index{Pressure!hydrostatic}%
energy\hyp{}tensor~\Eq{(53.5)} accordingly becomes
\[
T^{\mu\nu} = \begin{array}[t]{@{}c@{\quad}c@{\quad}c@{\quad}c@{}}
  p & 0 & 0 & 0\\
  0 & p & 0 & 0\\
  0 & 0 & p & 0\\
  0 & 0 & 0 & \rho_{00}\\
\end{array}
\]
Writing $\rho_{00} = \rho_{000} - p$, the pressure\hyp{}terms give a tensor~$-g^{\mu\nu} p$. Accordingly
we have have the tensor equation applicable to any coordinate\hyp{}system
\[
T^{\mu\nu} = \rho_{000}\, \frac{dx_{\mu}}{ds}\, \frac{dx_{\nu}}{ds} - g^{\mu\nu} p.
\Tag{(54.81)}
\]
Thus if the energy\hyp{}tensor is analysed into two terms depending respectively
on two invariants specifying the state of the fluid, we must take these invariants
to be $p$ and~$\rho_{000}$.

The three quantities are related by
\[
\rho_{0} = \rho_{00} - 3p = \rho_{000} - 4p.
\Tag{(54.82)}
\]

If a fluid is \emph{incompressible,} i.e.\ if the closeness of packing of the particles
is independent of~$p$, the condition must be that $\rho_{0}$~is constant\footnotemark.\footnotetext
  {Many writers seem to have defined incompressibility by the condition $\rho_{00} = \text{constant}$. This
\index{Incompressibility}%
  is surely a most misleading definition.}
Incompressibility
is concerned with constancy not of mass\hyp{}density but of particle\hyp{}density,
so that no account should be taken of increases of mass of the particles due
to motion relative to the centre of mass of the matter as a whole.

For a liquid or solid the stress does not arise entirely from molecular
motions, but is due mainly to direct repulsive forces between the molecules held
in proximity. These stresses must, of course, be included in the energy\hyp{}tensor
(which would otherwise not be conserved) just as the gaseous pressure is
included. It will be shown later that if these repulsive forces are Maxwellian
electrical forces they contribute nothing to~$\rho_{0}$, so that $\rho_{0}$~arises entirely from
the molecules individually (probably from the electrons individually) and is
independent of the circumstances of packing.

Since $\rho_{0}$~is the most useful of the three quantities in theoretical investigations
we shall in future call it the proper\hyp{}density (or invariant density)
without qualification.

\Section{55.}{The force}
\index{Force, covariant and contravariant components!expressed by $3$-index symbols}%

By \Eq{(51.2)} the equation $(T_{\mu}^{\nu})_{\nu} = 0$ becomes
\[
\frac{1}{\sqrt{-g}}\, \frac{\dd}{\dd x_{\nu}}(T_{\mu}^{\nu} \sqrt{-g})
= \{\mu\nu, \alpha\} T_{\alpha}^{\nu}.
\Tag{(55.1)}
\]
Let us choose coordinates so that $\sqrt{-g} = 1$; then
\[
\frac{\dd}{\dd x_{\nu}}T_{\mu}^{\nu} = \{\mu\nu, \alpha\} T_{\alpha}^{\nu}.
\Tag{(55.2)}
\]

In most applications the velocity of the matter is extremely small compared
with the velocity of light, so that on the right of this equation $T_{4}^{4} = \rho$ is
much larger than the other components of~$T_{\alpha}^{\nu}$. As a first approximation we
neglect the other components, so that
\[
\frac{\dd}{\dd x_{\nu}} T_{\mu}^{\nu} = \{\mu4, 4\} \rho.
\Tag{(55.3)}
\]
This will agree with classical mechanics~\Eq{(53.92)} if
\[
-X,\ -Y,\ -Z = \{14, 4\},\ \{24, 4\},\ \{34, 4\}.
\Tag{(55.4)}
\]

The $3$-index symbols can thus be interpreted as components of the field
of force. The three quoted are the leading components which act proportionately
to the mass or energy; the others, neglected in Newtonian mechanics,
are evoked by the momenta and stresses which form the remaining components
of the energy\hyp{}tensor.

The limitation $\sqrt{-g} = 1$ is not essential if we take account of the confusion
of tensor\hyp{}densities with tensors referred to at the end of \SecRef{49}. It will
be remembered that the force $(X, Y, Z)$ occurs because we attribute to our
mesh\hyp{}system an abstract Galilean geometry which is not the natural geometry.
Either inadvertently or deliberately we place ourselves in the position
of an observer who has mistaken his non\hyp{}Galilean mesh\hyp{}system for rectangular
coordinates and time. We therefore mistake the unit mesh for the unit of
natural volume, and the density of the energy\hyp{}tensor~$\mf{T}_{\mu}^{\nu}$ reckoned per unit
mesh is mistaken for the energy\hyp{}tensor itself~$T_{\mu}^{\nu}$ reckoned per unit natural
volume. For this reason the conservation of the supposed energy\hyp{}tensor
should be expressed analytically by $\dd\mf{T}_{\mu}^{\nu}/\dd x_{\nu} = 0$ and when a field of force
intervenes the equations of classical hydrodynamics should be written
\[
\frac{\dd}{\dd x_{\nu}} \mf{T}_{\mu}^{\nu}
= \mf{T}_{4}^{4}(-X, -Y, -Z, 0),
\Tag{(55.51)}
\]
the supposed density~$\rho$ being really the ``density\hyp{}density'' $\rho\sqrt{-g}$ or~$\mf{T}_{4}^{4}$\footnotemark.\footnotetext
  {It might seem preferable to avoid this confusion by immediately identifying the energy,
  momentum and stress with the components of~$\mf{T}_{\mu}^{\nu}$, instead of adopting the roundabout procedure
  of identifying them with~$T_{\mu}^{\nu}$ and noting that in practice $\mf{T}_{\mu}^{\nu}$~is inadvertently substituted. The
  inconvenience is that we do not always attribute abstract Galilean geometry to our coordinate\hyp{}system.
  For example, if polar coordinates are used, there is no tendency to confuse the mesh
  $dr\, d\theta\, d\phi$ with the natural volume $r^{2} \sin\theta\, dr\, d\theta\, d\phi$; in such a case it is much more convenient to
  take $T_{\mu}^{\nu}$ as the measure of the density of energy, momentum and stress. It is when by our
  attitude of mind we attribute abstract Galilean geometry to coordinates whose natural geometry
  is not accurately Galilean, that the automatic substitution of~$\mf{T}_{\mu}^{\nu}$ for the quantity intended to
  represent~$T_{\mu}^{\nu}$ occurs.}

Since \Eq{(55.1)}~is equivalent to
\[
\frac{\dd}{\dd x_{\nu}} \mf{T}_{\mu}^{\nu}
= \{\mu\nu, \alpha\} \mf{T}_{\alpha}^{\nu},
\Tag{(55.52)}
\]
the result~\Eq{(55.4)} follows irrespective of the value of~$\sqrt{-g}$.

The alternative formula~\Eq{(51.51)} may be used to calculate~$T_{\mu\nu}^{\nu}$, giving
\[
\frac{\dd}{\dd x_{\nu}} \mf{T}_{\mu}^{\nu}
= \tfrac{1}{2} \mf{T}^{\alpha\beta}\, \frac{\dd g_{\alpha\beta}}{\dd x_{\mu}}.
\Tag{(55.6)}
\]
Retaining on the right only~$\mf{T}^{44}$, we have by comparison with~\Eq{(55.51)}
\[
X,\ Y,\ Z = -\frac{1}{2}\, \frac{\dd g_{44}}{\dd x},\
-\frac{1}{2}\, \frac{\dd g_{44}}{\dd y},\
-\frac{1}{2}\, \frac{\dd g_{44}}{\dd z}.
\Tag{(55.7)}
\]
Hence, for a static coordinate\hyp{}system
\begin{align*}
  X\, dx + Y\, dy + Z\, dz
  &= -\frac{1}{2}\left(\frac{\dd g_{44}}{\dd x}\, dx + \frac{\dd g_{44}}{\dd y}\, dy + \frac{\dd g_{44}}{\dd z}\, dz\right) \\
  &= -\tfrac{1}{2} dg_{44},
\end{align*}
so that $X$, $Y$, $Z$ are derivable from a potential
\index{Potential!gravitational}%
\[
\Omega = -\tfrac{1}{2} g_{44} + \text{const.}
\]
Choosing the constant so that $g_{44} = 1$ when $\Omega = 0$
\[
g_{44} = 1 - 2\Omega.
\Tag{(55.8)}
\]
Special cases of this result will be found in~\Eq{(15.4)} and~\Eq{(38.8)}, $\Omega$~being the
potential of the centrifugal force and of the Newtonian gravitational force
respectively.

Let us now briefly review the principal steps in our new derivation of the
laws of mechanics and gravitation. We concentrate attention on the world\hyp{}tensor~$T_{\mu}^{\nu}$
\emph{defined} by
\[
T_{\mu}^{\nu} = -\frac{1}{8\pi} (G_{\mu}^{\nu} - \tfrac{1}{2} g_{\mu}^{\nu} G).
\]
The question arises how this tensor would be recognised in nature---what
names has the practical observer given to its components? We suppose
tentatively that when Galilean or natural coordinates are used $T_{4}^{4}$~is recognised
as the amount of mass or energy per unit volume, $T_{1}^{4}$, $T_{2}^{4}$, $T_{3}^{4}$ as the negative
momentum per unit volume, and the remaining components contain the
stresses according to the detailed specifications in~\Eq{(53.91)}. This can only be
tested by examining whether the components of~$T_{\mu}^{\nu}$ do actually obey the laws
which mass, momentum and stress are known by observation to obey. For
natural coordinates the empirical laws are expressed by $\dd T_{\mu}^{\nu}/\dd x_{\nu} = 0$, which is
satisfied because our tensor from its definition has been proved to satisfy
$(T_{\mu}^{\nu})_{\nu} = 0$ identically. When the coordinates are not natural, the identity
$T_{\mu\nu}^{\nu} = 0$ gives the more general law
\[
\frac{\dd}{\dd x_{\nu}} \mf{T}_{\mu}^{\nu}
= \frac{1}{2}\, \frac{\dd g_{\alpha\beta}}{\dd x_{\mu}}\, \mf{T}^{\alpha\beta}.
\]

We attribute an abstract Galilean geometry to these coordinates, and
should accordingly identify the components of~$T_{\mu}^{\nu}$ as before, just as though
the coordinates were natural; but owing to the resulting confusion of unit
mesh with unit natural volume, the tensor\hyp{}densities $\mf{T}_{1}^{4}$, $\mf{T}_{2}^{4}$, $\mf{T}_{3}^{4}$, $\mf{T}_{4}^{4}$ will now
be taken to represent the negative momentum and energy per unit volume.

In accordance with the definition of force as rate of change of momentum,
the quantity on the right will be recognised as the (negative) body-force
acting on unit volume, the three components of the force being given by
$\mu = 1$, $2$,~$3$. When the velocity of the matter is very small compared with the
velocity of light as in most ordinary problems, we need only consider on the
right the component~$\mf{T}^{44}$ or~$\rho$; and the force is then due to a field of acceleration
of the usual type with components $-\frac{1}{2}\dd g_{44}/\dd x_{1}$, $-\frac{1}{2}\dd g_{44}/\dd x_{2}$, $-\frac{1}{2}\dd g_{44}/\dd x_{3}$.
The potential~$\Omega$ of the field of acceleration is thus connected with~$g_{44}$ by the
relation $g_{44} = 1 - 2\Omega$. When this approximation is not sufficient there is no
simple field of acceleration; the acceleration of the matter depends not only
on its position but also on its velocity and even on its state of stress.
Einstein's law of gravitation for empty space $G_{\mu\nu} = 0$ follows at once from the
above identification of~$T_{\mu}^{\nu}$.

\Section{56.}{Dynamics of a particle}
\index{Dynamics of a particle}%
\index{Particle!dynamics of}%
\index{Particle!symmetry of}%

An isolated particle is a narrow tube in four dimensions containing a non-zero
energy\hyp{}tensor and surrounded by a region where the energy\hyp{}tensor is
zero. The tube is the world-line or track of the particle in space-time.
\index{World-line}%

The momentum and mass of the particle are obtained by integrating~$\mf{T}_{\mu}^{4}$
over a three\hyp{}dimensional volume; if the result is written in the form
\[
-Mu,\ -Mv,\ -Mw,\ M,
\]
then $M$~is the mass (relative to the coordinate system), and $(u, v, w)$ is the
\emph{dynamical velocity} of the particle, i.e.\ the ratio of the momenta to the mass.
\index{Dynamical velocity}%

The \emph{kinematical velocity} of the particle is given by the direction of the
\index{Kinematical velocity}%
tube in four dimensions, viz.\ $\left(\dfrac{dx_{1}}{dx_{4}}, \dfrac{dx_{2}}{dx_{4}}, \dfrac{dx_{3}}{dx_{4}}\right)$ along the tube. For completely
continuous matter there is no division of the energy\hyp{}tensor into tubes and the
notion of kinematical velocity does not arise.

It does not seem to be possible to deduce without special assumptions that
the dynamical velocity of a particle is equal to the kinematical velocity. The
law of conservation merely shows that $(Mu, Mv, Mw, M)$ is constant along the
tube when no field of force is acting; it does not show that the direction of
this vector is the direction of the tube.

I think there is no doubt that in nature the dynamical and kinematical
velocities are the same; but the reason for this must be sought in the symmetrical
properties of the ultimate particles of matter. If we assume as in
\SecRef{38} that the particle is the nucleus of a symmetrical field, the result becomes
obvious. A symmetrical particle which is kinematically at rest cannot have
any momentum since there is no preferential direction in which the momentum
could point; in that case the tube is along the $t$-axis, and so also is the vector
$(0, 0, 0, M)$. It is not necessary to assume complete spherical symmetry;
\index{Symmetry!of a particle}%
three perpendicular planes of symmetry would suffice. The ultimate particle
may for example have the symmetry of an anchor\hyp{}ring.

It might perhaps be considered sufficient to point out that a ``particle'' in
practical dynamics always consists of a large number of ultimate particles or
atoms, so that the symmetry may be merely a consequence of haphazard
averages. But we shall find in \SecRef{80}, that the same difficulty occurs in understanding
how an electrical field affects the direction of the world-line of a
charged particle, and the two problems seem to be precisely analogous. In
the electrical problem the motions of the ultimate particles (electrons) have
been experimented on individually, and there has been no opportunity of
introducing the symmetry by averaging. I think therefore that the symmetry
exists in each particle independently.

It seems necessary to suppose that it is an essential condition for the
existence of an actual particle that it should be the nucleus of a \emph{symmetrical}
field, and its world-line must be so directed and curved as to assure this
symmetry. A satisfactory explanation of this property will be reached in \SecRef{66}.

With this understanding we may use the equation~\Eq{(53.1)}, involving kinematical
velocity,
\[
T^{\mu\nu} = \rho_{0}\, \frac{dx_{\mu}}{ds}\, \frac{dx_{\nu}}{ds},
\Tag{(56.1)}
\]
in place of~\Eq{(53.4)}, involving dynamical velocity. From the identity $T_{\nu}^{\mu\nu} = 0$, we
have by~\Eq{(51.41)}
\[
\frac{\dd}{\dd x_{\nu}} (T^{\mu\nu} \sqrt{-g})
= -\{\alpha\nu, \mu\} T^{\alpha\nu} \sqrt{-g}.
\Tag{(56.2)}
\]

Integrate this through a very small four\hyp{}dimensional volume. The left-hand
side can be integrated once, giving
\begin{multline*}
  \left[\iiint T^{\mu1} \sqrt{-g}\, dx_{2}\, dx_{3}\, dx_{4}
    + \iiint T^{\mu2} \sqrt{-g}\, dx_{1}\, dx_{3}\, dx_{4} + \cdots\right] \\
  = -\iiiint \{\alpha\nu, \mu\} T^{\alpha\nu} \cdot \sqrt{-g}\, d\tau.
  \Tag{(56.3)}
\end{multline*}

Suppose that in this volume there is only a single particle, so that the
energy\hyp{}tensor vanishes everywhere except in a narrow tube. By~\Eq{(56.1)} the
quadruple integral becomes
\[
-\iiiint \{\alpha\nu, \mu\} \frac{dx_{\alpha}}{ds}\, \frac{dx_{\nu}}{ds}\, \rho_{0} \sqrt{-g}\, d\tau
= -\{\alpha\beta, \mu\} \frac{dx_{\alpha}}{ds}\, \frac{dx_{\beta}}{ds}\, m\, ds,
\Tag{(56.4)}
\]
since $\rho_{0} \sqrt{-g}\, d\tau = \rho_{0}\, dW \cdot ds = dm \cdot ds$, where $dm$~is the proper\hyp{}mass.

On the left the triple integrals vanish except at the two points where the
world-line intersects the boundary of the region. For convenience we draw
the boundary near these two points in the planes $dx_{1} = 0$, so that only the first
of the four integrals survives. The left-hand side of~\Eq{(56.3)} becomes
\[
\left[\iiint \rho_{0} \sqrt{-g}\, \frac{dx_{\mu}}{ds}\, \frac{dx_{1}}{ds}\, dx_{2}\, dx_{3}\, dx_{4}\right],
\Tag{(56.51)}
\]
the bracket denoting the difference at the two ends of the world-line.

The geometrical volume of the oblique cylinder cut off from the tube by
sections $dx_{2}\, dx_{3}\, dx_{4}$ at a distance apart~$ds$ measured along the tube is
\[
\frac{dx_{1}}{ds} \cdot ds\, dx_{2}\, dx_{3}\, dx_{4}.
\]

Multiplying by $\rho_{0} \sqrt{-g}$ we get the amount of~$\rho_{0}$ contained\footnotemark,\footnotetext
  {The amount of density in a four\hyp{}dimensional volume is, of course, not the mass but a
  quantity of dimensions $\text{mass} \times \text{time}$.}
which is~$dm\, ds$.
Hence \Eq{(56.51)}~reduces to
\[
\left[m\, \frac{dx_{\mu}}{ds}\right].
\]

The difference at the two limits is
\[
\frac{d}{ds} \left(m\, \frac{dx_{\mu}}{ds}\right) ds,
\Tag{(56.52)}
\]
where $ds$~is now the length of track between the two limits as in~\Eq{(56.4)}.

By \Eq{(56.4)} and~\Eq{(56.52)} the equation reduces to
\[
\frac{d}{ds} \left(m\, \frac{dx_{\mu}}{ds}\right)
= -m \{\alpha\beta, \mu\}\, \frac{dx_{\alpha}}{ds}\, \frac{dx_{\beta}}{ds}.
\Tag{(56.6)}
\]

Provided that $m$~is constant this gives the equations of a geodesic \Eq{(28.5)},
showing that the track of an isolated particle is a geodesic. The constancy of~$m$
can be proved formally as follows---

From~\Eq{(56.6)}
\begin{align*}
  mg_{\mu\nu}\, \frac{dx_{\nu}}{ds} \cdot \frac{d}{ds} \left(m\, \frac{dx_{\mu}}{ds}\right)
  &= -m^{2} [\alpha\beta, \nu]\, \frac{dx_{\nu}}{ds}\, \frac{dx_{\alpha}}{ds}\, \frac{dx_{\beta}}{ds} \\
  &= -\tfrac{1}{2} m^{2} \, \frac{\dd g_{\alpha\nu}}{\dd x_{\beta}}\, \frac{dx_{\beta}}{ds}\, \frac{dx_{\alpha}}{ds}\, \frac{dx_{\nu}}{ds} \\
  &= -\tfrac{1}{2} m^{2} \, \frac{dg_{\alpha\nu}}{ds}\, \frac{dx_{\alpha}}{ds}\, \frac{dx_{\nu}}{ds} \\
  &= -\tfrac{1}{2} m^{2} \, \frac{dg_{\mu\nu}}{ds}\, \frac{dx_{\mu}}{ds}\, \frac{dx_{\nu}}{ds}.
\end{align*}
Adding the same equation with $\mu$ and $\nu$ interchanged
%[** TN: Not broken in the original]
\begin{multline*}
g_{\mu\nu} \cdot m\, \frac{dx_{\nu}}{ds} \cdot \frac{d}{ds} \left(m\, \frac{dx_{\mu}}{ds}\right)
+ g_{\mu\nu} \cdot m\, \frac{dx_{\mu}}{ds} \cdot \frac{d}{ds} \left(m\, \frac{dx_{\nu}}{ds}\right) \\
+ m \cdot \frac{dx_{\mu}}{ds} \cdot m\, \frac{dx_{\nu}}{ds} \cdot \frac{dg_{\mu\nu}}{ds} = 0
\end{multline*}
or
\[
\frac{d}{ds} \left(g_{\mu\nu} \cdot m\, \frac{dx_{\mu}}{ds} \cdot m\, \frac{dx_{\nu}}{ds}\right) = 0.
\]
By~\Eq{(22.1)} this gives $dm^{2}/ds = 0$. Accordingly the invariant mass of an isolated
particle remains constant.

The present proof does not add very much to the argument in \SecRef{17} that
the particle follows a geodesic because that is the only track which is absolutely
defined. Here we postulate symmetrical properties for the particle
(referred to proper\hyp{}coordinates); this has the effect that there is no means of
fixing a direction in which it could deviate from a geodesic. For further
enlightenment we must wait until Chapter~\ChapNum{V}\@.

On reconsideration I think that it is unnecessary to assume that a particle has symmetrical properties in order
to prove that the dynamical velocity is equal to the kinematical velocity.
Possibly some limitations must be imposed on the structure of the particle, beyond the definition in the text,
viz. that a particle is a tube containing non\hyp{}vanishing energy\hyp{}tensor surrounded by a region of zero energy\hyp{}tensor;
but these limitations will be much less stringent than the assumption of symmetry.

In natural coordinates~$\dd\mf{T}^\nu_\mu/\dd x_\nu = 0$, so that
\[
\frac{\dd\mf{T}^1_4}{\dd x_1} + \frac{\dd\mf{T}^2_4}{\dd x_2} + \frac{\dd\mf{T}^2_4}{\dd x_2} + \frac{\dd\mf{T}^2_4}{\dd x_2} = 0,
\]
which may be compared with the equation in elementary electrostatics
\[
\frac{\dd E_x}{\dd x} + \frac{\dd E_y}{\dd y} + \frac{\dd E_z}{\dd z} = 0.
\]
The latter equation leads by Gauss's theorem to the conception of unit tubes of force, the whole space being divided
into tubes running in the direction of~$(E_x,E_y,E_z)$ and the flux of this vector across any section of a tube
remaining constant.
Similarly in four dimensions we shall have unit tubes of~$\mf{T}^\mu_4$ running in the direction of the vector and
containing constant flux.
Obviously such a tube cannot stray into a region where~$\mf{T}^\mu_4 = 0$, since the constancy of flux could not
then be maintained.
Hence the unit tubes must run along inside the world-tube bounding the particle, and for an infinitesimal particle
their direction cannot deviate appreciably from the direction of the world-tube.
But the unit tubes have the direction of~$\mf{T}^\mu_4$, i.e.\ the dynamical velocity, and the world-tube has the
direction of~$dx_\mu/ds$, i.e.\ the kinematical velocity.

I believe this argument is unassailable if we assume that every portion of the particle has the same dynamical
velocity, so that the unit tubes run parallel to one another.
In the more general case complications are conceivable which require fuller discussion\footnotemark,\footnotetext
   {The worst complications are avoided if we refuse to admit \emph{negative} mass.
     This prevents the tubes from doubling back.}
e.g.\ the tubes may spiral round inside the world-tube in a screw of narrow thread.
Or the world\hyp{}curvature inside the particle may be so large and variable that natural coordinates are inadmissible.
I think, however, that few if any of these cases will prove to be genuine exceptions, when the dynamical velocity
of the separate elements has been averaged over the particle.

\Section[Equality of gravitational and inertial mass]{57.}{Equality of gravitational and inertial mass. Gravitational waves}
\index{Gravitation, Newtonian constant of}%
\index{Inertial mass}%
\index{Mass!gravitational and inertial}%

The term gravitational mass can be used in two senses; it may refer to
\Item{(a)}~the response of a particle to a gravitational field of force, or \Item{(b)}~to its
power of producing a gravitational field of force. In the sense~\Item{(a)} its identity
with inertial mass is axiomatic in our theory, the separation of the field of
force from the inertial field being dependent on our arbitrary choice of
an abstract geometry. We accordingly use the term exclusively in the sense~\Item{(b)},
and we have shown in \SecRefs{38}, \SecNum{39} that the constant of integration~$m$ represents
the gravitational mass. But in the present discussion the $\rho_{0}$ which
occurs in the tensor~$T_{\mu\nu}$ refers to inertial mass defined by the conservation of
energy and momentum. The connection is made \Foreign{via} equation~\Eq{(54.3)}, where
on the left the mass appears in terms of~$g_{\mu\nu}$, i.e.\ in terms of its power of
exerting (or being accompanied by) a gravitational field; and on the right it
appears in the energy\hyp{}tensor which comprises~$\rho_{0}$ according to~\Eq{(53.1)}. But it
will be remembered that the factor~$8\pi$ in~\Eq{(54.3)} was chosen arbitrarily, and
this must now be justified\footnotemark.\footnotetext
  {It has been justified in \SecRef{46}, which has a close connection with the present paragraph; but
  the argument is now proceeding in the reverse direction.}
This coefficient of proportionality corresponds to
the Newtonian constant of gravitation.

The proportionality of gravitational and inertial mass, and the ``constant
of gravitation'' which connects them, are conceptions belonging to the approximate
Newtonian scheme, and therefore presuppose that the gravitational
fields are so weak that the equations can be treated as linear. For more
intense fields the Newtonian terminology becomes ambiguous, and it is idle
to inquire whether the constant of gravitation really remains constant when
the mass is enormously great. Accordingly we here discuss only the limiting
case of very weak fields, and set
\[
g_{\mu\nu} = \delta_{\mu\nu} + h_{\mu\nu},
\Tag{(57.1)}
\]
where $\delta_{\mu\nu}$~represents Galilean values, and $h_{\mu\nu}$~will be a small quantity of the
first order whose square is neglected. The derivatives of the~$g_{\mu\nu}$ will be small
quantities of the first order.

We have, correct to the first order,
%[** TN: Not broken in the original]
\begin{align*}
  G_{\mu\nu}
  &= g^{\sigma\rho} B_{\mu\nu\sigma\rho} \\
  &= \tfrac{1}{2}g^{\sigma\rho} \left(
    \frac{\dd^{2} g_{\mu\nu}}{\dd x_{\sigma}\, \dd x_{\rho}}
  + \frac{\dd^{2} g_{\sigma\rho}}{\dd x_{\mu}\, \dd x_{\nu}}
  - \frac{\dd^{2} g_{\mu\sigma}}{\dd x_{\nu}\, \dd x_{\rho}}
  - \frac{\dd^{2} g_{\nu\rho}}{\dd x_{\mu}\, \dd x_{\sigma}}\right)
\Tag{(57.2)}
\end{align*}
by~\Eq{(34.5)}.

We shall try to satisfy this by breaking it up into two equations
\[
G_{\mu\nu}
= \tfrac{1}{2}g^{\sigma\rho}\, \frac{\dd^{2} g_{\mu\nu}}{\dd x_{\sigma}\, \dd x_{\rho}}
\Tag{(57.31)}
\]
and
\[
0 = g^{\sigma\rho} \left(\frac{\dd^{2} g_{\sigma\rho}}{\dd x_{\mu}\, \dd x_{\nu}}
  - \frac{\dd^{2} g_{\mu\sigma}}{\dd x_{\nu}\, \dd x_{\rho}}
  - \frac{\dd^{2} g_{\nu\rho}}{\dd x_{\mu}\, \dd x_{\sigma}}\right).
\Tag{(57.32)}
\]

The second equation becomes, correct to the first order,
\begin{align*}
  0 &= \delta^{\sigma\rho} \left(
    \frac{\dd^{2} h_{\sigma\rho}}{\dd x_{\mu}\, \dd x_{\nu}}
  - \frac{\dd^{2} h_{\mu\sigma}}{\dd x_{\nu}\, \dd x_{\rho}}
  - \frac{\dd^{2} h_{\nu\rho}}{\dd x_{\mu}\, \dd x_{\sigma}}\right) \\
  &= \frac{\dd^{2} h}{\dd x_{\mu}\, \dd x_{\nu}}
  - \frac{\dd^{2} h_{\mu}^{\sigma}}{\dd x_{\nu}\, \dd x_{\rho}}
  - \frac{\dd^{2} h_{\nu}^{\rho}}{\dd x_{\mu}\, \dd x_{\sigma}},
\end{align*}
where
\[
h_{\mu}^{\rho} = \delta^{\sigma\rho} h_{\mu\sigma};\quad
h = h_{\rho}^{\rho} = \delta^{\sigma\rho} h_{\sigma\rho}.
\]

This is satisfied if
\[
\frac{\dd h_{\mu}^{\alpha}}{\dd x_{\alpha}}
= \frac{1}{2}\, \frac{\dd h}{\dd x_{\mu}}
\]
or
\[
\frac{\dd}{\dd x_{\alpha}} (h_{\mu}^{\alpha} - \tfrac{1}{2} \delta_{\mu}^{\alpha} h) = 0.
\Tag{(57.4)}
\]

The other equation~\Eq{(57.31)} may be written
\[
\Wave h_{\mu\nu} = 2G_{\mu\nu}
\]
or
\[
\Wave h_{\mu}^{\alpha} = 2G_{\mu}^{\alpha},
\]
showing that $G_{\mu}^{\alpha}$~is a small quantity of the first order. Hence
\begin{align*}
  \Wave (h_{\mu}^{\alpha} - \tfrac{1}{2} \delta_{\mu}^{\alpha} h)
  &= 2(G_{\mu}^{\alpha} - \tfrac{1}{2} g_{\mu}^{\alpha} G) \\
  &= -16\pi T_{\mu}^{\alpha}.
  \Tag{(57.5)}
\end{align*}

This ``equation of wave\hyp{}motion'' can be integrated. Since we are dealing
with small quantities of the first order, the effect of the deviations from
Galilean geometry will only affect the results to the second order; accordingly
the well-known solution\footnote
  {Rayleigh, \Title{Theory of Sound,} vol.~\Vol{II}, p.~104, equation~(3).}
may be used, viz.\
\[
h_{\mu}^{\alpha} - \tfrac{1}{2} \delta_{\mu}^{\alpha} h
= \frac{1}{4\pi} \int \frac{(-16\pi T_{\mu}^{\alpha})'\, dV'}{r'},
\Tag{(57.6)}
\]
the integral being taken over each element of space\hyp{}volume~$dV'$ at a coordinate
distance~$r'$ from the point considered and at a time $t - r'$, i.e.\ at a time
such that waves propagated from~$dV'$ with unit velocity can reach the point
at the time considered.

If we calculate from~\Eq{(57.6)} the value of
\[
\frac{\dd}{\dd x_{\alpha}} (h_{\mu}^{\alpha} - \tfrac{1}{2} \delta_{\mu}^{\alpha} h),
\]
the operator~$\dd/\dd x_{\alpha}$ indicates a displacement in space and time of the point
considered, involving a change of~$r'$. We may, however, keep $r'$ constant on the
right-hand side and displace to the same extent the element~$dV'$ where $(T_{\mu}^{\alpha})'$~is
calculated. Thus
\[
\frac{\dd}{\dd x_{\alpha}} (h_{\mu}^{\alpha} - \tfrac{1}{2} \delta_{\mu}^{\alpha} h)
= -4\int \left\{\frac{\dd}{\dd x_{\alpha}}(T_{\mu}^{\alpha})\right\}' \frac{dV'}{r'}.
\]
But by~\Eq{(55.2)} $\dd T_{\mu}^{\alpha}/\dd x_{\alpha}$ is of the second order of small quantities, so that to our
approximation \Eq{(57.4)}~is satisfied.

The result is that
\[
\Wave h_{\mu\nu} = 2G_{\mu\nu}
\Tag{(57.7)}
\]
satisfies the gravitational equations correctly to the first order, because both
the equations into which we have divided~\Eq{(57.2)} then become satisfied. Of
course there may be other solutions of~\Eq{(57.2)}, which do not satisfy \Eq{(57.31)}~and
\Eq{(57.32)} separately.

For a static field \Eq{(57.7)}~reduces to
\begin{align*}
  -\nabla^2 h_{\mu\nu}
  &= 2G_{\mu\nu} \\
  &= -16\pi (T_{\mu\nu} - \tfrac{1}{2} \delta_{\mu\nu}T)
  \quad\text{by~\Eq{(54.5)}.}
\end{align*}
Also for matter at rest $T = T_{44} = \rho$ (the \emph{inertial} density) and the other components
of~$T_{\mu\nu}$ vanish; thus
\[
\nabla^2 (h_{11}, h_{22}, h_{33}, h_{44}) = 8\pi\rho(1, 1, 1, 1).
\]
For a single particle the solution of this equation is well known to be
\[
h_{11}, h_{22}, h_{33}, h_{44} = -\frac{2m}{r}.
\]
Hence by~\Eq{(57.1)} the complete expression for the interval is
\[
ds^{2} = -\left(1 + \frac{2m}{r}\right)(dx^{2} + dy^{2} + dz^{2})
  + \left(1 - \frac{2m}{r}\right) dt^{2},
\Tag{(57.8)}
\]
agreeing with~\Eq{(46.15)}. But $m$~as here introduced is the inertial mass and not
\index{Gravitational mass of sun!equality with inertial mass}%
\index{Inertial mass!equal to gravitational mass}%
\index{Mass!gravitational and inertial}%
merely a constant of integration. We have shown in \SecRefs{38}, \SecNum{39} that the~$m$ in~\Eq{(46.15)}
is the gravitational mass reckoned with constant of gravitation unity.
Hence we see that inertial mass and gravitational mass are equal and expressed
in the same units, when the constant of proportionality between the
world\hyp{}tensor and the physical\hyp{}tensor is chosen to be~$8\pi$ as in~\Eq{(54.3)}.

In empty space \Eq{(57.7)}~becomes
\[
\Wave h_{\mu\nu} = 0,
\]
showing that the deviations of the gravitational potentials are propagated
as waves with unit velocity, i.e.\ the velocity of light (\SecRef{30}). But it must be
\index{Waves!gravitational}%
remembered that this representation of the propagation, though always permissible,
\index{Propagation!of gravitational waves}%
is not unique. In replacing \Eq{(57.2)} by \Eq{(57.31)} and \Eq{(57.32)}, we introduce
a restriction which amounts to choosing a special coordinate\hyp{}system. Other
solutions of~\Eq{(57.2)} are possible, corresponding to other coordinate\hyp{}systems.
All the coordinate\hyp{}systems differ from Galilean coordinates by small quantities
of the first order. The potentials~$g_{\mu\nu}$ pertain not only to the gravitational
influence which has objective reality, but also to the coordinate\hyp{}system which
we select arbitrarily. We can ``propagate'' coordinate\hyp{}changes with the
\emph{speed of thought,} and these may be mixed up at will with the more dilatory
propagation discussed above. There does not seem to be any way of distinguishing
a physical and a conventional part in the changes of the~$g_{\mu\nu}$.

The statement that in the relativity theory gravitational waves are propagated
with the speed of light has, I believe, been based entirely on the
foregoing investigation; but it will be seen that it is only true in a very
conventional sense. If coordinates are chosen so as to satisfy a certain condition
which has no very clear geometrical importance, the speed is that of
light; if the coordinates are slightly different the speed is altogether different
from that of light. The result stands or falls by the choice of coordinates and,
so far as can be judged, the coordinates here used were purposely introduced
in order to obtain the simplification which results from representing the
propagation as occurring with the speed of light. The argument thus follows
a vicious circle.

Must we then conclude that the speed of propagation of gravitation is
necessarily a conventional conception without absolute meaning? I think not.
The speed of gravitation is quite definite; only the problem of determining
it does not seem to have yet been tackled correctly. To obtain a speed independent
of the coordinate\hyp{}system chosen, we must consider the propagation
not of a world\hyp{}tensor but of a world\hyp{}invariant. The simplest world\hyp{}invariant
for this purpose is $B_{\mu\nu\sigma}^{\epsilon} B_{\epsilon}^{\mu\nu\sigma}$, since $G$~and $G_{\mu\nu}G^{\mu\nu}$ vanish in empty space. It is
scarcely possible to treat of the propagation of an isolated pulse of gravitational
influence, because there seems to be no way of starting a sudden pulse
without calling in supernatural agencies which violate the equations of
mechanics. We may consider the regular train of waves caused by the earth
in its motion round the sun. At a distant point in the ecliptic $B_{\mu\nu\sigma}^{\epsilon} B_{\epsilon}^{\mu\nu\sigma}$ will
vary with an annual periodicity; if it has a maximum or minimum value at
the instant when the earth is \emph{seen} to transit the sun, the inference is that the
wave of disturbance has travelled to us at the same speed as the light. (It
may perhaps be objected that there is no proof that the disturbance has been
propagated from the earth; it might be a stationary wave permanently
located round the sun which is as much the cause as the effect of the earth's
annual motion. I do not think the objection is valid, but it requires examination.)
There does not seem to be any grave difficulty in treating this problem;
and it deserves investigation.

Further light has been obtained on the problem whether the propagation of gravitation with the fundamental velocity
is more than a conventional representation.
We can show that the absolute disturbance, measured independently of the particular coordinate\hyp{}system employed above,
is propagated with the velocity of light.

Let us make a small transformation of the coordinate\hyp{}system, viz.
\[
x_\alpha = g^\alpha_\mu x'_\mu + \xi_\alpha,
\]
where the~$\xi_\alpha$ are small quantities of the first order, i.e.\ of the same order as~$h_{\mu\nu}$.
Then by~\Eq{(23.22)}
\begin{align*}
\delta_{\mu\nu} + h'_{\mu\nu} & = (\delta_{\alpha\beta} + h_{\alpha\beta})
                                  \left(g^\alpha_\mu + \frac{\dd\xi_\alpha}{\dd x'_\mu}\right)\\
                              & = \delta_{\mu\nu} + h_{\mu\nu} + \delta_{\mu\beta}\frac{\dd\xi_\beta}{\dd x'_\nu} +
                                  \delta_{\alpha\nu}\frac{\dd\xi_\alpha}{\dd x'_\mu},
\end{align*}
correct to the first order.
Hence the difference between~$h'_{\mu\nu}$ and~$h_{\mu\nu}$ is of the same order as the quantities themselves,
and the law of propagation of~$h_{\mu\nu}$ will not apply even approximately to~$h'_{\mu\nu}$.

Contrast this with the transformation of the Riemann\hyp{}Christoffel tensor, which is also of the first order of small
quantities
\begin{align*}
B'_{\mu\nu\sigma\rho} & = B_{\alpha\beta\gamma\delta}
                              \left(g^\alpha_\mu    + \frac{\dd\xi_\alpha}{\dd x'_\mu}\right)
                              \left(g^\beta_\nu     + \frac{\dd\xi_\beta} {\dd x'_\nu}\right)
                              \left(g^\gamma_\sigma + \frac{\dd\xi_\gamma}{\dd x'_\sigma}\right)
                              \left(g^\delta_\rho   + \frac{\dd\xi_\delta}{\dd x'_\rho}\right)\\
                      & = B_{\mu\nu\sigma\rho},
\end{align*}
correct to the first order.
Accordingly the law of propagation of~$B_{\mu\nu\sigma\rho}$ will apply approximately to~$B'_{\mu\nu\sigma\rho}$.
(We do not expect it to apply accurately since the velocity of light is altered to the first order by the
transformation.)

Hence, whereas the propagation of~$h_{\mu\nu}$ with the velocity of light is the property of a particular
coordinate\hyp{}system, the propagation of~$B_{\mu\nu\sigma\rho}$ with this speed is general.

It is instructive to consider this problem in detail for the case of plane gravitational waves.
We consider plane waves travelling with velocity~$V$ in the direction~$x_1$.
The coefficients~$h_{\mu\nu}$ can be grouped so as to correspond to three kinds of waves, which can exist
independently of one another, viz.
%\begin{align*}% XXX fix this alignment!
%& h_{22}, h_{23}, h_{33}         & \text{transverse-transverse (TT) waves,}\\
%& h_{12}, h_{24}, h_{13}, h_{34} & \text{longitudinal-transverse (LT) waves,}\\
%& h_{11}, h_{14}, h_{44}         & \text{longitudinal-longitudinal (LL) waves.}\\
%\end{align*}
\begin{align*}% XXX fix this alignment!
& \text{transverse-transverse (TT) waves,}     & h_{22}, h_{23}, h_{33}\\
& \text{longitudinal-transverse (LT) waves,}   & h_{12}, h_{24}, h_{13}, h_{34}\\
& \text{longitudinal-longitudinal (LL) waves.} & h_{11}, h_{14}, h_{44}\\
\end{align*}

The condition for empty space~$G_{\mu\nu} = 0$ leads to the following set of equations which must be
satisfied\footnotemark:\footnotetext
      {\Title{Proc.\ Roy.\ Soc.}\ 102~A. p.~268.}
\[
h_{22} + h_{33} = 0,\Tag{(57.91)}
\]
\[
(1-V^2)(h_{22}, h_{23}, h_{33}) = 0,\Tag{(57.92)}
\]
\[
h_{24} = V h_{12}; \quad h_{34} = V h_{13},\Tag{(57.93)}
\]
\[
h_{44} - 2 V h_{14} + V^2 h_{11} = 0.\Tag{(57.94)}
\]
It follows from~\Eq{(57.92)} that for $TT$ waves~$V=1$, so that these waves travel with the velocity of light.
For the other two classes of waves there is no reason why~$V$ should be unity.

To understand the nature of $LT$ and $LL$ waves which do not travel with the velocity of light we
suppose~$h_{22}, h_{23}, h_{33} = 0$, so that no $TT$ waves are present.
Then in consequence of conditions~\Eq{(57.93)} and~\Eq{(57.94)} it is found that the Riemann\hyp{}Christoffel
tensor vanishes altogether.
Accordingly space-time is flat, and no absolute disturbance is occurring.
$LT$ and $LL$ waves are spurious;
they are merely sinuosities of our coordinate\hyp{}system.
They exist, not in the world, but in our mental attitude, and the only speed relevant to their propagation is
the ``speed of thought.'' $TT$ waves contribute to the Riemann\hyp{}Christoffel tensor and involve a disturbance of the
curvatures of space-time; we have seen that these genuine waves have the speed of light.

The special coordinate\hyp{}system used above does not necessarily eliminate all $LL$ and $LT$ waves,
but it permits them only if they travel with the speed of light.
Spurious waves with this speed can take advantage of their resemblance to the genuine waves so as to slip
through the censorship.

If we group the coefficients of $TT$ waves in the triad,~$h_{22}+h_{33},h_{22}-h_{33},h_{23}$, equation~\Eq{(57.91)}
shows that the first of these is a type of wave which cannot exist in empty space.
This is because such a wave carries energy (real energy~$T^\nu_\mu$ not the pseudo\hyp{}energy~$\mf{t}^\nu_\mu$
carried by all $TT$ waves), and space containing real energy is not to be regarded as empty.
Light-waves and other kinds of electromagnetic waves belong to this class and involve
a propagation of~$h_{22}+h_{33}$.

A spinning rod sets up a train of gravitational waves which travel away towards infinity.
The interesting question arises whether these waves will carry away the energy of the rod so that it will
gradually come to rest of its own accord.
The analogous problem of spontaneous loss of energy of rotation of a double star is of considerable
astronomical interest.
The double star problem is still unsolved; but the result for a spinning rod, or for any rotating material
bound together by cohesive force, has been obtained by Einstein
(\Title{Berlin. Sitzungsberichte, 1916, p.~688; 1918, p.~154}).
Reference may also be made to the author's discussion
(\Title{Proc.\ Roy.\ Soc.\ 102~A, p.~268}) which in principle follows Einstein's method and except for a factor $2$
confirms his calculation.

The result is that a rod of moment of inertia~$I$, spinning with angular velocity~$\omega$, loses energy at the rate
\[
\frac{32}{5} k I^2 \omega^6
\Tag{(57.101)}
\]
per unit time, where~$k = 1$ for gravitational units and~$k=2.7\cdot10^{-60}$ for C.G.S. units.
The rate of decay of the rotation is in all practical cases exceedingly small.

The gravitational waves constitute a genuine disturbance of space-time, but their energy, represented by the
pseudo\hyp{}tensor~$\mf{t}^\nu_\mu$, is regarded as an analytical fiction as will be explained in~\SecRef{59}.
In Einstein's original method the outward flow of this pseudo\hyp{}energy is calculated.
Criticism was directed against his investigation owing to the employment of this fiction,
but Einstein had no difficulty in defending its validity.
We may, however, look at the problem from another point of view which ignores the fate of the lost energy,
and has a peculiar historic interest of its own.
If gravitation is not propagated instantaneously the lag may cause tangential components of the force to occur,
so that there will be a couple presumably opposing the rotation.
 Laplace anticipated that if gravitation were propagated with the speed of light this disturbing couple
would be large enough to be appreciable in astronomical systems, and deduced from its absence that gravitation
must have a much greater speed.
We now know that the first order effect which Laplace expected is compensated;
but the loss of energy~\Eq{(57.101)} is actually the residual Laplace effect of the third order of small quantities,
as determined by modern theory.
The rod comes to rest because, taking account of the propagation from one end to the other,
the gravitational attraction of its particles on one another is not exactly in the line of the rod and thus
creates a couple destroying the rotation---in short the action and reaction are not equal and opposite.
The following new deduction of~\Eq{(57.101)}, which is somewhat shorter than the investigations above quoted,
brings out this aspect of the problem.

Setting~$\mu=4$ in~\Eq{(55.6)}
\[
\frac{\dd\mf{T}^\nu_4}{\dd x_\nu} = \tfrac{1}{2}\mf{T}^{\alpha\beta}\frac{g_{\alpha\beta}}{\dd t}.
\]
Hence integrating over a three\hyp{}dimensional region enclosing the rod
\[
\frac{\dd}{\dd t} \int\mf{T}^4_4 dV = \tfrac{1}{2}\int\mf{T}^{\alpha\beta}\frac{\dd g_{\alpha\beta}}{\dd t} dV,
\Tag{(57.102)}
\]
since the other terms on the left yield surface integrals which vanish because the boundary does not pass through
matter.
Equation~\Eq{(57.102)} expresses the rate of change of material energy~$T^4_4$ within the region,
i.e.\ in the rod which is the only material system there.

In order to calculate the value of~$\dd g_{\alpha\beta}/\dd t$ to be substituted in~\Eq{(57.102)} we use~\Eq{(57.7)}
\[
\Wave h_{\alpha\beta} = 2 G_{\alpha\beta} = -16\pi(T_{\alpha\beta} - \tfrac{1}{2}\delta_{\alpha\beta} T).
\Tag{(57.103)}
\]
The solution of this wave\hyp{}equation is studied fully in~\SecRef{74}(d).
We have by~\Eq{(74.71)}
\[
h_{\alpha\beta} = -4 \int\left[\frac{T'_{\alpha\beta} - \tfrac{1}{2}\delta_{\alpha\beta} T'}{r(1-v_r)}\right] dV'.
\Tag{(57.104)}
\]
Here the square bracket is used to indicate appropriately antedated values; $r$ is the distance from the moving
source~$dV'$ at the appropriate moment to the point where~$h_{\alpha\beta}$ is to be calculated;
and~$v_r$ is the component velocity of~$dV'$ towards this point.
Although we shall find it necessary to retain somewhat high powers of the velocity in the coefficients of
periodic terms, it is unnecessary to take account of the FitzGerald factor~$\beta$ occurring as a constant multiplier
independent of the time.
For the same reason we can replace~$\mf{T}^{\alpha\beta}$ by~$T^{\alpha\beta}$ in~\Eq{(57.102)}.
Squares of~$h_{\alpha\beta}$ have been neglected.

Hence by~\Eq{(57.102)} and~\Eq{(57.104)} the rate of loss of energy of the rod is
\[
2\int\int\left\{T^{\alpha\beta}\frac{\dd}{\dd t}\left[\frac{T'_{\alpha\beta}}{r(1-v_r)}\right] - \tfrac{1}{2} T\frac{\dd}{\dd t}\left[\frac{T'}{r(1-v_r)}\right]\right\} dV dV',
\Tag{(57.105)}
\]
since~$T^{\alpha\beta}\delta_{\alpha\beta}=T$ to the adopted order of approximation.
This integral exhibits the loss of energy as arising from the mutual action of pairs of elements of
the rod,~$dV$ and~$dV'$.

The antedated values can be expanded in terms of contemporaneous values of~$r$ and~$T'_{\alpha\beta}$
by the series\footnotemark\footnotetext
      {See~\Eq{(74.94)}.}
\[
\left[\frac{T'_{\alpha\beta}}{r(1-v_r)}\right] = \frac{T'_{\alpha\beta}}{r} - \frac{d}{dt}T'_{\alpha\beta} +
                 \sum_2^\infty\frac{(-1)^n}{n!}\frac{d^n}{dt^n}(r^{n-1}T'_{\alpha\beta}),
\Tag{(57.106)}
\]
the quantities on the right \emph{not} being antedated.

Let the rod, spinning in the plane of~$xy$, be along the axis of~$x$ at the instant~$t=0$, the origin being at
the centre.
Let~$dV$ then be at~$x$ and~$dV'$ at~$x'$.
The varying distance of the source~$dV'$ from the fixed point~$x$ instantaneously occupied by~$dV$, is
\[
r=\sqrt{x^2 + x'^2 - 2 x x' \cos\omega t}.
\]
This must be used in~\Eq{(57.106)} and~$t$ must be made zero after the differentiations.
In the present application we can simplify~\Eq{(57.106)} by noting that if~$T^{\alpha\beta}$ is any component
which does not vanish when~$t=0$, $T'_{\alpha\beta}$ will be an even function of~$t$,
so that derivatives of odd order disappear.
Accordingly for our application
\begin{multline*}
\frac{\dd}{\dd t}\left[\frac{T'_{\alpha\beta}}{r(1-v_r)}\right] = -\frac{d^2}{dt^2}T'_{\alpha\beta} -
                        \frac{1}{6}\frac{d^4}{dt^4}\{T'_{\alpha\beta}(x^2 + x'^2 - 2 x x' \cos\omega t)\}\\
                   -\frac{1}{120}\frac{d^6}{dt^6}\{T'_{\alpha\beta}(x^2 + x'^2 - 2 x x' \cos\omega t)^2\} - \ldots.
\Tag{(57.107)}
\end{multline*}
This must be substituted in~\Eq{(57.105)} and the earliest non\hyp{}vanishing terms picked out.
Assuming that the rod is symmetrical (but not necessarily of uniform density) terms containing an
odd power of~$x$ or~$x'$ will vanish on integration.

\Subsection[(a)]{Stress components, $T^{11}, T^{22}$.}

These are small compared with the momentum and mass components and only the first term of the expansion~\Eq{(57.107)}
is required.
Since~$r$ does not appear, the double integration breaks up into the product of two independent integrals.
The contribution to~\Eq{(57.105)} is
\[
-2\int T^{11} dV \cdot \frac{d^2}{dt^2}\int T'_{11} dV' - 2\int T^{22} dV \cdot \frac{d^2}{dt^2}\int T'_{22} dV'.
\Tag{(57.108)}
\]

If~$\sigma$ is the line\hyp{}density of the rod~$T^{22} dV = \sigma\omega^2 x^2 dx$, so that
\[
\int T^{22} dV = I\omega^2.
\]
The component~$T^{11}$ represents the tension of the rod and it is easily found by elementary dynamics that its
integral is $-I\omega^2$.

For the moving source the corresponding integrals are~$I\omega^2\cos 2\omega t$ and~$-I\omega^2\cos 2\omega t$.
Hence~\Eq{(57.108)} gives the result~$16I^2\omega^6$.

\Subsection[(b)]{Momentum components, $T^{24}, T^{42}$.}

We have $T^{24}dV = \sigma\omega x dx$, $T'_{24}dV' = -\sigma'\omega x' dx' \cos\omega t$.

The first term of~\Eq{(57.107)} now yields nothing, owing to the odd powers of~$x$ and~$x'$.
We take the second term and obtain
\[
-2\int\int\sigma\omega x dx \cdot \sigma'\omega x' dx'\cdot\tfrac{1}{6} (2\omega)^4 x x' = -\tfrac{16}{3}I^2\omega^6.
\]
$T^{42}$ gives an equal contribution, making a total of~$-\tfrac{32}{3}I^2\omega^6$.

\Subsection[(c)]{Mass components, $T^{44}, T$.}
\[
T^{44} dV = \sigma dx, \quad T'_{44} dV' = \sigma dx'.
\]
The third term of~\Eq{(57.107)} is now required, giving
\[
2\int\int\sigma dx\cdot\sigma' dx' \cdot\tfrac{1}{120}(2\omega)^6\cdot 2 x^2 x'^2 = \tfrac{32}{15}I^2\omega^6.
\]
The proper\hyp{}density~$T$ and the coordinate\hyp{}density~$T^{44}$ are practically the same, so that the term in~$T$
cancels half the above amount, leaving~$\tfrac{16}{15}I^2\omega^6$.

Gathering together~$(a)$,~$(b)$ and~$(c)$ the rate of loss of energy is
\[
(16-\tfrac{32}{3}+\tfrac{16}{15})I^2\omega^6=\tfrac{32}{5}I^2\omega^6,
\]
agreeing with the result already stated.

If~$a$ is the order of magnitude of the linear dimensions of the system and~$v$ of the velocities,
this result is of order~$(M/a)^2 v^6$.
Since~$h_{44}$ is of order~$(M/a)$ the neglect of higher powers of~$h_{\mu\nu}$ excludes from the discussion terms
of order~$(M/a)^3 v^2$ and~$(M/a)^4 v^2$.
The former may possibly, and the latter will almost certainly occur;
the approximation accordingly assumes that such terms are negligible in comparison with~$(M/a)^2 v^6$.
There is no theoretical difficulty in the existence of cohesive systems with small mass and large velocities
for which our approximation is valid; but for gravitational systems,~$M/a$ is necessarily of the same order
of magnitude as~$v^2$, and the approximation fails.
Thus the decay of energy (if any) of a double star cannot be investigated by this method.

In a sense it is true that our success in solving the problem for cohesive systems and our failure for
gravitational systems is due to our comparative ignorance of the nature of cohesive forces.
Presumably cohesive forces are propagated with the fundamental velocity and our assumption that the tension
in the spinning rod lies wholly in the line of the rod may not be strictly true.
On the other hand the cohesion is between neighbouring particles and we must not think of it as propagated
from end to end of the rod in the way that the gravitational attraction is propagated.
For this reason it seems plausible to neglect the propagation of cohesion; but even if the effect is
appreciable we can scarcely suppose that the lag of the cohesive forces taken alone would \emph{accelerate}
the rotation of the rod, so that there seems no possibility of the gravitational loss of energy found in this
discussion being neutralised.
The problem of the double star is more difficult; we should have to take account of the effect of the
gravitational field in disturbing the propagation of its own potentials and we cannot be sure that even the
sign of~\Eq{(57.101)} is correct.

The spontaneous loss of energy of a spinning rod is interesting in connection with the problem of absolute rotation.
We used often to hear the suggestion that a moving star would gradually be brought to rest owing to the
back\hyp{}pressure of its own radiation.
Obviously there must be a fallacy\footnotemark\footnotetext
     {The actual fallacy lay in the neglect of the gradual loss of mass of the star which is radiating energy---a
      non\hyp{}vanishing force~$d(Mv)/dt$ is not inconsistent with uniform velocity if~$M$ varies.}
in the argument, since there is no ``rest'' for the star to be brought to.
Similarly it might be thought that the conclusion that a spinning rod spontaneously comes to rest must be fallacious.
But the relativity theory does not deny absolute rotation;
or at least if it does, its denial has not the same plain meaning as its denial of absolute translation.

\Section{58.}{Lagrangian form of the gravitational equations}

The Lagrangian function~$\mf{L}$ is defined by
\index{Lagrangian function}%
\[
\mf{L} = g^{\mu\nu} \sqrt{-g}
\bigl(\{\mu\alpha, \beta\}\{\nu\beta, \alpha\}
    - \{\mu\nu, \alpha\} \{\alpha\beta, \beta\}\bigr),
\Tag{(58.1)}
\]
which forms part of the expression for~$\mf{G}$ ($= g^{\mu\nu} G_{\mu\nu} \sqrt{-g}$). For any small
variation of~$\mf{L}$
\begin{multline*}
  \delta\mf{L}
  = \{\mu\alpha, \beta\}\, \delta\bigl(g^{\mu\nu} \sqrt{-g}\, \{\nu\beta, \alpha\}\bigr)
  + \{\nu\beta, \alpha\}\, \delta\bigl(g^{\mu\nu} \sqrt{-g}\, \{\mu\alpha, \beta\}\bigr) \\
  - \{\mu\nu, \alpha\}\, \delta\bigl(g^{\mu\nu} \sqrt{-g}\, \{\alpha\beta, \beta\}\bigr)
  - \{\alpha\beta, \beta\}\, \delta\bigl(g^{\mu\nu} \sqrt{-g}\, \{\mu\nu, \alpha\}\bigr) \\
  - \bigl(\{\mu\alpha, \beta\}\, \{\nu\beta, \alpha\} - \{\mu\nu, \alpha\}\{\alpha\beta, \beta\}\bigr)
  \delta\bigl(g^{\mu\nu} \sqrt{-g}\bigr).
  \Tag{(58.2)}
\end{multline*}

The first term in~\Eq{(58.2)}
\begin{align*}
  &= \tfrac{1}{2} \{\mu\alpha, \beta\}\, \delta\left(\sqrt{-g} \cdot g^{\mu\nu} g^{\alpha\epsilon}
  \left(\frac{\dd g_{\nu\epsilon}}{\dd x_{\beta}}
  + \frac{\dd g_{\beta\epsilon}}{\dd x_{\nu}}
  - \frac{\dd g_{\beta\nu}}{\dd x_{\epsilon}}\right)\right) \\
  &= \tfrac{1}{2} \{\mu\alpha, \beta\}\, \delta\left(\sqrt{-g} \cdot g^{\mu\nu} g^{\alpha\epsilon}\, \frac{\dd g_{\nu\epsilon}}{\dd x_{\beta}}\right) \\
  &= -\tfrac{1}{2} \{\mu\alpha, \beta\}\, \delta\left(\sqrt{-g} \cdot \frac{\dd g^{\mu\alpha}}{\dd x_{\beta}}\right)
  \quad\text{by \Eq{(35.11)}} \\
  &= -\tfrac{1}{2} \{\mu\nu, \alpha\}\, \delta\left(\sqrt{-g} \cdot \frac{\dd g^{\mu\nu}}{\dd x_{\alpha}}\right).
  \Tag{(58.31)}
\end{align*}

The second term reduces to the same.

The third term becomes by~\Eq{(35.4)}
\[
-\{\mu\nu, \alpha\}\, \delta\left(g^{\mu\nu}\, \frac{\dd}{\dd x_{\alpha}} \sqrt{-g}\right).
\Tag{(58.32)}
\]

In the fourth term we have
\[
g^{\mu\nu}\, \sqrt{-g}\, \{\mu\nu, \alpha\}
= -\frac{\dd}{\dd x_{\nu}}(g^{\alpha\nu} \sqrt{-g}),
\]
by~\Eq{(51.41)}, since the divergence of~$g^{\alpha\nu}$ vanishes. Hence with some alterations
of dummy suffixes, the fourth term becomes
\[
\{\nu\beta, \beta\}\, g_{\mu}^{\alpha}\, \delta\left(\frac{\dd}{\dd x_{\alpha}} (g^{\mu\nu}\, \sqrt{-g})\right).
\Tag{(58.33)}
\]

Substituting these values in~\Eq{(58.2)}, we have
\begin{multline*}
  \delta\mf{L} = \bigl[-\{\mu\nu, \alpha\} + g_{\mu}^{\alpha} \{\nu\beta, \beta\}\bigr]\,
  \delta\left(\frac{\dd}{\dd x_{\alpha}} (g^{\mu\nu} \sqrt{-g})\right) \\
  - \bigl[\{\mu\alpha, \beta\} \{\nu\beta, \alpha\} - \{\mu\nu, \alpha\} \{\alpha\beta, \beta\}\bigr]\,
  \delta(g^{\mu\nu} \sqrt{-g}).
  \Tag{(58.4)}
\end{multline*}

We write
\[
\mf{g}^{\mu\nu} = g^{\mu\nu} \sqrt{-g};\quad
\mf{g}_{\alpha}^{\mu\nu} = \frac{\dd}{\dd x_{\alpha}} (g^{\mu\nu} \sqrt{-g}).
\Tag{(58.45)}
\]
Then when $\mf{L}$~is expressed as a function of the~$\mf{g}^{\mu\nu}$ and~$\mf{g}_{\alpha}^{\mu\nu}$, \Eq{(58.4)}~gives
\begin{align*}
  \frac{\dd\mf{L}}{\dd\mf{g}^{\mu\nu}}
  &= -\bigl[\{\mu\alpha, \beta\} \{\nu\beta, \alpha\} - \{\mu\nu, \alpha\} \{\alpha\beta, \beta\}\bigr],
  \Tag{(58.51)} \\
  \frac{\dd\mf{L}}{\dd\mf{g}_{\alpha}^{\mu\nu}}
  &= \Neg\bigl[-\{\mu\nu, \alpha\} + g_{\mu}^{\alpha} \{\nu\beta, \beta\}\bigr].
  \Tag{(58.52)}
\end{align*}

Comparing with~\Eq{(37.2)} we have
\[
G_{\mu\nu} = \frac{\dd}{\dd x_{\alpha}}\, \frac{\dd\mf{L}}{\dd\mf{g}_{\alpha}^{\mu\nu}} - \frac{\dd\mf{L}}{\dd\mf{g}^{\mu\nu}}.
\Tag{(58.6)}
\]

This form resembles that of Lagrange's equations in dynamics. Regarding
\index{Lagrange's equations}%
$\mf{g}^{\mu\nu}$ as a coordinate~$q$, and $x_{\alpha}$~as a four\hyp{}dimensional time~$t$, so that $\mf{g}_{\alpha}^{\mu\nu}$ is a velocity~$q'$,
the gravitational equations $G_{\mu\nu} = 0$ correspond to the well-known form
\[
\frac{d}{dt}\, \frac{\dd\mf{L}}{\dd q'} - \frac{\dd\mf{L}}{\dd q} = 0.
\]

The two following formulae express important properties of the Lagrangian
function:
\begin{align*}
  \mf{g}^{\mu\nu}\, \frac{\dd\mf{L}}{\dd\mf{g}^{\mu\nu}}
  &= -\mf{L},
  \Tag{(58.71)} \\
  \mf{g}_{\alpha}^{\mu\nu}\, \frac{\dd\mf{L}}{\dd\mf{g}_{\alpha}^{\mu\nu}}
  &= 2\mf{L}.
  \Tag{(58.72)}
\end{align*}
The first is obvious from~\Eq{(58.51)}. To prove the second, we have
\begin{align*}
  \mf{g}_{\alpha}^{\mu\nu}
  &= \frac{\dd}{\dd x_{\alpha}} (g^{\mu\nu} \sqrt{-g})
  = \sqrt{-g}\, \frac{\dd g^{\mu\nu}}{\dd x_{\alpha}} + g^{\mu\nu} \sqrt{-g}\, \{\alpha\epsilon, \epsilon\} \\
  &= \sqrt{-g} \bigl[-\{\epsilon\alpha, \mu\} g^{\epsilon\nu}
    - \{\epsilon\alpha, \nu\} g^{\mu\epsilon}
    + \{\alpha\epsilon, \epsilon\} g^{\mu\nu}\bigr]
\end{align*}
by~\Eq{(30.1)} since the covariant derivative of~$g^{\mu\nu}$ vanishes.

Hence by~\Eq{(58.52)}
%[** TN: Re-breaking]
\begin{multline*}
  \mf{g}_{\alpha}^{\mu\nu}\, \frac{\dd\mf{L}}{\dd\mf{g}_{\alpha}^{\mu\nu}} \\
  = \sqrt{-g} \bigl[
    \{\mu\nu, \alpha\} \{\epsilon\alpha, \mu\} g^{\epsilon\nu}
    + \{\mu\nu, \alpha\} \{\epsilon\alpha, \nu\} g^{\epsilon\mu}
    - \{\mu\nu, \alpha\} \{\alpha\epsilon, \epsilon\} g^{\mu\nu} \\
    - \{\nu\beta, \beta\} g_{\mu}^{\alpha}\, \{\epsilon\alpha, \mu\} g^{\epsilon\nu}
    - \{\nu\beta, \beta\} g_{\mu}^{\alpha}\, \{\epsilon\alpha, \nu\} g^{\epsilon\mu}
    + \{\nu\beta, \beta\} g_{\mu}^{\alpha}\, \{\alpha\epsilon, \epsilon\} g^{\mu\nu}\bigr],
\end{multline*}
which by change of dummy suffixes becomes
\begin{align*}
  &= \sqrt{-g} \bigl[
      \{\beta\nu, \alpha\} \{\mu\alpha, \beta\} g^{\mu\nu}
    + \{\mu\beta, \alpha\} \{\nu\alpha, \beta\} g^{\nu\mu}
    - \{\mu\nu, \alpha\} \{\alpha\beta, \beta\} g^{\mu\nu} \\
    &\qquad
    - \{\nu\beta, \beta\} \{\mu\alpha, \alpha\} g^{\mu\nu}
    - \{\alpha\beta, \beta\} \{\nu\mu, \alpha\} g^{\nu\mu}
    + \{\nu\beta, \beta\} \{\mu\epsilon, \epsilon\} g^{\mu\nu}\bigr] \\
  &= 2\mf{L}\quad\text{by~\Eq{(58.1)}.}
\end{align*}

The equations \Eq{(58.71)} and \Eq{(58.72)} show that the Lagrangian function is a
homogeneous function of degree~$-1$ in the ``coordinates'' and of degree~$2$ in
the ``velocities.''

We can derive a useful expression for~$\mf{G}$
\begin{align*}
  \mf{G}
  &= \mf{g}^{\mu\nu} G_{\mu\nu} \\
  &= \mf{g}^{\mu\nu}\, \frac{\dd}{\dd x_{\alpha}}\, \frac{\dd\mf{L}}{\dd\mf{g}_{\alpha}^{\mu\nu}}
  - \mf{g}^{\mu\nu}\, \frac{\dd\mf{L}}{\dd\mf{g}^{\mu\nu}}\quad\text{by~\Eq{(58.6)}} \\
  &= \frac{\dd}{\dd x_{\alpha}} \left(\mf{g}^{\mu\nu}\, \frac{\dd\mf{L}}{\dd\mf{g}_{\alpha}^{\mu\nu}}\right)
  - \mf{g}_{\alpha}^{\mu\nu}\, \frac{\dd\mf{L}}{\dd\mf{g}_{\alpha}^{\mu\nu}}
  - \mf{g}^{\mu\nu}\, \frac{\dd\mf{L}}{\dd\mf{g}^{\mu\nu}} \\
  &= \frac{\dd}{\dd x_{\alpha}} \left(\mf{g}^{\mu\nu}\, \frac{\dd\mf{L}}{\dd\mf{g}_{\alpha}^{\mu\nu}}\right) - \mf{L}
  \Tag{(58.8)}
\end{align*}
by \Eq{(58.71)} and~\Eq{(58.72)}.

It will be seen that $(\mf{G} + \mf{L})$ has the form of a \emph{divergence}~\Eq{(51.12)}; but the
quantity of which it is the divergence is not a vector\hyp{}density, nor is $\mf{L}$~a scalar\hyp{}density.

We shall derive another formula which will be needed in \SecRef{59},
\[
d(g^{\mu\nu} \sqrt{-g})
= \sqrt{-g} (dg^{\mu\nu} + g^{\mu\nu} \cdot \tfrac{1}{2} g^{\alpha\beta}\, dg_{\alpha\beta})
\quad\text{by~\Eq{(35.3)}.}
\]
Hence, using~\Eq{(35.2)},
\begin{align*}
  G_{\mu\nu}\, d(g^{\mu\nu} \sqrt{-g})
  &= \sqrt{-g} (-G^{\mu\nu}\, dg_{\mu\nu} + \tfrac{1}{2} Gg^{\alpha\beta}\, dg_{\alpha\beta}) \\
  &= -(G^{\mu\nu} - \tfrac{1}{2}g^{\mu\nu}G) \sqrt{-g} \cdot dg_{\mu\nu} \\
  &= 8\pi\, \mf{T}^{\mu\nu}\, dg_{\mu\nu}.
  \Tag{(58.91)}
\end{align*}
Accordingly
\begin{align*}
  8\pi\, \mf{T}^{\mu\nu}\, \frac{\dd g_{\mu\nu}}{\dd x_{\alpha}}
  &= G_{\mu\nu} \mf{g}_{\alpha}^{\mu\nu} \\
  &= \mf{g}_{\alpha}^{\mu\nu} \biggl(
  \frac{\dd}{\dd x_{\beta}}\, \frac{\dd\mf{L}}{\dd\mf{g}_{\beta}^{\mu\nu}}
  - \frac{\dd\mf{L}}{\dd\mf{g}^{\mu\nu}}\biggr) \\
  &= \frac{\dd}{\dd x_{\beta}} \biggl(\mf{g}_{\alpha}^{\mu\nu}\, \frac{\dd\mf{L}}{\dd\mf{g}_{\beta}^{\mu\nu}}\biggr)
  - \frac{\dd}{\dd x_{\beta}}\, \mf{g}_{\alpha}^{\mu\nu} \cdot \frac{\dd\mf{L}}{\dd\mf{g}_{\beta}^{\mu\nu}}
  - \mf{g}_{\alpha}^{\mu\nu}\, \frac{\dd\mf{L}}{\dd\mf{g}^{\mu\nu}}.
  \Tag{(58.92)}
\end{align*}

Now
\[
\frac{\dd\mf{L}}{\dd x_{\alpha}}
= \frac{\dd\mf{L}}{\dd\mf{g}^{\mu\nu}}\, \frac{\dd\mf{g}^{\mu\nu}}{\dd x_{\alpha}}
+ \frac{\dd\mf{L}}{\dd\mf{g}_{\beta}^{\mu\nu}}\, \frac{\dd\mf{g}_{\beta}^{\mu\nu}}{\dd x_{\alpha}},
\]
and since
\[
\frac{\dd\mf{g}_{\beta}^{\mu\nu}}{\dd x_{\alpha}}
= \frac{\dd^{2}\mf{g}^{\mu\nu}}{\dd x_{\alpha}\, \dd x_{\beta}}
= \frac{\dd\mf{g}_{\alpha}^{\mu\nu}}{\dd x_{\beta}},
\]
we see that \Eq{(58.92)}~reduces to
\begin{align*}
  8\pi\, \mf{T}^{\mu\nu}\, \frac{\dd g_{\mu\nu}}{\dd x_{\alpha}}
  &= \frac{\dd}{\dd x_{\beta}} \biggl(\mf{g}_{\alpha}^{\mu\nu}\, \frac{\dd\mf{L}}{\dd\mf{g}_{\beta}^{\mu\nu}}\biggr)
  - \frac{\dd\mf{L}}{\dd x_{\alpha}} \\
  &= \frac{\dd}{\dd x_{\beta}} \biggl\{\mf{g}_{\alpha}^{\mu\nu}\, \frac{\dd\mf{L}}{\dd\mf{g}_{\beta}^{\mu\nu}}
  - g_{\alpha}^{\beta}\mf{L}\biggr\}.
  \Tag{(58.93)}
\end{align*}

\Section{59.}{Pseudo\hyp{}energy\hyp{}tensor of the gravitational field}

The formal expression of the conservation of the material energy and
\index{Conservation, formal law of}%
momentum is contained in the equations
\[
\frac{\dd\mf{T}_{\mu}^{\nu}}{\dd x_{\nu}} = 0,
\Tag{(59.1)}
\]
or, if we name the coordinates $x$, $y$, $z$, $t$,
\[
\frac{\dd}{\dd x}\mf{T}_{\mu}^{1}
+ \frac{\dd}{\dd y}\mf{T}_{\mu}^{2}
+ \frac{\dd}{\dd z}\mf{T}_{\mu}^{3}
+ \frac{\dd}{\dd t}\mf{T}_{\mu}^{4} = 0.
\]
Multiply by $dx\, dy\, dz$ and integrate through a given three\hyp{}dimensional region.
The last term is
\[
\frac{\dd}{\dd t} \iiint \mf{T}_{\mu}^{4}\, dx\, dy\, dz.
\]
The other three terms yield surface\hyp{}integrals over the boundary of the region.
Thus the law~\Eq{(59.1)} states that the rate of change of $\iiint \mf{T}_{\mu}^{4}\, dx\, dy\, dz$ is equal to
certain terms which describe something going on at the boundary of the region.
In other words, changes of this integral cannot be created in the interior of
the region, but are always traceable to transmission across the boundary. This
is clearly what is meant by conservation of the integral.

This equation~\Eq{(59.1)} applies only in the special case when the coordinates
are such that there is no field of force. We have generalised it by substituting
the corresponding tensor equation $T_{\mu\nu}^{\nu} = 0$; but this is no longer a formal expression
of the conservation of anything. It is of interest to compare the
traditional method of generalising~\Eq{(59.1)} in which formal conservation is
adhered to.

In classical mechanics the law of conservation is restored by recognising
\index{Energy, potential}%
\index{Potential energy}%
another form of energy---potential energy---which is not included in~$\mf{T}_{\mu}^{\nu}$. This
is supposed to be stored up in the gravitational field; and similarly the momentum
and stress components may have their invisible complements in the
gravitational field. We have therefore to add to~$\mf{T}_{\mu}^{\nu}$ a complementary expression~$\mf{t}_{\mu}^{\nu}$
denoting potential energy, momentum and stress; and conservation is only
asserted for the sum. If
\[
\mf{S}_{\mu}^{\nu} = \mf{T}_{\mu}^{\nu} + \mf{t}_{\mu}^{\nu},
\Tag{(59.2)}
\]
then \Eq{(59.1)}~is generalised in the form
\[
\frac{\dd\mf{S}_{\mu}^{\nu}}{\dd x_{\nu}} = 0.
\Tag{(59.3)}
\]

Accordingly the difference between the relativity treatment and the
classical treatment is as follows. In both theories it is recognised that in
certain cases $\mf{T}_{\mu}^{\nu}$~is conserved, but that in the general case this conservation
breaks down. The relativity theory treats the general case by discovering a
more exact formulation of what happens to~$\mf{T}_{\mu}^{\nu}$ when it is not strictly conserved,
viz.\ $\mf{T}_{\mu\nu}^{\nu} = 0$. The classical theory treats it by introducing a supplementary
energy, so that conservation is still maintained but for a different quantity,
viz.\ $\dd\mf{S}_{\mu}^{\nu}/\dd x_{\nu} = 0$. The relativity treatment adheres to the physical quantity and
modifies the law; the classical treatment adheres to the law and modifies the
physical quantity. Of course, both methods should be expressible by equivalent
formulae; and we have in our previous work spoken of $\mf{T}_{\mu\nu}^{\nu} = 0$ as the law of
conservation of energy and momentum, because, although it is not formally
a law of conservation, it expresses exactly the phenomena which classical
mechanics attributes to conservation.

The relativity treatment has enabled us to discover the exact equations,
and we may now apply these to obtain the corresponding exact expression for
the quantity~$\mf{S}_{\mu}^{\nu}$ introduced in the classical treatment.

It is clear that $\mf{t}_{\mu}^{\nu}$ and therefore~$\mf{S}_{\mu}^{\nu}$ cannot be tensor\hyp{}densities, because $\mf{t}_{\mu}^{\nu}$~vanishes
\index{Pseudo\hyp{}energy\hyp{}tensor}%
when natural coordinates are used at a point, and would therefore
always vanish if it were a tensor\hyp{}density. We call~$\mf{t}_{\mu}^{\nu}$ the pseudo\hyp{}tensor\hyp{}density
of potential energy.

The explicit value of~$\mf{t}_{\mu}^{\nu}$ must be calculated from the condition~\Eq{(59.3)}, or
\begin{align*}
  \frac{\dd\mf{t}_{\mu}^{\nu}}{\dd x_{\nu}}
  &= -\frac{\dd\mf{T}_{\mu}^{\nu}}{\dd x_{\nu}} \\
  &= -\tfrac{1}{2}\mf{T}^{\alpha\beta} \frac{\dd g_{\alpha\beta}}{\dd x_{\mu}}
  \quad\text{by~\Eq{(55.6)}} \\
  &= -\frac{1}{16\pi}\, \frac{\dd}{\dd x_{\nu}} \biggl\{\mf{g}_{\mu}^{\alpha\beta}\, \frac{\dd\mf{L}}{\dd\mf{g}_{\nu}^{\alpha\beta}}
  - g_{\mu}^{\nu} \mf{L}\biggr\}\quad\text{by~\Eq{(58.93)}.}
\end{align*}

Hence
\[
16\pi \mf{t}_{\mu}^{\nu}
= g_{\mu}^{\nu} \mf{L} - \mf{g}_{\mu}^{\alpha\beta}\, \frac{\dd\mf{L}}{\dd\mf{g}_{\nu}^{\alpha\beta}}.
\Tag{(59.4)}
\]
This may remind us of the Hamiltonian integral of energy
\[
-h = L - \sum q'\, \frac{\dd L}{\dd q'}
\]
in general dynamics.

We can form a pseudo\hyp{}scalar\hyp{}density by contraction of~\Eq{(59.4)}
\begin{align*}
  16\pi\mf{t}
  &= 4\mf{L} - \mf{g}_{\mu}^{\alpha\beta}\, \frac{\dd\mf{L}}{\dd\mf{g}_{\mu}^{\alpha\beta}} \\
  &= 2\mf{L}\quad\text{by~\Eq{(58.72)}.}
\end{align*}

Thus we obtain the interesting comparison with~\Eq{(54.4)}
\[
\left.
\begin{aligned}
  \mf{L} &= 8\pi\mf{t}\\
  \mf{G} &= 8\pi\mf{T}
\end{aligned}
\right\}
\Tag{(59.5)}
\]

It should be understood that in this section we have been occupied
with the transition between the old and new points of view. The quantity~$\mf{t}_{\mu}^{\nu}$
represents the potential energy of classical mechanics, but we do not ourselves
recognise it as an energy of any kind. It is not a tensor\hyp{}density and it can
be made to vanish at any point by suitably choosing the coordinates; we do
not associate it with any absolute feature of world\hyp{}structure. In fact finite
values of~$\mf{t}_{\mu}^{\nu}$ can be produced in an empty world containing no gravitating
matter merely by choice of coordinates. The tensor\hyp{}density~$\mf{T}_{\mu}^{\nu}$ comprises all
the energy which we recognise; and we call it gravitational or material energy
indiscriminately according as it is expressed in terms of~$g_{\mu\nu}$ or $\rho_{0}$, $u$, $v$,~$w$.

This difference between the classical and the relativity view of energy
recalls the remarks on the definition of physical quantities made in the Introduction.
As soon as the principle of conservation of energy was grasped, the
physicist practically made it his definition of energy, so that energy was that
\emph{something} which obeyed the law of conservation. He followed the practice of
the pure mathematician, defining energy by the properties he wished it to
have, instead of describing how he had measured it. This procedure has turned
out to be rather unlucky in the light of the new developments. It is true that
a quantity~$\mf{S}_{\mu}^{\nu}$ can be found which obeys the definition, but it is not a tensor
and is therefore not a direct measure of an intrinsic condition of the world.
Rather than saddle ourselves with this quantity, which is not now of primary
interest, we go back to the more primitive idea of \Foreign{vis viva}---generalised, it is
true, by admitting heat or molecular \Foreign{vis viva} but not potential energy. We
find that this is not in all cases formally conserved, but it obeys the law that
its divergence vanishes; and from our new point of view this is a simpler and
more significant property than strict conservation.

Integrating over an isolated material body we may set
\begin{align*}
\iiint \mf{T}_{\mu}^{4}\, dx\, dy\, dz &= -Mu,\ -Mv,\ -Mw,\ M, \\
\iiint \mf{S}_{\mu}^{4}\, dx\, dy\, dz &= -M'u',\ -M'v',\ -M'w',\ M',
\end{align*}
where the latter expression includes the potential energy and momentum of
the body. Changes of~$M'u'$, etc.\ can only occur by transfer from regions outside
the body by action passing through the boundary; whereas changes of~$Mu$,
\index{Action, material or gravitational}%
etc.\ can be produced by the mutual attractions of the particles of the
body. It is clear that the kinematical velocity, or direction of the world-line
of the body, corresponds to $u : v : w : 1$; the direction of $u' : v': w' : 1$ can be varied
at will by choosing different coordinate\hyp{}systems.

The components $(\mf{t}_4^1,\mf{t}_4^2,\mf{t}_4^3)$ constitute a ``Poynting vector'' representing the flow of
potential energy at any point.
No physical significance can be attached to the localisation of the energy flow, but the \emph{total} flux of this
vector through a closed surface in empty space will by~\Eq{(59.3)} give correctly the rate of diminution of material
and potential energy $(\mf{T}_4^4 + \mf{t}_4^4)$ within the surface.
If there is within the surface a material system in periodic motion, coordinates will naturally be chosen so that
$\mf{t}_4^4$ undergoes no secular change; the flux will then give the secular change of $\mf{T}_4^4$, i.e. the loss of
energy from the material system due to the gravitational waves produced by it.

\Section{60.}{Action}

The invariant integral
\[
A = \iiiint \rho_{0} \sqrt{-g}\, d\tau
\Tag{(60.11)}
\]
represents the \emph{action} of the matter in a four\hyp{}dimensional region.

By~\Eq{(49.42)},
\begin{align*}
  A &= \iiiint \rho_{0}\, dW\, ds \\
  &= \iint m\, ds,
  \Tag{(60.12)}
\end{align*}
where $m$~is the invariant mass or energy.

Thus the action of a particle having energy~$m$ for a proper\hyp{}time~$ds$ is
equal to~$m\, ds$, agreeing with the definition of action in ordinary mechanics as
energy multiplied by time. By~\Eq{(54.6)} another form is
\[
A = \frac{1}{8\pi} \iiiint G\sqrt{-g}\, d\tau,
\Tag{(60.2)}
\]
so that (ignoring the numerical factor) $G\sqrt{-g}$, or~$\mf{G}$, represents the action\hyp{}density
of the gravitational field. Note that material action and gravitational
action are alternative aspects of the same thing; they are not to be added
together to give a total action.

But in stating that the gravitational action and the material action are
necessarily the same thing, we have to bear in mind a very peculiar conception
which is almost always associated with the term Action. From its first introduction,
action has always been looked upon as something whose sole \Foreign{raison d'être}
is to be varied---and, moreover, \emph{varied in such a way as to defy the laws
of nature!} We have thus to remember that when a writer begins to talk
about action, he is probably going to consider impossible conditions of the
world. (That does not mean that he is talking nonsense---he brings out the
important features of the possible conditions by comparing them with impossible
conditions.) Thus we may not always \emph{disregard} the difference between material
and gravitational action; it is impossible that there should be any difference,
but then we are about to discuss impossibilities.

We have to bear in mind the two aspects of action in this subject. It is
primarily a physical quantity having a definite numerical value, given indifferently
by \Eq{(60.11)} or~\Eq{(60.2)}, which is of special importance because it is
invariant. But it also denotes a mathematical function of the variables; the
functional form, which is all important, will differ according to which of the
two expressions is used. In particular we have to consider the partial derivatives,
and these will depend on the variables in terms of which the action is
expressed.

The Hamiltonian method of variation of an integral is of great importance
in this subject; several examples of it will be given presently. I think it is
unfortunate that this valuable method is nearly always applied in the form of
a principle of stationary action. By considering the variation of the integral
for small variations of the~$g_{\mu\nu}$, or other variables, we obtain a kind of generalised
differential coefficient which I will call the Hamiltonian derivative. It
may be possible to construct integrals for which the Hamiltonian derivatives
vanish, so that the integral has the stationary property. But just as in the
ordinary differential calculus we are not solely concerned with problems of
maxima and minima, and we take some interest in differential coefficients
which do not vanish; so Hamiltonian derivatives may be worthy of attention
even when they disappoint us by failing to vanish.

Let us consider the variation of the gravitational action in a region, viz.\
\[
8\pi\, \delta A = \delta \int G\sqrt{-g}\, d\tau,
\]
for arbitrary small variations~$\delta g_{\mu\nu}$ which vanish at and near\footnote
  {So that their first derivatives also vanish.}
the boundary of
the region. By~\Eq{(58.8)}
\[
\delta \int G\sqrt{-g}\, d\tau
= -\delta \int \mf{L}\, d\tau
+ \delta \int \frac{\dd}{\dd x_{\alpha}} \left(\mf{g}^{\mu\nu}\, \frac{\dd\mf{L}}{\dd\mf{g}_{\alpha}^{\mu\nu}}\right) d\tau.
\]
Also since $\mf{L}$~is a function of $\mf{g}^{\mu\nu}$ and~$\mf{g}_{\alpha}^{\mu\nu}$
\[
\int \delta\mf{L}\, d\tau
  = \int \left(
  \frac{\dd\mf{L}}{\dd\mf{g}^{\mu\nu}}\, \delta\mf{g}^{\mu\nu}
+ \frac{\dd\mf{L}}{\dd\mf{g}_{\alpha}^{\mu\nu}}\, \delta\mf{g}_{\alpha}^{\mu\nu}
\right) d\tau,
\]
and, by partial integration of the second term,
\[
= \int \left(
\frac{\dd\mf{L}}{\dd\mf{g}^{\mu\nu}}
- \frac{\dd}{\dd x_{\alpha}}\, \frac{\dd\mf{L}}{\dd\mf{g}_{\alpha}^{\mu\nu}}\right) \delta\mf{g}^{\mu\nu}\, d\tau
+ \int \frac{\dd}{\dd x_{\alpha}} \left(\frac{\dd\mf{L}}{\dd\mf{g}_{\alpha}^{\mu\nu}}\, \delta\mf{g}^{\mu\nu}\right) d\tau.
\]
By~\Eq{(58.6)} the first integrand becomes $-G_{\mu\nu}\, \delta\mf{g}^{\mu\nu}$, so that we have
\[
\delta \int G\sqrt{-g}\, d\tau
= \int G_{\mu\nu}\, \delta(g^{\mu\nu} \sqrt{-g})\, d\tau
+ \int \frac{\dd}{\dd x_{\alpha}} \left(\mf{g}^{\mu\nu}\, \delta\left(\frac{\dd\mf{L}}{\dd\mf{g}_{\alpha}^{\mu\nu}}\right)\right) d\tau.
\Tag{(60.3)}
\]
The second term can be integrated immediately giving a triple integral over
the boundary of the four\hyp{}dimensional region; and it vanishes because all
variations vanish at the boundary by hypothesis. Hence
\begin{align*}
  \delta \int G\sqrt{-g}\, d\tau
  &= \Neg \int G_{\mu\nu}\, \delta(g^{\mu\nu} \sqrt{-g})\, d\tau
  \Tag{(60.41)} \\
  &= -\int (G_{\mu\nu} - \tfrac{1}{2}g^{\mu\nu}G)\, \delta g_{\mu\nu} \sqrt{-g}\, d\tau
\Tag{(60.42)}
\end{align*}
by~\Eq{(58.91)}.

I call the coefficient $-(G^{\mu\nu} - \frac{1}{2}g^{\mu\nu}G)$ the \emph{Hamiltonian derivative} of~$G$ with
\index{H@$\Ham$ (Hamiltonian operator)}%
\index{Hamiltonian derivative}%
\index{Operators!H@$\Ham$}%
respect to~$g_{\mu\nu}$, writing it symbolically
\[
\frac{\Ham G}{\Ham g_{\mu\nu}}
  = -(G^{\mu\nu} - \tfrac{1}{2}g^{\mu\nu}G)
  = 8\pi T^{\mu\nu}.
  \Tag{(60.43)}
\]

We see from~\Eq{(60.42)} that the action~$A$ is only stationary when the energy\hyp{}tensor~$T^{\mu\nu}$
\index{Action, principle of Stationary}%
\index{Stationary action, principle of}%
vanishes, that is to say in empty space. In fact action is only
stationary when it does not exist---and not always then.

It would thus appear that the Principle of Stationary Action is in general
\index{Principle!of least action}%
untrue. Nevertheless some modified statement of the principle appears to
have considerable significance. In the actual world the space occupied by
matter (electrons) is extremely small compared with the empty regions. Thus
the Principle of Stationary Action, although not universally true, expresses a
very general tendency---a tendency with exceptions\footnotemark.\footnotetext
  {I do not regard electromagnetic fields as constituting an exception, because they have not
  yet been taken into account in our work. But the action of matter has been fully included, so
  that the break\hyp{}down of the principle as applied to matter is a definite exception.}
Our theory does not
account for this atomicity of matter; and in the stationary variation of action
\index{Atomicity}%
we seem to have an indication of a way of approaching this difficult problem,
although the precise formulation of the law of atomicity is not yet achieved.
It is suspected that it may involve an ``action'' which is capable only of
discontinuous variation.

It is not suggested that there is anything incorrect in the principle of
least action as used in classical mechanics. The break-down occurs when we
attempt to generalise it for variations of the state of the system beyond those
hitherto contemplated. Indeed it is obvious that the principle must break
down if pressed to extreme generality. We may discriminate \Item{(a)}~possible
states of the world, \Item{(b)}~states which although impossible are contemplated,
\Item{(c)}~impossible states which are not contemplated. Generalisation of the principle
consists in transferring states from class~\Item{(c)} to class~\Item{(b)}; there must be
some limit to this, for otherwise we should find ourselves asserting that the
equation $\delta A \neq 0$ is not merely not a possible equation but also not even an
impossible equation.

\Section{61.}{A property of invariants}

Let $K$~be any invariant function of the~$g_{\mu\nu}$ and their derivatives up to any
order, so that
\[
\int K \sqrt{-g}\, d\tau\quad\text{is an invariant.}
\]

The small variations $\delta(K \sqrt{-g})$ can be expressed as a linear sum of terms
involving $\delta g_{\mu\nu}$, $\delta(\dd g_{\mu\nu}/dx_{\alpha})$, $\delta(\dd^{2} g_{\mu\nu}/dx_{\alpha}\, dx_{\beta})$, etc. By the usual method of partial
integration employed in the calculus of variations, these can all be reduced to
terms in~$\delta g_{\mu\nu}$, together with complete differentials.

Thus for variations which vanish at the boundary of the region, we can
write
\[
\delta \int K \sqrt{-g}\, d\tau
= \int P^{\mu\nu}\, \delta g_{\mu\nu}\, \sqrt{-g}\, d\tau,
\Tag{(61.1)}
\]
where the coefficients, here written~$P^{\mu\nu}$, can be evaluated when the analytical
expression for~$K$ is given. The complete differentials yield surface\hyp{}integrals
over the boundary, so that they do not contribute to the variations. In
accordance with our previous notation \Eq{(60.43)}, we have
\[
P^{\mu\nu} = \frac{\Ham K}{\Ham g_{\mu\nu}}.
\Tag{(61.2)}
\]

We take $P^{\mu\nu}$ to be symmetrical in $\mu$ and~$\nu$, since any antisymmetrical part
would be meaningless owing to the inner multiplication by~$\delta g_{\mu\nu}$. Also since
$\delta g_{\mu\nu}$ is an arbitrary tensor $P^{\mu\nu}$~must be a tensor.

Consider the case in which the $\delta g_{\mu\nu}$ arise merely from a transformation of
coordinates. Then \Eq{(61.1)}~vanishes, not from any stationary property, but
because of the invariance of~$K$. The $\delta g_{\mu\nu}$ are not now arbitrary independent
variations, so that it does not follow that $P^{\mu\nu}$~vanishes.

Comparing $g_{\mu\nu}$ and $g_{\mu\nu} + \delta g_{\mu\nu}$ by~\Eq{(23.22)}, since they correspond to a transformation
of coordinates,
\begin{align*}
  g_{\mu\nu} &= (g_{\alpha\beta} + \delta g_{\alpha\beta})
        \frac{\dd(x_{\alpha} + \delta x_{\alpha})}{\dd x_{\mu}}
  \cdot \frac{\dd(x_{\beta} + \delta x_{\beta})}{\dd x_{\nu}} \\
  &= (g_{\alpha\beta} + \delta g_{\alpha\beta})\,
  \frac{\dd x_{\alpha}}{\dd x_{\mu}}\, \frac{\dd x_{\beta}}{\dd x_{\nu}}
  + g_{\alpha\beta}\,
  \frac{\dd x_{\alpha}}{\dd x_{\mu}}\, \frac{\dd (\delta x_{\beta})}{\dd x_{\nu}}
  + g_{\alpha\beta}\,
  \frac{\dd x_{\beta}}{\dd x_{\nu}}\, \frac{\dd (\delta x_{\alpha})}{\dd x_{\mu}}.
\end{align*}
But
\[
\frac{\dd x_{\alpha}}{\dd x_{\mu}} = g_{\mu}^{\alpha},\quad
\frac{\dd x_{\beta}}{\dd x_{\nu}} = g_{\nu}^{\beta}\quad\text{by~\Eq{(22.3)}.}
\]
Hence
\[
g_{\mu\nu} = g_{\mu\nu} + \delta g_{\mu\nu}
+ g_{\mu\beta}\, \frac{\dd(\delta x_{\beta})}{\dd x_{\nu}}
+ g_{\alpha\nu}\, \frac{\dd(\delta x_{\alpha})}{\dd x_{\mu}}.
\]
This is a comparison of the fundamental tensor at $x_{\alpha} + \delta x_{\alpha}$ in the new
coordinate\hyp{}system with the value at~$x_{\alpha}$ in the old system. There would be no
objection to using this value of~$\delta g_{\mu\nu}$ provided that we took account of the
corresponding~$\delta(d\tau)$. We prefer, however, to keep $d\tau$~fixed in the comparison,
and must compare the values at~$x_{\alpha}$ in both systems. It is therefore necessary
to subtract the change $\delta x_{\alpha} \cdot \dd g_{\mu\nu}/\dd x_{\alpha}$ of~$g_{\mu\nu}$ in the distance~$\delta x_{\alpha}$; hence
\[
-\delta g_{\mu\nu}
= g_{\mu\beta}\, \frac{\dd(\delta x_{\beta})}{\dd x_{\nu}}
+ g_{\alpha\nu}\, \frac{\dd(\delta x_{\alpha})}{\dd x_{\mu}}
+ \frac{\dd g_{\mu\nu}}{\dd x_{\alpha}}\, \delta x_{\alpha}.
\Tag{(61.3)}
\]
Hence \Eq{(61.1)}~becomes
%[** TN: Not broken in the original]
\begin{multline*}
\delta \int K \sqrt{-g}\, d\tau \\
= -\int P^{\mu\nu} \sqrt{-g} \left(
g_{\mu\alpha}\, \frac{\dd}{\dd x_{\nu}} (\delta x_{\alpha})
+ g_{\nu\alpha}\, \frac{\dd}{\dd x_{\mu}} (\delta x_{\alpha})
+ \frac{\dd g_{\mu\nu}}{\dd x_{\alpha}}\, \delta x_{\alpha}
\right) d\tau
\end{multline*}
which, by partial integration,
\begin{align*}
  &= \int\! \left\{\!\frac{\dd}{\dd x_{\nu}} (g_{\mu\alpha} P^{\mu\nu} \sqrt{-g})
  + \frac{\dd}{\dd x_{\mu}} (g_{\nu\alpha} P^{\mu\nu} \sqrt{-g})
  - P^{\mu\nu} \sqrt{-g}\, \frac{\dd g_{\mu\nu}}{\dd x_{\alpha}}\!\right\} \delta x_{\alpha}\, d\tau \\
  &= 2\int \left\{\frac{\dd}{\dd x_{\nu}} \mf{P}_{\mu}^{\nu}
  - \tfrac{1}{2} \mf{P}^{\mu\nu}\, \frac{\dd g_{\mu\nu}}{\dd x_{\alpha}}\right\} \delta x_{\alpha}\, d\tau \\
  &= 2\int P_{\alpha\nu}^{\nu}\, \delta x_{\alpha}\, \sqrt{-g}\, d\tau
  \quad\text{by~\Eq{(51.51)}.}
  \Tag{(61.4)}
\end{align*}
This has to vanish for all arbitrary variations~$\delta x_{\alpha}$---deformations of the mesh\hyp{}system---and
accordingly
\[
(P_{\alpha}^{\nu})_{\nu} = 0.
\Tag{(61.5)}
\]

We have thus demonstrated the general theorem---

\emph{The Hamiltonian derivative of any fundamental invariant is a tensor whose
\index{Hamiltonian derivative!of fundamental invariants}%
divergence vanishes.}
\index{Divergence!Hamiltonian@of Hamiltonian derivative of an invariant}%

The theorem of \SecRef{52} is a particular case, since $T^{\mu\nu}$~is the Hamiltonian
derivative of~$G$ by~\Eq{(60.43)}.

\Section{62.}{Alternative energy\hyp{}tensors}
\index{Energy\hyp{}tensor of matter}%
\index{Fundamental velocity!invariants}%

We have hitherto identified the energy\hyp{}tensor with $G_{\mu}^{\nu} - \tfrac{1}{2} g_{\mu}^{\nu}G$ mainly
because the divergence of the latter vanishes identically; but the theorem
just proved enables us to derive other fundamental tensors whose divergence
vanishes, so that alternative identifications of the energy\hyp{}tensor would seem
to be possible. The three simplest fundamental invariants are
\[
K = G,\quad
K' = G_{\mu\nu} G^{\mu\nu},\quad
K'' = B_{\mu\nu\sigma}^{\rho} B_{\rho}^{\mu\nu\sigma}.
\Tag{(62.1)}
\]
Hitherto we have taken $\Ham K/\Ham g_{\mu\nu}$ to be the energy\hyp{}tensor; but if $\Ham K'/\Ham g_{\mu\nu}$
were substituted, the laws of conservation of energy and momentum would be
satisfied, since the divergence vanishes. Similarly $\Ham K''/\Ham g_{\mu\nu}$ could be used.

The condition for empty space is given by the vanishing of the energy\hyp{}tensor.
Hence for the three possible hypotheses, the law of gravitation in
empty space is
\[
\frac{\Ham K}{\Ham g_{\mu\nu}},\quad
\frac{\Ham K'}{\Ham g_{\mu\nu}},\quad
\frac{\Ham K''}{\Ham g_{\mu\nu}} = 0
\Tag{(62.2)}
\]
respectively.

It is easy to see that the last two tensors contain fourth derivatives of the~$g_{\mu\nu}$;
so that if we can lay it down as an essential condition that the law of
gravitation in empty space must be expressed by differential equations of the
second order, the only possible energy\hyp{}tensor is the one hitherto accepted.
For fourth\hyp{}order equations the question of the nature of the boundary conditions
necessary to supplement the differential equations would become very
difficult; but this does not seem to be a conclusive reason for rejecting such
equations.

The two alternative tensors are excessively complicated expressions; but
when applied to determine the field of an isolated particle, they become not
unmanageable. The field, being symmetrical, must be of the general form~\Eq{(38.2)},
so that we have only to determine the disposable coefficients $\lambda$ and~$\nu$
both of which must be functions of $r$~only. $K'$~can be calculated in terms of
$\lambda$ and~$\nu$ without difficulty from equations~\Eq{(38.6)}; but the expression for~$K''$
turns out to be rather simpler and I shall deal with it. By the method of
\SecRef{38}, we find
%[** TN: Re-breaking]
\begin{multline*}
  \mf{K}''
  = K'' \sqrt{-g}
  = 2e^{\frac{1}{2}(\lambda + \nu)} \sin\theta
  \bigl\{e^{-2\lambda} (\lambda'^{2} + \nu'^{2}) \\
  + 2r^{2} e^{-2\lambda} (\tfrac{1}{4}\lambda'\nu' - \tfrac{1}{4}\nu'^{2} - \tfrac{1}{2}\nu'')^{2}
  + 2(1 - e^{-\lambda})^{2}/r^{2}\bigr\}.
  \Tag{(62.3)}
\end{multline*}
It is clear that the integral of~$\mf{K}''$ will be stationary for variations from the
symmetrical condition, so that we need only consider variations of $\lambda$ and $\nu$
and their derivatives with respect to~$r$. Thus the gravitational equations
$\Ham\mf{K}''/\Ham g_{\mu\nu} = 0$ are equivalent to
\[
\frac{\Ham K''}{\Ham \lambda} = 0,\quad
\frac{\Ham K''}{\Ham \nu} = 0.
\Tag{(62.4)}
\]
Now for a variation of~$\lambda$
\begin{align*}
  \delta\! \int\!\! \mf{K}\, d\tau
  &=\!\! \int\!\! \biggl(\frac{\dd\mf{K}}{\dd\lambda}\, \delta\lambda
  + \frac{\dd\mf{K}}{\dd\lambda'}\, \delta\lambda'
  + \frac{\dd\mf{K}}{\dd\lambda''}\, \delta\lambda''\biggr) d\tau\displaybreak[0] \\
  &=\!\! \int\!\! \left\{\frac{\dd\mf{K}}{\dd\lambda}
  - \frac{\dd}{\dd r} \biggl(\frac{\dd\mf{K}}{\dd\lambda'}\biggr)
  + \frac{\dd^{2}}{\dd r^{2}} \biggl(\frac{\dd\mf{K}}{\dd\lambda''}\biggr)\!\!\right\} \delta\lambda\, d\tau
  + \text{surface-integrals.}
\end{align*}
Hence our equations~\Eq{(62.4)} take the Lagrangian form
\[
\left.
\begin{alignedat}{4}
  \frac{\Ham K''}{\Ham\lambda}
  &= \frac{\dd\mf{K}''}{\dd\lambda}
  &&-\frac{\dd}{\dd r} \frac{\dd\mf{K}''}{\dd\lambda'}
  &&+ \frac{\dd^{2}}{\dd r^{2}} \frac{\dd\mf{K}''}{\dd\lambda''}
  &&= 0\\
  \frac{\Ham K''}{\Ham\nu}
  &= \frac{\dd\mf{K}''}{\dd\nu}
  &&-\frac{\dd}{\dd r} \frac{\dd\mf{K}''}{\dd\nu'}
  &&+ \frac{\dd^{2}}{\dd r^{2}} \frac{\dd\mf{K}''}{\dd\nu''}
  &&= 0
\end{alignedat}
\right\}
\Tag{(62.5)}
\]
From these $\lambda$ and $\nu$ are to be determined.

It can be shown that one exact solution is the same as in \SecRef{38}, viz.\
\[
e^{-\lambda} = e^{\nu} = \gamma = 1 - 2m/r.
\Tag{(62.6)}
\]
For taking the partial derivatives of~\Eq{(62.3)}, and applying \Eq{(62.6)} after the
differentiation,
\begin{align*}
  \frac{\dd\mf{K}''}{\dd\lambda}
  &= -\tfrac{3}{2} \mf{K}'' + 4 \frac{(1 - e^{-\lambda})}{r^{4}} \cdot 2e^{\frac{1}{2}(\lambda + \nu)} \sin\theta
  = \left(-72 \frac{m^{2}}{r^{4}} + 16 \frac{m}{r^{3}}\right) \sin\theta,\displaybreak[0] \\
%
  \frac{\dd\mf{K}''}{\dd\lambda'}
  &= 2e^{\frac{1}{2}(\lambda + \nu)} \sin\theta
  \bigl\{2e^{-2\lambda} \lambda' + r^{2} e^{-2\lambda}(\tfrac{1}{4}\lambda'\nu' - \tfrac{1}{4}\nu'^{2} - \tfrac{1}{2}\nu'')\nu'\bigr\} \\
  &= \left(24 \frac{m^{2}}{r^{3}} - 8 \frac{m}{r^{2}}\right) \sin\theta,\displaybreak[0] \\
\index{Einstein's law of gravitation!alternatives to}%
  \frac{\dd\mf{K}''}{\dd\lambda''} &= 0,\displaybreak[0] \\
%
  \frac{\dd\mf{K}''}{\dd\nu}
  &= \tfrac{1}{2} \mf{K}''
  = 24 \frac{m^{2}}{r^{4}} \sin\theta,\displaybreak[0] \\
%
  \frac{\dd\mf{K}''}{\dd\nu'}
  &= 2e^{\frac{1}{2}(\lambda + \nu)} \sin\theta
  \bigl\{2e^{-2\lambda} \nu' + 4r^{2} e^{-2\lambda}(\tfrac{1}{4}\lambda'\nu' - \tfrac{1}{4}\nu'^{2} - \tfrac{1}{2}\nu'') (\tfrac{1}{4}\lambda' - \tfrac{1}{2}\nu')\bigr\} \\
  &= \left(-40 \frac{m^{2}}{r^{3}} + 8\frac{m}{r^{2}}\right) \sin\theta,\displaybreak[0] \\
%
  \frac{\dd\mf{K}''}{\dd\nu''}
  &= -2e^{\frac{1}{2}(\lambda + \nu)} \sin\theta
  \cdot 2r^{2} e^{-2\lambda}(\tfrac{1}{4}\lambda'\nu' - \tfrac{1}{4}\nu'^{2} - \tfrac{1}{2}\nu'') \\
  &= \left(16 \frac{m^{2}}{r^{2}} - 8\frac{m}{r}\right) \sin\theta.
\end{align*}
On substituting these values, \Eq{(62.5)}~is verified exactly.

The alternative law $\Ham K'/\Ham g_{\mu\nu} = 0$ is also satisfied by the same solution.
For
\[
\delta(G_{\mu\nu} G^{\mu\nu} \sqrt{-g})
= G_{\mu\nu}\, \delta(G^{\mu\nu} \sqrt{-g}) + G^{\mu\nu} \sqrt{-g}\, \delta G_{\mu\nu},
\]
hence the variation of $K' \sqrt{-g}$ vanishes wherever $G_{\mu\nu} = 0$. Any field of gravitation
agreeing with Einstein's law will satisfy the alternative law proposed,
but not usually \Foreign{vice versa}.

There are doubtless other symmetrical solutions for the alternative laws
of gravitation which are not permitted by Einstein's law, since the differential
equations are now of the fourth order and involve two extra boundary conditions
either at the particle or at infinity. It may be asked, Why should
these be excluded in nature? We can only answer that it may be for the
same reason that negative mass, doublets, electrons of other than standard
mass, or other theoretically possible singularities in the world, do not occur;
the ultimate particle satisfies conditions which are at present unknown to us.

It would seem therefore that there are three admissible laws of gravitation~\Eq{(62.2)}.
Each can give precisely the same gravitational field of the sun, and
all astronomical phenomena are the same whichever law is used. Small
differences may appear in the cross-terms due to two or more attracting
bodies; but as was shown in our discussion of the lunar theory these are too
small to be detected by astronomical observation. Each law gives precisely
the same mechanical phenomena, since the conservation of energy and
momentum is satisfied. When we ask which of the three is the law of the
actual world, I am not sure that the question has any meaning. The subject
is very mystifying, and the following suggestions are put forward very
tentatively.

The energy\hyp{}tensor has been regarded as giving the definition of matter,
since it comprises the properties by which matter is described in physics.
Our three energy\hyp{}tensors give us three alternative material worlds; and the
question is which of the three are we looking at when we contemplate the
world around us; but if these three material worlds are each doing the same
thing (within the limits of observational accuracy) it seems impossible to
decide whether we are observing one or other or all three.

To put it another way, an observation involves the relation of the~$T_{\mu}^{\nu}$ of
our bodies to the~$T_{\mu}^{\nu}$ of external objects, or alternatively of the respective
${T'}_{\mu}^{\nu}$ or~${T''}_{\mu}^{\nu}$. If these are the same relation it seems meaningless to ask which
of the three bodies and corresponding worlds the relation is between. After
all it is the relation which is the reality. In accepting $T_{\mu}^{\nu}$ as the energy\hyp{}tensor
we are simply choosing the simplest of three possible modes of representing
the observation.

One cannot but suspect that there is some identical relation between the
Hamiltonian derivatives of the three fundamental invariants. If this relation
were discovered it would perhaps clear up a rather mysterious subject.

\Section{63.}{Gravitational flux from a particle}
\index{Flux!gravitational}%
\index{Gravitational flux}%

Let us consider an empty region of the world, and try to create in it one
or more particles of small mass~$\delta m$ by variations of the~$g_{\mu\nu}$ within the region.
By \Eq{(60.12)} and~\Eq{(60.2)},
\[
\delta \int G\sqrt{-g}\, d\tau = 8\pi \sum \delta m \cdot ds,
\Tag{(63.1)}
\]
and by~\Eq{(60.42)} the left-hand side is zero because the space is initially empty.
In the actual world particles for which $\delta m \cdot ds$ is negative do not exist; hence it
is impossible to create any particles in an empty region, so long as we adhere
to the condition that the~$g_{\mu\nu}$ and their first derivatives must not be varied on
the boundary. To permit the creation of particles we must give up this
restriction and accordingly resurrect the term
\[
\delta \int G\sqrt{-g}\, d\tau
= \int \frac{\dd}{\dd x_{\alpha}} \left(\mf{g}^{\mu\nu}\, \delta\left(\frac{\dd\mf{L}}{\dd\mf{g}_{\alpha}^{\mu\nu}}\right)\right) d\tau,
\Tag{(63.2)}
\]
which was discarded from~\Eq{(60.3)}. On performing the first integration, \Eq{(63.2)}~gives
the flux of the normal component of
\[
\mf{g}^{\mu\nu}\, \delta\left(\frac{\dd\mf{L}}{\dd\mf{g}_{\alpha}^{\mu\nu}}\right)
= g^{\mu\nu} \sqrt{-g}\, \delta\bigl[-\{\mu\nu, \alpha\} + g_{\mu}^{\alpha} \{\nu\beta, \beta\}\bigr]
\Tag{(63.3)}
\]
across the three\hyp{}dimensional surface of the region. The close connection of
this expression with the value of~$\mf{t}_{\mu}^{\nu}$ in~\Eq{(59.6)} should be noticed.

Take the region in the form of a long tube and create a particle of gravitational
mass~$\delta m$ along its axis. The flux~\Eq{(63.3)} is an invariant, since $\delta m \cdot ds$
is invariant, so we may choose the special coordinates of \SecRef{38} for which the
particle is at rest. Take the tube to be of radius~$r$ and calculate the flux for
a length of tube $dt = ds$. The normal component of~\Eq{(63.3)} is given by $\alpha = 1$
and accordingly the flux is
\begin{multline*}
  \int g^{\mu\nu} \sqrt{-g}\, \delta\bigl[-\{\mu\nu, 1\} + g_{\mu}^{1} \{\nu\beta, \beta\}\bigr]\, d\theta\, d\phi\, dt \\
  = 4\pi r^{2}\, ds
  \cdot \left[ -g^{\mu\nu}\, \delta\{\mu\nu, 1\} + g^{\mu1}\, \delta \left(\frac{\dd}{\dd x_{\nu}} \log \sqrt{-g}\right)\right],
  \Tag{(63.4)}
\end{multline*}
which by~\Eq{(38.5)}
%[** TN: Re-breaking]
\begin{multline*}
  = 4\pi r^{2}\, ds\,
  \biggl\{e^{-\lambda}\, \delta(\tfrac{1}{2}\lambda')
  - \frac{1}{r^{2}}\, \delta(re^{-\lambda})
  - \frac{1}{r^{2} \sin^{2}\theta}\, \delta(r\sin^{2}\theta\, e^{-\lambda}) \\
  - e^{-\nu}\, \delta(\tfrac{1}{2} e^{\nu - \lambda} \nu')
  - e^{-\lambda}\, \delta\left(\frac{2}{r}\right)\biggr\}.
  \Tag{(63.5)}
\end{multline*}
Remembering that the variations involve only~$\delta m$, this reduces to
\begin{align*}
&4\pi r^{2}\, ds \left(-\delta\gamma' - \frac{2}{r}\, \delta\gamma\right) \\
  &\qquad= 8\pi\, \delta m \cdot ds.
  \Tag{(63.6)}
\end{align*}

We have ignored the flux across the two ends of the tube. It is clear
that these will counterbalance one another.

This verification of the general result~\Eq{(63.1)} for the case of a single particle
gives another proof of the identity of gravitational mass with inertial mass.
\index{Gravitational mass of sun!equality with inertial mass}%
\index{Inertial mass!equal to gravitational mass}%
\index{Mass!gravitational and inertial}%

We see then that a particle is attended by a certain flux of the quantity~\Eq{(63.3)}
across all surrounding surfaces. It is this flux which makes the presence
of a massive particle known to us, and characterises it; in an observational
sense the flux \emph{is} the particle. So long as the space is empty the flux is the
same across all surrounding surfaces however distant, the radius~$r$ of the tube
having disappeared in the result; so that in a sense the Newtonian law of
the inverse square has a direct analogue in Einstein's theory.

In general the flux is modified in passing through a region containing
other particles or continuous matter, since the first term on the right of~\Eq{(60.3)}
no longer vanishes. This may be ascribed analytically to the non\hyp{}linearity of
the field equations, or physically to the fact that the outflowing influence can
scarcely exert its action on other matter without being modified in the process.
In our verification for the single particle the flux due to~$\delta m$ was independent
of the value of~$m$ originally present; but this is an exceptional case due to
symmetrical conditions which cause the integral of $T^{\mu\nu}\, \delta g_{\mu\nu}$ to vanish although
$T^{\mu\nu}$~is not zero. Usually the flux due to~$\delta m$ will be modified if other matter
is initially present.

For an isolated particle $m\, ds$~in any region is stationary for variations of
its track, this condition being equivalent to~\Eq{(56.6)}. Hence for this kind of
variation the action $8\pi \sum m\, ds$ in a region is stationary. The question arises
how this is to be reconciled with our previous result (\SecRef{60}) that the principle
of stationary action is untrue for regions containing matter. The reason is
this:---when we give arbitrary variations to the~$g_{\mu\nu}$, the matter in the tube
will in general cease to be describable as a \emph{particle,} because it has lost the
symmetry of its field\footnotemark.\footnotetext
  {It will be remembered that in deriving~\Eq{(56.6)} we had to assume the symmetry of the particle.}
The action therefore is only stationary for a special
kind of variation of~$g_{\mu\nu}$ in the neighbourhood of each particle which deforms
the track without destroying the symmetry of the particle; it is not stationary
for unlimited variations of the~$g_{\mu\nu}$.

The fact that the variations which cause the failure of the principle of
stationary action---those which violate the symmetry of the particles---are
impossible in the actual world is irrelevant. Variations of the track of the
particle are equally impossible, since in the actual world a particle cannot
move in any other way than that in which it does move. The whole point of
the Principle of Stationary Action is to show the relation of an actual state
of the world to slightly varied states which cannot occur. Thus the break-down
of the principle cannot be excused. But we can see now why it gives
correct results in ordinary mechanics, which takes the tracks of the particles
as the sole quantities to be varied, and disregards the more general variations
of the state of the world for which the principle ceases to be true.

\Section{64.}{Retrospect}

We have developed the mathematical theory of a continuum of four
dimensions in which the points are connected in pairs by an absolute relation
called the interval. In order that this theory may not be merely an exercise
in pure mathematics, but may be applicable to the actual world, the quantities
appearing in the theory must at some point be tied on to the things of
experience. In the earlier chapters this was done by identifying the mathematical
interval with a quantity which is the result of practical measurement
with scales and clocks. In the present chapter this point of contact of theory
and experience has passed into the background, and attention has been
focussed on another opportunity of making the connection. The quantity
$G_{\mu}^{\nu} - \frac{1}{2} g_{\mu}^{\nu}G$ appearing in the theory is, on account of its property of conservation,
now identified with matter, or rather with the mechanical abstraction
\index{Matter!identification of}%
of matter which comprises the measurable properties of mass, momentum and
stress sufficing for all mechanical phenomena. By making the connection
between mathematical theory and the actual world at this point, we obtain a
great lift forward.

Having now two points of contact with the physical world, it should
become possible to construct a complete cycle of reasoning. There is one
chain of pure deduction passing from the mathematical interval to the mathematical
energy\hyp{}tensor. The other chain binds the physical manifestations of
the energy\hyp{}tensor and the interval; it passes from matter as now defined by
the energy\hyp{}tensor to the interval regarded as the result of measurements made
with this matter. The discussion of this second chain still lies ahead of us.

If actual matter had no other properties save such as are implied in the
functional form of $G_{\mu}^{\nu} - \frac{1}{2} g_{\mu}^{\nu}G$, it would, I think, be impossible to make measurements
with it. The property which makes it serviceable for measurement is
discontinuity (not necessarily in the strict sense, but discontinuity from the
macroscopic standpoint, i.e.\ atomicity). So far our only attempt to employ
\index{Atomicity}%
the new-found matter for measuring intervals has been in the study of the
dynamics of a particle in \SecRef{56}; we had there to assume that discrete particles
exist and further that they have necessarily a symmetry of field; on this
understanding we have completed the cycle for one of our most important
test\hyp{}bodies---the moving particle---the geodesic motion of which is used, especially
in astronomy, for comparing intervals. But the theory of the use of
matter for the purpose of measuring intervals will be taken up in a more
general way at the beginning of the next chapter, and it will be seen how
profoundly the existence of the complete cycle has determined that outlook
on the world which we express in our formulation of the laws of mechanics.

It is a feature of our attitude towards nature that we pay great regard to
\index{Energy\hyp{}tensor of matter!obtained by Hamiltonian differentiation}%
\index{Principle!of least action}%
\index{Stationary action, principle of}%
that which is permanent; and for the same reason the creation of anything
\index{Creation of the physical world}%
in the midst of a region is signalised by us as more worthy of remark than
its entry in the orthodox manner through the boundary. Thus when we
consider how an invariant depends on the variables used to describe the
world, we attach more importance to changes which result in creation than
to changes which merely involve transfer from elsewhere. It is perhaps for
this reason that the Hamiltonian derivative of an invariant gives a quantity
\index{Hamiltonian derivative!creative aspect of}%
of greater significance for us than, for example, the ordinary derivative. The
Hamiltonian derivative has a creative quality, and thus stands out in our
minds as an active agent working in the passive field of space-time. Unless
this idiosyncrasy of our practical outlook is understood, the Hamiltonian
method with its casting away of boundary integrals appears somewhat artificial;
but it is actually the natural method of deriving physical quantities
prominent in our survey of the world, because it is guided by those principles
which have determined their prominence. The particular form of the
Hamiltonian method known as Least Action, in which special search is made
\index{Action, principle of Stationary}%
for Hamiltonian derivatives which vanish, does not appear at present to admit
of any very general application. In any case it seems better adapted to give
neat mathematical formulae than to give physical insight; to grasp the
equality or identity of two physical quantities is simpler than to ponder over
the behaviour of the quantity which is their difference---distinguished though
it may be by the important property of being incapable of existing!

According to the views reached in this chapter the law of gravitation
$G_{\mu\nu} = 0$ is not to be regarded as an expression for the natural texture of the
continuum, which can only be forcibly broken at points where some extraneous
agent (matter) is inserted. The differentiation of occupied and unoccupied
space arises from our particular outlook on the continuum, which, as explained
above, is such that the Hamiltonian derivatives of the principal invariant~$G$
stand out as active agents against the passive background. It is therefore
the regions in which these derivatives vanish which are regarded by us as
unoccupied; and the law $G_{\mu\nu} = 0$ merely expresses the discrimination made
by this process.

Among the minor points discussed, we have considered the speed of propagation
of gravitational influence. It is presumed that the speed is that
of light, but this does not appear to have been established rigorously. Any
absolute influence must be measured by an invariant, particularly the invariant
$B_{\mu\nu\sigma}^{\rho} B_{\rho}^{\mu\nu\sigma}$. The propagation of this invariant does not seem to have
been investigated.

The ordinary potential energy of a weight raised to a height is not counted
\index{Potential energy}%
as energy in our theory and does not appear in our energy\hyp{}tensor. It is found
superfluous because the property of our energy\hyp{}tensor has been formulated
as a general law which from the absolute point of view is simpler than the
formal law of conservation. The potential energy and momentum~$\mf{t}_{\mu}^{\nu}$ needed
if the formal law of conservation is preserved is not a tensor, and must be
regarded as a mathematical fiction, not as representing any significant condition
of the world. The pseudo\hyp{}energy\hyp{}tensor~$\mf{t}_{\mu}^{\nu}$ can be created and destroyed
at will by changes of coordinates; and even in a world containing no attracting
matter (flat space-time) it does not necessarily vanish. It is therefore impossible
to regard it as of a nature homogeneous with the proper energy\hyp{}tensor.

\Chapter{V}{Curvature of Space and Time}
\index{Curvature!of 4-dimensional manifold}%

\Section{65.}{Curvature of a four\hyp{}dimensional manifold}

\lettrine{\textcolor{lettrinecolour}{I}}{n} the general Riemannian geometry admitted in our theory the~$g_{\mu\nu}$ may
be any $10$~functions of the four coordinates~$x_{\mu}$.

A four\hyp{}dimensional continuum obeying Riemannian geometry can be
represented graphically as a surface of four dimensions drawn in a Euclidean
hyperspace of a sufficient number of dimensions. Actually $10$~dimensions are
required, corresponding to the number of the~$g_{\mu\nu}$. For let $(y_{1}, y_{2}, y_{3}, \dots, y_{10})$ be
rectangular Euclidean coordinates, and $(x_{1}, x_{2}, x_{3}, x_{4})$ parameters on the surface;
the equations of the surface will be of the form
\[
y_{1} = f_{1}(x_{1}, x_{2}, x_{3}, x_{4}),\ \dots\dots,\
y_{10} = f_{10}(x_{1}, x_{2}, x_{3}, x_{4}).
\]

For an interval on the surface, the Euclidean geometry of the~$y$'s gives
\begin{align*}
  -ds^{2}
  &= dy_{1}^{2} + dy_{2}^{2} + dy_{3}^{2} + \cdots + dy_{10}^{2} \\
  &= \biggl\{\left(\frac{\dd f_{1}}{\dd x_{1}}\right)^{2}
  + \left(\frac{\dd f_{2}}{\dd x_{1}}\right)^{2} + \cdots
  + \left(\frac{\dd f_{10}}{\dd x_{1}}\right)^{2}\biggr\} dx_{1}^{2} + \cdots \\
  &\qquad + \biggl\{\frac{\dd f_{1}}{\dd x_{1}}\, \frac{\dd f_{1}}{\dd x_{2}}
  + \cdots + \frac{\dd f_{10}}{\dd x_{1}}\, \frac{\dd f_{10}}{\dd x_{2}}\biggr\}
  2\, dx_{1}\, dx_{2} + \cdots.
\end{align*}
Equating the coefficients to the given functions~$g_{\mu\nu}$, we have $10$~partial differential
equations of the form
\[
\frac{\dd f_{1}}{\dd x_{\mu}}\, \frac{\dd f_{1}}{\dd x_{\nu}}
+ \cdots + \frac{\dd f_{10}}{\dd x_{\mu}}\, \frac{\dd f_{10}}{\dd x_{\nu}}
= g_{\mu\nu},
\]
to be satisfied by the $10$ $f$'s. Clearly it would not be possible to satisfy these
equations with less than~$10$ $f$'s except in special cases.

When we use the phrase ``curvature'' in connection with space-time, we
always think of it as embedded in this way in a Euclidean space of higher
dimensions. It is not suggested that the higher space has any existence; the
purpose of the representation is to picture more vividly the metrical properties
of the world. It must be remembered too that a great variety of
four\hyp{}dimensional surfaces in $10$~dimensions will possess the same metric, i.e.\ be
applicable to one another by bending without stretching, and any one of these
can be chosen to represent the metric of space-time. Thus a geometrical property
of the chosen representative surface need not necessarily be a property
belonging intrinsically to the space-time continuum.

A four\hyp{}dimensional surface free to twist about in six additional dimensions
has bewildering possibilities. We consider first the simple case in which the
surface, or at least a small portion of it, can be drawn in Euclidean space of
five dimensions.

Take a point on the surface as origin. Let $(x_{1}, x_{2}, x_{3}, x_{4})$ be rectangular
coordinates in the tangent plane (four\hyp{}dimensional) at the origin; and let the
fifth rectangular axis along the normal be~$z$. Then by Euclidean geometry
\[
-ds^{2} = dx_{1}^{2} + dx_{2}^{2} + dx_{3}^{2} + dx_{4}^{2} + dz^{2},
\Tag{(65.1)}
\]
imaginary values of~$ds$ corresponding as usual to real distances in space. The
four\hyp{}dimensional surface will be specified by a single equation between the
five coordinates, which we may take to be
\[
z = f(x_{1}, x_{2}, x_{3}, x_{4}).
\]
If the origin is a regular point this can be expanded in powers of the~$x$'s. The
deviation from the tangent plane is of the second order compared with distances
parallel to the plane; consequently $z$~does not contain linear terms in
the~$x$'s. The expansion accordingly starts with a homogeneous quadratic
function, and the equation is of the form
\[
2z = a_{\mu\nu} x_{\mu} x_{\nu},
\Tag{(65.2)}
\]
correct to the second order. For a fixed value of~$z$ the quadric~\Eq{(65.2)} is called
the \emph{indicatrix}.
\index{Indicatrix}%

The radius of curvature of any normal section of the surface is found by
the well-known method. If $t$~is the radius of the indicatrix in the direction of
the section (direction cosines $l_{1}$, $l_{2}$, $l_{3}$, $l_{4}$), the radius of curvature is
\[
\rho = \frac{t^{2}}{2z} = \frac{1}{a_{\mu\nu} l_{\mu} l_{\nu}}.
\]
In particular, if the axes are rotated so as to coincide with the principal axes
of the indicatrix, \Eq{(65.2)}~becomes
\[
2z = k_{1}\, dx_{1}^{2} + k_{2}\, dx_{2}^{2} + k_{3}\, dx_{3}^{2} + k_{4}\, dx_{4}^{2},
\Tag{(65.3)}
\]
and the principal radii of curvature of the surface are the reciprocals of
$k_{1}$, $k_{2}$, $k_{3}$,~$k_{4}$.

Differentiating~\Eq{(65.2)}
\[
dz = a_{\mu\nu} x_{\mu}\, dx_{\nu},\quad
dz^{2} = a_{\mu\nu} x_{\mu}\, dx_{\nu} \cdot a_{\sigma\tau} x_{\sigma}\, dx_{\tau}.
\]
Hence, substituting in~\Eq{(65.1)}
\[
-ds^{2} = dx_{1}^{2} + dx_{2}^{2} + dx_{3}^{2} + dx_{4}^{2}
+ (a_{\mu\nu} a_{\sigma\tau} x_{\mu} x_{\sigma})\, dx_{\nu}\, dx_{\tau}
\]
for points in the four\hyp{}dimensional continuum. Accordingly
\[
-g_{\nu\tau} = g_{\nu}^{\tau} + a_{\mu\nu} a_{\sigma\tau} x_{\mu} x_{\sigma}.
\Tag{(65.4)}
\]

Hence at the origin the~$g_{\mu\nu}$ are Euclidean; their first derivatives vanish;
and their second derivatives are given by
\[
\frac{\dd^{2} g_{\nu\tau}}{\dd x_{\mu}\, \dd x_{\sigma}}
= -(a_{\mu\nu} a_{\sigma\tau} + a_{\sigma\nu} a_{\mu\tau}),
\]
by~\Eq{(35.5)}.

Calculating the Riemann\hyp{}Christoffel tensor by~\Eq{(34.5)}, since the first derivatives
vanish,
\begin{align*}
  B_{\mu\nu\sigma\rho}
  &= \frac{1}{2}\left(\frac{\dd g_{\sigma\rho}}{\dd x_{\mu}\, \dd x_{\nu}}
  + \frac{\dd g_{\mu\nu}}{\dd x_{\sigma}\, \dd x_{\rho}}
  - \frac{\dd g_{\mu\sigma}}{\dd x_{\nu}\, \dd x_{\rho}}
  - \frac{\dd g_{\nu\rho}}{\dd x_{\mu}\, \dd x_{\sigma}}\right) \\
  &= a_{\mu\nu} a_{\sigma\rho} - a_{\mu\sigma} a_{\nu\rho}.
  \Tag{(65.51)}
\end{align*}
Hence, remembering that the~$g^{\tau\rho}$ have Euclidean values~$-g_{\sigma}^{\rho}$,
\[
G_{\mu\nu} = g^{\sigma\rho} B_{\mu\nu\sigma\rho}
= -a_{\mu\nu} (a_{11} + a_{22} + a_{33} + a_{44}) + a_{\mu\sigma} a_{\nu\sigma}.
\Tag{(65.52)}
\]
In particular
\begin{align*}
  G_{11}
  &= -a_{11} (a_{11} + a_{22} + a_{33} + a_{44}) + a_{11}^{2} + a_{12}^{2} + a_{13}^{2} + a_{14}^{2} \\
  &= (a_{12}^{2} - a_{11} a_{22}) + (a_{13}^{2} - a_{11} a_{33}) + (a_{14}^{2} - a_{11} a_{44}).
  \Tag{(65.53)}
\end{align*}
Also
%[** TN: Re-breaking]
\begin{align*}
  G &= g^{\mu\nu} G_{\mu\nu}
  = -G_{11} - G_{22} - G_{33} - G_{44} \\
  &= -2 \bigl\{(a_{12}^{2} - a_{11} a_{22}) + (a_{13}^{2} - a_{11} a_{33}) + (a_{14}^{2} - a_{11} a_{44}) \\
  &\qquad + (a_{23}^{2} - a_{22} a_{33}) + (a_{24}^{2} - a_{22} a_{44}) + (a_{34}^{2} - a_{33} a_{44})\bigr\}.
  \Tag{(65.54)}
\end{align*}

When the principal axes are taken as in~\Eq{(65.3)}, these results become
\[
\left.
\begin{aligned}
  G_{11} &= -k_{1}(k_{2} + k_{3} + k_{4})\\
  G_{22} &= -k_{2}(k_{1} + k_{3} + k_{4});\ \text{etc.}
\end{aligned}
\right\}
\Tag{(65.55)}
\]
and
\[
G = 2(k_{1}k_{2} + k_{1}k_{3} + k_{1} k_{4} + k_{2} k_{3} + k_{2} k_{4} + k_{3} k_{4}).
\Tag{(65.6)}
\]

The invariant~$G$ has thus a comparatively simple interpretation in terms
of the principal radii of curvature. It is a generalisation of the well-known
\index{Curvature!radius of spherical}%
\index{Spherical curvature, radius of}%
invariant for two\hyp{}dimensional surfaces $1/\rho_{1}\rho_{2}$, or~$k_{1} k_{2}$. But this interpretation
is only possible in the simple case of five dimensions. In general five dimensions
are not sufficient to represent even the small portion of the surface near the
origin; for if we set $G_{\mu\nu} = 0$ in~\Eq{(65.55)}, we obtain $k_{\mu} = 0$, and hence by~\Eq{(65.51)}
$B_{\mu\nu\sigma\rho} = 0$. Thus it is not possible to represent a natural gravitational field
($G_{\mu\nu} = 0$, $B_{\mu\nu\sigma\rho} \neq 0$) in five Euclidean dimensions.

In the more general case we continue to call the invariant~$G$ the Gaussian
curvature although the interpretation in terms of normal curvatures no longer
\index{Normal, $6$-dimensional}%
holds. It is convenient also to introduce a quantity called the \emph{radius of
spherical curvature,} viz.\ the radius of a hypersphere which has the same
\index{Gaussian curvature}%
Gaussian curvature as the surface considered\footnotemark.\footnotetext
  {A hypersphere of four dimensions is by definition a four\hyp{}dimensional surface drawn in five
  dimensions so that \Eq{(65.6)} applies to it. Accordingly if its radius is~$R$, we have $G = 12/R^{2}$. For
  three dimensions $G = 6/R^{2}$; for two dimensions $G = 2/R^{2}$.}

Considering the geometry of the general case, in $10$~dimensions the normal is
a six\hyp{}dimensional continuum in which we can take rectangular axes $z_{1}$, $z_{2}$, \dots, $z_{6}$.
The surface is then defined by six equations which near the origin take the
form
\[
2z_{r} = a_{r\mu\nu} x_{\mu} x_{\nu}\quad (r = 1, 2, \dots, 6).
\]
The radius of curvature of a normal section in the direction~$l_{\mu}$ is then
\[
\rho = \frac{t^{2}}{2\sqrt{z_{1}^{2} + z_{2}^{2} + \cdots + z_{6}^{2}}}
 = \frac{1}{\sqrt{(a_{1\mu\nu} l_{\mu} l_{\nu})^{2} + \cdots + (a_{6\mu\nu} l_{\mu} l_{\nu})^{2}}}.
\]
It is, however, of little profit to develop the properties of normal curvature,
which depend on the surface chosen to represent the metric of space-time
and are not intrinsic in the metric itself. We therefore follow a different plan,
introducing the radius of spherical curvature which has invariant properties.

Reverting for the moment to five dimensions, consider the \emph{three\hyp{}dimensional}
space formed by the section of our surface by $x_{1} = 0$. Let $G_{(1)}$~be its Gaussian
curvature. Then $G_{(1)}$~is formed from~$G$ by dropping all terms containing the
\index{Curvature!quadric of}%
suffix~$1$---a dimension which no longer enters into consideration. Accordingly
$G - G_{(1)}$ consists of those terms of~$G$ which contain the suffix~$1$; and by~\Eq{(65.53)}
and~\Eq{(65.54)} we have
\[
\tfrac{1}{2}(G - G_{(1)}) = -G_{11}.
\Tag{(65.71)}
\]
Introducing the value $g_{11} = -1$ at the origin
\[
G_{11} - \tfrac{1}{2}g_{11} G = \tfrac{1}{2}G_{(1)}.
\Tag{(65.72)}
\]

This result obtained for five dimensions is perfectly general. From the
manner in which \Eq{(65.4)}~was obtained, it will be seen that each of the six~$z$'s will
make contributions to~$g_{\nu\tau}$ which are simply additive; we have merely to sum
$a_{\mu\nu} a_{\sigma\tau} x_{\mu} x_{\sigma}$ for the six values of~$a_{\mu\nu} a_{\sigma\tau}$ contributed by the six terms~$dz_{r}^{2}$. All the
subsequent steps involve linear equations and the work will hold for six~$z$'s
just as well as for one~$z$. Hence \Eq{(65.72)}~is true in the general case when the
representation requires $10$~dimensions.

Now consider the invariant quadric
\[
(G_{\mu\nu} - \tfrac{1}{2} g_{\mu\nu} G)\, dx_{\mu}\, dx_{\nu} = 3.
\Tag{(65.81)}
\]
Let $\rho_{1}$~be the radius of this quadric in the $x_{1}$~direction, so that $dx_{\mu} = (\rho_{1} , 0, 0, 0)$
is a point on the quadric; the equation gives
\[
(G_{11} - \tfrac{1}{2} g_{11}G) \rho_{1}^{2} = 3,
\]
so that by~\Eq{(65.72)}
\[
G_{(1)} = \frac{6}{\rho_{1}^{2}}.
\Tag{(65.82)}
\]

But for a hypersphere of radius~$R$ of \emph{three} dimensions ($k_{1} = k_{2} = k_{3} = 1/R$;
$k_{4}$~disappears) the Gaussian curvature is~$6/R^{2}$. Hence $\rho_{1}$~is the radius of
spherical curvature of the three\hyp{}dimensional section of the world perpendicular
to the axis~$x_{1}$.

Now the quadric~\Eq{(65.81)} is invariant, so that the axis~$x_{1}$ may be taken in
any arbitrary direction. Accordingly we see that---

\emph{The radius of the quadric $(G_{\mu\nu} = \tfrac{1}{2} g_{\mu\nu}G)\, dx_{\mu}\, dx_{\nu} = 3$ in any direction is equal
to the radius of spherical curvature of the corresponding three\hyp{}dimensional
section of the world.}

We call this quadric the \emph{quadric of curvature}.
\index{Quadric of curvature}%

\Section{66.}{Interpretation of Einstein's law of gravitation}

We take the later form of Einstein's law~\Eq{(37.4)}
\[
G_{\mu\nu} = \lambda g_{\mu\nu},
\Tag{(66.1)}
\]
in empty space, $\lambda$~being a universal constant at present unknown but so small
as not to upset the agreement with observation established for the original
form $G_{\mu\nu} = 0$. We at once obtain $G = 4\lambda$, and hence
\[
G_{\mu\nu} - \tfrac{1}{2} g_{\mu\nu} G = -\lambda g_{\mu\nu}.
\]
Substituting in~\Eq{(65.81)} the quadric of curvature becomes
\[
-\lambda g_{\mu\nu}\, dx_{\mu}\, dx_{\nu} = 3,
\]
or
\[
-ds^{2} = 3/\lambda.
\Tag{(66.2)}
\]
That is to say, the quadric of curvature is a sphere of radius~$\sqrt{3/\lambda}$, and the
radius of curvature in every direction\footnote
  {For brevity I use the phrase ``radius of curvature in a direction'' to mean the radius of
  spherical curvature of the three\hyp{}dimensional section of the world at right angles to that direction.
  There is no other radius of curvature \emph{associated with a direction} likely to be confused with it.}
and at every point in empty space has
the constant length~$\sqrt{3/\lambda}$.

Conversely if the directed radius of curvature in empty space is homogeneous
and isotropic Einstein's law will hold.

The statement that the radius of curvature is a constant length requires
more consideration before its full significance is appreciated. Length is not
absolute, and the result can only mean \emph{constant relative to the material standards
of length} used in all our measurements and in particular in those measurements
which verify $G_{\mu\nu} = \lambda g_{\mu\nu}$. In order to make a direct comparison the material
unit must be conveyed to the place and pointed in the direction of the length
to be measured. It is true that we often use indirect methods avoiding actual
transfer or orientation; but the justification of these indirect methods is that
they give the same result as a direct comparison, and their validity depends
on the truth of the fundamental laws of nature. We are here discussing the
most fundamental of these laws, and to admit the validity of the indirect
methods of comparison at this stage would land us in a vicious circle. Accordingly
the precise statement of our result is that the radius of curvature
at any point and in any direction is in constant proportion to the length of a
specified material unit placed at the same point and orientated in the same
direction.

This becomes more illuminating if we invert the comparison---

\emph{The length of a specified material structure bears a constant ratio to the
radius of curvature of the world at the place and in the direction in which it
lies}.\hfill\tTag{(66.3)}\quad\null

The law no longer appears to have any reference to the constitution of an
empty continuum. It is a law of material structure showing what dimensions
a specified collection of molecules must take up in order to adjust itself to
equilibrium with surrounding conditions of the world.

The possibility of the existence of an electron in space is a remarkable
phenomenon which we do not yet understand. The details of its structure
must be determined by some unknown set of equations, which apparently
admit of only two discrete solutions, the one giving a negative electron and
the other a positive electron or proton. If we solve these equations to find
the radius of the electron in any direction, the result must necessarily take
the form
\begin{quote}
  radius of electron in given direction $=$ numerical constant $\times$ some
  function of the conditions in the space into which the electron
  was inserted.
\end{quote}
And since the quantity on the left is a directed length, the quantity on the
right must be a directed length. We have just found one directed length
characteristic of the empty space in which the electron was introduced, viz.\
the radius of spherical curvature of a corresponding section of the world.
Presumably by going to third or fourth derivatives of the~$g_{\mu\nu}$ other independent
directed lengths could be constructed; but that seems to involve an unlikely
complication. There is strong ground then for anticipating that the solution
of the unknown equations will be
\begin{quote}
radius of electron in any direction $=$ numerical constant $\times$ radius of
curvature of space-time in that direction.
\end{quote}
This leads at once to the law~\Eq{(66.3)}.

As with the electron, so with the atom and aggregations of atoms forming
the practical units of material structure. Thus we see that Einstein's law of
gravitation is the almost inevitable outcome of the use of material measuring\hyp{}appliances
for surveying the world, whatever may be the actual laws under
which material structures are adjusted in equilibrium with the empty space
around them.

Imagine first a world in which the curvature, referred to some chosen
(non\hyp{}material) standard of measurement, was not isotropic. An electron inserted
in this would need to have the same anisotropy in order that it might
obey the same detailed conditions of equilibrium as a symmetrical electron in
an isotropic world. The same anisotropy persists in any material structure
formed of these electrons. Finally when we \emph{measure} the world, i.e.\ make comparisons
with material structures, the anisotropy occurs on both sides of the
comparison and is eliminated. Einstein's law of gravitation expresses the
\index{Einstein's law of gravitation!interpretation of}%
result of this elimination. The symmetry and homogeneity expressed by
Einstein's law is not a property of the external world, but a property of the
operation of measurement.

From this point of view it is inevitable that the constant~$\lambda$ cannot be
zero; so that empty space has a finite radius of curvature relative to familiar
standards. An electron could never decide how large it ought to be unless
there existed some length independent of itself for it to compare itself with.

It will be noticed that our rectangular coordinates $(x_{1}, x_{2}, x_{3}, x_{4})$ in this
and the previous section approximate to Euclidean, not Galilean, coordinates.
Consequently $x_{4}$~is imaginary time, and $G_{(1)}$~is not in any real direction
in the world. There is no radius of curvature in a real timelike direction.
This does not mean that our discussion is limited to three dimensions; it
includes all directions in the four\hyp{}dimensional world outside the light-cone,
and applies to the space\hyp{}dimensions of material structures moving with any
speed up to the speed of light. The real quadric of curvature terminates at
the light-cone, and the mathematical continuation of it lies not inside the
cone but in directions of imaginary time which do not concern us.

By consideration of extension in timelike directions we obtain a confirmation
of these views, which is, I think, not entirely fantastic. We have said that
an electron would not know how large it ought to be unless there existed independent
lengths in space for it to measure itself against. Similarly it would
not know how long it ought to exist unless there existed a length in time for
it to measure itself against. But there is no radius of curvature in a timelike
direction; so the electron does \emph{not} know how long it ought to exist. Therefore
it just goes on existing indefinitely.

The alternative laws of gravitation discussed in \SecRef{62} would be obtained if
the radius of the unit of material structure adjusted itself as a definite fraction
not of the radius of curvature, but of other directed lengths (of a more complex
origin) characteristic of empty space-time.

In \SecRef{56} it was necessary to postulate that the gravitational field due to an
ultimate particle of matter has symmetrical properties. This has now been
\index{Particle!symmetry of}%
justified. We have introduced a new and far\hyp{}reaching principle into the
relativity theory, viz.\ that symmetry itself can only be relative; and the
\index{Symmetry!a relative attribute}%
\index{Symmetry!of a particle}%
particle, which so far as mechanics is concerned is to be identified with its
gravitational field, is the standard of symmetry. We reach the same result if
we attempt to define symmetry by the propagation of light, so that the cone
$ds = 0$ is taken as the standard of symmetry. It is clear that if the locus
$ds = 0$ has complete symmetry about an axis (taken as the axis of~$t$) $ds^{2}$~must
be expressible by the formula~\Eq{(38.12)}.

The double\hyp{}linkage of field and matter, matter and field, will now be
realised. Matter is derived from the fundamental tensor~$g_{\mu\nu}$ by the expression
$G_{\mu}^{\nu} - \tfrac{1}{2}g_{\mu}^{\nu}G$; but it is matter so derived which is initially used to measure
the fundamental tensor~$g_{\mu\nu}$. We have in this section considered one simple
consequence of this cycle---the law of gravitation. It needs a broader analysis
to follow out the full consequences, and this will be attempted in Chapter~\ChapNum{VII},
Part~II\@.

\Section{67.}{Cylindrical and spherical space-time}

According to the foregoing section $\lambda$~does not vanish, and there is a
small but finite curvature at every point of space and time. This suggests
the consideration of the shape and size of the world as a whole.
\index{World!shape of}%

Two forms of the world have been suggested---

(1) Einstein's cylindrical world. Here the space\hyp{}dimensions correspond
\index{Cylindrical world}%
\index{Einstein's cylindrical world}%
to a sphere, but the time\hyp{}dimension is uncurved.

(2) De~Sitter's spherical world. Here all dimensions are spherical; but
\index{de Sitter's spherical world}%
\index{Sitter@de Sitter's spherical world}%
\index{Spherical world}%
since it is imaginary time which is homogeneous with the space\hyp{}coordinates,
sections containing real time become hyperbolas instead of circles.

We must describe these two forms analytically. A point on the surface of
a sphere of radius~$R$ is described by two angular variables $\theta$,~$\phi$, such that
\[
-ds^{2} = R^{2}(d\theta^2 + \sin^{2}\theta\, d\phi^{2}).
\]
Extending this to three dimensions, we have three angular variables such that
\[
-ds^{2} = R^{2}\bigl\{d\chi^{2} + \sin^{2}\chi\, (d\theta^2 + \sin^{2}\theta\, d\phi^{2})\bigr\}.
\Tag{(67.11)}
\]

Accordingly in Einstein's form the interval is given by
\[
ds^{2} = -R^{2}\, d\chi^{2} - R^{2} \sin^{2}\chi\, (d\theta^2 + \sin^{2}\theta\, d\phi^{2}) + dt^{2}.
\Tag{(67.12)}
\]

Of course this form applies only to a survey of the world on the grand
scale. Trifling irregularities due to the aggregation of matter into stars and
stellar systems are treated as local deviations which can be disregarded.

Proceeding from the origin in any direction, $R\chi$~is the distance determined
\index{Finiteness of space}%
by measurement with rigid scales. But the measured area of a sphere of radius~$R\chi$
is not $4\pi R^{2}\chi^{2}$ but $4\pi R^{2} \sin^{2}\chi$. There is not so much elbow\hyp{}room in distant
parts as Euclid supposed. We reach a ``greatest sphere'' at the distance~$\frac{1}{2}\pi R$;
proceeding further, successive spheres contract and decrease to a single point
at a distance~$\pi R$---the greatest distance which can exist.

The whole volume of space (determined by rigid scales) is finite and equal
to
\[
\int_{0}^{\pi} 4\pi R^{2} \sin^{2}\chi \cdot R\, d\chi = 2\pi^{2} R^{3}.
\Tag{(67.2)}
\]
Although the volume of space is finite, there is no boundary; nor is there any
centre of spherical space. Every point stands in the same relation to the rest
of space as every other point.

To obtain de~Sitter's form, we generalise~\Eq{(67.11)} to four dimensions (i.e.\ a
spherical four\hyp{}dimensional surface drawn in Euclidean space of five dimensions).
We have four angular variables $\omega$, $\zeta$, $\theta$,~$\phi$, and
\[
-ds^{2} = R^{2} \bigl[d\omega^2
  + \sin^{2}\omega \bigl\{d\zeta^2
  + \sin^{2}\zeta (d\theta + \sin^{2}\theta\, d\phi^{2})\bigr\}\bigr].
\Tag{(67.31)}
\]
In order to obtain a coordinate\hyp{}system whose physical interpretation is more
easily recognisable, we make the transformation
\begin{align*}
  \cos\omega &= \cos\chi \cos it, \\
  \cot\zeta &= \cot\chi \sin it,
\end{align*}
which gives
\[
\left.
\begin{aligned}
  \sin\chi &= \sin\zeta \sin\omega\\
  \tan it &= \cos\zeta \tan\omega.
\end{aligned}
\right\}
\Tag{(67.32)}
\]

Working out the results of this substitution, we obtain
\[
ds^{2} = -R^{2}\, d\chi^{2} - R^{2} \sin^{2}\chi\, (d\theta^2 + \sin^{2}\theta\, d\phi^{2})
+ R^{2} \cos^{2}\chi \cdot dt^{2}.
\Tag{(67.33)}
\]

So far as space $(\chi, \theta, \phi)$ is concerned, this agrees with Einstein's form~\Eq{(67.12)};
but the variable~$t$, which will be regarded as the ``time''\footnote
  {The velocity of light at the origin is now~$R$. In the usual units the time would be~$Rt$.}
in this
world, has different properties. For a clock at rest ($\chi$, $\theta$, $\phi = \text{const.}$) we have
\[
ds = R\cos\chi\, dt,
\Tag{(67.4)}
\]
\index{Atom, time of vibration of!in de Sitter's world}%
\index{Horizon of world}%
\index{Recession of spiral nebulae}%
so that the ``time'' of any cycle is proportional to~$\sec\chi$. The clock-beats
become longer and longer as we recede from the origin; in particular the
vibrations of an atom become slower. Moreover we can detect by practical
measurement this slowing down of atomic vibrations, because it is preserved
in the transmission of the light to us. The coordinates \Eq{(67.33)} form a statical
system, the velocity of light being independent of~$t$; hence the light\hyp{}pulses
are all delayed in transmission by the same ``time'' and reach us at the same
intervals of~$t$ as they were emitted. Spectral lines emanating from distant
\index{Displacement of spectral lines to red, in sun!in nebulae}%
\index{Red-shift!in nebulae}%
\index{Spectral lines, displacement!in nebulae}%
sources at rest should consequently appear displaced towards the red.

At the ``horizon'' $\chi = \frac{1}{2}\pi$, any finite value of~$ds$ corresponds to an infinite~$dt$.
It takes an infinite ``time'' for anything to happen. All the processes of
nature have come to a standstill so far as the observer at the origin can have
evidence of them.

But we must recall that by the symmetry of the original formula~\Eq{(67.31)},
any point of space and time could be chosen as origin with similar results.
Thus there can be no actual difference in the natural phenomena at the horizon
and at the origin. The observer on the horizon does not perceive the stoppage---in
fact he has a horizon of his own at a distance~$\frac{1}{2}\pi R$ where things appear
to him to have come to a standstill.

Let us send a ray of light from the origin to the horizon and back again.
(We take the double journey because the time-lapse can then be recorded by
a single clock at the origin; the physical significance of the time for a single
journey is less obvious.) Setting $ds = 0$, the velocity of the light is given by
\[
0 = -R^{2}\, d\chi^{2} + R^{2} \cos^{2}\chi\, dt^{2},
\]
so that
\[
dt = \pm\sec\chi\, d\chi,
\]
whence
\[
t = \pm\log\tan(\tfrac{1}{4}\pi + \tfrac{1}{2}\chi).
\Tag{(67.5)}
\]
This must be taken between the limits $\chi = 0$ and~$\frac{1}{2}\pi$; and again with reversed
sign between the limits $\frac{1}{2}\pi$ and~$0$. The result is infinite, and the journey can
never be completed.

De~Sitter accordingly dismisses the paradox of the arrest of time at the
horizon with the remark that it only affects events which happen before the
beginning or after the end of eternity. But we shall discuss this in greater
detail in \SecRef{70}.

\Section{68.}{Elliptical space}
\index{Elliptical space}%

The equation~\Eq{(67.11)} for spherical space, which appears in both de~Sitter's
and Einstein's form of the interval, can also be construed as representing a
slightly modified kind of space called ``elliptical space.'' From the modern
standpoint the name is rather unfortunate, and does not in any way suggest
its actual nature. We can approach the problem of elliptical space in the
following way---

Suppose that in spherical space the physical processes going on at every
point are exactly the same as those going on at the antipodal point, so that
one half of the world is an exact replica of the other half. Let $ABA'B'$ be
four points $90^{\circ}$~apart on a great circle. Let us proceed from~$B'$, \Foreign{via}~$A$, to~$B$;
on continuing the journey along~$BA'$ it is impossible to tell that we are not
repeating the journey $B'A$~already performed. We should be tempted to think
that the arc~$B'A$ was in fact the immediate continuation of~$AB$, $B$~and $B'$
being the same point and only represented as wide apart through some fault
in our projective representation---just as in a Mercator Chart we see the same
Behring Sea represented at both edges of the map. We may leave to the
metaphysicist the question whether two objects can be exactly alike, both
intrinsically and in relation to all surroundings, and yet differ in identity;
physics has no conception of what is meant by this mysterious differentiation
of identity; and in the case supposed, physics would unhesitatingly declare
that the observer was re\hyp{}exploring the same hemisphere.

Thus the spherical world in the case considered does not consist of two
similar halves, but of a single hemisphere imagined to be repeated twice over
for convenience of projective representation. The differential geometry is the
same as for a sphere, as given by~\Eq{(67.11)}, but the \emph{connectivity} is different; just
as a plane and a cylinder have the same differential geometry but different
connectivity. At the limiting circle of any hemisphere there is a cross\hyp{}connection
of opposite ends of the diameters which it is impossible to represent
graphically; but that is, of course, no reason against the existence of the
cross\hyp{}connection.

This hemisphere which returns on itself by cross\hyp{}con\-nec\-tions is the type
of elliptical space. In what follows we shall not need to give separate consideration
to elliptical space. It is sufficient to bear in mind that in adopting
spherical space we may be representing the physical world in duplicate; for
example, the volume $2\pi^{2} R^{3}$ already given may refer to the duplicated world.

The difficulty in conceiving spherical or elliptical space arises mainly because
\index{Space, a network of intervals}%
we think of space as a continuum in which objects are \emph{located}. But it
was explained in \SecRef{1} that location is not the primitive conception, and is of
the nature of a computational result based on the more fundamental notion
of extension or distance. In using the word ``space'' it is difficult to repress
irrelevant ideas; therefore let us abandon the word and state explicitly that
we are considering a \emph{network of intervals} (or distances, since at present we
are not dealing with time). The relation of interval or distance between two
points is of some transcendental character comparable, for example, with a
difference of potential or with a chemical affinity; the reason why this particular
relation is always associated with geometrical ideas must be sought in
human psychology rather than in its intrinsic nature. We apply measure\hyp{}numbers
to the interval as we should apply them to any other relation of the
two points; and we thus obtain a network with a number attached to every
chord of the net. We could then make a string model of the network, the
length of each string corresponding to the measure\hyp{}number of the interval.
Clearly the form of this model---the existence or non\hyp{}existence of unexpected
cross\hyp{}connections---cannot be predicted \Foreign{a~priori}; it must be the subject of
observation and experiment. It may turn out to correspond to a lattice drawn
by the mathematician in a Euclidean space; or it may be cross\hyp{}connected in
a way which cannot be represented in a lattice of that kind. Graphical representation
is serviceable as a tool but is dangerous as an obsession. If we can
find a graphical representation which conforms to the actual character of the
network, we may employ it; but we must not imagine that any considerations
as to suitability for graphical representation have determined the design of
the network. From experience we know that small portions of the network
do admit of easy representation as a lattice in flat space, just as small portions
of the earth's surface can be mapped on a flat sheet. It does not follow that
the whole earth is flat, or that the whole network can be represented in a
space without multiple connection.

\Section{69.}{Law of gravitation for curved space-time}

By means of the results~\Eq{(43.5)} the~$G_{\mu\nu}$ can be calculated for either Einstein's
or de~Sitter's forms of the world. De~Sitter's equation~\Eq{(67.33)} is of the standard
form with $\chi$~substituted for~$r$, and
\[
e^{\lambda} = R^{2},\quad
e^{\mu} = R^{2} \sin^{2}\chi/\chi^{2},\quad
e^{\nu} = R^{2} \cos^{2}\chi,
\]
thus
\begin{gather*}
  \lambda' = 0,\quad
  \mu' = 2\cot\chi - 2\chi,\quad
  \nu' = -2\tan\chi,\displaybreak[0] \\
  \mu'' = -2\cosec^2\chi + 2/\chi^{2},\quad
  \nu'' = -2\sec^2\chi.
\end{gather*}
Hence by~\Eq{(43.5)} we find after an easy reduction
\[
G_{11} = -3,\
G_{22} = -3\sin^{2}\chi,\
G_{33} = -3\sin^{2}\chi \sin^{2}\theta,\
G_{44} = 3\cos^{2}\chi.
\]

These are equivalent to
\[
G_{\mu\nu} = \frac{3}{R^{2}}\, g_{\mu\nu}.
\Tag{(69.11)}
\]
De~Sitter's world thus corresponds to the revised form of the law of gravitation
\[
G_{\mu\nu} = \lambda g_{\mu\nu},
\]
and its radius is given by
\[
\lambda = \frac{3}{R^{2}}.
\Tag{(69.12)}
\]

Einstein's form~\Eq{(67.12)} gives similarly
\[
e^{\lambda} = R^{2},\quad
e^{\mu} = R^{2}\sin^{2}\chi/\chi^{2},\quad
e^{\nu} = 1,
\]
from which by~\Eq{(43.5)}
\begin{gather*}
  G_{11} = -2,\
  G_{22} = -2\sin^{2}\chi,\
  G_{33} = -2\sin^{2}\chi \sin^{2}\theta,\
  G_{44} = 0,
  \Tag{(69.21)} \\
  G = 6/R^{2}.
  \Tag{(69.22)}
\end{gather*}
It is not possible to reconcile these values with the law $G_{\mu\nu} = \lambda g_{\mu\nu}$, owing to
the vanishing of~$G_{44}$. Einstein's form cannot be the natural form of empty
space; but it may nevertheless be the actual form of the world if the matter
in the world is suitably distributed. To determine the necessary distribution
we must calculate the energy\hyp{}tensor~\Eq{(54.71)}
\[
-8\pi T_{\mu\nu} = G_{\mu\nu} - \tfrac{1}{2}g_{\mu\nu}G + \lambda g_{\mu\nu}.
\]

We find
\[
\left.
\begin{aligned}
  - 8\pi\, T_{11} &= \left(-\frac{1}{R^{2}} + \lambda\right) g_{11}\\
  - 8\pi\, T_{22} &= \left(-\frac{1}{R^{2}} + \lambda\right) g_{22}\\
  - 8\pi\, T_{33} &= \left(-\frac{1}{R^{2}} + \lambda\right) g_{33}\\
  - 8\pi\, T_{44} &= \left(-\frac{3}{R^{2}} + \lambda\right) g_{44}
\end{aligned}
\right\}
\Tag{(69.3)}
\]
Since $\lambda$~is still at our disposal, the distribution of this energy\hyp{}tensor is indeterminate.
But it is noted that within the stellar system the speed of matter,
whether of molecules or of stars, is generally small compared with the velocity
of light. There is perhaps a danger of overstressing this evidence, since astronomical
research seems to show that the greater the scale of our exploration
the more divergent are the velocities; thus the spiral nebulae, which are
perhaps the most remote objects observed, have speeds of the order $500$~km.\
per~sec.---at least ten times greater than the speeds observed in the stellar
system. It seems possible that at still greater distances the velocities may
increase further. However, in Einstein's solution we assume that the average
velocity of the material particles is always small compared with the velocity
of light; so the general features of the world correspond to
\index{World!mass of}%
\[
T_{11} = T_{22} = T_{33} = 0,\quad
T_{44} = \rho,\quad
T = \rho_{0},
\]
where $\rho_{0}$~is the average density (in natural measure) of the matter in space.

Hence by~\Eq{(69.3)}
\[
\lambda = \frac{1}{R^{2}},\quad
8\pi\rho_{0} = \frac{2}{R^{2}}.
\Tag{(69.4)}
\]
Accordingly if $M$~is the total mass in the universe, we have by~\Eq{(67.2)}
\begin{align*}
  M &= 2\pi^{2} R^{2} \rho_{0} \\
  &= \tfrac{1}{2}\pi R^{2}.
  \Tag{(69.5)}
\end{align*}
$R$~can scarcely be less than $10^{18}$ kilometres since the distances of some of the
globular clusters exceed this. Remembering that the gravitational mass of
the sun is $1.5$~kilometres, the mass of the matter in the world must be equivalent
to at least a trillion suns, if Einstein's form of the world is correct.

It seems natural to regard de~Sitter's and Einstein's forms as two limiting
cases, the circumstances of the actual world being intermediate between them.
De~Sitter's empty world is obviously intended only as a limiting case; and
the presence of stars and nebulae must modify it, if only slightly, in the
direction of Einstein's solution. Einstein's world containing masses far exceeding
anything imagined by astronomers, might be regarded as the other
extreme---a world containing as much matter as it can hold. This view denies
any fundamental cleavage of the theory in regard to the two forms, regarding
it as a mere accident, depending on the amount of matter which happens to
have been created, whether de~Sitter's or Einstein's form is the nearer approximation
to the truth. But this compromise has been strongly challenged,
as we shall see.

\Section{70.}{Properties of de Sitter's spherical world}
\index{de Sitter's spherical world}%
\index{Recession of spiral nebulae}%
\index{Red-shift!in nebulae}%
\index{Sitter@de Sitter's spherical world}%
\index{Spectral lines, displacement!in nebulae}%
\index{Spherical world}%

If in~\Eq{(67.33)} we write~$r = R\sin\chi$, we obtain
\[
\left.
\begin{gathered}
  ds^{2} = -\gamma^{-1}\, dr^{2} - r^{2}\, d\theta^2 - r^{2}\sin^{2}\theta\, d\phi^{2} + \gamma\, dt^{2} \\
%[** TN: Hard-coded length]
\llap{\text{where}\rule{0.87in}{0pt}}
\gamma = 1 - r^{2}/R^{2} = 1 - \tfrac{1}{3}\lambda r^{2},
\end{gathered}\right\}
\Tag{(70.1)}
\]
and the customary unit of~$t$ has been restored. This solution for empty space
has already been given, equation~\Eq{(45.6)}.

We have merely to substitute this value of~$\gamma$ in the investigations of
\SecRefs{38}, \SecNum{39}, in order to obtain the motion of material particles and of light-waves
in de~Sitter's empty world. Thus \Eq{(39.31)}~may be written
\[
\frac{d^{2}r}{ds^{2}}
  - \frac{1}{2}\, \frac{\gamma'}{\gamma} \left(\frac{dr}{ds}\right)^{2}
  - r\gamma \left(\frac{d\phi}{ds}\right)^{2}
  + \frac{1}{2}\gamma\gamma' \left(\frac{dt}{ds}\right)^{2} = 0.
\]
Whence
\[
\frac{d^{2}r}{ds^{2}}
  = -\frac{\frac{1}{3}\lambda r}{1 - \frac{1}{3}\lambda r^{2}} \left(\frac{dr}{ds}\right)^{2}
  + r(1 - \tfrac{1}{3}\lambda r^{2}) \left(\frac{d\phi}{ds}\right)^{2}
  + \tfrac{1}{3}\lambda r(1 - \tfrac{1}{3}\lambda r^{2}) \left(\frac{dt}{ds}\right)^{2}.
  \Tag{(70.21)}
\]

Equation~\Eq{(70.21)} can at once be simplified by means of~\Eq{(70.1)} giving
\[
\frac{d^2 r}{ds^2} - r\left(\frac{d\phi}{ds}\right)^2 = \tfrac{1}{3}\lambda r.
\]
Since the integral of areas
\[
r^2\frac{d\phi}{ds} = h
\]
is unaltered, the motion in de Sitter's world is the same as under a central repulsion varying directly with~$r$,
except that the time in the orbit corresponds to~$s$ not~$t$.
Some of the conclusions in this section are reached more directly by noticing this.

For a particle at rest
\[
\frac{dr}{ds} = 0,\quad
\frac{d\phi}{ds} = 0,\quad
\left(\frac{dt}{ds}\right)^{2} = \gamma^{-1}.
\]
Hence
\[
\frac{d^{2}r}{ds^{2}} = \tfrac{1}{3}\lambda r.
\Tag{(70.22)}
\]

Thus a particle at rest will not remain at rest unless it is at the origin;
but will be repelled from the origin with an acceleration increasing with the
distance. A number of particles initially at rest will tend to scatter, unless
their mutual gravitation is sufficient to overcome this tendency.

It can easily be verified that there is no such tendency in Einstein's world.
A particle placed anywhere will remain at rest. This indeed is necessary for
the self\hyp{}consistency of Einstein's solution, for he requires the world to be
filled with matter having negligible velocity. It is sometimes urged against
de~Sitter's world that it becomes non\hyp{}statical as soon as any matter is inserted
in it. But this property is perhaps rather in favour of de~Sitter's theory than
against it.

It is not impossible that the dimensions of our galaxy may be such that in the remoter parts this cosmical
repulsion exceeds the ordinary gravitation of the system, thus setting a limit to the extent of the permanent
aggregation of stars.
Possibly also the same condition may have a bearing on the development of spiral nebulae if these are external
galaxies.

One of the most perplexing problems of cosmogony is the great speed of
the spiral nebulae. Their radial velocities average about $600$~km.\ per~sec.\ and
there is a great preponderance of velocities of recession from the solar system.
It is usually supposed that these are the most remote objects known (though
this view is opposed by some authorities), so that here if anywhere we might
look for effects due to a general curvature of the world. De~Sitter's theory
gives a double explanation of this motion of recession; first, there is the
general tendency to scatter according to equation~\Eq{(70.22)}; second, there is
the general displacement of spectral lines to the red in distant objects due to
\index{Displacement of spectral lines to red, in sun!in nebulae}%
the slowing down of atomic vibrations~\Eq{(67.4)} which would be erroneously interpreted
as a motion of recession.

The most extensive measurements of radial velocities of spiral nebulae
\index{Nebulae, velocities of}%
\index{Spiral nebulae, velocities of}%
have been made by Prof.\ V.~M. Slipher at the Lowell Observatory. He has
kindly prepared for me the following \hyperref[table:70.1]{table}, containing many unpublished
results. It is believed to be complete up to date (Feb.~1922). For the nebulae
marked~(*) the results have been closely confirmed at other observatories;
those marked~(\dag) are not so accurate as the others. The number in the first
column refers to the ``New General Catalogue,'' \Title{Memoirs R.A.S.,} vol.~49. One
additional nebula \NGC\ 1700 has been observed by Pease, who found a large
receding velocity but gave no numerical estimate.
\begin{table}[hb]
  \centering
\textsc{Radial Velocities of Spiral Nebulae} \\
$+$ indicates receding, $-$ approaching
\phantomsection\label{table:70.1}%
\[
\tablefontsize
\begin{array}{@{}r@{\qquad}r@{\ }r@{\quad\ }r@{\ }rc|r@{\qquad}r@{\ }r@{\quad\ }r@{\ }rc@{}}
  \multicolumn{1}{c}{\text{\NGC}} & \multicolumn{2}{c}{\RA} & \multicolumn{2}{c}{\text{\quad Dec.}} & \multicolumn{1}{c|}{\text{Rad.\ Vel.}} &
  \multicolumn{1}{c}{\text{\NGC}} & \multicolumn{2}{c}{\RA} & \multicolumn{2}{c}{\text{\quad Dec.}} & \multicolumn{1}{c}{\text{Rad.\ Vel.}} \\
  & h & m & \degree & ' & \multicolumn{1}{c|}{\text{km.\ per sec.}} &
  & h & m & \degree & ' & \multicolumn{1}{c}{\text{km.\ per sec.}} \\
  221 & 0 & 38 & +40 & 26 & -\9 300 & 4151\rlap{*} & 12 & 6 & +39 & 51 & +\9 980 \\
  224\rlap{*} & 0 & 38 & +40 & 50 & -\9 300 & 4214 & 12 & 12 & +36 & 46 & +\9 300 \\

  278\rlap{\dag} & 0 & 47 & +47 & 7 & +\9 650 & 4258 & 12 & 15 & +47 & 45 & +\9 500 \\
  404 & 1 & 5 & +35 & 17 & -\9\9 25 & 4382\rlap{\dag} & 12 & 21 & +18 & 38 & +\9 500 \\
  584\rlap{\dag} & 1 & 27 & -\9 7 & 17 & +1800 & 4449 & 12 & 24 & +44 & 32 & +\9 200 \\
  598\rlap{*} & 1 & 29 & +30 & 15 & -\9 260 & 4472 & 12 & 25 & +\9 8 & 27 & +\9 850 \\
  936 & 2 & 24 & -\9 1 & 31 & +1300 & 4486\rlap{\dag} & 12 & 27 & +12 & 50 & +\9 800 \\
  1023 & 2 & 35 & +38 & 43 & +\9 300 & 4526 & 12 & 30 & +\9 8 & 9 & +\9 580 \\
  1068\rlap{*} & 2 & 39 & -\9 0 & 21 & +1120 & 4565\rlap{\dag} & 12 & 32 & +26 & 26 & +1100 \\
  2683 & 8 & 48 & +33 & 43 & +\9 400 & 4594\rlap{*} & 12 & 36 & -11 & 11 & +1100 \\
  2841\rlap{\dag} & 9 & 16 & +51 & 19 & +\9 600 & 4649 & 12 & 40 & +12 & 0 & +1090 \\
  3031 & 9 & 49 & +69 & 27 & -\9\9 30 & 4736 & 12 & 47 & +41 & 33 & +\9 290 \\
  3034 & 9 & 49 & +70 & 5 & +\9 290 & 4826 & 12 & 53 & +22 & 7 & +\9 150 \\
  3115\rlap{\dag} & 10 & 1 & -\9 7 & 20 & +\9 600 & 5005 & 13 & 7 & +37 & 29 & +\9 900 \\
  3368 & 10 & 42 & +12 & 14 & +\9 940 & 5055 & 13 & 12 & +42 & 37 & +\9 450 \\
  3379\rlap{*} & 10 & 43 & +13 & 0 & +\9 780 & 5194 & 13 & 26 & +47 & 36 & +\9 270 \\
  3489\rlap{\dag} & 10 & 56 & +14 & 20 & +\9 600 & 5195\rlap{\dag} & 13 & 27 & +47 & 41 & +\9 240 \\
  3521 & 11 & 2 & +\9 0 & 24 & +\9 730 & 5236\rlap{\dag} & 13 & 32 & -29 & 27 & +\9 500 \\
  3623 & 11 & 15 & +13 & 32 & +\9 800 & 5866 & 15 & 4 & +56 & 4 & +\9 650 \\
  3627 & 11 & 16 & +13 & 26 & +\9 650 & 7331 & 22 & 33 & +33 & 23 & +\9 500 \\
  4111\rlap{\dag} & 12 & 3 & +43 & 31 & +\9 800 & \\
\end{array}
\]
\end{table}

The great preponderance of positive (receding) velocities is very striking;
but the lack of observations of southern nebulae is unfortunate, and forbids a
final conclusion. Even if these also show a preponderance of receding velocities
the cosmogonical difficulty is perhaps not entirely removed by de~Sitter's
theory. It will be seen that two\footnote
  {\NGC~221 and 224 may probably be counted as one system. The two approaching nebulae
  are the largest spirals in the sky.}
nebulae (including the great Andromeda
nebula) are approaching with rather high velocity and these velocities happen
to be exceptionally well determined. In the full formula~\Eq{(70.21)} there are no
terms which under any reasonable conditions encourage motion towards the
origin\footnotemark.\footnotetext
  {We are limited to the region in which $(1 - \frac{1}{3}\lambda r^{2})$ is positive since light cannot cross the barrier.}
It is therefore difficult to account for these motions even as exceptional
phenomena; on the other hand an approaching velocity of $300$~km.\ per~sec.\
is about the limit occasionally attained by individual stars or star clusters.

The conservation of energy is satisfied in de~Sitter's world; but from the
practical standpoint it is abrogated in large scale problems such as that of
the system of the spirals, since these are able to withdraw kinetic energy
from a source not generally taken into account.

Equation~\Eq{(39.44)}
\[
\frac{1}{\gamma} \left(\frac{h}{r^{2}}\, \frac{dr}{d\phi}\right)^{2}
  + \frac{h^{2}}{r^{2}} - \frac{c^2}{\gamma} = -1
\]
becomes on substituting for~$\gamma$
\[
\left(\frac{h}{r^{2}}\, \frac{dr}{d\phi}\right)^{2}
+ \frac{h^{2}}{r^{2}} = c^2 - 1 - \tfrac{1}{3}\lambda h^{2} + \tfrac{1}{3} \lambda r^{2},
\]
or writing $u = 1/r$
\[
\left(\frac{du}{d\phi}\right)^{2} + u^2
= \frac{c^2 - 1}{h^{2}} - \tfrac{1}{3}\lambda + \frac{\tfrac{1}{3} \lambda}{h^{2} u^2}.
\]
Whence, differentiating
\[
\frac{d^{2}u}{d\phi^{2}} + u = -\frac{\tfrac{1}{3} \lambda}{h^{2}} u^{-3}.
\Tag{(70.3)}
\]
The orbit is the same as that of a particle under a repulsive force varying
directly as the distance. (This applies only to the form of the orbit, not to
the velocity in the orbit.) For the motion of light the constant of areas~$h$ is
infinite, and the tracks of light-rays are the solutions of
\[
\frac{d^{2}u}{d\phi^{2}} + u = 0,
\]
i.e.\ straight lines. Determination of distance by parallax\hyp{}measurements rests
on the assumption that light is propagated in straight lines, and hence the
method is exact in this system of coordinates. In so far as the distances of
celestial objects are determined by parallaxes or parallactic motions, the
coordinate~$r$ will agree with their accepted distances. This result may be
contrasted with the solution for the field of a particle in \SecRef{38} where the coordinate~$r$
has no immediate observational significance. Radial distances determined
by direct operations with measuring\hyp{}rods correspond to~$R\chi$, not~$r$.

Some readers have found the above argument that astronomical measurements of parallax determine the coordinate~$r$
(not the proper\hyp{}distance~$R\chi$) too condensed.
It may be stated more fully as follows.
We make a \emph{map} of the curved world in a Euclidean space;
$r$ and $\phi$ are Euclidean polar coordinates in the map, each star being represented in the map at
the point so determined.
In general, distances and angles measured in thfe world will not agree with the distances and angles represented in
the map; but within the limits of the solar system, where~$\tfrac{1}{3}\lambda r^2$ is negligibly small,
the map and the world coalesce.
Since any practical measurements are made within the solar system, all \emph{direct} measurements may be immediately
transferred to the map.
The astronomer completes his triangulation by assuming (1) that space is Euclidean and (2) that light travels in
straight lines.
These assumptions are true in the map; hence the astronomer's results refer to the map, and his deduced distance
of the star is the map\hyp{}distance~$r$.
If he measured the distance with measuring rods he would have to go outside the solar system, and his measures
of length could not then be immediately transferred to the map;
this method gives the distance~$R\chi$, disagreeing with the map\hyp{}distance.

The spectroscopic radial velocity is not exactly equivalent to~$dr/dt$, but
\index{Light-pulse!in curved world}%
the divergence is unimportant. A pulse of light emitted by an atom situated
at $r = R\sin\chi$ at time~$t$ will reach the observer at the origin at time~$t'$, where
by~\Eq{(67.5)}
\[
t' = t + \log\tan(\tfrac{1}{4}\pi + \tfrac{1}{2}\chi),
\]
so that for the time\hyp{}interval between two pulses
\begin{align*}
dt' &= dt + \sec\chi\, d\chi \\
&= \left(1 + \sec\chi\, \frac{d\chi}{dt}\right) \frac{dt}{ds}\, ds \\
&= \left(\sec\chi + \sec^2\chi\, \frac{d\chi}{dt}\right) ds,
\quad\text{by~\Eq{(67.33)}}
\end{align*}
\index{Atom, time of vibration of!in de Sitter's world}%
neglecting the square of the velocity of the atom. If $dt_{0}'$~is the time for a
similar atom at rest at the origin,
\begin{align*}
  \frac{dt'}{dt_{0}'}
  &= \sec\chi + \sec^2\chi\, \frac{d\chi}{dt} \\
  &= \sec\chi + \sec^2\chi\, \frac{1}{R}\, \frac{dr}{dt}.
  \Tag{(70.4)}
\end{align*}
The first term represents the general shift to the red dependent on position
and not on velocity. Assuming that it has been allowed for, the remaining
part of the shift corresponds to a velocity of $\smash[b]{\sec^{3}\chi\, \dfrac{dr}{dt}}$ instead of~$\dfrac{dr}{dt}$. The
correction is scarcely of practical importance.

The acceleration $\frac{1}{3}\lambda r$ found in~\Eq{(70.22)}, if continued for the time~$R\chi$ taken
by the light from the object to reach the origin, would cause a change of
velocity of the order $\frac{1}{3}\lambda r^{2}$ or~$r^{2}/R^{2}$. The Doppler effect of this velocity would
be roughly the same as the shift to the red caused by the slowing down of
atomic vibrations. We may thus regard the red shift for distant objects at
rest as \emph{an anticipation} of the motion of recession which will have been attained
before we receive the light. If de~Sitter's interpretation of the red shift in
the spiral nebulae is correct, we need not regard the deduced large motions
of recession as entirely fallacious; it is true that the nebulae had not these
motions when they emitted the light which is now examined, but they have
acquired them by now. Even the standing still of time on the horizon becomes
intelligible from this point of view; we are supposed to be observing a system
which has \emph{now} the velocity of light, having acquired it during the infinite
time which has elapsed since the observed light was emitted.

The following paradox is sometimes found puzzling. Take coordinates for
an observer~$A$ at rest at the origin, and let $B$~be at rest at the time~$t$ at a
considerable distance from the origin. The vibrations of an atom at~$B$ are
slower (as measured in the time~$t$) than those of an atom at~$A$, and since the
coordinate\hyp{}system is static this difference will be detected experimentally by
\emph{any} observer who measures the frequency of the light he receives. Accordingly
$B$~must detect the difference, and conclude that the light from~$A$ is displaced
towards the violet relatively to his standard atom. This is absurd since, if we
choose $B$ as origin, the light from~$A$ should be displaced towards the red. The
fallacy lies in ignoring what has happened during the long time of propagation
from~$A$ to~$B$ or $B$ to~$A$; during this time the two observers have ceased
to be in relative rest, so that compensating Doppler effects are superposed.

To obtain a clearer geometrical idea of de~Sitter's world, we consider only
one dimension of space, neglecting the coordinates $\theta$ and~$\phi$. Then by~\Eq{(67.31)}
\begin{align*}
  -ds^{2} &= R^{2} (d\omega^2 + \sin^{2}\omega\, d\zeta^2)
  = R^{2} (d\chi^{2} - \cos^{2}\chi\, dt^{2}) \\
  &= dx^{2} + dy^{2} + dz^{2},
\end{align*}
where
\begin{alignat*}{2}
  x &= R\sin\omega \cos\zeta &&= R\cos\chi \sin it, \\
  y &= R\sin\omega \sin\zeta &&= R\sin\chi, \\
  z &= R\cos\omega &&= R\cos\chi \cos it,
\end{alignat*}
and
\[
x^{2} + y^{2} + z^{2} = R^{2}.
\]

It will be seen that real values of $\chi$ and~$t$ correspond to imaginary
values of~$\omega$ and~$\zeta$ and accordingly for real events $x$~is imaginary and $y$~and
$z$ are real. Introducing a real coordinate $\xi = -ix$, real space-time will be
represented by the hyperboloid of one sheet with its axis along the axis of~$\xi$,
\[
y^{2} + z^{2} - \xi^{2} = R^{2},
\]
the geometry being of the Galilean type
\[
ds^{2} = d\xi^{2} - dy^{2} - dz^{2}.
\]

We have
\begin{gather*}
  r = R\sin\chi = y, \\
  \tanh t = -i\tan it = -ix/z = \xi/z,
\end{gather*}
so that the space\hyp{}partitions are made by planes perpendicular to the axis of~$y$,
and the time\hyp{}partitions by planes through the axis of~$y$ cutting the hyperboloid
into lunes.

The light-tracks, $ds = 0$, are the generators of the hyperboloid. The tracks
of undisturbed particles are (non\hyp{}Euclidean) geodesics on the hyperboloid;
and, except for $y = 0$, the space\hyp{}partitions will not be geodesics, so that
particles do not remain at rest.

The coordinate\hyp{}frame $(r, t)$ of a single observer does not cover the whole
world. The range from $t = -\infty$ to $t = +\infty$ corresponds to values of~$\xi/z$
between~$\pm1$. The whole experience of any one observer of infinite longevity
is comprised within a $90^{\circ}$~lune. Changing the origin we can have another
observer whose experience covers a different lune. The two observers cannot
communicate the non\hyp{}overlapping parts of their experience, since there are
no light-tracks (generators) taking the necessary course.

A further question has been raised, Is de~Sitter's world really empty? In
\index{Horizon of world}%
\index{Mass\hyp{}horizon of world}%
formula~\Eq{(70.1)} there is a singularity at $r = \sqrt{3/\lambda}$ similar to the singularity
at $r = 2m$ in the solution for a particle of matter. Must we not suppose that
the former singularity also indicates matter---a ``mass\hyp{}horizon'' or ring of
peripheral matter necessary in order to distend the empty region within. If
so, it would seem that de~Sitter's world cannot exist without large quantities
of matter any more than Einstein's; he has merely swept the dust away into
unobserved corners.

A singularity of~$ds^{2}$ does not necessarily indicate material particles, for
we can introduce or remove such singularities by making transformations of
coordinates. It is impossible to know whether to blame the world\hyp{}structure
or the inappropriateness of the coordinate\hyp{}system. In a finite region we avoid
this difficulty by choosing a coordinate\hyp{}system initially appropriate---how this
is done is very little understood---and permitting only transformations which
have no singularity in the region. But we can scarcely apply this to a
consideration of the whole finite world since all the ordinary analytical transformations
\index{Mass of the world, total}%
\index{World!mass of}%
(even a change of origin) introduce a singularity somewhere. If
de~Sitter's form for an empty world is right it is impossible to find any
coordinate\hyp{}system which represents the whole of real space-time regularly.
This is no doubt inconvenient for the mathematician, but I do not see that
the objection has any other consequences.

The whole of de~Sitter's world can be reached by a process of continuation;
that is to say the finite experience of an observer~$A$ extends over a certain
lune; he must then hand over the description to~$B$ whose experience is partly
overlapping and partly new; and so on by overlapping lunes. The equation
$G_{\mu\nu} = \lambda g_{\mu\nu}$ rests on the considerations of \SecRef{66}, and simply by continuation of
this equation from point to point we arrive at de~Sitter's complete world
without encountering any barrier or mass\hyp{}horizon.

A possible indication that there is no real mass in de~Sitter's world is
afforded by a calculation of the gravitational flux~\Eq{(63.4)}. By~\Eq{(63.6)} this is
\[
4\pi r^{2} \left(-\delta y - \frac{2}{r}\, \delta\gamma\right) dt,
\]
since $dt$~can no longer be replaced by~$ds$. On substituting for~$\gamma$ it is found
that the flux vanishes for all values of~$r$. It is true that as we approach the
boundary $dt/ds$~becomes very great, but the complete absence of flux right up
to the boundary seems inconsistent with the existence of a genuine mass\hyp{}horizon.

I believe then that the mass\hyp{}horizon is merely an illusion of the observer
at the origin, and that it continually recedes as we move towards it.

\Section{71.}{Properties of Einstein's cylindrical world}
\index{Einstein's cylindrical world}%

Einstein does not regard the relation~\Eq{(69.5)}
\[
M = \tfrac{1}{2}\pi R = \tfrac{1}{2}\pi \lambda^{-\frac{1}{2}}
\Tag{(71.1)}
\]
as merely referring to the limiting case when the amount of matter in the
world happens to be sufficient to make the form cylindrical. He considers it
to be a necessary relation between $\lambda$ and~$M$; so that the constant~$\lambda$ occurring
in the law of gravitation is a function of the total mass of matter in the world,
and the volume of space is conditioned by the amount of matter contained
in it.

The question at once arises, By what mechanism can the value of~$\lambda$ be
adjusted to correspond with~$M$? The creation of a new stellar system in a
distant part of the world would have to propagate to us, not merely a gravitational
field, but a modification of the law of gravitation itself. We cannot
trace the propagation of any such influence, and the dependence of~$\lambda$ upon
distant masses looks like sheer action at a distance.

But the suggestion is perhaps more plausible if we look at the inverse
relation, viz.\ $M$~as a function of~$\lambda$. If we can imagine the gradual destruction
of matter in the world (e.g.\ by coalescence of positive and negative electrons),
we see by~\Eq{(71.1)} that the radius of space gradually contracts; but it is not
clear what is the fixed standard of length by which $R$~is supposed to be
measured. The natural standard of length in a theoretical discussion is the
radius $R$ itself. Choosing it as unit, we have $M = \frac{1}{2}\pi$, whatever the number
of elementary particles in the world. Thus with this unit the mass of a particle
must be inversely proportional to the number of particles. Now the gravitational
mass is the radius of a sphere which has some intimate relation to
the structure of the particle; and we must conclude that as the destruction
of particles proceeds, this sphere must swell up as though some pressure were
being relaxed. We might try to represent this pressure by the gravitational
flux (\SecRef{63}) which proceeds from every particle; but I doubt whether that
leads to a satisfactory solution. However that may be, the idea that the
particles each endeavour to monopolise all space, and restrain one another by
a mutual pressure, seems to be the simplest interpretation of~\Eq{(71.1)} if it is to
be accepted.

We do not know whether the actual (or electrical) radius of the particle
would swell in the same proportion---by a rough guess I should anticipate
that it would depend on the square root of the above ratio. But this radius,
on which the scale of ordinary material standards depends, has nothing to do
with equation~\Eq{(71.1)}; and if we suppose that it remains constant, the argument
of \SecRef{66} need not be affected.

In favour of Einstein's hypothesis is the fact that among the constants of
\index{Number of electrons in the world}%
nature there is one which is a very large pure number; this is typified by the
ratio of the radius of an electron to its gravitational mass $= 3 \cdot 10^{42}$. It is difficult
to account for the occurrence of a pure number (of order greatly different
from unity) in the scheme of things; but this difficulty would be removed if
we could connect it with the number of particles in the world---a number
presumably decided by pure accident\footnotemark.\footnotetext
  {The square of $3 \cdot 10^{42}$ might well be of the same order as the total number of positive and
  negative electrons. The corresponding radius is $10^{14}$~parsecs. But the result is considerably
  altered if we take the proton instead of the electron as the more fundamental structure.}
There is an attractiveness in the
idea that the total number of the particles may play a part in determining
the constants of the laws of nature; we can more readily admit that the laws
of the actual world are specialised by the accidental circumstance of a particular
number of particles occurring in it, than that they are specialised by
the same number occurring as a mysterious ratio in the fine-grained structure
of the continuum.

In Einstein's world one direction is uncurved and this gives a kind of
absolute time. Our critic who has been waiting ever since \SecRef{1} with his blank
label ``true time'' will no doubt seize this opportunity of affixing it. Moreover
absolute velocity is to some extent restored, for there is by hypothesis a
frame of reference with respect to which material bodies on the average have
only small velocities. Matter is essential to the existence of a space-time
frame according to Einstein's view; and it is inevitable that the space-time
frame should become to some extent materialised, thereby losing some of the
valuable elusiveness of a purely aetherial frame. It has been suggested that
since the amount of matter necessary for Einstein's world greatly exceeds that
known to astronomers, most of it is spread uniformly through space and is
undetectable by its uniformity. This is dangerously like restoring a crudely
material aether---regulated, however, by the strict injunction that it must on
no account perform any useful function lest it upset the principle of relativity.
We may leave aside this suggestion, which creates unnecessary difficulties.
I think that the matter contemplated in Einstein's theory is ordinary stellar
matter. Owing to the irregularity of distribution of stars, the actual form of
space is not at all a smooth sphere, and the formulae are only intended to give
an approximation to the general shape.

The Lorentz transformation continues to hold for a limited region. Since
the advent of the general theory, it has been recognised that the special theory
only applies to particular regions where the~$g_{\mu\nu}$ can be treated as constants, so
that it scarcely suffers by the fact that it cannot be applied to the whole
domain of spherical space. Moreover the special principle is now brought into
line with the general principle. The transformations of the theory of relativity
relate to the differential equations of physics; and our tendency to choose
simple illustrations in which these equations are integrable over the whole of
space-time (as simplified in the mathematical example) is responsible for much
misconception on this point.

The remaining features of Einstein's world require little comment. His
spherical space is commonplace compared with de~Sitter's. Each observer's
coordinate\hyp{}system covers the whole world; so that the fields of their finite
experience coincide. There is no scattering force to cause divergent motions.
Light performs the finite journey round the world in a finite time. There is
no passive ``horizon,'' and in particular no mass\hyp{}horizon, real or fictitious.
Einstein's world offers no explanation of the red shift of the spectra of distant
objects; and to the astronomer this must appear a drawback. For this and
other reasons I should be inclined to discard Einstein's view in favour of
de~Sitter's, if it were not for the fact that the former appears to offer a distant
hope of accounting for the occurrence of a very large pure number as one of
the constants of nature.

\Section{72.}{The problem of the homogeneous sphere}
\index{Homogeneous sphere, problem of}%
\index{Problem!of homogeneous sphere}%
\index{Sphere, problem of homogeneous}%

For comparison with the results for naturally curved space, we consider a
problem in which the curvature is due to the presence of ordinary matter.

The problem of determining~$ds^{2}$ at points within a sphere of fluid of uniform
density has been treated by Schwarzschild, Nordström and de~Donder.
Schwarzschild's solution\footnote
  {Schwarzschild's solution is of considerable interest; but I do not think that he solved exactly
  the problem which he intended to solve, viz.\ that of an incompressible fluid. For that reason I do
  not give the arguments which led to the solution, but content myself with discussing what distribution
  of matter his solution represents. A full account is given by de~Donder, \Title{La Gravifique
  Einsteinienne,} p.~169 (Gauthier\hyp{}Villars, 1921). The original gravitational equations are used, the
  natural curvature of space being considered negligible compared with that superposed by the
  material sphere.}
is
\[
ds^{2} = -e^{\lambda}\, dr^{2} - r^{2}\, d\theta^2 - r^{2}\sin^{2}\theta\, d\phi^{2} + e^{\nu}\, dt^{2},
\]
where
\[
\left.
\begin{aligned}
  e^{\lambda} &= 1/(1 - \alpha r^{2}) \\
  e^{\nu} &= \tfrac{1}{4}\bigl(3\sqrt{1 - \alpha a^2} - \sqrt{1 - \alpha r^{2}}\bigr)^{2}
\end{aligned}
\right\}
\Tag{(72.1)}
\]
and $a$ and~$\alpha$ are constants.

The formulae~\Eq{(46.9)}, which apply to this form of~$ds$, become on raising one
suffix
\[
\left.
\begin{aligned}
  -8\pi T_{1}^{1} &= e^{-\lambda} \bigl(\nu'/r - (e^{\lambda} - 1)/r^{2}\bigr)\\
  -8\pi T_{2}^{2} &= e^{-\lambda} \bigl(\tfrac{1}{2} \nu'' - \tfrac{1}{4}\lambda'\nu' + \tfrac{1}{4}\nu'^{2} + \tfrac{1}{2}(\nu' - \lambda')/r\bigr)\\
  -8\pi T_{3}^{3} &= -8\pi T_{2}^{2}\\
  -8\pi T_{4}^{4} &= e^{-\lambda} \bigl(-\lambda'/r - (e^{\lambda} - 1)/r^{2}\bigr)
\end{aligned}
\right\}
\Tag{(72.2)}
\]
We find from~\Eq{(72.1)} that
\[
(e^{\lambda} - 1)/r^{2} = \tfrac{1}{2}\lambda'/r;\quad
\tfrac{1}{2} \nu'' - \tfrac{1}{4}\lambda'\nu' + \tfrac{1}{4}\nu'^{2}  = \tfrac{1}{2}\nu'/r.
\]
Hence
\begin{gather*}
  T_{1}^{1} = T_{2}^{2} = T_{3}^{3} = \frac{1}{8\pi} e^{\lambda}(\tfrac{1}{2}\lambda' - \nu')/r,
\Tag{(72.31)} \\
T_{4}^{4} = \frac{1}{8\pi} e^{\lambda} \cdot \tfrac{3}{2}\lambda'/r
= 3\alpha/8\pi.
\Tag{(72.32)}
\end{gather*}

Referred to the coordinate\hyp{}system $(r, \theta, \phi)$, $T_{4}^{4}$~represents the density and
$T_{1}^{1}$, $T_{2}^{2}$, $T_{3}^{3}$ the stress\hyp{}system. Hence Schwarzschild's solution gives uniform
density and isotropic hydrostatic pressure at every point.
\index{Pressure!in homogeneous sphere}%

On further working out~\Eq{(72.31)}, we find that the pressure is
\[
p = -T_{1}^{1}
= \frac{\alpha}{8\pi}\,
\frac{\bigl\{\frac{3}{2}(1 - \alpha r^{2})^{\frac{1}{2}} - \frac{3}{2}(1 - \alpha a^2)^{\frac{1}{2}}\bigr\}}
     {\bigl\{\frac{3}{2}(1 - \alpha a^2)^{\frac{1}{2}} - \frac{1}{2}(1 - \alpha r^{2})^{\frac{1}{2}}\bigr\}}.
\Tag{(72.4)}
\]

We see that the pressure vanishes at $r = a$, and would become negative if
we attempted to continue the solution beyond $r = a$. Hence the sphere $r = a$
gives the boundary of the fluid. If it is desired to continue the solution outside
the sphere, another form of~$ds^{2}$ must be taken corresponding to the
equations for empty space.

Unless $a > \sqrt{8/9\alpha}$ the pressure will everywhere be finite. This condition
sets an upper limit to the possible size of a fluid sphere of given density. The
limit exists because the presence of dense matter increases the curvature of
space, and makes the total volume of space smaller. Clearly the volume of the
material sphere cannot be larger than the volume of space.

For spheres which are not unduly large (e.g.\ not much larger than the
stars) this solution corresponds approximately to the problem of the equilibrium
of an incompressible fluid. The necessary conditions are satisfied, viz.\

(1) The density is uniform.

(2) The pressure is zero at the surface.

(3) The stress\hyp{}system is an isotropic hydrostatic pressure, and therefore
satisfies the conditions of a perfect fluid.

(4) The pressure is nowhere infinite, negative, or imaginary.

Further equation~\Eq{(72.4)} determines the pressure at any distance from the
centre.

The components of an energy\hyp{}tensor are usually altered when a transformation of coordinates is made, so that before
interpreting them in terms of pressure and density it is necessary to ascertain that the appropriate coordinate\hyp{}system
has been used, viz. natural coordinates.
To pass from the coordinates $(r,\theta,\phi,t)$ to the natural coordinates at any point it is necessary to make a
transformation of scale of $r$ and $t$ at that point.
This transformation leaves the \emph{mixed} tensor $T_{\mu}^{\nu}$ unaltered, although it would alter $T^{\mu\nu}$
and $T_{\mu\nu}$.
Accordingly the results~\Eq{(72.31)} and~\Eq{(72.32)} are valid for natural measure, and the arbitrariness of our
original coordinate\hyp{}system does not affect the definiteness of our conclusions.

But in concluding that the solution represents a perfect fluid of uniform density, the density referred to is $T_4^4$
or $\rho_{00}$. For reasons explained in~\SecRef{54} this condition does not seem correct for an incompressible fluid.
We need a solution in which $T$ or $\rho_0$ is constant throughout the sphere; Schwarzschild has not solved this
problem.
For large spheres the central pressure is enormous and the difference of the two solutions may be considerable.

I make this comment with some hesitation because it is difficult to be certain what limit an actual liquid would
approach when the enormously increasing pressure is unable to bring the ultimate particles appreciably nearer
together.
In a gas the pressure is represented by molecular velocities and its nature is understood.
Here it may be a Maxwellian electromagnetic stress, in which case the conclusion that $\rho_0$ is constant continues
to hold true. But it may be some more mysterious quantum manifestation, as to which we can make no prediction.

If it is assumed that Schwarzschild's result
\[
a < \sqrt{8/9\alpha}
\]
is correct as regards order of magnitude, the radius of the greatest possible
mass of water would be 370 million kilometres.
The radius of the star Betelgeuse is something like half of this; but its density is much too small to lead
to any interesting applications of the foregoing result.

Admitting Einstein's modification of the law of gravitation, with $\lambda$~depending
on the total amount of matter in the world, the size of the greatest
sphere is easily determined. By~\Eq{(69.4)} $R^{2} = 1/4\pi\rho_{0}$, from which $R$~(for water)
is very nearly 300~million kilometres.

\Chapter{VI}{Electricity}

\Section{73.}{The electromagnetic equations}
\index{Electromagnetic action!force}%
\index{Electromagnetic action!potential}%
\index{F@$F_{\mu\nu}$ (electromagnetic force)}%

\lettrine{\textcolor{lettrinecolour}{I}}{n} the classical theory the electromagnetic field is described by a scalar
potential~$\Phi$ and a vector potential $(F, G, H)$. The electric force $(X, Y, Z)$
\index{Force!electromagnetic}%
\index{Potential!electromagnetic}%
and the magnetic force $(\alpha, \beta, \gamma)$ are derived from these according to the
equations
\[
\left.
\begin{aligned}
  X &= -\frac{\dd\Phi}{\dd x} - \frac{\dd F}{\dd t}\\
  \alpha &= \Neg\frac{\dd H}{\dd y} - \frac{\dd G}{\dd z}
\end{aligned}
\right\}
\Tag{(73.1)}
\]

The classical theory does not consider any possible interaction between the
gravitational and electromagnetic fields. Accordingly these definitions, together
with Maxwell's equations, are intended to refer to the case in which no
field of force is acting, i.e.\ to Galilean coordinates. We take a special system
of Galilean coordinates and set
\[
\kappa^{\mu} = (F, G, H, \Phi)
\Tag{(73.21)}
\]
for that system. Having decided to make $\kappa^{\mu}$ a contravariant vector we can
find its components in any other system of coordinates, Galilean or otherwise,
by the usual transformation law; but, of course, we cannot tell without investigation
what would be the physical interpretation of those components. In
particular we must not assume without proof that the components of~$\kappa^{\mu}$ in
another Galilean system would agree with the new $F$, $G$, $H$,~$\Phi$ determined
experimentally for that system. At the present stage, we have defined~$\kappa^{\mu}$ in all
systems of coordinates, but the equation~\Eq{(73.21)} connecting it with experimental
quantities is only known to hold for one particular Galilean system.

Lowering the suffix with Galilean~$g_{\mu\nu}$, we have
\[
\kappa_{\mu} = (-F, -G, -H, \Phi).
\Tag{(73.22)}
\]
Let the tensor
\[
F_{\mu\nu} \equiv \kappa_{\mu\nu} - \kappa_{\nu\mu}
= \frac{\dd\kappa_{\mu}}{\dd x_{\nu}} - \frac{\dd\kappa_{\nu}}{\dd x_{\mu}}
\Tag{(73.3)}
\]
as in~\Eq{(32.2)}.

Then by~\Eq{(73.1)}
\begin{alignat*}{2}
  F_{14} &= \frac{\dd\kappa_{1}}{\dd x_{4}} - \frac{\dd\kappa_{4}}{\dd x_{1}}
  &&= \frac{\dd(-F)}{\dd t} - \frac{\dd\Phi}{\dd x} = X, \\
  F_{23} &= \frac{\dd\kappa_{2}}{\dd x_{3}} - \frac{\dd\kappa_{3}}{\dd x_{2}}
  &&= \frac{\dd(-G)}{\dd z} - \frac{\dd(-H)}{\dd y} = \alpha.
\end{alignat*}

Accordingly the electric and magnetic forces together form the curl of the
electromagnetic potential. The complete scheme for~$F_{\mu\nu}$ is
\[
\begin{array}[t]{r}
  F_{\mu\nu} \\
  \MuNuarrow
\end{array}
=
\begin{array}[t]{@{}r@{\quad}r@{\quad}r@{\quad}r@{}}
  0 & -\gamma & \beta & -X\\
  \gamma & 0 & -\alpha & -Y\\
  -\beta & \alpha & 0 & -Z\\
  X & Y & Z & 0\\
\end{array}
\Tag{(73.41)}
\]

Using Galilean values of $g^{\mu\nu}$ to raise the two suffixes,
\[
F^{\mu\nu} = \begin{array}[t]{@{}r@{\quad}r@{\quad}r@{\quad}r@{}}
  0 & -\gamma & \beta & \Neg X\\
  \gamma & 0 & -\alpha & Y\\
  -\beta & \alpha & 0 & Z\\
  -X & -Y & -Z & 0\\
\end{array}
\Tag{(73.42)}
\]

Let $\rho$~be the density of electric charge and $\sigma_{x}$, $\sigma_{y}$, $\sigma_{z}$ the density of electric
current. We set
\index{Charge\hyp{}and\hyp{}current vector}%
\index{Current, electric}%
\index{J@$J^{\mu}$ (charge\hyp{}and\hyp{}current vector)}%
\[
J^{\mu} = (\sigma_{x}, \sigma_{y}, \sigma_{z}, \rho).
\Tag{(73.5)}
\]
Here again we must not assume that the components of~$J^{\mu}$ will be recognised
experimentally as electric charge and current\hyp{}density except in the original
system of coordinates.

The universally accepted laws of the electromagnetic field are those given
by Maxwell. Maxwell's equations are
\index{Maxwell's equations}%
\begin{gather*}
  \frac{\dd Z}{\dd y} - \frac{\dd Y}{\dd z} = -\frac{\dd\alpha}{\dd t},\
  \frac{\dd X}{\dd z} - \frac{\dd Z}{\dd x} = -\frac{\dd\beta}{\dd t},\
  \frac{\dd Y}{\dd x} - \frac{\dd X}{\dd y} = -\frac{\dd\gamma}{\dd t},
  \Tag{(73.61)}\displaybreak[0] \\
%
  \frac{\dd\gamma}{\dd y} - \frac{\dd\beta}{\dd z} = \frac{\dd X}{\dd t} + \sigma_{x},\
  \frac{\dd\alpha}{\dd z} - \frac{\dd\gamma}{\dd x} = \frac{\dd Y}{\dd t} + \sigma_{y},\
  \frac{\dd\beta}{\dd x} - \frac{\dd\alpha}{\dd y} = \frac{\dd Z}{\dd t} + \sigma_{z},
  \Tag{(73.62)}\displaybreak[0] \\
%
  \frac{\dd X}{\dd x} + \frac{\dd Y}{\dd y} + \frac{\dd Z}{\dd z} = \rho,
  \Tag{(73.63)}\displaybreak[0] \\
  \frac{\dd \alpha}{\dd x} + \frac{\dd \beta}{\dd y} + \frac{\dd \gamma}{\dd z} = 0.
  \Tag{(73.64)}
\end{gather*}
The Heaviside\hyp{}Lorentz unit of charge is used so that the factor~$4\pi$ does not
appear. The velocity of light is as usual taken to be unity. Specific inductive
capacity and magnetic permeability are merely devices employed in obtaining
macroscopic equations, and do not occur in the exact theory.

It will be seen by reference to \Eq{(73.41)} and~\Eq{(73.42)} that Maxwell's equations
are equivalent to
\begin{gather*}
  \frac{\dd F_{\mu\nu}}{\dd x_{\sigma}} +
  \frac{\dd F_{\nu\sigma}}{\dd x_{\mu}} +
  \frac{\dd F_{\sigma\mu}}{\dd x_{\nu}} = 0,
  \Tag{(73.71)}\displaybreak[0] \\
\frac{\dd F^{\mu\nu}}{\dd x_{\nu}} = J^{\mu}.
\Tag{(73.72)}
\end{gather*}
The first comprises the four equations \Eq{(73.61)} and~\Eq{(73.64)}; and the second
comprises \Eq{(73.62)} and~\Eq{(73.63)}.

On substituting $F_{\mu\nu} = \dd\kappa_{\mu}/\dd x_{\nu} - \dd\kappa_{\nu}\dd x_{\mu}$ in~\Eq{(73.71)} it will be seen that the
equation is satisfied identically. Also \Eq{(73.72)}~is the simplified form for Galilean
coordinates of $(F^{\mu\nu})_{\nu} = J^{\mu}$. Hence Maxwell's laws reduce to the simple form
\begin{align*}
  F_{\mu\nu} &= \frac{\dd\kappa_{\mu}}{\dd x_{\nu}} - \frac{\dd\kappa_{\nu}}{\dd x_{\mu}},
  \Tag{(73.73)} \\
  F_{\nu}^{\mu\nu} &= J^{\mu},
  \Tag{(73.74)}
  \end{align*}
which are \emph{tensor equations}.

By~\Eq{(51.52)} the second equation becomes
\[
\frac{\dd\mf{F}^{\mu\nu}}{\dd x_{\nu}} = \mf{J}^{\mu}.
\Tag{(73.75)}
\]
Owing to the antisymmetry of~$\mf{F}^{\mu\nu}$, $\dd^{2}\mf{F}^{\mu\nu}/\dd x_{\mu}\, \dd x_{\nu}$ vanishes, the terms in the
summation cancelling in pairs. Hence
\[
\frac{\dd^{2}\mf{F}^{\mu\nu}}{\dd x_{\mu}\, \dd x_{\nu}}
= \frac{\dd\mf{J}^{\mu}}{\dd x_{\mu}} = 0,
\Tag{(73.76)}
\]
whence, by~\Eq{(51.12)},
\[
(J^{\mu})_{\mu} = 0.
\Tag{(73.77)}
\]
The divergence of the charge\hyp{}and\hyp{}current vector vanishes.
\index{Charge, electric, conservation of}%
\index{Electric charge!conservation of}%

For our original coordinates \Eq{(73.77)}~becomes
\[
\frac{\dd\sigma_{x}}{\dd x}
+ \frac{\dd\sigma_{y}}{\dd y}
+ \frac{\dd\sigma_{z}}{\dd z}
+ \frac{\dd\rho}{\dd t} = 0.
\Tag{(73.78)}
\]
If the current is produced by the motion of the charge with velocity $(u, v, w)$,
we have $\sigma_{x}, \sigma_{y}, \sigma_{z} = \rho u, \rho v, \rho w$, so that
\[
\frac{\dd(\rho u)}{\dd x}
+ \frac{\dd(\rho v)}{\dd y}
+ \frac{\dd(\rho w)}{\dd z}
+ \frac{\dd\rho}{\dd t} = 0,
\]
which is the usual equation of continuity (cf.~\Eq{(53.71)}), showing that electric
\index{Continuity, equation of!in electric flow}%
charge is conserved.

It may be noted that even in non\hyp{}Galilean coordinates the charge\hyp{}and\hyp{}current
vector satisfies the strict law of conservation
\index{Conservation!of electric charge}%
\[
\frac{\dd\mf{J}^{\mu}}{\dd x_{\mu}} = 0.
\]
This may be contrasted with the material energy and momentum which, it will
be remembered, do not in the general case satisfy
\[
\frac{\dd\mf{T}_{\mu}^{\nu}}{\dd x_{\nu}} = 0,
\]
so that it becomes necessary to supplement them by the pseudo\hyp{}energy\hyp{}tensor~$\mf{t}_{\mu}^{\nu}$
(\SecRef{59}) in order to maintain the formal law. Both $T^{\mu\nu}$ and~$J^{\mu}$ have the
property which in the relativity theory we recognise as the natural generalisation
of conservation, viz.\ $T_{\nu}^{\mu\nu} = 0$, $J_{\mu}^{\nu} = 0$.

If the charge is moving with velocity
\[
\frac{dx}{dt},\
\frac{dy}{dt},\
\frac{dz}{dt},
\]
we have
\begin{align*}
  J^{\mu}
  &= \rho\, \frac{dx}{dt}, \rho\, \frac{dy}{dt}, \rho\, \frac{dz}{dt}, \rho \\
  &= \rho\, \frac{ds}{dt}\left(\frac{dx}{ds}, \frac{dy}{ds}, \frac{dz}{ds}, \frac{dt}{ds}\right).
  \Tag{(73.81)}
\end{align*}
The bracket constitutes a contravariant vector;
consequently $\rho\, ds/dt$ is an invariant.
Now $ds/dt$ represents the FitzGerald contraction, so that a volume
which would be measured as unity by an observer moving with the charge
\index{Charge, electric, conservation of!invariance of}%
\index{Electric charge!invariance of}%
will be measured as $ds/dt$ by an observer at rest in the coordinates chosen.
The invariant $\rho\, ds/dt$ is the amount of charge in this volume, i.e.\ unit proper\hyp{}volume.

We write
\[
\rho_{0} = \rho\, \frac{ds}{dt},
\]
so that $\rho_{0}$~is the proper\hyp{}density of the charge. If $A^{\mu}$~is the velocity\hyp{}vector
$dx_{\mu}/ds$ of the charge, then \Eq{(73.81)}~becomes
\[
J^{\mu} = \rho_{0} A^{\mu}.
\Tag{(73.82)}
\]

\emph{Charge, unlike mass, is not altered by motion relative to the observer.} This
follows from the foregoing result that the amount of charge in an absolutely
defined volume (unit proper\hyp{}volume) is an invariant. The reason for this
difference of behaviour of charge and mass will be understood by reference to~\Eq{(53.2)}
where the FitzGerald factor $ds/dt$ occurs squared.

For the observer~$S$ using our original system of Galilean coordinates, the
quantities $k_{\mu}$, $F_{\mu\nu}$ and~$J^{\mu}$ represent the electromagnetic potential, force, and
current, according to definition. For another observer~$S'$ with different velocity,
we have corresponding quantities $\kappa_{\mu}'$, $F_{\mu\nu}'$, $J'^{\mu}$, obtained by the transformation\hyp{}laws;
but we have not yet shown that these are the quantities which $S'$~will
measure when he makes experimental determinations of potential, force, and
current relative to his moving apparatus. Now if $S'$~recognises certain measured
quantities as potential, force, and current it must be because they play the
same part in the world relative to him, as $\kappa_{\mu}$, $F_{\mu\nu}$ and~$J^{\mu}$ play in the world
relative to~$S$. To play the same part means to have the same properties, or
fulfil the same relations or equations. But $\kappa_{\mu}'$, $F_{\mu\nu}'$ and~$J'^{\mu}$ fulfil the same
equations in~$S'$'s coordinates as $\kappa_{\mu}$, $F_{\mu\nu}$ and~$J^{\mu}$ do in $S$'s~coordinates, \emph{because
the fundamental equations \Eq{(73.73)}, \Eq{(73.74)} and \Eq{(73.77)} are tensor equations
holding in all systems of coordinates}. The fact that Maxwell's equations are
tensor equations, enables us to make the identification of $\kappa_{\mu}$, $F_{\mu\nu}$, $J^{\mu}$ with the
experimental potential, force, and current in all systems of Galilean coordinates
and not merely in the system initially chosen.

In one sense our proof is not yet complete. There are other equations
obeyed by the electromagnetic variables which have not yet been discussed.
In particular there is the equation which prescribes the motion of a particle
carrying a charge in the electromagnetic field. We shall show in \SecRef{76} that
this also is of the tensor form, so that the accented variables continue to play
the same part in $S'$'s~experience which the unaccented variables play in~$S$'s
experience. But even as it stands our proof is sufficient to show that \emph{if} there
exists for~$S'$ a potential, force, and current precisely analogous to the potential,
\index{Potential!electromagnetic}%
force, and current of~$S$, these must be expressed by $\kappa_{\mu}'$, $F_{\mu\nu}$, $J'^{\mu}$, because other
quantities would not satisfy the equations already obtained. The proviso must
clearly be fulfilled unless the special principle of relativity is violated.

When an observer uses non\hyp{}Galilean coordinates, he will as usual treat
them as though they were Galilean and attribute all discrepancies to the
effects of the field of force which is introduced. $\kappa_{\mu}$, $F_{\mu\nu}$ and~$J^{\mu}$ will be identified
with the potential, force, and current, just as though the coordinates were
Galilean. These quantities will no longer accurately obey Maxwell's original
form of the equations, but will conform to our generalised tensor equations
\Eq{(73.73)} and~\Eq{(73.74)}. The replacement of~\Eq{(73.72)} by the more general form~\Eq{(73.74)}
extends the classical equations to the case in which a gravitational
field of force is acting in addition to the electromagnetic field.

\Section{74.}{Electromagnetic waves}
\index{Electromagnetic action!waves, propagation of}%
\index{Light!propagation of}%
\index{Waves!electromagnetic}%

\Subsection[(a)]{Propagation of electromagnetic potential.}
\index{Propagation!of electromagnetic waves}%

It is well known that the electromagnetic potentials $F$, $G$, $H$, $\Phi$ are not
determinate. They are concerned in actual phenomena only through their
curl---the electromagnetic force. The curl is unaltered, if we replace
\[
-F,\ -G,\ -H,\ \Phi\quad\text{by}\quad
-F + \frac{\dd V}{\dd x},\
-G + \frac{\dd V}{\dd \smash[b]{y}},\
-H + \frac{\dd V}{\dd z},\
\Phi + \frac{\dd V}{\dd t},
\]
where $V$~is an arbitrary function of the coordinates. The latter expression
gives the same field of electromagnetic force and may thus equally well be
adopted for the electromagnetic potentials.

It is usual to avoid this arbitrariness by selecting from the possible values
the set which satisfies
\[
\frac{\dd F}{\dd x} + \frac{\dd G}{\dd y} + \frac{\dd H}{\dd z} + \frac{\dd\Phi}{\dd t} = 0.
\]
Similarly in general coordinates we remove the arbitrariness of~$\kappa_{\mu}$ by imposing
the condition
\[
(\kappa^{\mu})_{\mu} = 0.
\Tag{(74.1)}
\]
When the boundary\hyp{}condition at infinity is added, the value of~$\kappa_{\mu}$ becomes
completely determinate.

By \Eq{(73.74)} and~\Eq{(73.3)}
\begin{align*}
  J = (F_{\mu}^{\alpha})_{\alpha}
  &= (g^{\alpha\beta} F_{\mu\beta})_{\alpha} = g^{\alpha\beta} (F_{\mu\beta})_{\alpha} \\
  &= g^{\alpha\beta} (\kappa_{\mu\beta\alpha} - \kappa_{\beta\mu\alpha})
  \Tag{(74.2)}\displaybreak[0] \\
  &= g^{\alpha\beta} (\kappa_{\mu\beta\alpha} - \kappa_{\beta\alpha\mu} + B_{\beta\alpha\mu}^{\epsilon} \kappa_{\epsilon})
  \quad\text{by~\Eq{(34.3)}}\displaybreak[0] \\
  &= g^{\alpha\beta} (\kappa_{\mu})_{\beta\alpha} - (\kappa_{\alpha}^{\alpha})_{\mu} + G_{\mu}^{\epsilon} \kappa_{\epsilon}.
\end{align*}

The operator $g^{\alpha\beta}(\dots)_{\beta\alpha}$ has been previously denoted by~$\Wave$. Also, by~\Eq{(74.1)}
$\kappa_{\alpha}^{\alpha} = 0$. Hence
\[
\Wave\kappa_{\mu} = J^{\mu} - G_{\mu}^{\epsilon}\kappa_{\epsilon}.
\Tag{(74.31)}
\]
In empty space this becomes
\[
\Wave \kappa_{\mu} = 0,
\Tag{(74.32)}
\]
showing that $\kappa_{\mu}$~is propagated with the fundamental velocity.

If the law of gravitation $G_{\mu\nu} = \lambda g_{\mu\nu}$ for curved space-time is adopted, the
equation in empty space becomes
\[
(\Wave + \lambda) \kappa_{\mu} = 0.
\Tag{(74.33)}
\]

\Subsection[(b)]{Propagation of electromagnetic force.}

To determine a corresponding law of propagation of~$F_{\mu\nu}$ we naturally try
to take the curl of~\Eq{(74.31)}; but care is necessary since the order of the operations
curl and~$\Wave$ is not interchangeable.

By~\Eq{(74.2)}
\begin{align*}
  J_{\mu\nu}
  &= g^{\alpha\beta}(\kappa_{\mu\beta\alpha\nu} - \kappa_{\beta\mu\alpha\nu}) \\
%
%[** TN: Next line not broken in the original]
  &= g^{\alpha\beta}(\kappa_{\mu\beta\nu\alpha} - \kappa_{\beta\mu\nu\alpha}) \\
  &\qquad- g^{\alpha\beta} (B_{\mu\nu\alpha}^{\epsilon} \kappa_{\epsilon\beta}
  + B_{\beta\nu\alpha}^{\epsilon} \kappa_{\mu\epsilon}
  - B_{\beta\nu\alpha}^{\epsilon} \kappa_{\epsilon\mu}
  - B_{\mu\nu\alpha}^{\epsilon} \kappa_{\beta\epsilon})
  \quad\text{by~\Eq{(34.8)}} \\
%
  &= g^{\alpha\beta}(\kappa_{\mu\beta\nu} - \kappa_{\beta\mu\nu})_{\alpha}
  - g^{\alpha\beta} (B_{\mu\nu\alpha}^{\epsilon} F_{\epsilon\beta}
  - B_{\beta\nu\alpha}^{\epsilon} F_{\epsilon\mu}) \\
%
    &= g^{\alpha\beta}(\kappa_{\mu\nu\beta} - \kappa_{\beta\mu\nu} + B_{\mu\beta\nu}^{\epsilon} \kappa_{\epsilon})_{\alpha}
  - B_{\mu\nu\alpha\epsilon} F^{\epsilon\alpha} - G_{\nu}^{\epsilon} F_{\epsilon\mu}.
\end{align*}
Hence
\begin{multline*}
  J_{\mu\nu} - J_{\nu\mu} = g^{\alpha\beta}(\kappa_{\mu\nu\beta} - \kappa_{\nu\mu\beta}
  - B_{\beta\mu\nu}^{\epsilon} \kappa_{\epsilon}
  + B_{\mu\beta\nu}^{\epsilon} \kappa_{\epsilon}
  + B_{\nu\beta\mu}^{\epsilon} \kappa_{\epsilon})_{\alpha} \\
  - (B_{\mu\nu\alpha\epsilon} - B_{\nu\mu\alpha\epsilon}) F^{\epsilon\alpha}
  - G_{\nu}^{\epsilon} F_{\epsilon\mu} + G_{\mu}^{\epsilon} F_{\epsilon\nu}.
\end{multline*}
But by the cyclic relation~\Eq{(34.6)}
\[
B_{\beta\mu\nu}^{\epsilon} + B_{\mu\nu\beta}^{\epsilon} + B_{\nu\beta\mu}^{\epsilon} = 0.
\]
Also by the antisymmetric properties
\[
(B_{\mu\nu\alpha\epsilon} - B_{\nu\mu\alpha\epsilon}) F^{\epsilon\alpha}
= 2B_{\mu\nu\alpha\epsilon} F^{\epsilon\alpha}.
\]
Hence the result reduces to
\[
J_{\mu\nu} - J_{\nu\mu}
= g^{\alpha\beta} (\kappa_{\mu\nu} - \kappa_{\nu\mu})_{\beta\alpha}
- G_{\nu}^{\epsilon} F_{\epsilon\mu}
+ G_{\mu}^{\epsilon} F_{\epsilon\nu}
- 2B_{\mu\nu\alpha\epsilon} F^{\epsilon\alpha},
\]
so that
\[
\Wave F_{\mu\nu} = J_{\mu\nu} - J_{\nu\mu}
- G_{\mu}^{\epsilon} F_{\epsilon\nu}
+ G_{\nu}^{\epsilon} F_{\epsilon\mu}
+ 2B_{\mu\nu\alpha\epsilon} F^{\epsilon\alpha}.
\Tag{(74.41)}
\]
In empty space this becomes
\[
\Wave F_{\mu\nu} = 2B_{\mu\nu\alpha\epsilon} F^{\epsilon\alpha}
\Tag{(74.42)}
\]
for an infinite world. For a curved world undisturbed by attracting matter,
in which $G_{\mu}^{\epsilon} = \lambda g_{\mu}^{\epsilon}$, $B_{\mu\nu\alpha\epsilon} = \frac{1}{3}\lambda(g_{\mu\nu} g_{\alpha\epsilon} - g_{\mu\alpha} g_{\nu\epsilon})$, the result is
\[
(\Wave + \tfrac{4}{3}\lambda)F_{\mu\nu} = 0.
\Tag{(74.43)}
\]
It need not surprise us that the velocity of propagation of electromagnetic
potential and of electromagnetic force is not the same (cf.~\Eq{(74.33)} and~\Eq{(74.43)}).
The former is not physically important since it involves the arbitrary convention
$\kappa_{\alpha}^{\alpha} = 0$.

But the result~\Eq{(74.42)} is, I think, unexpected. It shows that the equations
of propagation of electromagnetic force involve the Riemann\hyp{}Christoffel tensor;
and therefore this is not one of the phenomena for which the ordinary Galilean
equations can be immediately generalised by the principle of equivalence.
This naturally makes us uneasy as to whether we have done right in adopting
the invariant equations of propagation of light ($ds = 0$, $\delta\int ds = 0$) as true in
all circumstances; but the investigation which follows is reassuring.

\Subsection[(c)]{Propagation of a wave-front.}

The conception of a ``ray'' of light in physical optics is by no means
elementary. Unless the wave-front is of infinite extent, the ray is an abstraction,
and to appreciate its meaning a full discussion of the phenomena of
interference fringes is necessary. We do not wish to enter on such a general
discussion here; and accordingly we shall not attempt to obtain the formulae
for the tracks of rays of light for the case of general coordinates \Foreign{ab initio}.
Our course will be to reduce the general formulae to such a form, that the
subsequent work will follow the ordinary treatment given in works on physical
optics.

The fundamental equation treated in the usual theory of electromagnetic
waves is
\[
\left(\frac{\dd^{2}}{\dd t^{2}}
- \frac{\dd^{2}}{\dd x^{2}}
- \frac{\dd^{2}}{\dd y^{2}}
- \frac{\dd^{2}}{\dd z^{2}}\right) \kappa_{\mu} = 0,
\Tag{(74.51)}
\]
which is the form taken by $\Wave\kappa_{\mu} = 0$ in Galilean coordinates. When the region
of space-time is not flat we cannot immediately simplify $\Wave\kappa_{\mu}$ in this way;
but we can make a considerable simplification by adopting natural coordinates
at the point considered. In that case the $3$-index symbols (but not their
derivatives) vanish, and
\begin{align*}
  \Wave\kappa_{\mu} &= g^{\alpha\beta}(\kappa_{\mu})_{\alpha\beta} \\
  &= g^{\alpha\beta} \left(\frac{\dd^{2}\kappa_{\mu}}{\dd x_{\alpha}\, \dd x_{\beta}}
  - \frac{\dd}{\dd x_{\alpha}} \{\mu\beta, \epsilon\} \cdot \kappa_{\epsilon}\right).
\end{align*}
Hence the law of propagation $\Wave\kappa_{\mu} = 0$ becomes in natural coordinates
\[
\left(\frac{\dd^{2}}{\dd t^{2}}
- \frac{\dd^{2}}{\dd x^{2}}
- \frac{\dd^{2}}{\dd y^{2}}
- \frac{\dd^{2}}{\dd z^{2}}\right) \kappa_{\mu}
= g^{\alpha\beta}\, \frac{\dd}{\dd x_{\alpha}} \{\mu\beta, \epsilon\} \cdot \kappa_{\epsilon}.
\Tag{(74.52)}
\]

At first sight this does not look very promising for a justification of the
principle of equivalence. We cannot make all the derivatives $\dd\{\mu\beta, \epsilon\}/\dd x_{\alpha}$
vanish by any choice of coordinates, since these determine the Riemann\hyp{}Christoffel
tensor. It looks as though the law of propagation in curved space-time
involves the Riemann\hyp{}Christoffel tensor, and consequently differs from
the law in flat space-time. But the inner multiplication by~$g^{\alpha\beta}$ saves the
situation. It is possible to choose coordinates such that $g^{\alpha\beta}\, \dd\{\mu\beta, \epsilon\}/\dd x_{\alpha}$ vanishes
for all the sixteen possible combinations of $\mu$ and~$\epsilon$\footnotemark.\footnotetext
  {According to~\Eq{(36.55)} it is possible by a transformation to increase $\dd\{\mu\beta, \epsilon\}/\dd x_{\alpha}$ by an
  arbitrary quantity~$a_{\mu\beta\alpha}^{\epsilon}$, symmetrical in $\mu$,~$\beta$ and~$\alpha$. The sixteen quantities $g^{\alpha\beta} a_{\mu\beta\alpha}^{\epsilon}$ ($\mu, \epsilon = 1, 2, 3, 4$)
  will not have to fulfil any conditions of symmetry, and may be chosen independently of one another
  Hence we can make the right-hand side of~\Eq{(74.52)} vanish by an appropriate transformation.}
For these coordinates
\Eq{(74.52)}~reduces to~\Eq{(74.51)}, and the usual solution for flat space-time will
apply at the point considered.

A solution of~\Eq{(74.51)}, giving plane waves, is
\index{Propagation with unit velocity!solution of equation}%
\index{Wave\hyp{}equation, solution of}%
\[
\kappa_{\mu} = A_{\mu} \exp\frac{2\pi i}{\lambda}(lx + my + nz - ct).
\Tag{(74.53)}
\]
Here $A_{\mu}$~is a constant vector; $l$, $m$, $n$ are direction cosines so that $l^{2} + m^{2} + n^{2} = 1$.
Substituting in~\Eq{(74.51)} we find that it will be satisfied if $c^2 = 1$ and the first
and second derivatives of $l$, $m$, $n$, $c$ vanish. According to the usual discussion
of this equation $(l, m, n)$ is the direction of the ray and $c$~the velocity of
propagation along the ray.

The vanishing of first and second derivatives of $(l, m, n)$ shows that the
direction of the ray is stationary at the point considered. (The light\hyp{}oscillations
correspond to~$F_{\mu\nu}$ (not~$\kappa_{\mu}$) and the direction of the ray would not
necessarily agree with $(l, m, n)$ if the first derivatives did not vanish; consequently
the stationary property depends on the vanishing of second derivatives
as well.) Further the velocity~$c$ along the ray is unity.

It follows that in any kind of space-time the ray is a geodesic, and the
velocity is such as to satisfy the equation $ds = 0$. Stated in this form, the
result deduced for a very special system of coordinates must hold for all
coordinate\hyp{}systems since it is expressed invariantly. The expression for the
potential~\Eq{(74.53)} is, of course, only valid for the special coordinate\hyp{}system.

We have thus arrived at a justification of the law for the track of a light-pulse
(\SecRef{47}~(4)) which has been adopted in our previous work.

\Subsection[(d)]{Solution of the equation $\Wave\kappa^{\mu} = J^{\mu}$.}

We assume that space-time is flat to the order of approximation required,
and accordingly adopt Galilean coordinates. The equation becomes
\[
\frac{\dd^{2}\kappa^{\mu}}{\dd t^{2}} - \nabla^2\kappa^{\mu} = J^{\mu},
\]
of which the solution (well known in the theory of sound) is
\[
\{\kappa^{\mu}\}_{x,y,z,t} = \frac{1}{4\pi} \iiint \{J^{\mu}\}_{\xi,\eta,\zeta,t-r} \cdot \frac{d\xi\, d\eta\, d\zeta}{r},
\Tag{(74.61)}
\]
where $r$~is the distance between $(x, y, z)$ and $(\xi,\eta, \zeta)$.

The contributions to~$\kappa^{\mu}$ of each element of charge or current are simply
additive; accordingly we shall consider a single element of charge~$de$ moving
with velocity~$A^{\mu}$, and determine the part of~$\kappa^{\mu}$ corresponding to it. By~\Eq{(73.81)}
the equation becomes
\[
\kappa^{\mu} = \frac{1}{4\pi}\, \frac{ds}{dt} A^{\mu} \iiint \rho\, \frac{d\xi\, d\eta\, d\zeta}{r},
\Tag{(74.62)}
\]
where all quantities on the right are taken for the time~$t - r$.

For an infinitesimal element we may take $\rho$~constant and insert limits of
integration; but these limits must be taken for the time~$t - r$, and this introduces
an important factor representing a kind of Doppler effect. If the element
of charge is bounded by two planes perpendicular to the direction of~$r$, the
limits of integration are from the front plane at time $t - r$ to the rear plane
\index{Retarded potential}%
at time $t - r - dr$. If $v_{r}$~is the component velocity in the direction of~$r$, the
front plane has had time to advance a distance~$v_{r}\, dr$. Consequently the
instantaneous thickness of the element of charge is less than the distance
between the limits of integration in the ratio $1 - v_{r}$; and the integration is
over a volume $(1 - v_{r})^{-1}$~times the instantaneous volume of the element of
charge. Hence
\[
\iiint \rho\, d\xi\, d\eta\, d\zeta = \frac{de}{1 - v_{r}}.
\]

Writing as usual $\beta$~for the FitzGerald factor~$dt/ds$, \Eq{(74.62)}~becomes
\[
\kappa^{\mu} = \left\{\frac{A^{\mu}\, de}{4\pi r\beta(1 - v_{r})}\right\}_{t-r}
= \left\{\frac{de(u, v, w, 1)}{4\pi r(1 - v_{r})}\right\}_{t-r}.
\Tag{(74.71)}
\]

In most applications the motion of the charge can be regarded as uniform
during the time of propagation of the potential through the distance~$r$. In
that case
\[
\{r(1 - v_{r})\}_{t-r} = \{r\}_{t},
\]
the present distance being less than the antedated distance by~$v_{r} r$. The result
then becomes
\[
\kappa^{\mu} = \left\{\frac{A^{\mu}\, de}{4\pi r\beta}\right\}_{t}
= \left\{\frac{de(u, v, w, 1)}{4\pi r}\right\}_{t}.
\Tag{(74.72)}
\]

It will be seen that the scalar potential~$\Phi$ of a charge is unaltered by
uniform motion, and must be reckoned for the present position of the charge,
\emph{not from the antedated position}.

The equation~\Eq{(74.71)} can be written in the pseudo\hyp{}tensor form
\[
\kappa^{\mu} = \left\{\frac{A^{\mu}\, de}{4\pi A^{\nu}R_{\nu}}\right\}_{R^{\alpha}R_{\alpha}=0},
\Tag{(74.8)}
\]
where $R^{\mu}$~is the pseudo\hyp{}vector representing the displacement from the charge
\index{Pseudo\hyp{}vector}%
$(\xi, \eta, \zeta, \tau)$ to the point $(x, y, z, t)$ where $\kappa^{\mu}$~is reckoned. The condition
$R^{\alpha}R_{\alpha} = 0$ gives
\[
-(x - \xi)^{2} - (y - \eta)^{2} - (z - \zeta)^{2} + (t - \tau)^{2} = 0,
\]
so that
\[
\tau = t - r.
\]
Also
\begin{align*}
  A^{\nu} R_{\nu}
  &= -\beta u(x - \xi) - \beta v(y - \eta) - \beta w(z - \zeta) + \beta(t - \tau) \\
  &= -\beta v_{r} r + \beta r \\
  &= r\beta(1 - v_{r}).
\end{align*}

A \emph{finite} displacement~$R^{\mu}$ is not a vector in the general theory. We call it
a pseudo\hyp{}vector because it behaves as a vector for Galilean coordinates and
Lorentz transformations. Thus the equation~\Eq{(74.8)} does not admit of application
to coordinates other than Galilean.

Equation~\Eq{(74.71)} expresses the potential at time~$t$ in terms of the positions and strengths of the sources
at a time~$t-r$, different for each source.
Evidently it will be useful for practical calculation to have a formula giving the potential at time~$t$ in terms
of the positions and strengths of the sources at the time~$t$.
This formula is obtained as follows.

Consider a fixed point~$P$ at time~$t$ and a moving source~$P'$ at time~$t-\tau$.
Let the motion of the source be prescribed so that the distance $PP'$ is given as a function of~$t-\tau$, viz.
\[
PP' = r = f(t-\tau).
\]
Then the component velocity of~$P'$ along $P'P$ is
\[
v_r = -dr/dt = -f'(t-\tau).
\]
Suppose that the wave emitted from~$P'$ at time~$t-\tau$ has at time~$t$ reached a point~$Q$ in the direction~$P'P$.
Let~$PQ=\alpha$, so that
\[
\alpha = \tau - r = \tau - f(t-\tau),
\Tag{(74.91)}
\]
the wave\hyp{}velocity being unity.
Then~$\alpha$ is a function of~$t$ and \Foreign{vice versa}.
By differentiating~\Eq{(74.91)}
\[
1 = \frac{d\tau}{d\alpha} + f'(t-\tau)\frac{d\tau}{d\alpha} = (1-v_r)\frac{d\tau}{d\alpha}.
\]
Hence, if~$\phi(t-\tau)$ is any quantity associated with the source~$P'$ at the time~$t-\tau$
\[
\left\{ \frac{\phi}{r(1-v_r)} \right\}_{t-\tau} = \frac{\phi(t-\tau)}{f(t-\tau)}\frac{d\tau}{d\alpha}=\frac{d}{d\alpha}F(t-\tau),
\Tag{(74.92)}
\]
where $F'=-\phi/f$.

The appropriate value of~$\tau$ required in calculating a retarded potential is given by the condition that~$Q$
coincides with~$P$, i.e.~$\alpha = 0$.
Hence
\[
\left[\frac{\phi}{r(1-v_r)}\right] = \left\{\frac{d}{d\alpha}F(t-\tau)\right\}_{\alpha=0},
\Tag{(74.93)}
\]
the square bracket indicating the antedated value.

By Lagrange's theorem on the expansion of implicit functions, writing~\Eq{(74.91)} in the form
\[
(t-\tau) = (t-\alpha)-f(t-\tau),
\]
we have
\begin{multline*}
F(t-\tau) = F(t-\alpha) - F'(t-\alpha)f(t-\alpha) -
                \sum_2^\infty\frac{1}{n!}\frac{\dd^{n-1}}{\dd\alpha^{n-1}}(F'(t-\alpha)\{f(t-\alpha)\}^n)\\
          = F(t-\alpha) + \phi(t-\alpha) +
              \sum_2^\infty\frac{1}{n!}\frac{\dd^{n-1}}{\dd\alpha^{n-1}}(\phi(t-\alpha)\{f(t-\alpha)\}^n)
\end{multline*}
by~\Eq{(74.92)}.
Substituting in~\Eq{(74.93)}, and noting that~$\dd/\dd\alpha = -\dd/\dd t$, we obtain
\[
\left[\frac{\phi}{r(1-v_r)}\right]= -F'(t) - \frac{d}{dt}\phi(t) +
                                      \sum_2^\infty\frac{(-1)^n}{n!}\frac{d^n}{dt^n}(\phi(t)\{f(t)\}^n).
\]
Hence
\[
\left[\frac{\phi}{r(1-v_r)}\right]= \frac{\phi}{r} - \frac{d\phi}{dt}\phi(t) +
                                      \sum_2^\infty\frac{(-1)^n}{n!}\frac{d^n}{dt^n}(r^{n-1}\phi).
\Tag{(74.94)}
\]
where on the right~$r$ and~$\phi$ are to be taken for the time~$t$.
This expansion has been used in~\SecRef{57}.

\Section[Lorentz transformation of electromagnetic force]{75.}{The Lorentz transformation of electromagnetic force}
\index{Force!Lorentz transformation of}%
\index{Lorentz transformation!for electromagnetic force}%

The Lorentz transformation for an observer~$S'$ moving relatively to~$S$ with
a velocity~$u$ along the $x$-axis is
\[
x_{1}' = q(x_1 - u x_4),\quad
x_{2}' = x_2,\quad
x_{3}' = x_3,\quad
x_{4}' = q(x_4 - u x_1),
\Tag{(75.1)}
\]
where
\[
q = (1 - u^2)^{-\frac{1}{2}}.
\]
We use $q$ instead of~$\beta$ in order to avoid confusion with the component~$\beta$ of
magnetic force.
\index{Force!mechanical force due to}%
\index{Mechanical force of electromagnetic field}%

We have
\[
\frac{\dd x_{1}'}{\dd x_{1}} = \frac{\dd x_{4}'}{\dd x_{4}} = q,\quad
\frac{\dd x_{1}'}{\dd x_{4}} = \frac{\dd x_{4}'}{\dd x_{1}} = -qu,\quad
\frac{\dd x_{2}'}{\dd x_{2}} = \frac{\dd x_{3}'}{\dd x_{3}} = 1,
\Tag{(75.2)}
\]
and all other derivatives vanish.

To calculate the electromagnetic force for~$S'$ in terms of the force for~$S$,
we apply the general formulae of transformation~\Eq{(23.21)}. Thus
\begin{align*}
  \gamma' = F'^{12}
  &= \frac{\dd x_{1}'}{\dd x_{\alpha}}\, \frac{\dd x_{2}'}{\dd x_{\beta}}\, F^{\alpha\beta} \\
  &= \frac{\dd x_{1}'}{\dd x_{1}}\, \frac{\dd x_{2}'}{\dd x_{2}}\, F^{12}
   + \frac{\dd x_{1}'}{\dd x_{4}}\, \frac{\dd x_{2}'}{\dd x_{2}}\, F^{42} \\
  &= q\gamma - quY.
\end{align*}
Working out the other components similarly, the result is
\[
\left.
\begin{aligned}
  X' &= X, & Y' &= q(Y - u\gamma), & Z' &= q(Z + u\beta)\\
  \alpha' &= \alpha, & \beta' &= q(\beta + uZ), & \gamma' &= q(\gamma - uY)
\end{aligned}
\right\}
\Tag{(75.3)}
\]
which are the formulae given by Lorentz.

The more general formulae when the velocity of the observer~$S'$ is $(u, v, w)$
become very complicated. We shall only consider the approximate results
when the square of the velocity is neglected. In that case $q = 1$, and the
formulae~\Eq{(75.3)} can be completed by symmetry, viz.\
\[
\left.
\begin{alignedat}{3}
  X' &= X &&- w\beta &&+ v\gamma\\
  \alpha' &= \alpha &&+ wY &&- vZ
\end{alignedat}
\right\}
\Tag{(75.4)}
\]

\Section{76.}{Mechanical effects of the electromagnetic field}

According to the elementary laws, a piece of matter carrying electric
charge of density~$\rho$ experiences in an electrostatic field a mechanical force
\[
\rho X,\quad \rho Y,\quad \rho Z
\]
per unit volume. Moving charges constituting electric currents of amount
$(\sigma_{x}, \sigma_{y}, \sigma_{z})$ per unit volume are acted on by a magnetic field, so that a
mechanical force
\[
\gamma\sigma_{y} - \beta\sigma_{z},\quad
\alpha\sigma_{z} - \gamma\sigma_{x},\quad
\beta\sigma_{x} - \alpha\sigma_{y}
\]
per unit volume is experienced.

Hence if $(P, Q, R)$ is the total mechanical force per unit volume
\[
\left.
\begin{alignedat}{3}
P &= \rho X &&+ \gamma \sigma_{y} &&- \beta \sigma_{z}\\
Q &= \rho Y &&+ \alpha \sigma_{z} &&- \gamma\sigma_{x}\\
R &= \rho Z &&+ \beta  \sigma_{x} &&- \alpha\sigma_{y}
\end{alignedat}
\right\}
\Tag{(76.1)}
\]
The rate at which the mechanical force does work is
\[
S = \sigma_{x} X + \sigma_{y} Y + \sigma_{z} Z.
\]
The magnetic part of the force does no work since it acts at right angles to
the current of charged particles.

By \Eq{(73.41)} and~\Eq{(73.5)} we find that these expressions are equivalent to
\index{ham@$h_{\mu}$ (ponderomotive force)}%
\[
(P, Q, R, -S) = F_{\mu\nu} J^{\nu}.
\]
We denote the vector~$F_{\mu\nu} J^{\nu}$ by~$h_{\mu}$. Raising the suffix with Galilean~$g_{\mu\nu}$ we
have
\[
(P, Q, R, S) = -h^{\mu} = -{F^{\mu}}_{\nu} J^{\nu}.
\Tag{(76.2)}
\]

The mechanical force will change the momentum and energy of the
material system; consequently the material energy\hyp{}tensor taken alone will
\index{M@$M_{\mu\nu}$ (material energy\hyp{}tensor)}%
no longer be conserved. In order to preserve the law of conservation of
momentum and energy, we must recognise that the electric field contains an
electromagnetic momentum and energy whose changes are equal and opposite
\index{E@$E_{\mu\nu}$ (electromagnetic energy\hyp{}tensor)}%
to those of the material system\footnotemark.\footnotetext
  {Notwithstanding the warning conveyed by the fate of potential energy (\SecRef{59}) we are again
  running into danger by generalising energy so as to conform to an assigned law. I am not sure
  that the danger is negligible. But we are on stronger ground now, because we know that there is a
  world\hyp{}tensor which satisfies the assigned law $T_{\nu}^{\mu\nu} = 0$; whereas the potential energy was introduced
  to satisfy $\dd\mf{S}_{\mu}^{\nu}/\dd x_{\nu} = 0$, and it was only a speculative possibility (now found to be untenable) that
  there existed a tensor with that property.}
The whole energy\hyp{}tensor will then consist
of two parts, $M_{\mu}^{\nu}$~due to the matter and $E_{\mu}^{\nu}$~due to the electromagnetic field.

{\Loosen We keep the notation~$T_{\mu}^{\nu}$ for the whole energy\hyp{}tensor---the thing which
is always conserved, and is therefore to be identified with $G_{\mu}^{\nu} - \frac{1}{2}g_{\mu}^{\nu}G$. Thus}
\[
T_{\mu}^{\nu} = M_{\mu}^{\nu} + E_{\mu}^{\nu}.
\Tag{(76.3)}
\]

Since $P$, $Q$, $R$, $S$ measure the rate of increase of momentum and energy
of the material system, they may be equated to~$\dd M^{\mu\nu}/\dd x_{\nu}$ as in~\Eq{(53.82)}. Thus
\[
\frac{\dd M^{\mu\nu}}{\dd x_{\nu}} = -h^{\mu}.
\]
The equal and opposite change of the momentum and energy of the electromagnetic
field is accordingly given by
\[
\frac{\dd E^{\mu\nu}}{\dd x_{\nu}} = +h^{\mu}.
\]
These equations apply to Galilean or to natural coordinates. We pass over to
general coordinates by substituting covariant derivatives, so as to obtain the
tensor equations
\[
M_{\nu}^{\mu\nu} = -h^{\mu} = -E_{\nu}^{\mu\nu},
\Tag{(76.4)}
\]
which are independent of the coordinates used. This satisfies
\[
T_{\nu}^{\mu\nu} = (M^{\mu\nu} + E^{\mu\nu})_{\nu} = 0.
\]

Consider a charge moving with velocity $(u, v, w)$. We have by~\Eq{(75.4)}
\begin{align*}
  \rho X' &= \rho X - (\rho w)\beta + (\rho v)\gamma \\
  &= \rho X - \sigma_{z}\beta + \sigma_{y}\gamma \\
&= P.
\end{align*}
The square of the velocity has been neglected, and to this order of approximation
$\rho' = \rho$. Thus to the first order in the velocities, the mechanical force on
a moving charge is $(\rho'X, \rho'Y, \rho'Z)$; just as the mechanical force on a charge
at rest is $(\rho X, \rho Y, \rho Z)$. We obtain the force on the moving charge either by
applying the formula~\Eq{(76.1)} in the original coordinates, or by transforming to
new coordinates in which the charge is at rest so that $\sigma_{x}$, $\sigma_{y}$, $\sigma_{z} = 0$. The
equivalence of the two calculations is in accordance with the principle of
relativity for uniform motion.

If the square of the velocity is not neglected, no such simple relation
exists. The mechanical force ($\text{mass} \times \text{acceleration}$) will not be exactly the same
in the accented and unaccented systems of coordinates, since the mass and
acceleration are altered by terms involving the square of the velocity. In
fact we could not expect any accurate relation between the mechanical force
$(P, Q, R)$ and the electric force $(X, Y, Z)$ in different systems of coordinates; the
former is part of a vector, and the latter part of a tensor of the second rank.

Perhaps it might have been expected that with the advent of the electron
\index{Non\hyp{}Maxwellian stresses}%
theory of matter it would become unnecessary to retain a separate material
energy\hyp{}tensor~$M^{\mu\nu}$, and that the whole energy and momentum could be
included in the energy\hyp{}tensor of the electromagnetic field. But we cannot
dispense with~$M^{\mu\nu}$. The fact is that an electron must not be regarded as a
purely electromagnetic phenomenon; that is to say, something enters into its
constitution which is not comprised in Maxwell's theory of the electromagnetic
field. In order to prevent the electronic charge from dispersing under its own
repulsion, non\hyp{}Maxwellian ``binding forces'' are necessary, and it is the energy,
stress and momentum of these binding forces which constitute the material
energy\hyp{}tensor~$M^{\mu\nu}$.

The equation~\Eq{(76.2)} giving the mechanical force due to an electromagnetic field is essentially a
\emph{macroscopic} equation; it expresses the results of experiments on matter in bulk.
We must not assume that it holds for a single electron---in fact it is probably untrue.

This limitation has been obscured by the fact that the law of motion of a single electron, which has been
discovered empirically, has a false resemblance to~\Eq{(76.2)}.
But as shall be explained in~\SecRef{80} the force on the electron is~${F'^\mu}_\nu J^\nu$, where~${F'^\mu}_\nu$ is
the applied external field, not the actual field~${F^\mu}_\nu$ modified by the presence of the accelerated electron.
In a macroscopically continuous distribution of charge and current the distinction between~$F$ and~$F'$ is of no
importance\footnotemark;\footnotetext
            {The material is assumed to be non\hyp{}polarisable.
             Polarisation introduces the complications discussed in~\SecRef{82}.}
but in an electronic distribution~$F-F'$ is of the same order of magnitude as~$F'$ at the only points which matter,
viz.\ where~$J^\mu\neq 0$. According to~\SecRef{80} the condition~$F_{\mu\nu}F^{\mu\nu}=0$ fulfilled in empty space
appears to be replaced by the condition that the integral of this expression over the electron vanishes,
or by some condition nearly akin to this.
The appearance of an integral equation in the microscopic laws of physics is in accordance with modern expectation.

From~\Eq{(76.2)} the expression~$E^\nu_\mu$ for the electromagnetic energy\hyp{}tensor is obtained.
This accordingly refers to the macroscopic electromagnetic field, and has nothing to do with the intense
forces within and between the atoms.
We then recognise that the whole energy\hyp{}tensor (which is conserved identically) must consist of the two
parts~$M^\nu_\mu$ and~$E^\nu_\mu$ supposed to be due respectively to the continuous matter and the continuous
electromagnetic field.
The question whether we could possibly dispense with~$M^\nu_\mu$ does not arise, because the macroscopic
conception does not admit local electronic fields.

In passing to the microscopic problem we assume that the state of the world at any point is still described by
the variables~$g_{\mu\nu}$ and~$F_{\mu\nu}$.
We continue to construct the formal quantities~$T_{\mu\nu}$ and~$E_{\mu\nu}$ according to~\Eq{(54.3)} and~\Eq{(77.2)}.
The former has an evident importance because it is identically conserved;
the latter is only of interest because it is continuous with the genuine electromagnetic energy\hyp{}tensor which still
exists in regions outside matter.
Outside matter~$T_{\mu\nu}$ and~$E_{\mu\nu}$ coalesce; inside matter they cannot be equal because~$E$ vanishes
identically.
The difference is called the non\hyp{}Maxwellian energy\hyp{}tensor~$M_{\mu\nu}$.
It will be seen that the separation of~$T_{\mu\nu}$ into~$M_{\mu\nu}+E_{\mu\nu}$ is somewhat different according
as the problem is treated macroscopically or microscopically, since in the former case the energy of the intense
Maxwellian fields around the electrons is included in~$M_{\mu\nu}$.

\Section{77.}{The electromagnetic energy\hyp{}tensor}
\index{Electromagnetic action!energy\hyp{}tensor}%
\index{Energy\hyp{}tensor of matter!electromagnetic@of electromagnetic field}%

To determine explicitly the value of~$E_{\mu}^{\nu}$ we have to rely on the relation
found in the preceding section
\[
E_{\mu\nu}^{\nu} = h_{\mu} = F_{\mu\nu} J^{\nu} = F_{\mu\nu} F_{\sigma}^{\nu\sigma}.
\Tag{(77.1)}
\]

The solution of this differential equation is
\[
E_{\mu}^{\nu}  = -F^{\nu\alpha} F_{\mu\alpha} + \tfrac{1}{4} g_{\mu}^{\nu} F^{\alpha\beta} F_{\alpha\beta}.
\Tag{(77.2)}
\]

To verify this we take the divergence, remembering that covariant differentiation
obeys the usual distributive law and that $g_{\mu}^{\nu}$~is a constant.
\begin{alignat*}{2}
  E_{\mu\nu}^{\nu}
  &= -F_{\nu}^{\nu\alpha} F_{\mu\alpha}
  &&- \phantom{\tfrac{1}{2}} F^{\nu\alpha} F_{\mu\alpha\nu}
  + \tfrac{1}{4} g_{\mu}^{\nu}
     (F_{\nu}^{\alpha\beta} F_{\alpha\beta} +  F^{\alpha\beta} F_{\alpha\beta\nu}) \\
%
  &= -F_{\nu}^{\nu\alpha} F_{\mu\alpha}
  &&- \phantom{\tfrac{1}{2}} F^{\nu\alpha} F_{\mu\alpha\nu}
  + \tfrac{1}{2} g_{\mu}^{\nu} F^{\alpha\beta} F_{\alpha\beta\nu}
  \qquad\text{by~\Eq{(26.3)}} \\
%
  &= -F_{\nu}^{\nu\alpha} F_{\mu\alpha}
  &&- \tfrac{1}{2} F^{\beta\alpha} F_{\mu\alpha\beta}
    - \tfrac{1}{2} F^{\alpha\beta} F_{\mu\beta\alpha}
    + \tfrac{1}{2} F^{\alpha\beta} F_{\alpha\beta\mu} \\
\intertext{by changes of dummy suffixes,}
  &= F_{\nu}^{\alpha\nu} F_{\mu\alpha}
  &&+ \tfrac{1}{2} F^{\alpha\beta}(F_{\mu\alpha\beta} + F_{\beta\mu\alpha} + F_{\alpha\beta\mu})
\end{alignat*}
by the antisymmetry of~$F^{\mu\nu}$.

It is easily verified that
\[
F_{\mu\alpha\beta} + F_{\beta\mu\alpha} + F_{\alpha\beta\mu}
= \frac{\dd F_{\mu\alpha}}{\dd x_{\beta}}
+ \frac{\dd F_{\beta\mu}}{\dd x_{\alpha}}
+ \frac{\dd F_{\alpha\beta}}{\dd x_{\mu}} = 0
\]
by \Eq{(30.3)} and~\Eq{(73.71)}; the terms containing the $3$-index symbols mutually
cancel.

Hence
\[
E_{\mu\nu}^{\nu} = F_{\nu}^{\alpha\nu} F_{\mu\alpha} = J^{\alpha} F_{\mu\alpha},
\]
agreeing with~\Eq{(77.1)}.

It is of interest to work out the components of the energy\hyp{}tensor \Eq{(77.2)}
\index{Stress\hyp{}system!electromagnetic}%
in Galilean coordinates by \Eq{(73.41)} and~\Eq{(73.42)}. We have
\begin{align*}
  &F^{\alpha\beta} F_{\alpha\beta}
  = 2(\alpha^2 + \beta^2 + \gamma^2 - X^{2} - Y^{2} - Z^{2}),
  \Tag{(77.3)} \\
  &E_{1}^{1} = \tfrac{1}{2}(\alpha^2 - \beta^2 - \gamma^2)
  + \tfrac{1}{2}(X^{2} - Y^{2} - Z^{2}),
  \Tag{(77.41)}\displaybreak[0] \\
  &E_{1}^{2} = \alpha\beta + XY,
  \Tag{(77.42)} \\
  &E_{1}^{4} = \beta Z - \gamma Y,
  \Tag{(77.43)}\displaybreak[0] \\
  &E_{4}^{4} = \tfrac{1}{2}(\alpha^2 + \beta^2 + \gamma^2)
  + \tfrac{1}{2}(X^{2} + Y^{2} + Z^{2}).
  \Tag{(77.44)}
\end{align*}
The last gives the energy or mass of the electromagnetic field; the third
\index{Mass!invariant and relative}%
\index{Mass!of electromagnetic field}%
expression gives the momentum; the first two give the stresses in the field.
\index{Momentum!electromagnetic}%
In all cases these formulae agree with those of the classical theory.

Momentum, being rate of flow of mass, is also the rate of flow of energy.
In the latter aspect it is often called Poynting's vector. It is seen from~\Eq{(77.43)}
\index{Poynting's vector}%
that the momentum is the vector\hyp{}product of the electric and magnetic forces---to
use the terminology of the elementary vector theory.

From $E_{\mu}^{\nu}$ we can form a scalar~$E$ by contraction, just as $T$~is formed from~$T_{\mu}^{\nu}$.
The invariant density~$T$ will be made up of the two parts $E$ and~$M$, the
former arising from the electromagnetic field and the latter from the matter
or non\hyp{}Maxwellian stresses involved in the electron. It turns out, however,
\index{Electron!non\hyp{}Maxwellian stresses in}%
that $E$~is identically zero, so that the electromagnetic field contributes nothing
to the invariant density. The invariant density must be attributed entirely
to the non\hyp{}Maxwellian binding stresses. Contracting~\Eq{(77.2)}
\[
E = -F^{\mu\alpha} F_{\mu\alpha} + \tfrac{1}{4} g_{\mu}^{\mu} F^{\alpha\beta} F_{\alpha\beta} = 0,
\Tag{(77.5)}
\]
since $g_{\mu}^{\mu} = 4$.

The question of the origin of the inertia of matter presents a very curious
\index{Inertia!electromagnetic origin of}%
paradox. We have to distinguish---
\begin{align*}
  &\text{the invariant mass~$m$ arising from the \PadTxt[l]{coordinate}{invariant} density~$T$, and} \\
\index{Invariant mass}%
  &\text{the \PadTxt[l]{invariant}{relative} mass~$M$ arising from the coordinate density~$T^{44}$.}
\end{align*}
As we have seen, the former cannot be attributed to the electromagnetic field.
But it is generally believed that the latter---which is the ordinary mass as
understood in physics---arises solely from the electromagnetic fields of the
electrons, the inertia of matter being simply the energy of the electromagnetic
fields contained in it. It is probable that this view, which arose in consequence
of J.~J. Thomson's researches\footnotemark,\footnotetext
  {\Title{Phil.\ Mag.}\ vol.~11 (1881), p.~229.}
is correct; so that ordinary or relative mass
may be regarded as entirely electromagnetic, whilst invariant mass is entirely
non\hyp{}electromagnetic.

How then does it happen that for an electron at rest, invariant mass and
relative mass are equal, and indeed synonymous?

Probably the distinction of Maxwellian and non\hyp{}Maxwellian stresses as
\index{Non\hyp{}Maxwellian stresses}%
\index{Stress\hyp{}system!non\hyp{}Maxwellian}%
tensors of different natures is artificial---like the distinction of gravitational
and inertial fields---and the real remedy is to remodel the electromagnetic
equations so as to comprehend both in an indissoluble connection. But so
long as we are ignorant of the laws obeyed by the non\hyp{}Maxwellian stresses, it
is scarcely possible to avoid making the separation. From the present point
of view we have to explain the paradox as follows---

Taking an electron at rest, the relative mass is determined solely by the
component~$E^{44}$; but the stress\hyp{}components of~$E^{\mu\nu}$ make a contribution to~$E$
which exactly cancels that of~$E^{44}$, so that $E = 0$. These stresses are balanced
by non\hyp{}Maxwellian stresses $M^{11}$, \dots, $M^{33}$; the balancing being not necessarily
exact in each element of volume, but exact for the region round the electron
taken as a whole. Thus the term which cancels~$E^{44}$ is itself cancelled, and $E^{44}$~becomes
reinstated. The final result is that the integral of~$T$ is equal to the
integral of~$E^{44}$ for the electron at rest.

It is usually assumed that the non\hyp{}Maxwellian stresses are confined to
the interior, or the close proximity, of the electrons, and do not wander about
in the detached way that the Maxwellian stresses do, e.g.\ in light-waves.
I shall adopt this view in order not to deviate too widely from other writers,
although I do not see any particular reason for believing it to be true\footnotemark.\footnotetext
  {We may evade the difficulty by extending the definition of electrons or matter to include all
  regions where Maxwell's equations are inadequate (e.g.\ regions containing quanta).}

If then all non\hyp{}Maxwellian stresses are closely bound to the electrons, it
follows that in regions containing no matter $E_{\mu}^{\nu}$~is the entire energy\hyp{}tensor.
Then \Eq{(54.3)}~becomes
\[
G_{\mu}^{\nu} - \tfrac{1}{2} g_{\mu}^{\nu} G = -8\pi E_{\mu}^{\nu}.
\Tag{(77.6)}
\]
Contracting,
\[
G = 8\pi E = 0,
\]
and the equation simplifies to
\[
G_{\mu\nu} = -8\pi E_{\mu\nu}
\Tag{(77.7)}
\]
for regions containing electromagnetic fields but no matter. We may notice
that the Gaussian curvature of space-time is zero even when electromagnetic
energy is present provided there are no electrons in the region.

Since for electromagnetic energy the invariant mass,~$m$, is zero, and the
relative mass,~$M$, is finite, the equation~\Eq{(12.3)}
\[
M = m \frac{dt}{ds}
\]
shows that $ds/dt$~is zero. Accordingly free electromagnetic energy must always
have the velocity of light.

\Section{78.}{The gravitational field of an electron}
\index{Electron!gravitational field of}%

This problem differs from that of the gravitational field of a particle (\SecRef{38})
\index{Gravitational field of a particle!of an electron}%
in that the electric field spreads through all space, and consequently the
energy\hyp{}tensor is not confined to a point or small sphere at the origin.

For the most general symmetrical field we take as before
\[
g_{11} = -e^{\lambda},\quad
g_{22} = -r^{2},\quad
g_{33} = -r^{2} \sin^{2}\theta,\quad
g_{44} = e^{\nu}.
\Tag{(78.1)}
\]
Since the electric field is static, we shall have
\[
F, G, H = \kappa_{1}, \kappa_{2}, \kappa_{3} = 0,
\]
and $\kappa_{4}$~will be a function of $r$~only. Hence the only surviving components of~$F_{\mu\nu}$
are
\[
F_{41} = -F_{14} = \kappa_{4}',
\Tag{(78.2)}
\]
the accent denoting differentiation with respect to~$r$. Then
\[
F^{41} = g^{44} g^{11} F_{41} = -e^{-(\lambda+\nu)} \kappa_{4}',
\]
and
\[
\mf{F}^{41} = F^{41} \sqrt{-g}
  = -e^{-\frac{1}{2}(\lambda+\nu)} r^{2} \sin\theta \cdot \kappa_{4}'.
\]
Hence by~\Eq{(73.75)} the condition for no electric charge and current (except at
the singularity at the origin) is
\[
\frac{\dd\mf{F}^{41}}{\dd x_{1}}
= -\sin\theta\, \frac{\dd}{\dd r}(e^{-\frac{1}{2}(\lambda+\nu)} r^{2}\kappa_{4}') = 0,
\Tag{(78.3)}
\]
so that
\[
\kappa_{4}' = \frac{\epsilon}{r^{2}}\, e^{\frac{1}{2}(\lambda+\nu)},
\Tag{(78.4)}
\]
where $\epsilon$~is a constant of integration.

Substituting in~\Eq{(77.2)} we find
\begin{align*}
  E_{1}^{1}
  = -E_{2}^{2}
  = -E_{3}^{3}
  = E_{4}^{4}
  &= \tfrac{1}{2} e^{-\lambda-\nu} \kappa_{4}'^{2} \\
  &= \frac{1}{2}\, \frac{\epsilon^{2}}{r^{4}}.
  \Tag{(78.5)}
\end{align*}
By \Eq{(77.7)} we have to substitute $-8\pi E_{\mu\nu}$ for zero on the right-hand side of
(\EqNo{38.61}--\EqNo{38.64}). The first and fourth equations give as before $\lambda' = -\nu'$; and the
second equation now becomes
\begin{align*}
  e^{\nu} (1 + r\nu') - 1
  &= -8\pi g_{22} E_{2}^{2} \\
  &= -4\pi \epsilon^{2}/r^{2}.
\end{align*}
Hence writing $e^{\nu} = \gamma$,
\[
\gamma + r\gamma' = 1 - 4\pi \epsilon^{2}/r^{2},
\]
so that
\[
r\gamma = r + 4\pi \epsilon^{2}/r - 2m,
\]
where $2m$~is a constant of integration.

Hence the gravitational field due to an electron is given by
\[
ds^{2} = -\gamma^{-1}\, dr^{2} - r^{2}\, d\theta^2 - r^{2}\sin^{2}\theta\, d\phi^{2} + \gamma\, dt^{2},
\]
with
\[
\gamma = 1 - \frac{2m}{r} + \frac{4\pi\epsilon^{2}}{r^{2}}.
\Tag{(78.6)}
\]

This result appears to have been first given by Nordström. I have here
followed the solution as given by G.~B. Jeffery\footnotemark.\footnotetext
  {\Title{Proc.\ Roy.\ Soc.}\ vol.~99\Vol{A}, p.~123.}

The effect of the term $4\pi\epsilon^{2}/r^{2}$ is that the effective mass decreases as $r$~decreases.
This is what we should naturally expect because the mass or energy
is spread throughout space. We cannot put the constant~$m$ equal to zero,
because that would leave a \emph{repulsive} force on an uncharged particle varying
as the inverse cube of the distance; by \Eq{(55.8)} the approximate Newtonian
potential is $m/r - 2\pi\epsilon^{2}/r^{2}$.

The constant~$m$ can be identified with the mass and $4\pi\epsilon$ with the electric
charge of the particle. The known experimental values for the negative
electron are
\begin{align*}
m &= 7 \cdot 10^{-56} \text{ cm.}, \\
a &= \frac{2\pi\epsilon^{2}}{m} = 1.5 \cdot 10^{-13} \text{ cm.}
\end{align*}
The quantity~$a$ is usually considered to be of the order of magnitude of the
radius of the electron, so that at all points outside the electron $m/r$~is of order
$10^{-40}$ or smaller. Since $\lambda + \nu = 0$, \Eq{(78.4)}~becomes
\[
F_{41} = \kappa_{4}' = \frac{\epsilon}{r^{2}},
\]
which justifies our identification of~$4\pi\epsilon$ with the electric charge.

This example shows how very slight is the gravitational effect of the
electronic energy. We can discuss most electromagnetic problems without
taking account of the non\hyp{}Euclidean character which an electromagnetic field
necessarily imparts to space-time, the deviations from Euclidean geometry
being usually so small as to be negligible in the cases we have to consider.

When $r$~is diminished the value of~$\gamma$ given by~\Eq{(78.6)} decreases to a minimum
for $r = 2a$, and then increases continually becoming infinite at $r = 0$. There is
no singularity in the electromagnetic and gravitational fields except at $r = 0$.
It is thus possible to have an electron which is strictly a point\hyp{}singularity,
but nevertheless has a finite mass and charge.

The solution for the gravitational field of an uncharged particle is quite
different in this respect. There is a singularity at $r = 2m$, so that the particle
must have a finite perimeter not less than~$4\pi m$. Moreover this singularity is
caused by $\gamma$~vanishing, whereas for the point\hyp{}electron the singularity is due
\index{Point\hyp{}electron}%
to $\gamma$~becoming infinite.

This demonstration that a point\hyp{}electron may have exactly the properties
which electrons are observed to have is a useful corrective to the general belief
that the radius of an electron is known with \emph{certainty}. But on the whole,
I think that it is more likely that an electron is a structure of finite size; our
solution will then only be valid until we enter the substance of the electron,
so that the question of a singularity at the origin does not arise.

Assuming that we do not encounter the substance of the electron outside
the sphere $r = a$, the total energy of the electromagnetic field beyond this
radius would be equal to the mass of the electron determined by observation.
For this reason $a$~is usually taken as the radius of the electron. If it is
admitted that the electromagnetic field continues undisturbed within this
limit, an excess of energy accumulates, and it is therefore necessary to suppose
that there exists negative energy in the inner portion, or that the effect of
the singularity is equivalent to a negative energy. The conception of negative
energy is not very welcome according to the usual outlook.

Another reason for believing that the charge of an electron is distributed
through a volume of radius roughly equal to~$a$ will be found in the investigation
of \SecRef{80}. Accordingly I am of opinion that the point\hyp{}electron is no more
than a mathematical curiosity, and that the solution~\Eq{(78.6)} should be limited
to values of~$r$ greater than~$a$.

\Section{79.}{Electromagnetic action}
\index{Action, material or gravitational!electromagnetic}%
\index{Electromagnetic action}%
\index{Hamiltonian derivative!of electromagnetic action}%

The invariant integral
\[
A = \tfrac{1}{4} \int F^{\mu\nu} F_{\mu\nu} \sqrt{-g}\, d\tau
\Tag{(79.1)}
\]
is called the action of the electromagnetic field. In Galilean coordinates it
becomes by~\Eq{(77.3)}
\[
\int dt \iiint \tfrac{1}{2}(\alpha^2 + \beta^2 + \gamma^2 - X^{2} - Y^{2} - Z^{2})\, dx\, dy\, dz.
\Tag{(79.2)}
\]
Regarding the magnetic energy as kinetic~($T$) and the electric energy as
potential~($V$) this is of the form
\[
\int (T - V)\, dt,
\]
i.e.\ the time\hyp{}integral of the Lagrangian function\footnotemark.\footnotetext
  {In dynamics there are two integrals which have the stationary property under proper restrictions,
  viz.\ $\int T\, dt$ and $\int (T - V)\, dt$. The first of these is the action as originally defined. In the
  general theory the term has been applied to both integrals somewhat indiscriminately, since there
  is no clear indication of energy which must be reckoned as potential.}
The derivation of the
electromagnetic equations by the stationary variation of this integral has been
investigated in the classical researches of Larmor\footnotemark.\footnotetext
  {\Title{Aether and Matter,} Chapter~\Vol{VI}.}

We shall now show that the two most important electromagnetic tensors,
viz.\ the energy\hyp{}tensor~$E^{\mu\nu}$ and the charge\hyp{}and\hyp{}current vector~$J^{\mu}$, are the
Hamiltonian derivatives of the action, the formulae being
\begin{align*}
  \frac{\Ham}{\Ham g_{\mu\nu}} (\tfrac{1}{4}F^{\mu\nu} F_{\mu\nu})
  &= \tfrac{1}{2} E^{\mu\nu},
  \Tag{(79.31)} \\
  \frac{\Ham}{\Ham \kappa_{\mu}} (\tfrac{1}{4}F^{\mu\nu} F_{\mu\nu})
  &= -J^{\mu}.
  \Tag{(79.32)}
\end{align*}

First consider small variations~$\delta g_{\mu\nu}$, the~$\kappa_{\mu}$ remaining constant. The~$F_{\mu\nu}$
(but not the~$F^{\mu\nu}$) will accordingly remain unvaried. We have then
\begin{multline*}
  \delta (F^{\mu\nu} F_{\mu\nu} \sqrt{-g}) \\
  \begin{aligned}
  &= F^{\mu\nu} F_{\mu\nu}\, \delta(\sqrt{-g})
  + F_{\alpha\beta} F_{\mu\nu} \sqrt{-g} \cdot \delta(g^{\mu\alpha} g^{\nu\beta}) \\
%
  &= F^{\sigma\tau} F_{\sigma\tau} \sqrt{-g} \cdot \frac{1}{2}\, \frac{\delta g}{g}
  + F_{\alpha\beta} F_{\mu\nu} \sqrt{-g} (g^{\mu\alpha}\, \delta g^{\nu\beta} +  g^{\nu\beta}\, \delta g^{\mu\alpha}) \\
  &= \sqrt{-g} \{-\tfrac{1}{2} F^{\sigma\tau} F_{\sigma\tau} g_{\nu\beta}\, \delta g^{\nu\beta} + 2F_{\alpha\beta} F_{\mu\nu} g^{\mu\alpha}\, \delta g^{\nu\beta}\} \\
  &= 2\sqrt{-g} \cdot \delta g^{\nu\beta} \{-\tfrac{1}{4} g_{\nu\beta} F^{\sigma\tau} F_{\sigma\tau} + {F^{\mu}}_{\beta} F_{\mu\nu}\} \\
  &= -2E_{\nu\beta} \sqrt{-g} \cdot \delta g^{\nu\beta} \qquad\text{by~\Eq{(77.2)}} \\
  &= \Neg2E^{\nu\beta} \sqrt{-g} \cdot \delta g_{\nu\beta} \qquad\text{by~\Eq{(35.2)}.}
  \end{aligned}
\end{multline*}
From this \Eq{(79.31)}~follows immediately.

Next consider variations~$\delta\kappa_{\mu}$, the $g_{\mu\nu}$~remaining constant. We have
\begin{align*}
  \delta (F^{\mu\nu} F_{\mu\nu} \sqrt{-g})
  &= 2F^{\mu\nu} \sqrt{-g} \cdot \delta F_{\mu\nu} \\
  &= 2F^{\mu\nu} \sqrt{-g} \left(\frac{\dd(\delta\kappa_{\mu})}{\dd x_{\nu}} - \frac{\dd(\delta\kappa_{\nu})}{\dd x_{\mu}}\right) \\
  &= 4F^{\mu\nu} \sqrt{-g} \cdot \frac{\dd(\delta\kappa_{\mu})}{\dd x_{\nu}}
\intertext{owing to the antisymmetry of~$F^{\mu\nu}$}
  &= -4\, \frac{\dd}{\dd x_{\nu}} (F^{\mu\nu} \sqrt{-g})\, \delta\kappa_{\mu} + 4\, \frac{\dd}{\dd x_{\nu}} (F^{\mu\nu} \sqrt{-g} \cdot \delta\kappa_{\mu}).
\end{align*}

The second term can be omitted since it is a complete differential, and
yields a surface\hyp{}integral over the boundary where the variations have to vanish.
Hence
\begin{align*}
  \delta \int F^{\mu\nu} F_{\mu\nu} \sqrt{-g}\, d\tau
  &= -4\int \frac{\dd}{\dd x_{\nu}} (F^{\mu\nu} \sqrt{-g}) \cdot \delta\kappa_{\mu}\, d\tau \\
  &= -4\int J^{\mu}\, \delta\kappa_{\mu} \cdot \sqrt{-g}\, d\tau
\end{align*}
by~\Eq{(73.75)}. This demonstrates~\Eq{(79.32)}.

In a region free from electrons
\[
T^{\mu\nu} - E^{\mu\nu} = 0.
\]
Hence by \Eq{(60.43)} and~\Eq{(79.31)}
\[
\frac{\Ham}{\Ham g_{\mu\nu}} (G - 4\pi F^{\mu\nu} F_{\mu\nu}) = 0.
\Tag{(79.4)}
\]

In the mechanical theory, neglecting electromagnetic fields, we found that
the action~$G$ was stationary in regions containing no matter. We now see
that when electromagnetic fields are included, the quantity which is stationary
is $G - 4\pi F^{\mu\nu} F_{\mu\nu}$. Moreover it is stationary for variations~$\delta\kappa_{\mu}$ as well as~$\delta g_{\mu\nu}$,
since when there are no electrons present $J^{\mu}$~must be zero.

The quantity $G - 4\pi F^{\mu\nu} F_{\mu\nu}$ thus appears to be highly significant from the
physical point of view, in the discrimination between matter (electrons) and
electromagnetic fields. But this significance fails to appear in the analytical
expression. Analytically the combination of the two invariants $G$ and $F^{\mu\nu} F_{\mu\nu}$---the
one a spur, and the other a square of a length---appears to be quite
nonsensical. We can only regard the present form of the expression as a
stepping\hyp{}stone to something simpler. It will appear later that $G - 4\pi F^{\mu\nu} F_{\mu\nu}$
is perhaps not the exact expression for the significant physical quantity; it
may be an approximation to a form which is analytically simpler, in which
the gravitational and electromagnetic variables appear in a more intelligible
combination.

Whereas material and gravitational actions are two aspects of the same
thing, electromagnetic action stands entirely apart. There is no gravitational
action associated with an electromagnetic field, owing to the identity $E = 0$.
Thus any material or gravitational action is additional to electromagnetic
action---if ``addition'' is appropriate in connection with quantities which are
apparently of dissimilar nature.

\Section{80.}{Explanation of the mechanical force}
\index{Force!mechanical force due to}%
\index{Mechanical force of electromagnetic field!explanation of}%

Why does a charged particle move when it is placed in an electromagnetic
field? We may be tempted to reply that the reason is obvious; there is an
electric force lying in wait, and it is the nature of a force to make bodies
move. But this is a confusion of terminology; electric force is not a force in
the mechanical sense of the term; it has nothing to do with pushing and
pulling. Electric force describes a world\hyp{}condition essentially different from
that described by a mechanical force or stress\hyp{}system; and the discussion in
\SecRef{76} was based on empirical laws without theoretical explanation.

If we wish for a representation of the state of the aether in terms of
mechanical forces, we must employ the stress\hyp{}system (\EqNo{77.41}, \EqNo{77.42}). In fact
the pulling and pushing property is described by the tensor~$E_{\mu\nu}$ not by~$F_{\mu\nu}$.
Our problem is to explain why a somewhat arbitrary combination of the
electromagnetic variables~$F_{\mu\nu}$ should have the properties of a mechanical
stress\hyp{}system.

To reduce the problem to its simplest form we consider an isolated electron.
\index{Acceleration of light\hyp{}pulse!of charged particle}%
\index{Electron!acceleration in electromagnetic field}%
In an electromagnetic field its world-line does not follow a geodesic, but
deviates according to laws which have been determined experimentally. It is
worth noticing that the behaviour of an isolated electron has been directly
determined by experiment, this being one of the few cases in which microscopic
laws have been found immediately and not inferred hypothetically from
macroscopic experiments. We want to know what the electron is trying to
accomplish by deviating from the geodesic---what condition of existence is
fulfilled, which makes the four\hyp{}dimensional structure of an accelerated electron
a possible one, whereas a similar structure ranged along a geodesic track would
be an impossible one.

The law which has to be explained is\footnote
  {In this and a succeeding equation I have a \emph{quantity} on the left-hand side and a \emph{density} on
  the right-hand side. I trust to the reader to amend this mentally. It would, I think, only make
  the equations more confusing if I attempted to indicate the amendment symbolically.}
\[
-m \left\{\frac{d^{2}x_{\mu}}{ds^{2}}
+ \{\alpha\beta, \mu\}\, \frac{dx_{\alpha}}{ds}\, \frac{dx_{\beta}}{ds}\right\}
= h^{\mu} = {F^{\mu}}_{\nu} J^{\nu},
\Tag{(80.1)}
\]
which is the tensor equation corresponding to the law of elementary electrostatics
\[
m\, \frac{dx^{2}}{dt^{2}} = Xe.
\]

Let $A^{\mu}$~be the velocity\hyp{}vector of the electron ($A^{\mu} = dx_{\mu}/ds$), and $\rho_{0}$~the
proper\hyp{}density of the charge, then by~\Eq{(73.82)}
\[
J^{\mu} = \rho_{0} A^{\mu},
\Tag{(80.21)}
\]
and
\[
\frac{d^{2}x_{\mu}}{ds^{2}}
+ \{\alpha\beta, \mu\}\, \frac{dx_{\alpha}}{ds}\, \frac{dx_{\beta}}{ds}
= A^{\nu} (A^{\mu})_{\nu},
\Tag{(80.22)}
\]
as in~\Eq{(33.4)}.

Considering the verification of~\Eq{(80.1)} by experiment we remark that $X$ or
$F_{\mu\nu}$ refers to the applied external field, no attention being paid to the possible
disturbance of this field caused by the accelerated electron itself. To distinguish
this we denote the external field by~${F'}_{\mu\nu}$. The equation to be explained
accordingly becomes
\[
mA^{\nu}(A^{\mu})_{\nu} = -{F^{\mu}}_{\nu} (\rho_{0} A^{\nu}),
\]
or, lowering the suffix~$\mu$,
\[
mA^{\nu} A_{\mu\nu} = -{F'^{\mu}}_{\nu} eA^{\nu}.
\Tag{(80.3)}
\]
We have replaced the \emph{density}~$\rho_{0}$ by the \emph{quantity}~$e$ for the reason explained in
the footnote.

Consider now the field due to the electron itself in its own neighbourhood.
This is determined by~\Eq{(74.41)}
\[
\Wave F_{\mu\nu} = J_{\mu\nu} - J_{\nu\mu}
- G_{\mu}^{\epsilon} F_{\epsilon\nu} + G_{\nu}^{\epsilon} F_{\epsilon\mu}
+ 2B_{\mu\nu\alpha\epsilon} F^{\epsilon\alpha}.
\]
The discussion of \SecRef{78} shows that we may safely neglect the gravitational
field caused by the energy of the electron or of the external field. Hence
approximately
\[
\Wave F_{\mu\nu} = J_{\mu\nu} - J_{\nu\mu}.
\]
The solution is as in~\Eq{(74.72)}
\begin{align*}
  F_{\mu\nu} &= \int \frac{de\, (A_{\mu\nu} - A_{\nu\mu})}{4\pi\beta r} \\
  &= \frac{1}{4\pi\beta}(A_{\mu\nu} - A_{\nu\mu}) \int \frac{de}{r},
  \Tag{(80.4)}
\end{align*}
if all parts of the electron have the same velocity~$A^{\mu}$. This result is obtained
primarily for Galilean coordinates; but it is a tensor equation applying to
all coordinate\hyp{}systems provided that $\int de/r$~is treated as an invariant and
calculated in natural measure. We shall reckon it in proper\hyp{}measure and
accordingly drop the factor~$\beta$.

Now suppose that the electron moves in such a way that its own field
on the average just neutralises the applied external field~${F'}_{\mu\nu}$ in the region
occupied by the electron. The value of~$F_{\mu\nu}$ averaged for all the elements of
charge constituting the electron is given by
\begin{align*}
  eF_{\mu\nu}
  &= \frac{1}{4\pi}(A_{\mu\nu} - A_{\nu\mu}) \iint \frac{de_{1}\, de_{2}}{r_{12}} \\
  &= \frac{1}{4\pi}(A_{\mu\nu} - A_{\nu\mu}) \frac{e^{2}}{a},
\end{align*}
where $1/a$~is an average value of~$1/r_{12}$ for every pair of points in the electron.
We may leave indeterminate the exact weighting of the pairs of points in
taking the average, merely noting that $a$~will be a length comparable with
the radius of the sphere throughout which the charge (or the greater part of
it) is spread.

If this value of~$F_{\mu\nu}$ is equal and opposite to~${F'}_{\mu\nu}$, we have
\begin{align*}
  -eA^{\nu} {F'}_{\mu\nu}
  &= \frac{1}{4\pi} A^{\nu} (A_{\mu\nu} - A_{\nu\mu}) \frac{e^{2}}{a} \\
  &= A^{\nu} A_{\mu\nu} \cdot \frac{e^{2}}{4\pi a},
  \Tag{(80.5)}
\end{align*}
because
\[
A^{\nu} A_{\nu\mu} = A_{\nu} (A^{\nu})_{\mu}
= \tfrac{1}{2} (A_{\nu} A^{\nu})_{\mu}
= \tfrac{1}{2} (1)_{\mu}
= 0,
\]
the square of the length of a velocity\hyp{}vector being necessarily unity.

The result~\Eq{(80.5)} will agree with~\Eq{(80.3)} if the mass of the electron is
\[
m = \frac{e^{2}}{4\pi a}.
\Tag{(80.6)}
\]

The observed law of motion of the electron thus corresponds to the condition
that it can be under no resultant electromagnetic field. We must not
imagine that a resultant electromagnetic force has anything of a tugging
nature that can deflect an electron. It never gets the chance of doing anything
to the electron, because if the resultant field existed the electron could not
exist---it would be an impossible structure.

The interest of this discussion is that it has led us to one of the conditions
for the existence of an electron, which turns out to be of a simple character---viz.\
that on the average the electromagnetic force throughout the electron
must be zero\footnotemark.\footnotetext
  {The exact region of zero force is not determined. The essential point is that on some critical
  surface or volume the field has to be symmetrical enough to give no resultant.}
This condition is clearly fulfilled for a symmetrical electron
at rest in no field of force; and the same condition applied generally leads to
the law of motion~\Eq{(80.1)}.

For the existence of an electron, non\hyp{}Maxwellian stresses are necessary,
and we are not yet in a position to state the laws of these additional stresses.
The existence of an electron contradicts the electromagnetic laws with which
we have to work at present, so that from the present standpoint an electron
at rest in no external field of force is a \emph{miracle}. Our calculation shows that an
electron in an external field of force having the acceleration given by~\Eq{(80.1)} is
\index{Electron!size of}%
\emph{precisely the same miracle}. That is as far as the explanation goes.

The electromagnetic field within the electron will vanish on the average
\index{Acceleration of light\hyp{}pulse!determined by symmetrical condition}%
if it has sufficient symmetry. There appears to be an analogy between this
\index{Symmetry!of an electron}%
and the condition which we found in \SecRef{56} to be necessary for the existence of
a particle, viz.\ that its gravitational field should have symmetrical properties.
There is further an analogy in the condition determining the acceleration in
the two cases. An uncharged undisturbed body takes such a course that
relative to it there is no resultant gravitational field; similarly an electron
takes such a course that relative to it there is no resultant electromagnetic
field. We have given a definite reason for the gravitational symmetry of a
particle, viz.\ because in practical measurement it is itself the standard of
symmetry; I presume that there is an analogous explanation of the electrical
symmetry of an electron, but it has not yet been formulated. The following
argument (which should be compared with \SecRefs{64}, \SecNum{66}) will show where the
difficulty occurs.

The analogue of the interval is the flux $F_{\mu\nu}\, dS^{\mu\nu}$. As the interval between
\index{Flux!electromagnetic}%
two adjacent points is the fundamental invariant of mechanics, so the flux
through a small surface is the fundamental invariant of electromagnetism.
Two electrical systems will be alike observationally if, and only if, all corresponding
fluxes are equal. Equality of flux can thus be tested absolutely; and
different fluxes can be measured (according to a conventional code) by apparatus
constituted with electrical material. From the flux we can pass by mathematical
processes to the charge\hyp{}and\hyp{}current vector, and this enables us to make
the second contact between mathematical theory and the actual world, viz.\ the
identification of electricity. We should now complete the cycle by showing
that with electricity so defined apparatus can be constructed which will measure
the original flux. Here, however, the analogy breaks down, at least temporarily.
The use of electricity for measuring electromagnetic fluxes requires discontinuity,
but this discontinuity is obtained in practice by complicated conditions
such as insulation, constant contact differences of potential, etc. We do not
seem able to reduce the theory of electrical measurement to direct dependence
on an innate discontinuity of electrical charge in the same way that geometrical
measurement depends on the discontinuity of matter. For this reason the last
chain of the cycle is incomplete, and it does not seem permissible to deduce
that the discontinuous unit of electric charge must become the standard of
electrical symmetry in the same way that the discontinuous unit of matter
(turned in different orientations) becomes the standard of geometrical symmetry.

According to~\Eq{(80.6)} the mass of the electron is~$e^{2}/4\pi a$, where $a$~is a length
comparable with the radius of the electron. This is in conformity with the
usual view as to the size of an electron, and is opposed to the point\hyp{}electron
suggested in \SecRef{78} as an alternative. But the mass here considered is a purely
electromagnetic constant, which only enters into equations in which electromagnetic
forces are concerned. When the right-hand side of~\Eq{(80.1)} vanishes,
the electron describes a geodesic just as an uncharged particle would; but
$m$~is now merely a constant multiplier which can be removed. We have still to
find the connection between this electromagnetic mass
\index{Mass!electromagnetic}%
\[
m_{e} = e^{2}/4\pi a
\Tag{(80.71)}
\]
and the gravitational (i.e.\ gravitation\hyp{}producing) mass~$m_{g}$, given by
\[
m_{g}\, ds = \frac{1}{8\pi} \int G\sqrt{-g}\, d\tau.
\Tag{(80.72)}
\]

Since we believe that all negative electrons are precisely alike, $m_{g}/m_{e}$ will
be a constant for the negative electron; similarly it will be a constant for the
positive electron. But positive and negative electrons are structures of very
different kinds, and it does not follow that $m_{g}/m_{e}$~is the same for both. As a
matter of fact there is no experimental evidence which suggests that the ratio
is the same for both. Any gravitational field perceptible to observation is
caused by practically equal numbers of positive and negative electrons, so that
no opportunity of distinguishing their contributions occurs. If, however, we
admit that the principle of conservation of energy is universally valid in cases
where the positive and negative electrons are separated to an extent never
yet realised experimentally, it is possible to prove that $m_{g}/m_{e}$~is the same for
both kinds.

From the equation~\Eq{(80.1)} we deduce the value of the electromagnetic
energy\hyp{}tensor as in \SecRefs{76},~\SecNum{77}; only, $E^{\mu\nu}$~will not be expressed in the same
units as the whole energy\hyp{}tensor $G_{\mu}^{\nu} - \frac{1}{2} g_{\mu}^{\nu} G$, since the mass appearing in~\Eq{(80.1)}
is~$m_{e}$ instead of~$m_{g}$. In consequence, the law for empty space~\Eq{(77.6)} must be
written
\[
G_{\mu}^{\nu} - \tfrac{1}{2} g_{\mu}^{\nu} G
= -8\pi\, \frac{m_{g}}{m_{e}} (-F^{\nu\alpha} F_{\mu\alpha} + \tfrac{1}{4}g_{\mu}^{\nu} F^{\alpha\beta} F_{\alpha\beta}).
\Tag{(80.8)}
\]
We can establish this equation firstly by considering the motion of a positive
electron and secondly by considering a negative electron. Evidently we shall
obtain inconsistent equations in the two cases unless $m_{g}/m_{e}$ for the positive
electron is the same as for the negative electron. Unless this condition is fulfilled,
we should violate the law of conservation of energy and momentum by
first converting kinetic energy of a negative electron into free electromagnetic
energy and then reconverting the free energy into kinetic energy of a positive
electron.

Accordingly $m_{g}/m_{e}$~is a constant of nature and it may be absorbed in
equation~\Eq{(80.8)} by properly choosing the unit of~$F_{\mu\nu}$.

\Section{81.}{Electromagnetic volume}

If $a_{\mu\nu}$~is any tensor, the determinant~$|a_{\mu\nu}|$ is transformed according to the
law
\[
|a_{\mu\nu}| = J^{2} |a_{\mu\nu}'|
\]
by~\Eq{(48.8)}, whence it follows as in~\Eq{(49.3)} that
\[
\int \sqrt{|a_{\mu\nu}|}\, d\tau
\Tag{(81.1)}
\]
for any four\hyp{}dimensional region is an invariant.

We have already considered the case a $a_{\mu\nu} = g_{\mu\nu}$, and it is natural now to
consider the case $a_{\mu\nu} = F_{\mu\nu}$. Since the tensor~$g_{\mu\nu}$ defines the metric of space-time,
and the corresponding invariant is the metrical volume (natural volume)
\index{Electromagnetic action!volume}%
\index{Volume!electromagnetic}%
of the region, it seems appropriate to call the invariant
\[
V_{e} = \int \sqrt{|F_{\mu\nu}|}\, d\tau
\Tag{(81.2)}
\]
the electromagnetic volume of the region. The resemblance to metrical volume
is purely analytical.

Since $|F_{\mu\nu}|$~is a skew\hyp{}symmetric determinant of even order, it is a perfect
square, and \Eq{(81.2)}~is rational. It easily reduces to
\[
V_{e} = \int (F_{23} F_{14} + F_{31} F_{24} + F_{12} F_{34})\, d\tau.
\Tag{(81.31)}
\]
In Galilean coordinates this becomes
\[
V_{e} = \int (\alpha X + \beta Y + \gamma Z)\, d\tau.
\Tag{(81.32)}
\]

It is somewhat curious that the scalar\hyp{}product of the electric and magnetic
forces is of so little importance in the classical theory, for \Eq{(81.32)} would seem
to be the most fundamental invariant of the field. Apart from the fact that
it vanishes for electromagnetic waves propagated in the absence of any bound
electric field (i.e.\ remote from electrons), this invariant seems to have no significant
properties. Perhaps it may turn out to have greater importance when
the study of electron\hyp{}structure is more advanced.

From~\Eq{(81.31)} we have
\begin{align*}
  V_{e}
  &= \int \sum \left(\frac{\dd\kappa_{1}}{\dd x_{4}}\, \frac{\dd\kappa_{2}}{\dd x_{3}}
  - \frac{\dd\kappa_{1}}{\dd x_{3}}\, \frac{\dd\kappa_{2}}{\dd x_{4}}\right) d\tau
\intertext{the summation being for all permutations of the suffixes}
  &= \int \sum \left\{\frac{\dd}{\dd x_{4}} \left(\kappa_{1}\, \frac{\dd\kappa_{2}}{\dd x_{3}}\right)
  - \frac{\dd}{\dd x_{3}} \left(\kappa_{1}\, \frac{\dd\kappa_{2}}{\dd x_{4}}\right)\right\} d\tau.
\end{align*}
Hence $V_{e}$~reduces to a surface\hyp{}integral over the boundary of the region, and
it is useless to consider its variations by the Hamiltonian method. The electromagnetic
volume of a region is of the nature of a flux through its three\hyp{}dimensional
boundary.

\Section{82.}{Macroscopic equations}
\index{Macroscopic electromagnetic equations}%

For macroscopic treatment the distribution and motion of the electrons are
averaged, and the equivalent continuous distribution is described by two new
quantities
\begin{align*}
&\text{the electric displacement, $P$, $Q$, $R$,} \\
&\text{the magnetic induction, $a$, $b$, $c$,}
\end{align*}
in addition to
\begin{align*}
&\text{the electric force, $X$, $Y$, $Z$,} \\
  &\text{the magnetic force, $\alpha$, $\beta$, $\gamma$.}
\end{align*}
These are grouped cross-wise to form the two principal electromagnetic tensors
\[
F_{\mu\nu} = \begin{array}[t]{@{}r@{\quad}r@{\quad}r@{\quad}r@{}}
   0 & -c &  b & -X,\\
   c &  0 & -a & -Y\\
  -b &  a &  0 & -Z\\
   X &  Y &  Z &  0\\
\end{array}\qquad
H^{\mu\nu} = \begin{array}[t]{@{}r@{\quad}r@{\quad}r@{\quad}r@{}}
  0 & -\gamma & \beta & \Neg P\\
  \gamma & 0 & -\alpha & Q\\
  -\beta & \alpha & 0 & R\\
  -P & -Q & -R & 0\\
\end{array}
\Tag{(82.1)}
\]
$H^{\mu\nu}$~now plays the part previously taken by~$F^{\mu\nu}$; but it is no longer derived
from~$F_{\mu\nu}$ by a mere raising of suffixes. The relation between the two tensors
is given by the constitutive equations of the material; in simple cases it is
\index{Constitutive equations}%
specified by two constants, the specific inductive capacity~$\kappa$ and the permeability~$\mu$.
\index{Permeability, magnetic}%

Equations \Eq{(73.73)} and~\Eq{(73.74)} are replaced by
\[
\left.
\begin{aligned}
  F_{\mu\nu} &= \frac{\dd\kappa_{\mu}}{\dd x_{\nu}} - \frac{\dd\kappa_{\nu}}{\dd x_{\mu}}\\
  H_{\nu}^{\mu\nu} &= J^{\mu}
\end{aligned}
\right\}
\Tag{(82.2)}
\]
These represent the usual equations of the classical theory. It should be
noticed that $\dd H/\dd y - \dd G/\dd z$ is now~$a$, not~$\alpha$.

In the simple case the constitutive equations are
\[
(P, Q, R) = K(X, Y, Z);\quad (a, b, c) = \mu(\alpha, \beta, \gamma),
\Tag{(82.3)}
\]
so that
%[** TN: Not broken in the original]
\begin{gather*}
H^{11}, H^{12}, \dots, H^{33}
= \frac{1}{\mu}(F^{11}, F^{12}, \dots, F^{33});\displaybreak[0] \\
H^{14}, H^{24}, H^{34} = K(F^{14}, F^{24}, F^{34}).
\end{gather*}
These simplified equations are not of tensor form, and refer only to coordinates
with respect to which the material is at rest. For general coordinates the
constitutive equations must be of the form
\[
H^{\mu\nu} = p^{\mu\alpha} p^{\nu\beta} F_{\alpha\beta},
\]
where $p^{\mu\nu}$~is a tensor.

The law of conservation of electric charge can be deduced from $H_{\nu}^{\mu\nu} = J^{\mu}$
just as in~\Eq{(73.76)}.

The macroscopic method is introduced for practical purposes rather than
as a contribution to the theory, and there seems to be no advantage in developing
it further here. The chief theoretical interest lies in the suggestion
of a possible generalisation of Maxwell's theory by admitting that the covariant
and contravariant electromagnetic tensors may in certain circumstances be
independent tensors, e.g.\ inside the electron. This is the basis of a theory of
matter developed by G.~Mie.

For a further discussion of the macroscopic electromagnetic equations
when the constitutive equations are not isotropic, see
\Title{T.~de~ Donder, Gomptes Rendus, 9 July 1923}.
It will be noticed that de Donder introduces quantities~$(\bar{F}_{\mu\nu})$
which appear to be intermediate between~$F_{\mu\nu}$ and~$H_{\mu\nu}$ as defined in this section.
This is due to a difficulty which arises in obtaining an action\hyp{}density and energy\hyp{}tensor.
The natural form of action\hyp{}density~$F_{\mu\nu}\mf{H}^{\mu\nu}$ can be associated with an energy\hyp{}tensor
\[
-F_{\mu\alpha}H^{\nu\alpha} + \tfrac{1}{4}g^\nu_\mu F_{\alpha\beta}H^{\alpha\beta}
\]
analogous to~\Eq{(77.2)};
but the divergence of this is not equal to the expected mechanical force~$F_{\mu\nu}H^{\nu\sigma}_\sigma$
or~$F_{\mu\nu}J^\mu$.
On the other hand, using the intermediate variable, the action\hyp{}density takes the symmetrical
form~$\bar{F}_{\mu\nu}\bar{\mf{F}}^{\mu\nu}$, and the derivation of a mechanical
force~$\bar{F}_{\mu\nu}\bar{F}^{\nu\sigma}_\sigma$ proceeds without difficulty.

The macroscopically continuous variables are in any case fictitious, and we have a certain amount of choice in
selecting the ``fictions'' which shall be regarded as homologous with the variables originally defined for
empty space.
In Maxwell's original theory the electric displacement is an essentially distinct quantity from the electric force
and the effect of the polarisation of the medium is to alter the constants in the equations connecting them.
The more modern view is that displacement and force are essentially the same quantity, and the polarisation of
the medium increases the average force and displacement alike.
De Donder's treatment corresponds to the latter view and it presumably represents a closer approximation to
the actual microscopic processes.
The chief interest in preserving the older notation in this section is in connection with Mie's theory which
resurrects the antiquated notion of independence of displacement and force as a possibly fruitful conception in
connection with the structure of the electron.

\Chapter{VII}{World Geometry}

\Part{I.}{Weyl's Theory}

\Section{83.}{Natural geometry and world geometry}
\index{Natural coordinates!geometry}%

\lettrine{\textcolor{lettrinecolour}{G}}{raphical} representation is a device commonly employed in dealing with
\index{Graphical representation}%
all kinds of physical quantities. It is most often used when we wish to set
before ourselves a mass of information in such a way that the eye can take it
in at a glance; but this is not the only use. We do not always draw the graphs
on a sheet of paper; the method is also serviceable when the representation
is in a conceptual mathematical space of any number of dimensions and possibly
non\hyp{}Euclidean geometry. One great advantage is that when the graphical
representation has been made, an extensive geometrical nomenclature becomes
available for description---straight line, gradient, curvature, etc.---and a self\hyp{}explanatory
nomenclature is a considerable aid in discussing an abstruse
subject.

It is therefore reasonable to seek enlightenment by giving a graphical
representation to all the physical quantities with which we have to deal. In
this way physics becomes geometrised. But graphical representation does not
assume any hypothesis as to the ultimate nature of the quantities represented.
The possibility of exhibiting the whole world of physics in a unified geometrical
representation is a test not of the nature of the world but of the ingenuity of
the mathematician.

There is no special rule for representing physical quantities such as electric
force, potential, temperature, etc.; we may draw the isotherms as straight
lines, ellipses, spheres, according to convenience of illustration. But there are
certain physical quantities (i.e.\ results of operations and calculations) which
have a natural graphical representation; we habitually think of them graphically,
and are almost unconscious that there is anything conventional in the
way we represent them. For example, measured distances and directions are
instinctively conceived by us graphically; and the space in which we represent
them is for us \emph{actual space}. These quantities are not in their intrinsic
nature dissimilar from other physical quantities which are not habitually represented
geometrically. If we eliminated the human element (or should we not
say, the pre\hyp{}human element?)\ in natural knowledge the device of graphical
representation of the results of measures or estimates of distance would appear
just as artificial as the graphical representation of thermometer readings. We
cannot predict that a superhuman intelligence would conceive of distance in
the way we conceive it; he would perhaps admit that our device of mentally
plotting the results of a survey in a three\hyp{}dimensional space is ingenious and
scientifically helpful, but it would not occur to him that this space was more
\emph{actual} than the $pv$~space of an indicator\hyp{}diagram.

In our previous work we have studied this unsophisticated graphical representation
of certain physical quantities, under the name Natural Geometry;
we have slightly extended the idea by the addition of a fourth dimension to
include time; and we have found that not only the quantities ordinarily
regarded as geometrical but also mechanical quantities, such as force, density,
energy, are fully represented in this natural geometry. For example the energy\hyp{}tensor
was found to be made up of the Gaussian curvatures of sections of actual
space-time~\Eq{(65.72)}. But the electromagnetic quantities introduced in the preceding
chapter have not as yet been graphically represented; the vector~$\kappa_{\mu}$ was
supposed to exist \emph{in} actual space, not to be the measure of any property \emph{of} actual
space. Thus up to the present the geometrisation of physics is not complete.

Two possible ways of generalising our geometrical outlook are open. It
may be that the Riemannian geometry assigned to actual space is not exact;
and that the true geometry is of a broader kind leaving room for the vector~$\kappa_{\mu}$
to play a fundamental part and so receive geometrical recognition as one
of the determining characters of actual space. For reasons which will appear
in the course of this chapter, I do not think that this is the correct solution.
The alternative is to give all our variables, including~$\kappa_{\mu}$, a suitable graphical
representation in some new conceptual space---not actual space. With sufficient
ingenuity it ought to be possible to accomplish this, for no hypothesis is implied
as to the nature of the quantities so represented. This generalised graphical
scheme may or may not be helpful to the progress of our knowledge; we
attempt it in the hope that it will render the interconnection of electromagnetic
and gravitational phenomena more intelligible. I think it will be found
that this hope is not disappointed.

In \Title{Space, Time and Gravitation,} Chapter~\Vol{XI}, Weyl's non\hyp{}Riemannian
geometry has been regarded throughout as expressing an amended and
exact Natural Geometry. That was the original intention of his theory\footnotemark.\footnotetext
  {The original paper (\Title{Berlin.\ Sitzungsberichte,} 30~May 1918) is rather obscure on this point.
  It states the mathematical development of the corrected Riemannian geometry---``the physical
  application is obvious.'' But it is explicitly stated that the absence of an electromagnetic field is
  the necessary condition for Einstein's theory to be valid---an opinion which, I think, is no longer
  held.}
For the present we shall continue to develop it on this understanding. But
we shall ultimately come to the second alternative, as Weyl himself has done,
and realise that his non\hyp{}Riemannian geometry is not to be applied to \emph{actual}
\index{Non\hyp{}Riemannian geometry}%
space-time; it refers to a graphical representation of that relation\hyp{}structure
which is the basis of all physics, and both electromagnetic and metrical
variables appear in it as interrelated. Having arrived at this standpoint we
pass naturally to the more general geometry of relation\hyp{}structure developed
in Part~II of this chapter.%

We have then to distinguish between Natural Geometry, which is the
single \emph{true} geometry in the sense understood by the physicist, and World
Geometry, which is the pure geometry applicable to a conceptual graphical
representation of all the quantities concerned in physics. We may perhaps go
so far as to say that the World Geometry is intended to be closely descriptive
\index{Geometry, Riemannian!world geometry}%
\index{World geometry}%
of the fundamental relation\hyp{}structure which underlies the various manifestations
of space, time, matter and electromagnetism; that statement, however,
is rather vague when we come to analyse it. Since the graphical representation
is in any case conventional we cannot say that one method rather than another
is right. Thus the two geometries discussed in Parts I and~II of this chapter
are not to be regarded as contradictory. My reason for introducing the second
treatment is that I find it to be more illuminating and far\hyp{}reaching, not that
I reject the first representation as inadmissible.

In the following account of Weyl's theory I have not adhered to the author's
\index{Weyl's theory}%
order of development, but have adapted it to the point of view here taken up,
which sometimes differs (though not, I believe, fundamentally) from that which
he adopts. It may be somewhat unfair to present a theory from the wrong
end---as its author might consider; but I trust that my treatment has not
unduly obscured the brilliance of what is unquestionably the greatest advance
in the relativity theory after Einstein's work.

\Section{84.}{Non\hyp{}integrability of length}
\index{Integrability of parallel displacement!of length and direction}%
\index{Length!non\hyp{}integrability of}%
\index{Non\hyp{}integrability of length and direction}%

We have found in \SecRef{33} that the change~$\delta A_{\mu}$ of a vector taken by parallel
displacement round a small circuit is
\begin{align*}
  \delta A_{\mu}
  &= \tfrac{1}{2} (A_{\mu\nu\sigma} - A_{\mu\sigma\nu})\, dS^{\nu\sigma}\displaybreak[0] \\
  &= \tfrac{1}{2} B_{\mu\nu\sigma}^{\epsilon} A_{\epsilon}\, dS^{\nu\sigma}\displaybreak[0] \\
  &= \tfrac{1}{2} B_{\mu\nu\sigma\epsilon} A^{\epsilon}\, dS^{\nu\sigma}.
  \Tag{(84.1)}\displaybreak[0]
  \intertext{Hence}
  A^{\mu}\, \delta A_{\mu}
  &= \tfrac{1}{2} B_{\mu\nu\sigma\epsilon} A^{\mu} A^{\epsilon}\, dS^{\nu\sigma} = 0,
\end{align*}
since $B_{\mu\nu\sigma\epsilon}$~is antisymmetrical in $\mu$ and~$\epsilon$.

Hence by~\Eq{(26.4)} $\delta A_{\mu}$~is perpendicular to~$A_{\mu}$, and the \emph{length} of the vector~$A_{\mu}$
is unaltered by its parallel displacement round the circuit. It is only the
direction which changes.

We endeavoured to explain how this change of direction can occur in a
curved world by the example of a ship sailing on a curved ocean (\SecRef{33}). Having
convinced ourselves that there is no logical impossibility in the result that the
direction changes, we cannot very well see anything self\hyp{}contradictory in the
length changing also. It is true that we have just given a mathematical proof
that the length does not change; but that only means that a change of length
is excluded by conditions which have been introduced, perhaps inadvertently,
in the postulates of Riemannian geometry. We can construct a geometry in
which the change of length occurs, without landing ourselves in a contradiction.

In the more general geometry, we have in place of~\Eq{(84.1)}
\[
\delta A_{\mu} = \tfrac{1}{2}\, \Star{B}_{\mu\nu\sigma\epsilon} A^{\epsilon}\, dS^{\nu\sigma},
\Tag{(84.21)}
\]
where $\Star{B}_{\mu\nu\sigma\epsilon}$~is a more general tensor which is \emph{not} antisymmetrical in $\mu$ and~$\epsilon$.
It will be antisymmetrical in $\nu$ and~$\sigma$ since a symmetrical part would be
meaningless in~\Eq{(84.21)}, and disappear owing to the antisymmetry of~$dS^{\nu\sigma}$.
Writing
\begin{gather*}
  R_{\mu\nu\sigma\epsilon} = \tfrac{1}{2}(\Star{B}_{\mu\nu\sigma\epsilon} - \Star{B}_{\epsilon\nu\sigma\mu});\quad
  F_{\mu\nu\sigma\epsilon} = \tfrac{1}{2}(\Star{B}_{\mu\nu\sigma\epsilon} + \Star{B}_{\epsilon\nu\sigma\mu}), \\
  \delta A_{\mu} = \tfrac{1}{2}(R_{\mu\nu\sigma\epsilon} + F_{\mu\nu\sigma\epsilon}) A^{\epsilon}\, dS^{\nu\sigma},
  \Tag{(84.22)}
\end{gather*}
where $R$~is antisymmetrical, and $F$~symmetrical, in $\mu$ and~$\epsilon$.

Then the change of length~$l$ is given by
\[
\delta(l^{2}) = 2A^{\mu}\, \delta A_{\mu}
= F_{\mu\nu\sigma\epsilon} A^{\mu} A^{\epsilon}\, dS^{\nu\sigma},
\Tag{(84.3)}
\]
which does not vanish.

To obtain Weyl's geometry we must impose two restrictions on~$F_{\mu\nu\sigma\epsilon}$: \\
\Indent\Item{(a)} $F_{\mu\nu\sigma\epsilon}$~is of the special form~$g_{\mu\epsilon}F_{\nu\sigma}$, \\
\Indent\Item{(b)} $F_{\nu\sigma}$~is the curl of a vector.

The second restriction is logically necessary. We have expressed the change
of a vector taken round a circuit by a formula involving a surface bounded by
the circuit. We may choose different surfaces, all bounded by the same circuit;
and these have to give the same result for~$\delta A_{\mu}$. It is easily seen, as in Stokes's
theorem, that these results will only be consistent if the co\hyp{}factor of~$dS^{\nu\sigma}$ is a
curl.

The first restriction is not imperatively demanded, and we shall discard it
in Part~II of this chapter. It has the following effect. Equation~\Eq{(84.3)} becomes
\begin{align*}
  \delta(l^{2}) &= F_{\nu\sigma} \cdot g_{\mu\epsilon} A^{\mu} A^{\epsilon} \cdot dS^{\nu\sigma} \\
  &= F_{\nu\sigma} l^{2}\, dS^{\nu\sigma},
  \intertext{so that}
  \frac{\delta l}{l} &= \tfrac{1}{2} F_{\nu\sigma}\, dS^{\nu\sigma}.
  \Tag{(84.4)}
\end{align*}
The change of length is proportional to the original length and is independent
of the direction of the vector; whereas in the more general formula~\Eq{(84.3)} the
change of length depends on the direction.

One result of the restriction is that zero-length is still zero-length after
parallel displacement round a circuit. If we have identified zero-length at one
point of the world we can transfer it without ambiguity to every other point
and so identify zero-length everywhere. Finite lengths cannot be transferred
without ambiguity; a route of parallel displacement must be specified.

Zero-length is of great importance in optical phenomena, because in
\index{Zero-length of light tracks}%
Einstein's geometry any element of the track of a light-pulse is a vector of
zero-length; so that if there were no definite zero-length a pulse of light would
not know what track it ought to take. It is because Weyl's theory makes no
attempt to re\hyp{}interpret this part of Einstein's theory that an absolute zero-length
is required, and the restriction~\Item{(a)} is therefore imposed.

Another result of the restriction is that lengths at the same point but in
different orientations become comparable without ambiguity. The ambiguity
is limited to the comparison of lengths at different places.

\Section{85.}{Transformation of gauge\hyp{}systems}

According to the foregoing section it is not possible to compare lengths
(except zero-length) at different places, because the result of the comparison
will depend on the route taken in bringing the two lengths into juxtaposition.

In Riemannian geometry we have taken for granted this possibility of
comparing lengths. The interval at any point has been assigned a definite
value, which implies comparison with a standard; it did not occur to us to
question how this comparison at a distance could be made. We have now to
define the geometry of the continuum in a way which recognises this difficulty.

We suppose that a definite but arbitrary \emph{gauge\hyp{}system} has been adopted;
\index{Gauge\hyp{}system}%
that is to say, at every point of space-time a standard of interval\hyp{}length has
been set up, and every interval is expressed in terms of the standard at the
point where it is. This avoids the ambiguity involved in transferring intervals
from one point to another to compare with a single standard.

Take a displacement at~$P$ (coordinates,~$x_{\mu}$) and transfer it by parallel displacement
to an infinitely near point~$P'$ (coordinates, $x_{\mu} + dx_{\mu}$). Let its initial
length measured by the gauge at~$P$ be~$l$, and its final length measured by the
gauge at~$P'$ be~$l + dl$. We may express the change of length by the formula
\[
d(\log l) = \kappa_{\mu}\, dx_{\mu},
\Tag{(85.1)}
\]
where $\kappa_{\mu}$~represents some vector\hyp{}field. If we alter the gauge\hyp{}system we shall,
of course, obtain different values of~$l$, and therefore of~$\kappa_{\mu}$.

It is not necessary to specify the route of transfer for the small distance
$P$ to~$P'$. The difference in the results obtained by taking different routes is
by~\Eq{(84.4)} proportional to the area enclosed by the routes, and is thus of the
second order in~$dx_{\mu}$. As $PP'$~is taken infinitely small this ambiguity becomes
negligible compared with the first\hyp{}order expression~$\kappa_{\mu}\, dx_{\mu}$.

Our system of reference can now be varied in two ways---by change of
coordinates and by change of gauge\hyp{}system. The behaviour of~$g_{\mu\nu}$ and~$\kappa_{\mu}$ for
transformation of coordinates has been fully studied; we have to examine how
they will be transformed by a transformation of gauge.

A new gauge\hyp{}system will be obtained by altering the length of the standard
at each point in the ratio~$\lambda$, where $\lambda$~is an arbitrary function of the coordinates.
If the standard is decreased in the ratio~$\lambda$, the length of a displacement will
be increased in the ratio~$\lambda$. If accents refer to the new system
\[
ds' = \lambda\, ds.
\Tag{(85.2)}
\]
The components $dx_{\mu}$ of a displacement will not be changed, since we are not
altering the coordinate\hyp{}system, thus
\[
dx_{\mu} = dx_{\mu}'.
\Tag{(85.3)}
\]
Hence
\[
g_{\mu\nu}'\, dx_{\mu}'\, dx_{\nu}'
= ds'^{2}
= \lambda^2\, ds^{2}
= \lambda^2 g_{\mu\nu}\, dx_{\mu}\, dx_{\nu}
= \lambda^2 g_{\mu\nu}\, dx_{\mu}'\, dx_{\nu}',
\]
so that
\[
g_{\mu\nu}' = \lambda^2 g_{\mu\nu}.
\Tag{(85.41)}
\]

It follows at once that
\index{Potential!electromagnetic}%
\begin{align*}
g' & = \lambda^{8} g,
\Tag{(85.42)} \\
{g'}^{\mu\nu} &= \lambda^{-2} g^{\mu\nu},
\Tag{(85.43)} \\
\sqrt{-g'} \cdot d\tau' &= \lambda^{4} \sqrt{-g} \cdot d\tau.
\Tag{(85.44)}
\end{align*}
Again, by~\Eq{(85.1)}
\begin{align*}
  \kappa_{\mu}'\, dx_{\mu}
  &= d(\log l') = d\{\log(\lambda l)\} \\
  &= d(\log l) + d(\log\lambda) \\
  &= \kappa_{\mu}\, dx_{\mu} + \frac{\dd(\log\lambda)}{\dd x_{\mu}}\, dx_{\mu}.
\end{align*}
Or, writing
\[
\phi = \log\lambda,
\Tag{(85.51)}
\]
then
\[
\kappa_{\mu}' = \kappa_{\mu} + \frac{\dd\phi}{\dd x_{\mu}}.
\Tag{(85.52)}
\]

The curl of~$\kappa_{\mu}$ has an important property; if
\[
F_{\mu\nu} = \frac{\dd\kappa_{\mu}}{\dd x_{\nu}} - \frac{\dd\kappa_{\nu}}{\dd x_{\mu}},
\]
we see by~\Eq{(85.52)} that
\[
F_{\mu\nu}' = F_{\mu\nu},
\Tag{(85.6)}
\]
so that $F_{\mu\nu}$~is independent of the gauge\hyp{}system. This is only true of the covariant
tensor; if we raise one or both suffixes the function~$\lambda$ is introduced
by~\Eq{(85.43)}.

It will be seen that the geometry of the continuum now involves $14$~functions
which vary from point to point, viz.\ ten~$g_{\mu\nu}$ and four~$\kappa_{\mu}$. These may be subjected
to transformations, viz.\ the transformations of gauge discussed above,
and the transformations of coordinates discussed in Chapter~\ChapNum{II}\@. Such transformations
will not alter any intrinsic properties of the world; but any changes
in the~$g_{\mu\nu}$ and~$\kappa_{\mu}$ other than gauge or coordinate transformations will alter the
intrinsic state of the world and may reasonably be expected to change its
physical manifestations.

The question then arises, How will the change manifest itself physically if
we alter the~$\kappa_{\mu}$? All the phenomena of mechanics have been traced to the~$g_{\mu\nu}$,
so that presumably the change is not shown in mechanics, or at least the
primary effect is not mechanical. We are left with the domain of electromagnetism
which is not expressible in terms of $g_{\mu\nu}$~alone; and the suggestion
arises that an alteration of~$\kappa_{\mu}$ may appear physically as an alteration of the
electromagnetic field.

We have seen that the electromagnetic field is described by a vector already
called~$\kappa_{\mu}$, and it is an obvious step to identify this with the $\kappa_{\mu}$ introduced in
Weyl's geometry. According to observation the physical condition of the world
is not completely defined by the~$g_{\mu\nu}$ and an additional vector must be specified;
according to theoretical geometry the nature of a continuum is not completely
indicated by the~$g_{\mu\nu}$ and an additional vector must be specified. The conclusion
is irresistible that the two vectors are to be identified.

Moreover according to~\Eq{(85.52)} we can change $\kappa_{\mu}$ to $\kappa_{\mu} + \dd\phi/\dd x_{\mu}$ by a change
of gauge without altering the intrinsic state of the world. It was explained at
the beginning of \SecRef{74} that we can make the same change of the electromagnetic
potential without altering the resulting electromagnetic field.

We accordingly accept this identification. The $\kappa_{\mu}$~and $F_{\mu\nu}$ of the present
geometrical theory will be the electromagnetic potential and force of Chapter~\ChapNum{VI}\@.
It will be best to suspend the convention $\kappa_{\mu}^{\mu} = 0$ \Eq{(74.1)} for the present, since
that would commit us prematurely to a particular gauge\hyp{}system.

It must be borne in mind that by this identification the electromagnetic
force becomes expressed in some natural unit whose relation to the \CGS\
system is at present unknown. For example the constant of proportionality
in~\Eq{(77.7)} may be altered. $F_{\mu\nu}$~is not altered by any change of gauge\hyp{}system~\Eq{(85.6)}
so that its value is a pure number. The question then arises, How many
volts per centimetre correspond to $F_{\mu\nu} = 1$ in any given coordinate\hyp{}system?
The problem is a difficult one, but we shall give a rough and rather dubious
estimate in \SecRef{102}.

I do not think that our subsequent discussion will add anything material
to the present argument in favour of the electromagnetic interpretation of~$\kappa_{\mu}$.
The case rests entirely on the apparently significant fact, that on removing an
artificial restriction in Riemannian geometry, we have just the right number
of variables at our disposal which are necessary for a physical description of
the world.

\Section{86.}{Gauge\hyp{}invariance}
\index{In- (prefix)}%

It will be useful to discover tensors and invariants which, besides possessing
their characteristic properties with regard to transformations of coordinates,
are unaltered by any transformation of gauge\hyp{}system. These will be called
\emph{in\hyp{}tensors} and \emph{in\hyp{}invariants}.
\index{In\hyp{}tensors}%

There are other tensors or invariants which merely become multiplied by
a power of~$\lambda$, when the gauge is altered. These will be called \emph{co\hyp{}tensors} and
\emph{co\hyp{}invariants}.

Change of gauge is a generalisation of change of unit in physical equations,
the unit being no longer a constant but an arbitrary function of position. We
have only one unit to consider---the unit of interval. Coordinates are merely
identification\hyp{}numbers and have no reference to our unit, so that a displacement~$dx_{\mu}$
is an in\hyp{}vector. It should be noticed that if we change the unit\hyp{}mesh
of a rectangular coordinate\hyp{}system from one mile to one kilometre, we make
a change of coordinates not a change of gauge. The distinction is more obvious
when coordinates other than Cartesian are used. The most confusing case is
that of Galilean coordinates, for then the special values of the~$g_{\mu\nu}$ fix the length
of side of unit mesh as equal to the unit of interval; and it is not easy to keep
in mind that the \emph{displacement} between two corners of the mesh is the number~$1$,
whilst the \emph{interval} between them is $1$~kilometre.

According to~\Eq{(85.6)} the electromagnetic force~$F_{\mu\nu}$ is an in\hyp{}tensor. $F_{\mu\nu}$~is
only a co\hyp{}tensor, and $F_{\mu\nu} F^{\mu\nu}$~a co\hyp{}invariant.

Transforming the $3$-index symbol $[\mu\nu, \sigma]$ by an alteration of gauge we have
\index{Christoffel's $3$-index symbols!generalisation of}%
\index{Derivative!in\hyp{}covariant}%
\index{In\hyp{}covariant derivative}%
\index{Three\hyp{}index symbol!generalised}%
by~\Eq{(85.41)}
\begin{align*}
  [\mu\nu, \sigma]'
  &= \frac{1}{2} \left(\frac{\dd(\lambda^2 g_{\mu\sigma})}{\dd x_{\nu}}
  + \frac{\dd(\lambda^2 g_{\nu\sigma})}{\dd x_{\mu}}
  - \frac{\dd(\lambda^2 g_{\mu\nu})}{\dd x_{\sigma}}\right)\displaybreak[0] \\
  &= \lambda^2 [\mu\nu, \sigma]
  + \tfrac{1}{2} g_{\mu\sigma}\, \frac{\dd\lambda^2}{\dd x_{\nu}}
  + \tfrac{1}{2} g_{\nu\sigma}\, \frac{\dd\lambda^2}{\dd x_{\mu}}
  - \tfrac{1}{2} g_{\mu\nu}\, \frac{\dd\lambda^2}{\dd x_{\sigma}}\displaybreak[0] \\
  &= \lambda^2 [\mu\nu, \sigma]
  + \lambda^2(g_{\mu\sigma} \phi_{\nu} + g_{\nu\sigma} \phi_{\mu} - g_{\mu\nu} \phi_{\sigma})
\end{align*}
by~\Eq{(85.51)}. We have written
\[
\phi_{\mu} \equiv \frac{\dd\phi}{\dd x_{\mu}}.
\]

Multiply through by ${g'}^{\sigma\alpha} = \lambda^{-2} g^{\sigma\alpha}$; we obtain
\[
\{\mu\nu, \alpha\}'
= \{\mu\nu, \alpha\} + g_{\mu}^{\alpha} \phi_{\nu} + g_{\nu}^{\alpha} \phi_{\mu} - g_{\mu\nu} \phi^{\alpha}.
\Tag{(86.1)}
\]

Let
\[
\Star{\{\mu\nu, \alpha\}}
\equiv \{\mu\nu, \alpha\} - g_{\mu}^{\alpha} \kappa_{\nu} - g_{\nu}^{\alpha} \kappa_{\mu} + g_{\mu\nu} \kappa^{\alpha}.
\Tag{(86.2)}
\]

Then by \Eq{(86.1)} and~\Eq{(85.52)}
\[
\Star{\{\mu\nu, \alpha\}}' = \Star{\{\mu\nu, \alpha\}}.
\Tag{(86.3)}
\]

The ``generalised $3$-index symbol'' $\Star{\{\mu\nu, \alpha\}}$ has the ``in-'' property, being
unaltered by any gauge\hyp{}transformation. It is, of course, not a tensor.

We shall generally indicate by a star~(${}^{*}$) quantities generalised from corresponding
expressions in Riemannian geometry in order to be independent of
(or covariant with) the gauge\hyp{}system. The following illustrates the general
method of procedure.

Let $A_{\mu}^{\nu}$~be a symmetrical in\hyp{}tensor; its divergence~\Eq{(51.31)} becomes on
gauge\hyp{}transformation
\begin{align*}
  {A'}_{\mu\nu}^{\nu}
  &= \frac{1}{\lambda^{4} \sqrt{-g}}\, \frac{\dd}{\dd x_{\nu}} (A_{\mu}^{\nu} \lambda^{4} \sqrt{-g})
  - \tfrac{1}{2} (\lambda^{-2} A^{\alpha\beta})\, \frac{\dd}{\dd x_{\mu}} (\lambda^2 g_{\alpha\beta}) \\
  &= \frac{1}{\sqrt{-g}}\, \frac{\dd}{\dd x_{\nu}} (A_{\mu}^{\nu}\sqrt{-g})
%[** TN: Not broken in the original]
  \begin{aligned}[t]
    &- \tfrac{1}{2} A^{\alpha\beta}\, \frac{\dd g_{\alpha\beta}}{\dd x_{\mu}}
    + A_{\mu}^{\nu} \cdot \frac{1}{\lambda^{4}}\, \frac{\dd\lambda^{4}}{\dd x_{\nu}} \\
    &- \tfrac{1}{2} A^{\alpha\beta} g_{\alpha\beta} \cdot \frac{1}{\lambda^2}\, \frac{\dd\lambda^2}{\dd x_{\mu}}
  \end{aligned} \\
  &= A_{\mu\nu}^{\nu} + 4A_{\mu}^{\nu} \phi_{\nu} - A\phi_{\mu}.
\end{align*}
Hence by~\Eq{(85.52)} the quantity
\[
\Star{A}_{\mu\nu}^{\nu} = A_{\mu\nu}^{\nu} - 4A_{\mu}^{\nu} \kappa_{\nu} + A\kappa_{\mu}
\Tag{(86.4)}
\]
is unaltered by any gauge\hyp{}transformation, and is accordingly an in\hyp{}vector.

This operation may be called in\hyp{}covariant differentiation, and the result is
the in\hyp{}divergence.

The result is modified if $A^{\mu\nu}$~is the in\hyp{}tensor, so that $A_{\mu}^{\nu}$~is a co\hyp{}tensor. The
different associated tensors are not equally fundamental in Weyl's geometry,
since only one of them can be an in\hyp{}tensor.

Unless expressly stated a final suffix will indicate ordinary covariant (not
in\hyp{}covariant) differentiation.

\Section{87.}{The generalised Riemann\hyp{}Christoffel tensor}
\index{Riemann\hyp{}Christoffel tensor!generalisation of}%

Corresponding to~\Eq{(34.4)} we write

\[
\Star{B}_{\mu\nu\sigma}^{\epsilon}
\!=\! -\frac{\dd}{\dd x_{\sigma}} \Star{\{\mu\nu, \epsilon\}}
  + \Star{\{\mu\sigma, \alpha\}} \Star{\{\alpha\nu, \epsilon\}}
  + \frac{\dd}{\dd x_{\nu}} \Star{\{\mu\sigma, \epsilon\}}
  - \Star{\{\mu\nu, \alpha\}} \Star{\{\alpha\sigma, \epsilon\}}.
  \Tag{(87.1)}
\]

This will be an in\hyp{}tensor since the starred symbols are all independent of
the gauge; and it will be evident when we reach~\Eq{(87.4)} that the generalisation
has not destroyed the ordinary tensor properties.

We consider the first two terms; the complete expression can then be
obtained at any stage by interchanging $\nu$ and~$\sigma$ and subtracting. The additional
terms introduced by the stars are by~\Eq{(86.2)}
%[** TN: Re-breaking]
\begin{multline*}
  -\frac{\dd}{\dd x_{\sigma}} (-g_{\mu}^{\epsilon} \kappa_{\nu} - g_{\nu}^{\epsilon} \kappa_{\mu} + g_{\mu\nu} \kappa^{\epsilon})
  + (-g_{\mu}^{\alpha} \kappa_{\sigma} - g_{\sigma}^{\alpha} \kappa_{\mu} + g_{\mu\sigma} \kappa^{\alpha}) \{\alpha\nu, \epsilon\} \\
  + (-g_{\alpha}^{\epsilon} \kappa_{\nu} - g_{\nu}^{\epsilon} \kappa_{\alpha} + g_{\alpha\nu} \kappa^{\epsilon}) \{\mu\sigma, \alpha\} \\
  + (-g_{\mu}^{\alpha} \kappa_{\sigma} - g_{\sigma}^{\alpha} \kappa_{\mu} + g_{\mu\sigma} \kappa^{\alpha})
  (-g_{\alpha}^{\epsilon} \kappa_{\nu} - g_{\nu}^{\epsilon} \kappa_{\alpha} + g_{\alpha\nu} \kappa^{\epsilon}) \\
  = g_{\mu}^{\epsilon}\, \frac{\dd\kappa_{\nu}}{\dd x_{\sigma}}
  + g_{\nu}^{\epsilon}\, \frac{\dd\kappa_{\mu}}{\dd x_{\sigma}}
  - g_{\mu\nu}\, \frac{\dd\kappa^{\epsilon}}{\dd x_{\sigma}} - \frac{\dd g_{\mu\nu}}{\dd x_{\sigma}} \kappa^{\epsilon} \\
  - \kappa_{\sigma} \{\mu\nu, \epsilon\} - \kappa_{\mu} \{\sigma\nu, \epsilon\} + g_{\mu\sigma} \{\alpha\nu, \epsilon\} \kappa^{\alpha} \\
  - \kappa_{\nu} \{\mu\sigma, \epsilon\} - g_{\nu}^{\epsilon} \{\mu\sigma, \alpha\} \kappa_{\alpha}
  + \kappa^{\epsilon} [\mu\sigma, \nu]
  + g_{\mu}^{\epsilon} \kappa_{\sigma} \kappa_{\nu}
  + g_{\nu}^{\epsilon} \kappa_{\sigma} \kappa_{\mu}
  - g_{\mu\nu} \kappa_{\sigma} \kappa^{\epsilon} \\
  + g_{\sigma}^{\epsilon} \kappa_{\mu} \kappa_{\nu}
  + g_{\nu}^{\epsilon} \kappa_{\mu} \kappa_{\sigma}
  - g_{\sigma\nu} \kappa_{\mu} \kappa^{\epsilon}
  - g_{\mu\sigma} \kappa^{\epsilon} \kappa_{\nu}
  - g_{\mu\sigma} g_{\nu}^{\epsilon} \kappa^{\alpha} \kappa_{\alpha}
  + g_{\mu\sigma} \kappa_{\nu} \kappa^{\epsilon},
  \Tag{(87.2)}
\end{multline*}
which is equivalent to
\[
  g_{\mu}^{\epsilon}\, \frac{\dd\kappa_{\nu}}{\dd x_{\sigma}}
  + g_{\nu}^{\epsilon} (\kappa_{\mu})_{\sigma}
  - g_{\mu\nu} (\kappa^{\epsilon})_{\sigma}
  + g_{\nu}^{\epsilon} \kappa_{\mu} \kappa_{\sigma}
  - g_{\nu}^{\epsilon} g_{\mu\sigma} \kappa_{\alpha} \kappa^{\alpha}
  + g_{\mu\sigma} \kappa_{\nu} \kappa^{\epsilon}.
  \Tag{(87.3)}
\]

[To follow this reduction let the terms in~\Eq{(87.2)} be numbered in order
from~$1$ to~$19$. It will be found that the following terms or pairs of terms are
symmetrical in $\nu$ and~$\sigma$, and therefore disappear when the expression is
completed, viz.\ $5$~and~$8$, $6$,~$11$, $12$ and $14$, $13$ and $17$,~$16$. Further $4$~and $10$
together give $-[\nu\sigma, \mu] \kappa^{\epsilon}$, which is rejected for the same reason. We combine
$2$~and $9$ to give $g_{\nu}^{\epsilon} (\kappa_{\mu})_{\sigma}$. We exchange $7$ for its counterpart $-g_{\mu\nu} \{\alpha\sigma, \epsilon\} \kappa^{\alpha}$
in the remaining half of the expression, and combine it with~$3$ to give
$-g_{\mu\nu} (\kappa^{\epsilon})_{\sigma}$.]

Hence interchanging $\nu$ and~$\sigma$, and subtracting, the complete expression is
\begin{multline*}
  \Star{B}_{\mu\nu\sigma}^{\epsilon}
  = B_{\mu\nu\sigma}^{\epsilon} + g_{\mu}^{\epsilon} \left(\frac{\dd\kappa_{\nu}}{\dd x_{\sigma}} - \frac{\dd\kappa_{\sigma}}{\dd x_{\nu}}\right)
  + (g_{\nu}^{\epsilon} \kappa_{\mu\sigma} - g_{\sigma}^{\epsilon} \kappa_{\mu\nu})
  + (g_{\mu\sigma} \kappa_{\nu}^{\epsilon} - g_{\mu\nu} \kappa_{\sigma}^{\epsilon}) \\
  + (g_{\nu}^{\epsilon} \kappa_{\mu} \kappa_{\sigma} - g_{\sigma}^{\epsilon} \kappa_{\mu} \kappa_{\nu})
  + (g_{\sigma}^{\epsilon} g_{\mu\nu} - g_{\nu}^{\epsilon} g_{\mu\sigma}) \kappa_{\alpha} \kappa^{\alpha}
  + (g_{\mu\sigma} \kappa_{\nu} - g_{\mu\nu} \kappa_{\sigma}) \kappa^{\epsilon}.
  \Tag{(87.4)}
\end{multline*}

Next set $\epsilon = \sigma$. We obtain the contracted in\hyp{}tensor
\begin{multline*}
  \Star{G}_{\mu\nu} = G_{\mu\nu} - F_{\mu\nu}
  + (\kappa_{\mu\nu} - 4\kappa_{\mu\nu})
  + (\kappa_{\mu\nu} - g_{\mu\nu} \kappa_{\alpha}^{\alpha})
  + (\kappa_{\mu} \kappa_{\nu} - 4\kappa_{\mu} \kappa_{\nu}) \\
  + (4g_{\mu\nu} - g_{\mu\nu}) \kappa_{\alpha} \kappa^{\alpha}
  + (\kappa_{\mu} \kappa_{\nu} - g_{\mu\nu} \kappa_{\alpha} \kappa^{\alpha}) \\
  = G_{\mu\nu} - 2F_{\mu\nu} - (\kappa_{\mu\nu} + \kappa_{\nu\mu})
  - g_{\mu\nu} \kappa_{\alpha}^{\alpha}
  - 2\kappa_{\mu} \kappa_{\nu}
  + 2g_{\mu\nu} \kappa_{\alpha}^{\alpha}.\footnotemark
  \Tag{(87.5)}
\end{multline*}

Finally multiply by~$g^{\mu\nu}$. We obtain the co\hyp{}invariant
\footnotetext{The unit of~$\kappa_{\mu}$ is arbitrary; and in the generalised theory in Part~II the $\kappa_{\mu}$ there employed
  corresponds to twice the $\kappa_{\mu}$ of these formulae. This must be borne in mind in comparing, for
  example, \Eq{(87.5)} and~\Eq{(94.3)}.}%
\[
\Star{G} = G - 6\kappa_{\alpha}^{\alpha} + 6\kappa_{\alpha} \kappa^{\alpha}.
\Tag{(87.6)}
\]

The multiplication by~$g^{\mu\nu}$ reintroduces the unit of gauge, so that $\Star{G}$~becomes
multiplied by~$\lambda^{-2}$ when the gauge is transformed.

If the suffix~$\epsilon$ is lowered in~\Eq{(87.4)} the only part of~$\Star{B}_{\mu\nu\sigma\epsilon}$ which is symmetrical
in $\mu$~and $\epsilon$ is $g_{\mu\nu} (\dd\kappa_{\nu}/\dd x_{\sigma} - \dd\kappa_{\sigma}/\dd x_{\nu}) = g_{\mu\epsilon} F_{\nu\sigma}$, which agrees with the
condition~\Item{(a)} of Weyl's geometry (\SecRef{84}).

\Section{88.}{The in\hyp{}invariants of a region}
\index{In\hyp{}invariants}%

There are no functions of the~$g_{\mu\nu}$ and~$\kappa_{\mu}$ at a point which are in\hyp{}invariants;
but functions which are in\hyp{}invariant\hyp{}densities may be found as
follows---

Since $\sqrt{-g}$~becomes multiplied by~$\lambda^{4}$ on gauge\hyp{}transformation we must
combine it with co\hyp{}invariants which become multiplied by~$\lambda^{-4}$. The following
are easily seen to be in\hyp{}invariant\hyp{}densities:
\begin{gather*}
  (\Star{G})^{2} \sqrt{-g};\quad
  \Star{G}_{\mu\nu}\, \Star{G}^{\mu\nu} \sqrt{-g};\quad
  \Star{B}_{\mu\nu\sigma}^{\epsilon}\, \Star{B}_{\epsilon}^{\mu\nu\sigma} \sqrt{-g},
  \Tag{(88.1)} \\
  F_{\mu\nu} F^{\mu\nu} \sqrt{-g}.
  \Tag{(88.2)}
\end{gather*}

We can also form in\hyp{}invariant\hyp{}densities from the fundamental tensor of
the sixth rank. Let $\Star{}(\Star{B}_{\mu\nu\sigma\rho})_{\alpha\beta}$ be the second co\hyp{}covariant derivative of the
co\hyp{}tensor $\Star{B}_{\mu\nu\sigma\rho}$; the spur formed by raising three suffixes and contracting
will vary as~$\lambda^{-4}$ and give an in\hyp{}invariant\hyp{}density on multiplication by~$\sqrt{-g}$.
\index{Density!in\hyp{}invariant-}%
There are three different spurs, according to the pairing of the suffixes, but
I believe that there are relations between them so that they give only one
independent expression. The simplest of them is
\[
g^{\mu\nu} g^{\sigma\rho} g^{\alpha\beta}\, \Star(\Star{B}_{\mu\nu\sigma\rho})_{\alpha\beta} \sqrt{-g}
= \Star{\,\Wave}\, \Star{G} \cdot \!\sqrt{-g}.
\Tag{(88.3)}
\]

If $\mf{A}$~stands for any in\hyp{}invariant\hyp{}density,
\index{Absolute change!properties of a region}%
\[
\int \mf{A}\, d\tau
\]
taken over a four\hyp{}dimensional region is a pure number independent of coordinate\hyp{}system
and gauge\hyp{}system. Such a number denotes a property of
the region which is absolute in the widest sense of the word; and it seems
likely that one or more of these numerical invariants of the region must
stand in a simple relation to all the physical quantities which measure the
more general properties of the world. The simplest operation which we can
perform on a regional invariant appears to be that of Hamiltonian differentiation,
and a particular importance will therefore be attached to the tensors
$\Ham A/\Ham g_{\mu\nu}$, $\Ham A/\Ham\kappa_{\mu}$.

It has been pointed out by Weyl that it is only in a four\hyp{}dimensional
world that a simple set of regional in\hyp{}invariants of this kind exists. In an
odd number of dimensions there are none; in two dimensions there is one,~$\Star{G} \sqrt{-g}$;
in six or eight dimensions the in\hyp{}invariants are all very complex
involving derivatives of at least the fourth order or else obviously artificial.
This may give some sort of reason for the four dimensions of the world. The
\index{Dimensions, world of $3 + 1$!reason for four}%
\index{Four dimensions of world}%
argument appears to be that a world with an odd number of dimensions
could contain nothing absolute, which would be unthinkable.

These conclusions are somewhat modified by the existence of a particularly
simple regional in\hyp{}invariant, which seems to have been generally overlooked
because it is not of the type which investigators have generally studied. The
quantity
\[
\int \sqrt{-|\Star{G}^{\mu\nu}|}\, d\tau
\Tag{(88.4)}
\]
is an invariant by~\Eq{(81.1)} and it contains nothing which depends on the
gauge. It is not \emph{more} irrational than the other in\hyp{}invariants since these
contain~$\sqrt{-g}$. We shall find later that it is closely analogous to the metrical
volume and the electromagnetic volume (\SecRef{81}) of the region. It will be
\index{Volume!generalised}%
called the \emph{generalised volume}. This in\hyp{}invariant would still exist if the world
\index{Generalised volume}%
had an odd number of dimensions.

It may be remarked that $F^{\mu\nu} \sqrt{-g}$, or~$\mf{F}^{\mu\nu}$, is an in\hyp{}tensor\hyp{}density.
Thus the factor~$\sqrt{-g}$ should always be associated with the contravariant tensor, if
the formulae are to have their full physical significance.
The electromagnetic action\hyp{}density should be written
\[
F_{\mu\nu} \mf{F}^{\mu\nu},
\]
and the energy\hyp{}density
\[
-F_{\mu\nu} \mf{F}^{\nu\alpha} + \tfrac{1}{4} g_{\mu}^{\nu} F_{\alpha\beta} \mf{F}^{\alpha\beta}.
\]
The field is thus characterised by an \emph{intensity~$F_{\mu\nu}$} or a \emph{quantity} of density~$\mf{F}^{\mu\nu}$;
both descriptions are then independent of the gauge\hyp{}system used.

\Section{89.}{The natural gauge}
\index{Natural coordinates!gauge}%

For the most part the laws of mechanics investigated in Chapters \ChapNum{III}--\ChapNum{V}
have been expressed by tensor equations but not in\hyp{}tensor equations. Hence
they can only hold when a particular gauge\hyp{}system is used, and will cease to
be true if a transformation of gauge\hyp{}system is made. The gauge\hyp{}system for
which our previous work is valid (if it is valid) is called the \emph{natural gauge};
it stands in somewhat the same position with respect to a general gauge as
Galilean coordinates stand with respect to general coordinates.

Just as we have generalised the equations of physics originally found for
Galilean coordinates, so we could generalise the equations for the natural
gauge by substituting the corresponding in\hyp{}tensor equations applicable to
any gauge. But before doing so, we stop to ask whether anything would be
gained by this generalisation. There is not much object in generalising the
Galilean formulae, so long as Galilean coordinates are available; we required
the general formulae because we discovered that there are regions of the
world where no Galilean coordinates exist. Similarly we shall only need the
in\hyp{}tensor equations of mechanics if there are regions where no natural
gauge exists; that is to say, if no gauge\hyp{}system can be found for which
Einstein's formulae are accurately true. It was, I think, the original idea of
Weyl's theory that electromagnetic fields were such regions, where accordingly
in\hyp{}tensor equations would be essential.

There is in any case a significant difference between Einstein's generalisation
of Galilean geometry and Weyl's generalisation of Riemannian
geometry. We have proved directly that the condition which renders Galilean
coordinates impossible \emph{must} manifest itself to us as a gravitational field of
force. That is the meaning of a field of force according to the definition of force.
But we cannot prove that the break\hyp{}down of the natural gauge would manifest
itself as an electromagnetic field; we have merely speculated that the world\hyp{}condition
measured by the vector~$\kappa_{\mu}$ which appears in the in\hyp{}tensor equations
may be the origin of electrical manifestations \emph{in addition} to causing the
failure of Riemannian geometry.

Accepting the original view of Weyl's theory, the ambiguity in the
comparison of lengths at a distance has hitherto only shown itself in practical
experiments by the electromagnetic phenomena supposed to be dependent on
it but not (so far as we can see) immediately implied by it. This is not
surprising when we attempt to estimate the order of magnitude of the
ambiguity. Taking formula~\Eq{(84.4)}, $dl/l = \frac{1}{2} F_{\nu\sigma}\, dS^{\nu\sigma}$, we might perhaps expect
that $dl/l$~would be comparable with unity, if the electromagnetic force~$F_{\nu\sigma}$
were comparable with that at the surface of an electron, $4 \cdot 10^{18}$~volts per~cm.,
and the side of the circuit were comparable with the radius of curvature of
space. Thus for ordinary experiments $dl/l$~would be far below the limits of
experimental detection. Accordingly we can have a gauge\hyp{}system specified
by the transfer of material standards which is for all practical purposes
unambiguous, and yet contains that minute theoretical ambiguity which is
only of practical consequence on account of its side\hyp{}manifestation as the
cause of electrical phenomena. The gauge\hyp{}system employed in practice is
the natural gauge\hyp{}system to which our previous mechanical formulae apply---or
rather, since the practical gauge\hyp{}system is slightly ambiguous and the
theoretical formulae are presumably exact, the natural gauge is an exact
gauge with which all practical gauges agree to an approximation sufficient
for all observable mechanical and metrical phenomena.

According to Weyl the natural gauge is determined by the condition
\[
\Star{G} = 4\lambda,
\Tag{(89.1)}
\]
where $\lambda$~is a constant everywhere.

This attempt to reconcile a theoretical ambiguity of our system of
measurement with its well-known practical efficiency seems to be tenable,
though perhaps a little overstrained. But an alternative view is possible.
This states that---

\emph{Comparison of lengths at different places is an unambiguous procedure
having nothing to do with parallel displacement of a vector.}

The practical operation of transferring a measuring\hyp{}scale from one place
to another is not to be confounded with the transfer by parallel displacement
of the vector representing the displacement between its two extremities. If
this is correct Einstein's Riemannian geometry, in which each interval has
a unique length, must be accepted as exact; the ambiguity of transfer by
parallel displacement does not affect his work. No attempt is to be made to
apply Weyl's geometry as a Natural Geometry; it refers to a different
subject of discussion.

Prof.\ Weyl himself has come to prefer the second alternative. He draws a
\index{Adjustment and persistence}%
\index{Persistence and adjustment}%
useful distinction between magnitudes which are determined by \emph{persistence}
(\Foreign{Beharrung}) and by \emph{adjustment} (\Foreign{Einstellung}); and concludes that the dimensions
of material objects are determined by adjustment. The size of an
electron is determined by adjustment in proportion to the radius of curvature
of the world, and not by persistence of anything in its past history. This is
the view taken in \SecRef{66}, and we have seen that it has great value in affording
an explanation of Einstein's law of gravitation.

The generalised theory of Part~II leads almost inevitably to the second
alternative. The first form of the theory has died rather from inanition
than by direct disproof; it ceases to offer temptation when the problem is
approached from a broader point of view. It now seems an unnecessary
speculation to introduce small ambiguities of length\hyp{}comparisons too small
to be practically detected, merely to afford the satisfaction of geometrising
the vector~$\kappa_{\mu}$ which has more important manifestations.

The new view entirely alters the status of Weyl's theory. Indeed it is no
\index{Weyl's theory!modified view of}%
longer a hypothesis, but a graphical representation of the facts, and its value
lies in the insight suggested by this graphical representation. We need not
now hesitate for a moment over the identification of the electromagnetic
potential with the geometrical vector~$\kappa_{\mu}$; the geometrical vector is the
potential because that is the way in which we choose to represent the
potential graphically. We take a conceptual space obeying Weyl's geometry
and represent in it the gravitational potential by the $g_{\mu\nu}$~for that space and
the electromagnetic potential by the $\kappa_{\mu}$~for that space. We find that all
other quantities concerned in physics are now represented by more or less
simple geometrical magnitudes in that space, and the whole picture enables
us to grasp in a comprehensive way the relations of physical quantities,
and more particularly those reactions in which both electromagnetic and
mechanical variables are involved. Parallel displacement of a vector in this
space is a definite operation, and may in certain cases have an immediate
physical interpretation; thus when an uncharged particle moves freely in a
geodesic its velocity\hyp{}vector is carried along by parallel displacement~\Eq{(33.4)};
but when a material measuring\hyp{}rod is moved the operation is not one of
parallel displacement, and must be described in different geometrical terms,
which have reference to the natural gauging\hyp{}equation~\Eq{(89.1)}.

When in Part~II we substitute a conceptual space with still more general
geometry, we shall not need to regard it as in opposition to the present
discussion. We may learn more from a different graphical picture of what is
going on; but we shall not have to abandon anything which we can perceive
clearly in the first picture.

We consider now the gauging\hyp{}equation $\Star{G} = 4\lambda$ assumed by Weyl. It is
probably the one which most naturally suggests itself. Suppose that we
have adopted initially some other gauge in which $\Star{G}$~is not constant. $\Star{G}$~is
a co\hyp{}invariant such that when the measure of interval is changed in the ratio~$\mu$,
$\Star{G}$~changes in the ratio~$\mu^{-2}$. Hence we can obtain a new gauge in which
$\Star{G}$~becomes constant by transforming the measure of the interval in the
ratio~$\Star{G}^{-\frac{1}{2}}$.

By~\Eq{(87.6)} the gauging\hyp{}equation is equivalent to
\[
G - 6\kappa_{\alpha}^{\alpha} + 6\kappa_{\alpha} \kappa^{\alpha} = 4\lambda.
\Tag{(89.2)}
\]
But by~\Eq{(54.72)} the proper\hyp{}density of matter is
\begin{align*}
  \rho_{0} &= \frac{1}{8\pi} (G - 4\lambda) \\
  &= \frac{3}{4\pi} (\kappa_{\alpha}^{\alpha} - \kappa_{\alpha} \kappa^{\alpha}).
  \Tag{(89.3)}
\end{align*}
For empty space, or for space containing free electromagnetic fields without
electrons, $\rho_{0} = 0$, so that
\[
\kappa_{\alpha}^{\alpha} = \kappa_{\alpha} \kappa^{\alpha},
\Tag{(89.4)}
\]
except within an electron. This condition should replace the equation $\kappa_{\alpha}^{\alpha} = 0$
which was formerly introduced in order to make the electromagnetic potential
determinate~\Eq{(74.1)}.

We cannot conceive of any kind of measurement with clocks, scales,
moving particles or light\hyp{}waves being made \emph{inside} an electron, so that any
gauge employed in such a region must be purely theoretical having no significance
in terms of practical measurement. For the sake of continuity we
define the natural gauge in this region by the same equation $\Star{G} = 4\lambda$; it is
as suitable as any other. Inside the electron $\kappa_{\alpha}^{\alpha}$~will not be equal to~$\kappa_{\alpha} \kappa^{\alpha}$ and
the difference will determine the mass of the electron in accordance with~\Eq{(89.3)}.
But it will be understood that this application of~\Eq{(89.3)} is merely
conventional; although it appears to refer to experimental quantities, the
conditions are such that it ceases to be possible for the experiments to be
made by any conceivable device.

\Section{90.}{Weyl's action\hyp{}principle}
\index{Action, material or gravitational!Weyl's formula}%
\index{Principle!of least action}%
\index{Stationary action, principle of}%

Weyl adopts an action\hyp{}density
\[
A \sqrt{-g} = (\Star{G}^{2} - \alpha F_{\mu\nu} F^{\mu\nu}) \sqrt{-g},
\Tag{(90.1)}
\]
the constant~$\alpha$ being a pure number. He makes the hypothesis that it obeys
the principle of stationary action for all variations $\delta g_{\mu\nu}$, $\delta\kappa_{\mu}$ which vanish at
the boundary of the region considered. Accordingly
\[
\frac{\Ham A}{\Ham g_{\mu\nu}} = 0,\quad
\frac{\Ham A}{\Ham \kappa_{\mu}} = 0.
\Tag{(90.2)}
\]

Weyl himself states that his action\hyp{}principle is probably not realised in
nature exactly in this form. But the procedure is instructive as showing the
kind of unifying principle which is aimed at according to one school of
thought.

The variation of $\Star{G}^{2} \sqrt{-g}$ is
\[
2\, \Star{G}\, \delta(\Star{G} \sqrt{-g}) - \Star{G}^{2}\, \delta(\sqrt{-g}),
\]
which in the natural gauge becomes by~\Eq{(89.1)}
\[
8\lambda\, \delta(\Star{G} \sqrt{-g}) - 16\lambda^2\, \delta(\sqrt{-g}).
\]
Hence by~\Eq{(87.6)}
\[
\frac{1}{8\lambda}\, \delta(A \sqrt{-g})
= \delta \bigl\{(G - 6\kappa_{\alpha}^{\alpha} + 6\kappa_{\alpha} \kappa^{\alpha} - 2\lambda - \beta F_{\mu\nu} F^{\mu\nu}) \sqrt{-g}\bigr\},
\Tag{(90.3)}
\]
where $\beta = \alpha/8\lambda$.

The term $\kappa_{\alpha}^{\alpha} \sqrt{-g}$ can be dropped, because by~\Eq{(51.11)}
\[
\kappa_{\alpha}^{\alpha} \sqrt{-g}
= \frac{\dd}{\dd x_{\alpha}} (\kappa^{\alpha} \sqrt{-g}).
\]
This can be integrated, and yields a surface\hyp{}integral over the boundary of the
region considered. Its Hamiltonian derivatives accordingly vanish.

Again
\begin{align*}
  \delta (\kappa_{\alpha} \kappa^{\alpha} \sqrt{-g})
  &= \kappa_{\alpha} \kappa_{\beta}\, \delta(g^{\alpha\beta} \sqrt{-g})
  + g^{\alpha\beta} \sqrt{-g} (\kappa_{\alpha}\, \delta\kappa_{\beta} + \kappa_{\beta}\, \delta\kappa_{\alpha}) \\
  &= \kappa_{\alpha} \kappa_{\beta} \sqrt{-g} (\delta g^{\alpha\beta} + \tfrac{1}{2} g^{\alpha\beta} g^{\mu\nu}\, \delta g_{\mu\nu})
  + 2g^{\alpha\beta} \sqrt{-g}\, \kappa_{\beta}\, \delta\kappa_{\alpha}) \\
  &= \kappa_{\alpha} \kappa_{\beta} \sqrt{-g} (-g^{\mu\alpha} g^{\nu\beta} + \tfrac{1}{2} g^{\alpha\beta} g^{\mu\nu})\, \delta g_{\mu\nu}
  + 2\kappa^{\alpha} \sqrt{-g}\, \delta\kappa_{\alpha} \\
  &= \sqrt{-g} (-\kappa^{\mu} \kappa^{\nu} + \tfrac{1}{2} g^{\mu\nu} \kappa_{\alpha} \kappa^{\alpha})\, \delta g_{\mu\nu}
  + 2\kappa^{\alpha} \sqrt{-g}\, \delta\kappa_{\alpha}.
\end{align*}
Hence
\begin{align*}
  \frac{\Ham}{\Ham g_{\mu\nu}} (\kappa_{\alpha} \kappa^{\alpha})
  &= (-\kappa^{\mu} \kappa^{\nu} + \tfrac{1}{2} g^{\mu\nu} \kappa_{\alpha} \kappa^{\alpha}),
  \Tag{(90.41)} \\
  \frac{\Ham}{\Ham \kappa_{\alpha}} (\kappa_{\alpha} \kappa^{\alpha})
  &= 2\kappa^{\alpha}.
  \Tag{(90.42)}
\end{align*}
Hamiltonian derivatives of the other terms in~\Eq{(90.3)} have already been found
in \Eq{(60.43)}, \Eq{(79.31)} and~\Eq{(79.32)}. Collecting these results we have
\begin{align*}
  \frac{1}{8\lambda}\, \frac{\Ham A}{\Ham g_{\mu\nu}}
  &= -(G^{\mu\nu} - \tfrac{1}{2} g^{\mu\nu} G)
  - 6(\kappa^{\mu} \kappa^{\nu} - \tfrac{1}{2} g^{\mu\nu} \kappa_{\alpha} \kappa^{\alpha})
  - \lambda g^{\mu\nu} - 2\beta E^{\mu\nu} \\
  &= 8\pi T^{\mu\nu} - 2\beta E^{\mu\nu} - 6(\kappa^{\mu} \kappa^{\nu} - \tfrac{1}{2} g^{\mu\nu} \kappa_{\alpha} \kappa^{\alpha})
  \Tag{(90.51)}
\end{align*}
by~\Eq{(54.71)}; and
\[
\frac{1}{8\lambda}\, \frac{\Ham A}{\Ham \kappa_{\mu}}
= 12\kappa^{\mu} + 4\beta J^{\mu}.
\Tag{(90.52)}
\]

If the hypothesis~\Eq{(90.2)} is correct, these must vanish. The vanishing of~\Eq{(90.51)}
shows that the whole energy\hyp{}tensor consists of the electromagnetic
energy\hyp{}tensor together with another term, which must presumably be identified
with the material energy\hyp{}tensor attributable to the binding forces of the
electrons\footnotemark.\footnotetext
  {I doubt if this is the right interpretation. See the end of \SecRef{100}.}
The constant~$2\beta/8\pi$ correlates the natural gravitational and
electromagnetic units. The material energy\hyp{}tensor, being the difference between
the whole tensor and the electromagnetic part, is accordingly
\[
M^{\mu\nu} = \frac{3}{4\pi} (\kappa^{\mu} \kappa^{\nu} - \tfrac{1}{2} g^{\mu\nu} \kappa_{\alpha} \kappa^{\alpha}).
\Tag{(90.61)}
\]
Hence, multiplying by~$g_{\mu\nu}$,
\[
\rho_{0} = M = -\frac{3}{4\pi}\, \kappa_{\alpha} \kappa^{\alpha}.
\Tag{(90.62)}
\]

The vanishing of~\Eq{(90.52)} gives the remarkable equation
\[
\kappa^{\mu} = -\tfrac{1}{3}\beta J^{\mu}.
\Tag{(90.71)}
\]
And since $J_{\mu}^{\mu} = 0$ \Eq{(73.77)}, we must have
\[
\kappa_{\mu}^{\mu} = 0,
\Tag{(90.72)}
\]
agreeing with the original limitation of~$\kappa_{\mu}$ in~\Eq{(74.1)}.

We see that the formula for~$\rho_{0}$ \Eq{(90.62)} agrees with that previously found~\Eq{(89.3)}
having regard to the limitation $\kappa_{\mu}^{\mu} = 0$.

The result~\Eq{(90.62)} becomes by~\Eq{(90.71)}
\[
\rho_{0} = -\frac{\beta^2}{12\pi}\, J_{\mu} J^{\mu}.
\]
This shows that matter cannot be constituted without electric charge and
current. But since the density of matter is always positive, the electric charge\hyp{}and\hyp{}current
inside an electron must be a \emph{spacelike} vector, the square of its
\index{Electron!magnetic constitution of}%
\index{Magnetic constitution of electron}%
length being negative. It would seem to follow that the electron cannot be
built up of elementary electrostatic charges but resolves itself into something
more akin to magnetic charges.

It will be noticed that the result~\Eq{(90.72)} is inconsistent with the formula
$\kappa_{\alpha} \kappa^{\alpha} = \kappa_{\alpha}^{\alpha}$ which we have found for empty space~\Eq{(89.4)}. The explanation is afforded
by~\Eq{(90.71)} which requires that a charge\hyp{}and\hyp{}current vector must exist wherever
$\kappa_{\mu}$~exists, so that no space is really empty. On Weyl's hypothesis $\kappa_{\alpha}^{\alpha} = 0$ is the
condition which holds in all circumstances; whilst the additional condition
$\kappa_{\alpha}^{\alpha} = \kappa_{\alpha} \kappa^{\alpha}$ holding in empty space reduces to the condition expressed by $J^{\alpha} = 0$.
It is supposed that outside what is ordinarily considered to be the boundary
of the electron there is a small charge and current $\dfrac{3}{\beta}\, \kappa^{\alpha}$ extending as far as the
electromagnetic potential extends.

For an isolated electron at rest in Galilean coordinates $\kappa_{4} = e/r$, so that
$\kappa_{\alpha} \kappa^{\alpha} = e^{2}/r^{2}$. On integrating throughout infinite space the result is apparently
infinite; but taking account of the finite radius of space, the result is of order~$e^{2}R$.
By~\Eq{(90.62)} this represents the part of the (negative) mass of the electron\footnote
  {This must not be confused with mass of the energy of the electromagnetic field. The present
  discussion relates to \emph{invariant mass} to which the field contributes nothing.}
which is not concentrated within the nucleus. The actual mass was found in
\SecRef{80} to be of order~$e^{2}/a$ where $a$~is the radius of the nucleus. The two masses
$e^{2}R$ and~$e^{2}/a$ are not immediately comparable since they are expressed in
different units, the connection being made by Weyl's constant~$\beta$ whose value
is left undecided. But since they differ in dimensions of length, they would
presumably become comparable if the natural unit of length were adopted,
viz.\ the radius of the world; in that case $e^{2}/a$~is at least $10^{36}$~times $e^{2}R$, so that
the portion of the mass outside the nucleus is quite insignificant.

The action\hyp{}principle here followed out is obviously speculative. Whether
the results are such as to encourage belief in this or some similar law, or whether
they tend to dispose of it by something like a \Foreign{reductio ad absurdum,} I will
leave to the judgment of the reader. There are, however, two points which
seem to call for special notice---

(1) When we compare the forms of the two principal energy\hyp{}tensors
\begin{align*}
  T_{\mu}^{\nu} &= -\frac{1}{8\pi} \bigl\{G_{\mu}^{\nu} - \tfrac{1}{2} g_{\mu}^{\nu} (G - 2\lambda)\bigr\}, \\
  E_{\mu}^{\nu} &= -F_{\mu\sigma} F^{\nu\sigma} + \tfrac{1}{4} g_{\mu}^{\nu} F_{\alpha\beta} F^{\alpha\beta},
\end{align*}
it is rather a mystery how the second can be contained in the first, since they
seem to be anything but homologous. The connection is simplified by observing
that the difference between them occurs in $\Ham A/\Ham g_{\mu\nu}$ \Eq{(90.51)} accompanied only
by a term which would presumably be insensible except inside the electrons.

But the connection though reduced to simpler terms is not in any way
explained by Weyl's action\hyp{}principle. It is obvious that his action as it stands
has no deep significance; it is a mere stringing together of two in\hyp{}invariants
of different forms. To subtract $F_{\mu\nu}F^{\mu\nu}$ from~$\Star{G}^{2}$ is a fantastic procedure which
has no more theoretical justification than subtracting~$E_{\mu}^{\nu}$ from~$T_{\mu}^{\nu}$. At the
most we can only regard the assumed form of action~$A$ as a step towards some
more natural combination of electromagnetic and gravitational variables.

(2) For the first term of the action, $\Star{G}^{2} \sqrt{-g}$~was chosen instead of the
simpler $\Star{G} \sqrt{-g}$, \emph{because the latter is not an in\hyp{}invariant\hyp{}density} and cannot
be regarded as a measure of any absolute property of the region. It is
interesting to trace how this improvement leads to the appearance of the term
$\delta (-2\lambda \sqrt{-g})$ in~\Eq{(90.3)}, so that the cosmical curvature\hyp{}term in the expression
for the energy\hyp{}tensor now appears quite naturally and inevitably. We may
contrast this with the variation of~$G \sqrt{-g}$ worked out in \SecRef{60}, where no such
term appears. In attributing more fundamental importance to the in\hyp{}invariant
$\Star{G}^{2} \sqrt{-g}$ than to the co\hyp{}invariant $\Star{G} \sqrt{-g}$, Weyl's theory makes an undoubted
advance towards the truth.

\Part{II.}{Generalised Theory}

\Section{91.}{Parallel displacement}
\index{Displacement!parallel}%
\index{Equivalence of displacements}%
\index{Generalisation of Weyl's theory}%
\index{Parallel displacement}%

Let an infinitesimal displacement~$A^{\mu}$ at the point~$P$ (coordinates,~$x_{\mu}$) be
carried by parallel displacement to a point~$P'$ (coordinates, $x_{\mu} + dx_{\mu}$) infinitely
near to~$P$. The most general possible continuous formula for the change of~$A^{\mu}$
is of the form
\[
dA^{\mu} = -\Gamma_{\nu\alpha}^{\mu} A^{\alpha}\, dx_{\nu},
\Tag{(91.1)}
\]
where $\Gamma_{\nu\alpha}^{\mu}$, which is not assumed to be a tensor, represents $64$~arbitrary
coefficients. Both $A^{\alpha}$ and~$dx_{\nu}$ are infinitesimals, so that there is no need to
insert any terms of higher order.

We are going to build the theory afresh starting from this notion of
infinitesimal parallel displacement; and by so doing we arrive at a generalisation
even wider than that of Weyl. Our fundamental axiom is that parallel
displacement has some significance in regard to the ultimate structure of the
world---it does not much matter what significance. The idea is that out of the
whole group of displacements radiating from~$P'$, we can select one $A^{\mu} + dA^{\mu}$
which has some kind of \emph{equivalence} to the displacement~$A^{\mu}$ at~$P$. We do not
define the nature of this equivalence, except that it shall have reference to the
part played by~$A^{\mu}$ in the relation\hyp{}structure which underlies the world of physics.
Notice that---

(1) This equivalence is only supposed to exist in the limit when $P$ and~$P'$
are infinitely near together. For more distant points equivalence can in general
only be approximate, and gradually becomes indeterminate as the distance is
increased. It can be made determinate by specifying a particular route of
connection, in which case the equivalence is traced step by step along the
route.

(2) The equivalence is not supposed to exist between any world\hyp{}relations
other than displacements. Hitherto we have applied parallel displacement to
any tensor, but in this theory we only use it for displacements.

(3) It is not assumed that there is any complete observational test of
equivalence. This is rather a difficult point which will be better appreciated
later. The idea is that the scheme of equivalence need not be determinate
observationally, and may have permissible transformations; just as the scheme
of coordinate\hyp{}reckoning is not determinate observationally and is subject to
transformations.

Let $PP_{1}$ represent the displacement $A^{\mu} = \delta x_{\mu}$ which on parallel displacement
to~$P'$ becomes~$P'P_{1}'$; then by~\Eq{(91.1)} the difference of coordinates of~$P_{1}'$
and~$P_{1}$ is
\[
A^{\mu} + dA^{\mu} = \delta x_{\mu} - \Gamma_{\nu\alpha}^{\mu}\, \delta x_{\alpha}\, dx_{\nu},
\]
so that the coordinates of~$P_{1}'$ relative to~$P$ are
\[
dx_{\mu} + \delta x_{\mu} - \Gamma_{\nu\alpha}^{\mu}\, \delta x_{\alpha}\, dx_{\nu}.
\Tag{(91.2)}
\]
Interchanging the two displacements, i.e.\ displacing $PP'$ along~$PP_{1}$ we shall
not arrive at the same point~$P_{1}'$ unless
\[
\Gamma_{\nu\alpha}^{\mu} = \Gamma_{\alpha\nu}^{\mu}.
\Tag{(91.3)}
\]
When \Eq{(91.3)}~is satisfied we have the parallelogram law, that if a displacement
$AB$ is equivalent to~$CD$, then $AC$~is equivalent to~$BD$.

This is the necessary condition for what is called \emph{affine geometry}. It is
\index{Affine geometry}%
\index{Geometry, Riemannian!affine geometry}%
adopted by Weyl and other writers; but J.~A. Schouten in a purely geometrical
investigation has dispensed with it. I shall adopt it here.

All questions of the fundamental axioms of a science are difficult. In
general we have to start somewhat above the fundamental plane and develop
the theory backwards towards fundamentals as well as forwards to results. I
shall defer until \SecRef{98} the examination of how far the axiom of parallel displacement
and the condition of affine geometry are essential in translating the
properties of a relation\hyp{}structure into mathematical expression; and I proceed
at once to develop the consequences of the specification here introduced.

By the symmetry condition the number of independent~$\Gamma_{\nu\alpha}^{\mu}$ is reduced to~$40$,
variable from point to point of space. They are descriptive of the relation\hyp{}structure
of the world, and should contain all that is relevant to physics. Our
immediate problem is to show how the more familiar variables of physics can
be extracted from this crude material.

\Section{92.}{Displacement round an infinitesimal circuit}
\index{Parallelogram\hyp{}law}%

Let a displacement~$A^{\mu}$ be carried by parallel displacement round a small
circuit~$C$. The condition for parallel displacement is by~\Eq{(91.1)}
\[
\frac{\dd A^{\mu}}{\dd x_{\nu}} = -\Gamma_{\nu\alpha}^{\mu} A^{\alpha}.
\Tag{(92.1)}
\]
Hence the difference of the initial and final values is
\begin{align*}
  \delta A^{\mu}
  &= \int_{C} \frac{\dd A^{\mu}}{\dd x_{\nu}}\, dx_{\nu} \\
  &= -\int_{C} \Gamma_{\nu\alpha}^{\mu} A^{\alpha}\, dx_{\nu} \\
  &= \frac{1}{2} \iint \left\{
  \frac{\dd}{\dd x_{\sigma}} (\Gamma_{\nu\alpha}^{\mu} A^{\alpha})
  - \frac{\dd}{\dd x_{\nu}} (\Gamma_{\sigma\alpha}^{\mu} A^{\alpha})\right\} dS^{\nu\sigma}
\end{align*}
by Stokes's theorem~\Eq{(32.3)}.
\index{Stokes's theorem!application of}%

The integrand is equal to
\begin{align*}
  &\phantom{{}={}} A^{\alpha} \left(\frac{\dd}{\dd x_{\sigma}} \Gamma_{\nu\alpha}^{\mu}
  - \frac{\dd}{\dd x_{\nu}} \Gamma_{\sigma\alpha}^{\mu}\right)
  + \Gamma_{\nu\alpha}^{\mu}\, \frac{\dd A^{\alpha}}{\dd x_{\sigma}}
  - \Gamma_{\sigma\alpha}^{\mu}\, \frac{\dd A^{\alpha}}{\dd x_{\nu}} \\
  &= A^{\epsilon} \left(\frac{\dd}{\dd x_{\sigma}} \Gamma_{\nu\epsilon}^{\mu}
  - \frac{\dd}{\dd x_{\nu}} \Gamma_{\sigma\epsilon}^{\mu}\right)
  - \Gamma_{\nu\alpha}^{\mu} \Gamma_{\sigma\epsilon}^{\alpha} A^{\epsilon}
  + \Gamma_{\sigma\alpha}^{\mu} \Gamma_{\nu\epsilon}^{\alpha} A^{\epsilon}
  \quad\text{by~\Eq{(92.1)}} \\
  &= -\Star{B}_{\epsilon\nu\sigma}^{\mu} A^{\epsilon},
\end{align*}
where
\[
\Star{B}_{\epsilon\nu\sigma}^{\mu}
= -\frac{\dd}{\dd x_{\sigma}} \Gamma_{\nu\epsilon}^{\mu}
+ \frac{\dd}{\dd x_{\nu}} \Gamma_{\sigma\epsilon}^{\mu}
+ \Gamma_{\nu\alpha}^{\mu} \Gamma_{\sigma\epsilon}^{\alpha}
- \Gamma_{\sigma\alpha}^{\mu} \Gamma_{\nu\epsilon}^{\alpha}.
\Tag{(92.2)}
\]

Hence
\[
\delta A^{\mu} = -\frac{1}{2} \iint \Star{B}_{\epsilon\nu\sigma}^{\mu} A^{\epsilon}\, dS^{\nu\sigma}.
\Tag{(92.31)}
\]

As in \SecRef{33} the formula applies only to infinitesimal circuits. In evaluating
the integrand we assumed that $A^{\alpha}$~satisfies the condition of parallel displacement~\Eq{(92.1)}
not only on the boundary but at all points within the circuit. No
single value of~$A^{\alpha}$ can satisfy this, since if it holds for one circuit of displacement
it will not hold for a second. But the discrepancies are of order proportional
to~$dS^{\nu\sigma}$, and another factor~$dS^{\nu\sigma}$ occurs in the integration; hence \Eq{(92.31)}~is
true when the square of the area of the circuit can be neglected.

Writing $\Sigma^{\nu\sigma} = \iint dS^{\nu\sigma}$ for a small circuit, \Eq{(92.31)}~approaches the limit
\[
\delta A^{\mu} = -\tfrac{1}{2}\, \Star{B}_{\epsilon\nu\sigma}^{\mu} A^{\epsilon} \Sigma^{\nu\sigma},
\Tag{(92.32)}
\]
which shows that $\Star{B}_{\epsilon\nu\sigma}^{\mu}$~is a tensor\footnotemark.\footnotetext
  {Another independent proof that $\Star{B}_{\epsilon\nu\sigma}^{\mu}$~is a tensor is obtained in equation~\Eq{(94.1)}; so that if
  the reader is uneasy about the rigour of the preceding analysis, he may regard it as merely
  suggesting consideration of the expression~\Eq{(92.2)} and use the alternative proof that it is a tensor.}
Moreover it is an in\hyp{}tensor, since we have
not yet introduced any gauge. In fact all quantities introduced at present
must have the ``in-''~property, for we have not begun to discuss the conception
of length.

We can form an in\hyp{}tensor of the second rank by contraction. With the
\index{Riemann\hyp{}Christoffel tensor!generalisation of}%
more familiar arrangement of suffixes,
\begin{align*}
  \Star{B}_{\mu\nu\sigma}^{\epsilon}
  &= -\frac{\dd}{\dd x_{\sigma}} \Gamma_{\nu\mu}^{\epsilon}
  + \frac{\dd}{\dd x_{\nu}} \Gamma_{\sigma\mu}^{\epsilon}
  + \Gamma_{\sigma\mu}^{\alpha} \Gamma_{\nu\alpha}^{\epsilon}
  - \Gamma_{\nu\mu}^{\alpha} \Gamma_{\sigma\alpha}^{\epsilon},
  \Tag{(92.41)} \\
  \Star{G}^{\mu\nu}
  &= -\frac{\dd}{\dd x_{\alpha}} \Gamma_{\nu\mu}^{\alpha}
  + \frac{\dd}{\dd x_{\nu}} \Gamma_{\alpha\mu}^{\alpha}
  + \Gamma_{\beta\mu}^{\alpha} \Gamma_{\nu\alpha}^{\beta}
  - \Gamma_{\nu\mu}^{\alpha} \Gamma_{\beta\alpha}^{\beta}.
  \Tag{(92.42)}
\end{align*}
Another contracted in\hyp{}tensor is obtained by setting $\epsilon = \mu$, viz.\
\[
-2F_{\nu\sigma}
= -\frac{\dd}{\dd x_{\sigma}} \Gamma_{\nu\alpha}^{\alpha}
+ \frac{\dd}{\dd x_{\nu}} \Gamma_{\sigma\alpha}^{\alpha}.
\Tag{(92.43)}
\]

We shall write
\[
\Gamma_{\nu} \equiv \Gamma_{\nu\alpha}^{\alpha}.
\Tag{(92.5)}
\]

Then
\[
2F_{\nu\sigma}
= \frac{\dd\Gamma_{\nu}}{\dd x_{\sigma}} - \frac{\dd\Gamma_{\sigma}}{\dd x_{\nu}}.
\Tag{(92.55)}
\]

It will be seen from~\Eq{(92.42)} that\footnote
  {Here for the first time we make use of the symmetrical property of~$\Gamma_{\mu\nu}^{\alpha}$. If $\Gamma_{\mu\nu}^{\alpha} \neq \Gamma_{\nu\mu}^{\alpha}$ the
  analysis at this point becomes highly complicated.}
\[
\Star{G}_{\mu\nu} - \Star{G}_{\nu\mu}
= \frac{\dd\Gamma_{\mu}}{\dd x_{\nu}} - \frac{\dd\Gamma_{\nu}}{\dd x_{\mu}}
= 2F_{\mu\nu},
\Tag{(92.6)}
\]
so that $F_{\mu\nu}$~is the antisymmetrical part of~$\Star{G}_{\mu\nu}$. Thus the second mode of
contraction of~$\Star{B}_{\mu\nu\sigma}^{\epsilon}$ does not add anything not obtainable by the first mode,
and we need not give $F_{\mu\nu}$ separate consideration.

According to this mode of development the in\hyp{}tensors $\Star{B}_{\mu\nu\sigma}^{\epsilon}$ and~$\Star{G}_{\mu\nu}$ are
\index{In\hyp{}tensors!fundamental}%
the most fundamental measures of the intrinsic structure of the world. They
take precedence of the~$g_{\mu\nu}$, which are only found at a later stage in our theory.
Notice that we are not yet in a position to raise or lower a suffix, or to define
an invariant such as~$\Star{G}$, because we have no~$g_{\mu\nu}$. If we wish at this stage to
form an invariant of a four\hyp{}dimensional region we must take its ``generalised
volume''
\[
\iiiint \sqrt{-|\Star{G}_{\mu\nu}|}\, d\tau,
\]
which is accordingly more elementary than the other regional invariants
enumerated in \SecRef{88}.

It may be asked whether there is any other way of obtaining tensors,
besides the consideration of parallel displacement round a closed circuit. I
think not; because unless our succession of displacements takes us back to the
starting\hyp{}point, we are left with initial and final displacements at a distance,
between which no comparability exists.

The equation~\Eq{(92.55)} does not prove immediately that $F_{\mu\nu}$~is the curl of a
vector, because, notwithstanding the notation, $\Gamma_{\mu}$~is not usually a vector. But
since $F_{\mu\nu}$~is a tensor
\begin{align*}
  2F_{\alpha\beta}'
  &= 2F_{\mu\nu}\, \frac{\dd x_{\mu}}{\dd x_{\alpha}'}\, \frac{\dd x_{\nu}}{\dd x_{\beta}'} \\
  &= \frac{\dd\Gamma_{\mu}}{\dd x_{\nu}}\, \frac{\dd x_{\nu}}{\dd x_{\beta}'}\, \frac{\dd x_{\mu}}{\dd x_{\alpha}'}
  - \frac{\dd\Gamma_{\nu}}{\dd x_{\mu}}\, \frac{\dd x_{\mu}}{\dd x_{\alpha}'}\, \frac{\dd x_{\nu}}{\dd x_{\beta}'} \\
  &= \frac{\dd}{\dd x_{\beta}'} \biggl(\Gamma_{\mu}\, \frac{\dd x_{\mu}}{\dd x_{\alpha}'}\biggr)
  - \frac{\dd}{\dd x_{\alpha}'} \biggl(\Gamma_{\nu}\, \frac{\dd x_{\nu}}{\dd x_{\beta}'}\biggr).
\end{align*}
Now by~\Eq{(23.12)} $\Gamma_{\mu}\, \dd x_{\mu}/\dd x_{\alpha}'$ is a vector. Let us denote it by~$2\kappa_{\alpha}'$. Then
\[
F_{\alpha\beta}'
= \frac{\dd\kappa_{\alpha}'}{\dd x_{\beta}'} - \frac{\dd\kappa_{\beta}'}{\dd x_{\alpha}'}.
\]
Thus $F_{\alpha\beta}'$~is actually the curl of a vector~$\kappa_{\alpha}'$, though that vector is not necessarily
equal to~$\Gamma_{\alpha}'$ in all systems of coordinates. The general solution of
\[
\frac{1}{2} \left(\frac{\dd\Gamma_{\alpha}'}{\dd x_{\beta}'} - \frac{\dd\Gamma_{\beta}'}{\dd x_{\alpha}'}\right)
= \frac{\dd\kappa_{\alpha}'}{\dd x_{\beta}'} - \frac{\dd\kappa_{\beta}'}{\dd x_{\alpha}'}
\]
is
\[
\Gamma_{\alpha}' = 2\kappa_{\alpha}' + \frac{\dd\Omega}{\dd x_{\alpha}'},
\Tag{(92.7)}
\]
and since $\Omega$~need not be an invariant, $\Gamma_{\alpha}'$~is not a vector.

\Section{93.}{Introduction of a metric}
\index{Metric!introduction of}%

Up to this point the interval~$ds$ between two points has not appeared in
our theory. It will be remembered that the interval is the length of the corresponding
displacement, and we have to consider how a length (an invariant)
is to be assigned to a displacement~$dx_{\mu}$ (a contravariant in\hyp{}vector). In this
section we shall assign it by the convention
\[
ds^{2} = g_{\mu\nu}\, dx_{\mu}\, dx_{\nu}.
\Tag{(93.11)}
\]
Here $g_{\mu\nu}$~must be a tensor, in order that the interval may be an invariant;
but the tensor is chosen by us arbitrarily.

The adoption of a particular tensor~$g_{\mu\nu}$ is equivalent to assigning a particular
gauge\hyp{}system---a system by which a unique measure is assigned to the interval
\index{Gauge\hyp{}system}%
between every two points. In Weyl's theory, a gauge\hyp{}system is partly physical
and partly conventional; lengths in different directions but at the same point
are supposed to be compared by experimental (optical) methods; but lengths
at different points are not supposed to be comparable by physical methods
(transfer of clocks and rods) and the unit of length at each point is laid down
\index{Length!definition of}%
by a convention. I think that this hybrid definition of length is undesirable,
and that length should be treated as a purely conventional or else a purely
physical conception. In the present section we treat it as a purely conventional
invariant whose properties we wish to discuss, so that length as here
defined is not anything which has to be consistent with ordinary physical tests.
Later on we shall consider how $g_{\mu\nu}$~must be chosen in order that conventional
length may obey the recognised physical tests and thereby become physical
length; but at present the tensor~$g_{\mu\nu}$ is unrestricted.

Without any loss of generality, we may take $g_{\mu\nu}$ to be a symmetrical tensor,
since any antisymmetrical part would drop out on multiplication by~$dx_{\mu}\, dx_{\nu}$
and would be meaningless in~\Eq{(93.11)}.

Let $l$~be the length of a displacement~$A^{\mu}$, so that
\[
l^{2} = g_{\mu\nu} A^{\mu} A^{\nu}.
\Tag{(93.12)}
\]
Move $A^{\mu}$ by parallel displacement through~$dx_{\sigma}$, then
\begin{align*}
  d(l^{2})
  &= \left(\frac{\dd g_{\mu\nu}}{\dd x_{\sigma}} A^{\mu} A^{\nu}
  + g_{\mu\nu} A^{\nu}\, \frac{\dd A^{\mu}}{\dd x_{\sigma}}
  + g_{\mu\nu} A^{\mu}\, \frac{\dd A^{\nu}}{\dd x_{\sigma}}\right) dx_{\sigma} \\
  &= \left(\frac{\dd g_{\mu\nu}}{\dd x_{\sigma}} A^{\mu} A^{\nu}
  - g_{\mu\nu} A^{\nu}\, \Gamma_{\sigma\alpha}^{\mu} A^{\alpha}
  - g_{\mu\nu} A^{\mu}\, \Gamma_{\sigma\alpha}^{\nu} A^{\alpha}\right) dx_{\sigma}
  \quad\text{by~\Eq{(91.1)}} \\
  &= \left(\frac{\dd g_{\mu\nu}}{\dd x_{\sigma}}
  - g_{\alpha\nu}\, \Gamma_{\sigma\mu}^{\alpha}
  - g_{\mu\alpha}\, \Gamma_{\sigma\nu}^{\alpha}\right) A^{\mu} A^{\nu}\, dx_{\sigma}
\end{align*}
by interchanging dummy suffixes.

In conformity with the usual rule for lowering suffixes, we write
\[
\Gamma_{\sigma\mu,\nu} = g_{\alpha\nu}\, \Gamma_{\sigma\mu}^{\alpha},
\]
so that
\[
d(l^{2}) = \left(\frac{\dd g_{\mu\nu}}{\dd x_{\sigma}}
  - \Gamma_{\sigma\mu,\nu} - \Gamma_{\sigma\nu,\mu}\right) A^{\mu} A^{\nu}\, (dx)^{\sigma}.
\Tag{(93.2)}
\]

But $d(l^{2})$, the difference of two invariants, is an invariant. Hence the
quantity in the bracket is a covariant tensor of the third rank which is evidently
symmetrical in $\mu$ and~$\nu$. We denote it by~$2\Kappa_{\mu\nu,\sigma}$. Thus
\begin{align*}
  2\Kappa_{\mu\nu,\sigma} = \frac{\dd g_{\mu\nu}}{\dd x_{\sigma}}
  - \Gamma_{\sigma\mu,\nu} - \Gamma_{\sigma\nu,\mu}.
  \Tag{(93.3)}\displaybreak[0] \\
\intertext{Similarly}
  2\Kappa_{\mu\sigma,\nu} = \frac{\dd g_{\mu\sigma}}{\dd x_{\nu}}
  - \Gamma_{\nu\mu,\sigma} - \Gamma_{\nu\sigma,\mu}, \\
  2\Kappa_{\nu\sigma,\mu} = \frac{\dd g_{\nu\sigma}}{\dd x_{\mu}}
  - \Gamma_{\mu\nu,\sigma} - \Gamma_{\mu\sigma,\nu}.
\end{align*}
Adding these and subtracting~\Eq{(93.3)} we have
\index{Three\hyp{}index symbol!generalised}%
\[
\Kappa_{\mu\sigma,\nu} + \Kappa_{\nu\sigma,\mu} - \Kappa_{\mu\nu,\sigma}
= \frac{1}{2} \left(\frac{\dd g_{\mu\sigma}}{\dd x_{\nu}}
+ \frac{\dd g_{\nu\sigma}}{\dd x_{\mu}}
- \frac{\dd g_{\mu\nu}}{\dd x_{\sigma}}\right) - \Gamma_{\mu\nu,\sigma}.
\Tag{(93.4)}
\]
Let
\[
S_{\mu\nu,\sigma}
= \Kappa_{\mu\nu,\sigma} - \Kappa_{\mu\sigma,\nu} - \Kappa_{\nu\sigma,\mu}.
\Tag{(93.5)}
\]
Then \Eq{(93.4)}~becomes
\[
\Gamma_{\mu\nu,\sigma} = [\mu\nu, \sigma] + S_{\mu\nu,\sigma},
\]
so that, raising the suffix,
\[
\Gamma_{\mu\nu}^{\sigma} = \{\mu\nu, \sigma\} + S_{\mu\nu}^{\sigma}.
\Tag{(93.6)}
\]
If $\Kappa_{\mu\nu,\sigma}$~has the particular form~$g_{\mu\nu} \kappa_{\sigma}$,
\[
S_{\mu\nu}^{\sigma} = g_{\mu\nu} \kappa^{\sigma} - g_{\mu}^{\sigma} \kappa_{\nu} - g_{\nu}^{\sigma} \kappa_{\mu},
\]
so that \Eq{(93.6)}~reduces to~\Eq{(86.2)} with $\Gamma_{\mu\nu}^{\sigma} = \Star{}\{\mu\nu, \sigma\}$.

Thus Weyl's geometry is a particular case of our general geometry of
parallel displacement. His restriction $K_{\mu\nu,\sigma} = g_{\mu\nu} \kappa_{\sigma}$ is equivalent to that already
explained in \SecRef{84}.

The formula~$\Gamma^\sigma_{\mu\nu} = \{\mu\nu,\sigma\} + S^\sigma_{\mu\nu}$ enables us to pass easily from
results obtained in metrical geometry to the corresponding results in affine geometry.
For example, corresponding to a tensor~$A_{\mu\nu}$ we know that there is a tensor~$A_{\mu\nu\sigma}$ given by
\[
A_{\mu\nu\sigma} =
   \frac{\dd A_{\mu\nu}}{\dd x_\sigma} - \{\mu\sigma,\epsilon\}A_{\epsilon\nu} - \{\nu\sigma,\epsilon\}A_{\mu\epsilon}.
\Tag{(93.7)}
\]
It follows at once that corresponding to an in\hyp{}tensor~$A_{\mu\nu}$ there will be an
in\hyp{}tensor~$(A_{\mu\nu})_\sigma$ given by
\[
(A_{\mu\nu})_\sigma =
   \frac{\dd A_{\mu\nu}}{\dd x_\sigma}
     - \Gamma^\epsilon_{\mu\sigma}A_{\epsilon\nu} - \Gamma^\epsilon_{\nu\sigma}A_{\mu\epsilon},
\Tag{(93.71)}
\]
since the difference of~\Eq{(93.7)} and~\Eq{(93.71)} is seen to be a tensor.

But in developing an affine geometry in which metrical conceptions play no part, it is not very satisfactory
to prove our theorems by introducing a provisional metric which is ultimately eliminated.
The proofs are valid, but they remind us of conceptions which we wish to keep out of our heads.

It is therefore desirable to notice that the operation of affine (or in\hyp{}covariant) differentiation can be
introduced without reference to any metric, provisional or otherwise.
In a vector\hyp{}field the difference between the \emph{actual} vector~$A^\mu + dA^\mu$ at~$x_\mu+dx_\mu$
and the vector~$A^\mu+\delta A^\mu$ at the same point \emph{equivalent} to~$A^\mu$ at~$x_\mu$, is a vector.
Hence by~\Eq{(91.1)}
\[
dA^\mu + \Gamma^\mu_{\nu\alpha} A^\alpha dx_\nu
\]
is a vector.
It follows that
\[
\frac{\dd A^\mu}{\dd x_\nu} + \Gamma^\mu_{\nu\alpha}A^\alpha,
\Tag{(93.8)}
\]
is a tensor, which we shall call the affine derivative of~$A^\mu$ and denote by~$(A^\mu)_\nu$.
Affine derivatives of other kinds of tensors are defined by the rules,
(a) the affine derivative of an invariant is its ordinary derivative,
(b) the affine derivative of a product is formed by the usual distributive rule.
These rules secure that the quantities so defined are tensors. E.g.
\begin{align*}
\frac{\dd}{\dd x_\sigma}(A_{\mu\nu} B^\mu C^\nu) & = (A_{\mu\nu} B^\mu C^\nu)_\sigma \, \text{by Rule (a)}\\
 & = (A_{\mu\nu})_\sigma B^\mu C^\nu + A_{\mu\nu} (B^\mu)_\sigma C^\nu + A_{\mu\nu} B^\mu (C^\nu)_\sigma \, \text{by Rule (b)}.
\end{align*}
Substituting for~$(B^\mu)_\sigma$ and~$(C^\nu)_\sigma$ according to~\Eq{(93.8)},
we find that $(A_{\mu\nu})_\sigma$ is the expression~\Eq{(93.71)}.

It will be found that
\[
((A_\mu)_\nu)_\sigma - ((A_\mu)_\sigma)_\nu) = \Star{B}^\epsilon_{\mu\nu\sigma}A_\epsilon,
\Tag{(93.9)}
\]
giving an immediate proof that~$\Star{B}^\epsilon_{\mu\nu\sigma}$ is an in\hyp{}tensor (cf.~\Eq{(34.3)}).

Although there is no constant relation between tensors and corresponding tensor\hyp{}densities when the
metrical quantity~$\sqrt{-g}$ does not exist, nevertheless tensor\hyp{}densities appear in the affine calculus
indepenfiently of any metrical conceptions.
This is because there exists a purely numerical
tensor\hyp{}density~$\mf{E}^{\alpha\beta\gamma\delta}=\epsilon_{\alpha\beta\gamma\delta}$ \Eq{(49.51)} which
obviously has no reference to a particular metric.

Affine derivatives of tensor\hyp{}densities may be formed, thus
\[
(\mf{A}^{\mu\nu})_\sigma = \frac{\dd\mf{A}^{\mu\nu}}{\dd x_\sigma}
                         +\Gamma^\mu_{\epsilon\sigma}\mf{A}^{\epsilon\nu}
                         +\Gamma^\nu_{\epsilon\sigma}\mf{A}^{\mu\epsilon}
                         -\Gamma^\epsilon_{\epsilon\sigma}\mf{A}^{\mu\nu},
\Tag{(93.91)}
\]
corresponding to
\[
\mf{A}^{\mu\nu}_\sigma = \frac{\dd\mf{A}^{\mu\nu}}{\dd x_\sigma}
                         +\{\epsilon\sigma,\mu\}\mf{A}^{\epsilon\nu}
                         +\{\epsilon\sigma,\nu\}\mf{A}^{\mu\epsilon}
                         -\{\epsilon\sigma,\epsilon\}\mf{A}^{\mu\nu},
\Tag{(93.92)}
\]
in the metrical calculus.

It is not difficult to verify that there exists a cyclic relation between the affine derivatives of the generalised
Riemann\hyp{}Christoffel tensor, which corresponds to~\Eq{(52.6)}, viz.
\[
(\Star{B}^\epsilon_{\mu\nu\sigma})_\tau +
(\Star{B}^\epsilon_{\mu\sigma\tau})_\nu +
(\Star{B}^\epsilon_{\mu\tau\nu})_\sigma = 0
\Tag{(93.93)}
\]

The tensor~$2K_{\mu\nu,\sigma}$ introduced in~\Eq{(93.3)} has now a simple geometrical interpretation as the
affine derivative of~$g_{\mu\nu}$.


\Section{94.}{Evaluation of the fundamental in\hyp{}tensors}

In~\Eq{(92.41)} $\Star{B}_{\mu\nu\sigma}^{\epsilon}$~is expressed in terms of the non\hyp{}tensor quantities~$\Gamma_{\mu\nu}^{\sigma}$.
By means of~\Eq{(93.6)} it can now be expressed in terms of tensors~$g_{\mu\nu}$ and~$S_{\mu\nu}^{\sigma}$.
Making the substitution the result is
\begin{multline*}
  \Star{B}_{\mu\nu\sigma}^{\epsilon}
  = -\frac{\dd}{\dd x_{\sigma}} \{\mu\nu, \epsilon\}
  + \frac{\dd}{\dd x_{\nu}} \{\mu\sigma, \epsilon\}
  + \{\mu\sigma, \alpha\} \{\nu\alpha, \epsilon\}
  - \{\mu\nu, \alpha\} \{\sigma\alpha, \epsilon\} \\
  - \frac{\dd}{\dd x_{\sigma}} S_{\mu\nu}^{\epsilon}
  + \frac{\dd}{\dd x_{\nu}} S_{\mu\sigma}^{\epsilon}
  + S_{\mu\sigma}^{\alpha} \{\nu\alpha, \epsilon\}
  + S_{\nu\alpha}^{\epsilon} \{\mu\sigma, \alpha\}
  - S_{\mu\nu}^{\alpha} \{\sigma\alpha, \epsilon\}
  - S_{\sigma\alpha}^{\epsilon} \{\mu\nu, \alpha\} \\
  + S_{\mu\sigma}^{\alpha} S_{\nu\alpha}^{\epsilon}
  - S_{\mu\nu}^{\alpha} S_{\sigma\alpha}^{\epsilon}.
\end{multline*}
The first four terms give the ordinary Riemann\hyp{}Christoffel tensor~\Eq{(34.4)}. The
next six terms reduce to
\[
-(S_{\mu\nu}^{\epsilon})_{\sigma} + (S_{\mu\sigma}^{\epsilon})_{\nu},
\]
where the final suffix represents ordinary covariant differentiation (not in\hyp{}covariant
differentiation), viz.\ by~\Eq{(30.4)},
\[
(S_{\mu\nu}^{\epsilon})_{\sigma}
= \frac{\dd}{\dd x_{\sigma}} S_{\mu\nu}^{\epsilon}
- \{\mu\sigma, \alpha\} S_{\alpha\nu}^{\epsilon}
- \{\nu\sigma, \alpha\} S_{\mu\alpha}^{\epsilon}
+ \{\alpha\sigma, \epsilon\}  S_{\mu\nu}^{\alpha} .
\]

Hence
\[
\Star{B}_{\mu\nu\sigma}^{\epsilon}
= B_{\mu\nu\sigma}^{\epsilon} - (S_{\mu\nu}^{\epsilon})_{\sigma} + (S_{\mu\sigma}^{\epsilon})_{\nu}
+ S_{\nu\sigma}^{\alpha} S_{\nu\alpha}^{\epsilon} - S_{\mu\nu}^{\alpha} S_{\sigma\alpha}^{\epsilon}.
\Tag{(94.1)}
\]
This form makes its tensor\hyp{}property obvious, whereas the form \Eq{(92.41)} made its ``in-''~property obvious.

We next contract by setting $\epsilon = \sigma$ and write
\[
S_{\mu\alpha}^{\alpha} = 2\kappa_{\mu},
\Tag{(94.2)}
\]
obtaining
\[
\Star{G}_{\mu\nu} = G_{\mu\nu} - (S_{\mu\nu}^{\alpha})_{\alpha}
+ 2\kappa_{\mu\nu} + S_{\mu\beta}^{\alpha} S_{\nu\alpha}^{\beta}
- 2\kappa_{\alpha} S_{\mu\nu}^{\alpha}.
\Tag{(94.3)}
\]

Again, multiplying by~$g^{\mu\nu}$,
\[
\Star{G} = G + 2\lambda_{\alpha}^{\alpha} + 2\kappa_{\alpha}^{\alpha}
+ 4\kappa_{\alpha} \lambda^{\alpha} + S_{\gamma}^{\alpha\beta} S_{\alpha,\beta}^{\gamma},
\Tag{(94.4)}
\]
where we have set
\[
S_{\alpha,\mu}^{\alpha} = -2\lambda_{\mu}.
\Tag{(94.5)}
\]

The difference between \Eq{(94.5)} and~\Eq{(94.2)} is that $\lambda_{\mu}$~is formed by equating
the two symmetrical suffixes, and $\kappa_{\mu}$~by equating one of the symmetrical
suffixes with the third suffix in the $S$-tensor. $\kappa_{\mu}$~and $\lambda_{\mu}$ are, of course, entirely
different vectors.

The only term on the right of~\Eq{(94.3)} which is not symmetrical in $\mu$ and~$\nu$
is~$2\kappa_{\mu\nu}$. We write
\begin{align*}
  R_{\mu\nu} &= G_{\mu\nu} + (\kappa_{\mu\nu} + \kappa_{\nu\mu})
  - (S_{\mu\nu}^{\alpha})_{\alpha} - 2\kappa_{\alpha} S_{\mu\nu}^{\alpha}
  + S_{\mu\beta}^{\alpha} S_{\nu\alpha}^{\beta},
  \Tag{(94.61)} \\
  F_{\mu\nu} &= \kappa_{\mu\nu} - \kappa_{\nu\mu},
  \Tag{(94.62)}
\end{align*}
so that
\[
\Star{G}_{\mu\nu} = R_{\mu\nu} + F_{\mu\nu},
\Tag{(94.63)}
\]
and $R_{\mu\nu}$ and~$F_{\mu\nu}$ are respectively its symmetrical and antisymmetrical parts.
Evidently $R_{\mu\nu}$~and $F_{\mu\nu}$ will both be in\hyp{}tensors.

We can also set
\[
\Star{B}_{\mu\nu\sigma\epsilon} = R_{\mu\nu\sigma\epsilon} + F_{\mu\nu\sigma\epsilon},
\]
where $R$~is antisymmetrical and $F$~is symmetrical in $\mu$ and~$\epsilon$. We find that
\index{F@$F_{\mu\nu}$ (electromagnetic force)}%
\[
F_{\mu\nu\sigma\epsilon} = (\Kappa_{\mu\epsilon,\nu})_{\sigma} - (\Kappa_{\mu\epsilon,\sigma})_{\nu},
\]
a result which is of interest in connection with the discussion of \SecRef{84}. But
$R_{\mu\nu\sigma\epsilon}$ and $F_{\mu\nu\sigma\epsilon}$ are not in\hyp{}tensors, since the $g_{\mu\nu}$ are needed to lower the suffix~$\epsilon$.

By \Eq{(92.5)} and~\Eq{(93.6)}
\begin{align*}
  \Gamma_{\mu} = \Gamma_{\mu\alpha}^{\alpha}
  &= \{\mu\alpha, \alpha\} + S_{\mu\alpha}^{\alpha} \\
  &= \frac{\dd}{\dd x_{\mu}} (\log \sqrt{-g}) + 2\kappa_{\mu}.
  \Tag{(94.7)}
\end{align*}
By comparison with~\Eq{(92.7)} we see that the indeterminate function~$\Omega$ is
$\log \sqrt{-g}$, which is not an invariant.

\Section{95.}{The natural gauge of the world}
\index{Gauging\hyp{}equation}%
\index{Natural coordinates!gauge}%

We now introduce the natural gauge of the world. The tensor~$g_{\mu\nu}$, which
has hitherto been arbitrary, must be chosen so that the lengths of displacements
agree with the lengths determined by measurements made with material
and optical appliances. Any apparatus used to measure the world is itself part
of the world, so that the natural gauge represents the world as self\hyp{}gauging.
\index{R@$R_{\mu\nu}$ (gauging\hyp{}tensor)}%
This can only mean that the tensor~$g_{\mu\nu}$ which defines the natural gauge is
not extraneous, but is a tensor already contained in the world\hyp{}geometry. Only
one such tensor of the second rank has been found, viz.~$\Star{G}_{\mu\nu}$. Hence natural
length is given by
\[
l^{2} = \Star{G}_{\mu\nu} A^{\mu} A^{\nu}.
\]
The antisymmetrical part drops out, giving
\[
l^{2} = R_{\mu\nu} A^{\mu} A^{\nu}.
\]
Accordingly by~\Eq{(93.12)} we must take
\[
\lambda g_{\mu\nu} = R_{\mu\nu},
\Tag{(95.1)}
\]
introducing a universal constant~$\lambda$, in order to remain free to use the centimetre
instead of the natural unit of length whose ratio to familiar standards
is unknown.

The manner in which the tensor~$R_{\mu\nu}$ is transferred \Foreign{via} material structure
to the measurements made with material structure, has been discussed in
\SecRef{66}. We have to replace the tensor~$G_{\mu\nu}$ used in that section by its more
general form~$R_{\mu\nu}$, since $G_{\mu\nu}$~is not an in\hyp{}tensor and has no definite value
until \emph{after} the gauging\hyp{}equation~\Eq{(95.1)} has been laid down. The gist of the
argument is as follows---

First adopt any arbitrary conventional gauge which has no relation to
\index{Measurement, principle of}%
\index{Principle!of measurement}%
physical measures. Let the displacement~$A^{\mu}$ represent the radius in a given
direction of some specified unit of material structure---e.g.\ an average electron,
an average oxygen atom, a drop of water containing $10^{20}$~molecules at temperature
of maximum density. $A^{\mu}$~is determined by laws which are in the main
unknown to us. But just as we can often determine the results of unknown
physical laws by the method of dimensions, after surveying the physical
constants which can enter into the results, so we can determine the condition
satisfied by~$A^{\mu}$ by surveying the world\hyp{}tensors at our disposal. This method
indicates that the condition is
\[
R_{\mu\nu} A^{\mu} A^{\nu} = \text{constant}.
\Tag{(95.11)}
\]
If now we begin to make measures of the world, using the radius of such a
material structure as unit, we are thereby adopting a gauge\hyp{}system in which
the length~$l$ of the radius is unity, i.e.\
\[
1 = l^{2} = g_{\mu\nu} A^{\mu} A^{\nu}.
\Tag{(95.12)}
\]
By comparing \Eq{(95.11)} and~\Eq{(95.12)} it follows that $g_{\mu\nu}$~must be a constant
multiple of~$R_{\mu\nu}$; accordingly we obtain~\Eq{(95.1)}\footnotemark.\footnotetext
  {Note that the isotropy of the material unit or of the electron is not necessarily a symmetry
  of form but an independence of orientation. Thus a metre\hyp{}rule has the required isotropy because
  it has (conventionally) the same length however it is orientated.}

Besides making comparisons with material units, we can also compare the
lengths of displacements by optical devices. We must show that these comparisons
will also fit into the gauge\hyp{}system~\Eq{(95.1)}. The light\hyp{}pulse diverging
\index{Light\hyp{}pulse!in\hyp{}invariant equation}%
from a point of space\hyp{}time occupies a unique conical locus. This locus exists
independently of gauge and coordinate systems, and there must therefore be
an in\hyp{}tensor equation defining it. The only in\hyp{}tensor equation giving a cone
of the second degree is
\[
R_{\mu\nu}\, dx_{\mu}\, dx_{\nu} = 0.
\Tag{(95.21)}
\]
Comparing this with Einstein's formula for the light\hyp{}cone
\[
ds^{2} = g_{\mu\nu}\, dx_{\mu}\, dx_{\nu} = 0.
\Tag{(95.22)}
\]
We see that again
\[
R_{\mu\nu} = \lambda g_{\mu\nu}.
\Tag{(95.23)}
\]

Note however that the optical comparison is less stringent than the
material comparison; because \Eq{(95.21)} and \Eq{(95.22)} would be consistent if $\lambda$~were
a function of position, whereas the material comparisons require that it
shall be a universal constant. That is why Weyl's theory of gauge\hyp{}transformation
occupies a position intermediate between pure mathematics and physics.
He admits the physical comparison of length by optical methods, so that his
gauge\hyp{}transformations are limited to those which do not infringe~\Eq{(95.23)}; but
he does not recognise physical comparison of length by material transfer, and
consequently he takes $\lambda$~to be a function fixed by arbitrary convention and
not necessarily a constant. There is thus both a physical and a conventional
element in his ``length.''

A hybrid gauge, even if illogical, may be useful in some problems, particularly
if we are describing the electromagnetic field without reference to
matter, or preparatory to the introduction of matter. Even without matter
the electromagnetic field is self\hyp{}gauging to the extent of~\Eq{(95.23)}, $\lambda$~being a
function of position; so that we can gauge our tensors to this extent without
tackling the problem of matter. Many of Weyl's in\hyp{}tensors and in\hyp{}invariants
are not invariant for the unlimited gauge\hyp{}transformations of the generalised
theory, but they become determinate if optical gauging alone is employed;
whereas the ordinary invariant or tensor is only determinate in virtue of
relations to material standards. In particular $\mf{F}^{\mu\nu}$~is not a complete in\hyp{}tensor\hyp{}density,
but it has a self\hyp{}contained absolute meaning, because it measures the
electromagnetic field and at the same time electromagnetic fields (light\hyp{}waves)
suffice to gauge it. It may be contrasted with~$F^{\mu\nu}$ which can only be gauged
by material standards; $F^{\mu\nu}$~has an absolute meaning, but the meaning is not
self\hyp{}contained. For this reason problems will arise for which Weyl's more
limited gauge\hyp{}transformations are specially appropriate; and we regard the
generalised theory as supplementing without superseding his theory.

Adopting the natural gauge of the world, we describe its condition by
\index{Einstein's law of gravitation!equivalent to the gauging\hyp{}equation}%
two tensors $g_{\mu\nu}$ and~$\Kappa_{\mu\nu}^{\sigma}$. If the latter vanishes we recognise nothing but~$g_{\mu\nu}$,
i.e.\ pure metric. Now metric is the one characteristic of space. I refer, of
\index{Metric!sole character of space and time}%
course, to the conception of space in physics and in everyday life---the mathematician
can attribute to his space whatever properties he wishes. If $\Kappa_{\mu\nu}^{\sigma}$~does
not vanish, then there is something else present not recognised as a property
\index{Things}%
of pure space; it must therefore be attributed to a ``thing\footnotemark.''\footnotetext
  {An electromagnetic field is a ``thing''; a gravitational field is not, Einstein's theory having
  shown that it is nothing more than the manifestation of the metric.}
Thus if there
is no ``thing'' present, i.e.\ if space is quite empty, $K_{\mu\nu}^{\sigma} = 0$, and by~\Eq{(94.61)} $R_{\mu\nu}$~reduces
to~$G_{\mu\nu}$. In empty space the gauging\hyp{}equation becomes accordingly
\index{Empty space}%
\[
G_{\mu\nu} = \lambda g_{\mu\nu},
\Tag{(95.3)}
\]
which is the law of gravitation~\Eq{(37.4)}. The gauging\hyp{}equation is an \Foreign{alias} of the
law of gravitation.

We see by~\Eq{(66.2)} that the natural unit of length ($\lambda= 1$) is $1/\sqrt{3}$~times the
radius of curvature of the world in any direction in empty space. We do not
know its value, but it must obviously be very large.

One reservation must be made with regard to the definition of empty
space by the condition $\Kappa_{\mu\nu}^{\sigma} = 0$. It is possible that we do not recognise $\Kappa_{\mu\nu}^{\sigma}$ by
any physical experiment, but only certain combinations of its components. In
that case definite values of~$\Kappa_{\mu\nu}^{\sigma}$ would not be recognised as constituting a
``thing,'' if the recognisable combinations of its components vanished; just as
finite values of~$\kappa_{\mu}$ do not constitute an electromagnetic field, if the curl
vanishes. This does not affect the validity of~\Eq{(95.3)}, because any breach of
this equation is capable of being recognised by physical experiment, and
therefore would be brought about by a combination of components of~$\Kappa_{\mu\nu}^{\sigma}$ which
had a physical significance.

\Section{96.}{The principle of identification}
\index{Identification, Principle of}%
\index{Principle!of identification}%

In \SecRefs{91}--\SecNum{93} we have developed a pure geometry, which is intended to be descriptive
of the relation\hyp{}structure of the world. The relation\hyp{}structure presents
itself in our experience as a physical world consisting of \emph{space,} \emph{time} and \emph{things}.
The transition from the geometrical description to the physical description
can only be made by identifying the tensors which measure physical quantities
with tensors occurring in the pure geometry; and we must proceed by
inquiring first what experimental properties the physical tensor possesses,
and then seeking a geometrical tensor which possesses these properties \emph{by
virtue of mathematical identities}.

If we can do this completely, we shall have constructed out of the primitive
relation\hyp{}structure a world of entities which behave in the same way and obey
the same laws as the quantities recognised in physical experiments. Physical
theory can scarcely go further than this. How the mind has cognisance of
these quantities, and how it has woven them into its vivid picture of a perceptual
world, is a problem of psychology rather than of physics.

The first step in our transition from mathematics to physics is the identification
of the geometrical tensor~$R_{\mu\nu}$ with the physical tensor~$g_{\mu\nu}$ giving the
metric of physical space and time. Since the metric is the only property of
space and time recognised in physics, we may be said to have identified space
and time in terms of relation\hyp{}structure. We have next to identify ``things,''
and the physical description of ``things'' falls under three heads.

(1) The energy\hyp{}tensor~$T_{\mu}^{\nu}$ comprises the energy momentum and stress in
unit volume. This has the property of conservation $(T_{\mu}^{\nu})_{\nu} = 0$, which enables
us to make the identification
\[
-8\pi T_{\mu}^{\nu} = G_{\mu}^{\nu} - \tfrac{1}{2} g_{\mu}^{\nu}(G - 2\lambda),
\Tag{(96.1)}
\]
satisfying the condition of conservation identically. Here $\lambda$~might be any
constant; but if we add the usual convention that the zero\hyp{}condition from
which energy, momentum and stress are to be reckoned is that of empty space
(not containing electromagnetic fields), we obtain the condition for empty
space by equating \Eq{(96.1)} to zero, viz.\
\[
G_{\mu\nu} = \lambda g_{\mu\nu},
\]
so that $\lambda$~must be the same constant as in~\Eq{(95.3)}.

(2) The electromagnetic force\hyp{}tensor~$F_{\mu\nu}$ has the property that it fulfils
the first half of Maxwell's equations
\[
\frac{\dd F_{\mu\nu}}{\dd x_{\sigma}}
+ \frac{\dd F_{\nu\sigma}}{\dd x_{\mu}}
+ \frac{\dd F_{\sigma\mu}}{\dd x_{\nu}} = 0.
\Tag{(96.2)}
\]
This will be an identity if $F_{\mu\nu}$~is the curl of any covariant vector; we
accordingly identify it with the in\hyp{}tensor already called~$F_{\mu\nu}$ in anticipation,
which we have seen is the curl of a vector~$\kappa_{\mu}$ \Eq{(94.62)}.

(3) The electric charge\hyp{}and\hyp{}current vector~$J^{\mu}$ has the property of conservation
of electric charge, viz.\
\[
J_{\mu}^{\mu} = 0.
\]
The divergence of~$J^{\mu}$ will vanish identically if $J^{\mu}$~is itself the divergence of any
antisymmetrical contravariant tensor. Accordingly we make the identification
\[
J^{\mu} = F_{\nu}^{\mu\nu},
\Tag{(96.3)}
\]
a formula which satisfies the remaining half of Maxwell's equations.

The correctness of these identifications should be checked by examining
whether the physical tensors thus defined have all the properties which
experiment requires us to attribute to them. There is, however, only one
further general physical law, which is not implicit in these definitions, viz.\ the
law of mechanical force of an electromagnetic field. We can only show in an
imperfect way that our tensors will conform to this law, because a complete
proof would require more knowledge as to the structure of an electron; but
the discussion of \SecRef{80} shows that the law follows in a very plausible way.

In identifying ``things'' we have not limited ourselves to in\hyp{}tensors,
because the ``things'' discussed in physics are in physical space and time and
therefore presuppose the natural gauge\hyp{}system. The laws of conservation and
Maxwell's equations, which we have used for identifying ``things,'' would not
hold true in an arbitrary gauge\hyp{}system.

No doubt alternative identifications would be conceivable. For example,
$F_{\mu\nu}$~might be identified with the curl of~$\lambda_{\mu}$\footnote
  {The curl of~$\lambda_{\mu}$ is not an in\hyp{}tensor, but there is no obvious reason why as in\hyp{}tensor should
  be required. If magnetic flux were measured in practice by comparison with that of a magneton
  transferred from point to point, as a length is measured by transfer of a scale, then an in\hyp{}tensor
  would be needed. But that is not the actual procedure.}
instead of the curl of~$\kappa_{\mu}$. That
would leave the fundamental in\hyp{}tensor apparently doing nothing to justify its
existence. We have chosen the most obvious identifications, and it seems
reasonable to adhere to them, unless a crucial test can be devised which shows
them to be untenable. In any case, with the material at our disposal the
number of possible identifications is very limited.

\Section{97.}{The bifurcation of geometry and electrodynamics}
\index{Bifurcation of geometry and electrodynamics}%

The fundamental in\hyp{}tensor~$\Star{G}_{\mu\nu}$ breaks up into a symmetrical part~$R_{\mu\nu}$
and an antisymmetrical part~$F_{\mu\nu}$. The former is~$\lambda g_{\mu\nu}$, or if the natural unit
of length ($\lambda = 1$) is used, it is simply~$g_{\mu\nu}$. We have then
\[
\Star{G}_{\mu\nu} = g_{\mu\nu} + F_{\mu\nu},
\]
showing at once how the field or aether contains two characteristics, the
\index{Aether}%
gravitational potential (or the metric) and the electromagnetic force. These
are connected in the most simple possible way in the tensor descriptive of
underlying relation\hyp{}structure; and we see in a general way the reason for this
\index{Relation\hyp{}structure}%
inevitable bifurcation into symmetrical and antisymmetrical---geometrical\hyp{}mechanical
and electromagnetic---characteristics.

Einstein approaches these two tensors from the physical side, having
recognised their existence in observational phenomena. We here approach
them from the deductive side endeavouring to show as completely as possible
that they must exist for almost any kind of underlying structure. We confirm
his assumption that the interval~$ds^{2}$ is an absolute quantity, for it is our in\hyp{}invariant
$R_{\mu\nu}\, dx_{\mu}\, dx_{\nu}$; we further confirm the well\hyp{}known property of~$F_{\mu\nu}$ that
it is the curl of a vector.

We not only justify the assumption that natural geometry is Riemannian
geometry and not the ultra\hyp{}Riemannian geometry of Weyl, but we can show
a reason why the quadratic formula for the interval is necessary. The only
simple absolute quantity relating to two points is
\[
\Star{G}_{\mu\nu}\, dx_{\mu}\, dx_{\nu}.
\]
To obtain another in\hyp{}invariant we should have to proceed to an expression like
\[
\Star{B}_{\mu\nu\sigma}^{\rho}\, \Star{B}_{\lambda\tau\rho}^{\sigma}\, dx_{\mu}\, dx_{\nu}\, dx_{\lambda}\, dx_{\tau}.
\]
Although the latter quartic expression does theoretically express some absolute
property associated with the two points, it can scarcely be expected that
we shall come across it in physical exploration of the world so immediately as
the former quadratic expression.

It is the new insight gained on these points which is the chief advantage
of the generalised theory.

\Section{98.}{General relation\hyp{}structure}
\index{Quadratic formula for interval!justification of}%
\index{Structure, represented by relations}%

We proceed to examine more minutely the conceptions on which the
fundamental axioms of parallel displacement and affine geometry depend.

The fundamental basis of all things must presumably have \emph{structure} and
\emph{substance}. We cannot describe substance; we can only give a name to it.
Any attempt to do more than give a name leads at once to an attribution of
structure. But structure can be described to some extent; and when reduced
to ultimate terms it appears to resolve itself into a complex of relations. And
further these relations cannot be entirely devoid of comparability; for if
nothing in the world is comparable with anything else, all parts of it are alike
in their unlikeness, and there cannot be even the rudiments of a structure.

The axiom of parallel displacement is the expression of this comparability,
and the comparability postulated seems to be almost the minimum conceivable.
Only relations which are close together, i.e.\ interlocked in the relation\hyp{}structure,
are supposed to be comparable, and the conception of equivalence
is applied only to one type of relation. This comparable relation is called
displacement. By representing this relation graphically we obtain the idea of
location in space; the reason why it is natural for us to represent this particular
relation graphically does not fall within the scope of physics.

Thus our axiom of parallel displacement is the geometrical garb of a
principle which may be called ``the comparability of proximate relations.''
\index{Comparability of proximate relations}%

There is a certain hiatus in the arguments of the relativity theory which
has never been thoroughly explored. We refer all phenomena to a system of
coordinates; but do not explain how a system of coordinates (a method of
\index{Coordinates!difficulty in the introduction of}%
numbering events for identification) is to be found in the first instance. It
may be asked, What does it matter how it is found, since the coordinate\hyp{}system
fortunately is entirely arbitrary in the relativity theory? But the
arbitrariness of the coordinate\hyp{}system is limited. We may apply any continuous
transformation; but our theory does not contemplate a discontinuous
transformation of coordinates, such as would correspond to a re\hyp{}shuffling of
\index{Order, coordinate agreeing with structural}%
the points of the continuum. There is something corresponding to an \emph{order of
enumeration} of the points which we desire to preserve, when we limit the
changes of coordinates to continuous transformations.

It seems clear that this order which we feel it necessary to preserve must
be a structural order of the points, i.e.\ an order determined by their mutual
relations in the world\hyp{}structure. Otherwise the tensors which represent
structural features, and have therefore a possible physical significance, will
become discontinuous with respect to the coordinate description of the world.
So far as I know the only attempt to derive a coordinate order from a postulated
structural relation is that of Robb\footnotemark;\footnotetext
  {\Title{The Absolute Relations of Time and Space} (Camb.\ Univ.\ Press). He uses the relation of
  ``before and after.''}
this appears to be successful in the
case of the ``special'' theory of relativity, but the investigation is very
laborious. In the general theory it is difficult to discern any method of
attacking the problem. It is by no means obvious that the interlocking of
relations would necessarily be such as to determine an order reducible to the
kind of order presumed in coordinate enumeration. I can throw no light on
this question. It is necessary to admit that there is something of a jump
from the recognition of a comparable relation called displacement to the
assumption that the ordering of points by this relation is homologous with
the ordering postulated when the displacement is represented graphically by
a coordinate difference~$dx_{\mu}$.

The hiatus probably indicates something more than a temporary weakness
of the rigorous deduction. It means that space and time are only approximate
conceptions, which must ultimately give way to a more general conception of
the ordering of events in nature not expressible in terms of a four-fold coordinate\hyp{}system.
It is in this direction that some physicists hope to find a solution
of the contradictions of the quantum theory. It is a fallacy to think that the
\index{Quantum!excluded from coordinate calculations}%
conception of location in space-time based on the observation of large-scale
phenomena can be applied unmodified to the happenings which involve only
a small number of quanta. Assuming that this is the right solution it is useless
to look for any means of introducing quantum phenomena into the later
formulae of our theory; these phenomena have been excluded at the outset
by the adoption of a coordinate frame of reference.

The relation of displacement between point\hyp{}events and the relation of
\index{Equivalence of displacements}%
``equivalence'' between displacements form parts of one idea, which are only
separated for convenience of mathematical manipulation. That the relation of
displacement between $A$ and~$B$ amounts to such\hyp{}and\hyp{}such a quantity conveys
no absolute meaning; but that the relation of displacement between $A$ and~$B$
is ``equivalent'' to the relation of displacement between $C$ and~$D$ is (or at
any rate may be) an absolute assertion. Thus four points is the minimum
number for which an assertion of absolute structural relation can be made.
The ultimate elements of structure are thus four-point elements. By adopting
the condition of affine geometry~\Eq{(91.3)}, I have limited the possible assertion
with regard to a four-point element to the statement that the four points do,
or do not, form a parallelogram. The defence of affine geometry thus rests on
the not unplausible view that four-point elements are recognised to be differentiated
from one another by a single character, viz.\ that they are or are not
of a particular kind which is conventionally named \emph{parallelogramical}. Then
\index{Parallelogramical property}%
the analysis of the parallelogram property into a double equivalence of $AB$ to~$CD$
and $AC$ to~$BD$, is merely a definition of what is meant by the equivalence
of displacements.

I do not lay overmuch stress on this justification of affine geometry. It
may well happen that four-point elements are differentiated by what might
be called trapezoidal characters in which the pairs of sides are not commutable;
so that we could distinguish an element $ABCD$ trapezoidal with respect to
$AB$,~$CD$ from one trapezoidal with respect to $AC$,~$BD$. I am quite prepared
to believe that the affine condition may not always be fulfilled---giving rise to
new phenomena not included in this theory. But it is probably best in aiming
at the widest generality to make the generalisation in successive steps, and
explore each step before ascending to the next.

In reference to the difficulties encountered in the most general description
of relation\hyp{}structure, the possibility may be borne in mind that in physics we
have not to deal with individual relations but with statistical averages; and
the simplifications adopted may have become possible because of the averaging.

\Section{99.}{The tensor $\Star{B}_{\mu\nu\sigma}^{\epsilon}$}
\index{B@$B_{\mu\nu\sigma}^{\epsilon}$ (Riemann\hyp{}Christoffel tensor)}%

Besides furnishing the two tensors~$g_{\mu\nu}$ and~$F_{\mu\nu}$ of which Einstein has
made good use, our investigation has dragged up from below a certain
amount of apparently useless lumber. We have obtained the full tensor~$\Star{B}_{\mu\nu\sigma}^{\epsilon}$
which has not been used except in the contracted form---that is to say
certain components have been ignored entirely, and others have not been
considered individually but as sums. Until the problem of electron\hyp{}structure
is more advanced it is premature to reject finally any material which could
conceivably be relevant; although at present there is no special reason for
anticipating that the full tensor will be helpful in constructing electrons.

Accordingly in the present state of knowledge the tensor~$\Star{B}_{\mu\nu\sigma}^{\epsilon}$ cannot
be considered to be a physical quantity; it \emph{contains} a physical quantity~$\Star{G}_{\mu\nu}$.
Two states of the world which are described by different $\Star{B}_{\mu\nu\sigma}^{\epsilon}$ but the
same~$\Star{G}_{\mu\nu}$ are so far as we know identical states; just as two configurations
of events described by different coordinates but the same intervals are
identical configurations. If this is so, the~$\Gamma_{\nu\alpha}^{\mu}$ must be capable of other transformations
besides coordinate transformations without altering anything in
the physical condition of the world.

Correspondingly the tensor~$\Kappa_{\mu\nu}^{\sigma}$ can take any one of an infinite series of
values without altering the physical state of the world. It would perhaps be
possible to show that among these values is~$g_{\mu\nu} \kappa^{\sigma}$, which gives Weyl's geometry;
but I am not sure that it necessarily follows. It has been suggested
that the occurrence of non\hyp{}physical quantities in the present theory is a
drawback, and that Weyl's geometry which contains precisely the observed
number of ``degrees of freedom'' of the world has the advantage. For some
purposes that may be so, but not for the problems which we are now considering.
In order to discuss why the structure of the world is such that the
observed phenomena appear, we must necessarily compare it with other
structures of a more general type; that involves the consideration of ``non\hyp{}physical''
quantities which exist in the hypothetical comparison\hyp{}worlds, but
are not of a physical nature because they do not exist in the actual world.
If we refuse to consider any condition which is conceivable but not actual,
we cannot account for the actual; we can only prescribe it dogmatically.

As an illustration of what is gained by the broader standpoint, we may
consider the question why the field is described by exactly $14$~potentials.
Our former explanation attributed this to the occurrence of $14$~variables in
the most general type of geometry. We now see that this is fallacious and
that a natural generalisation of Riemannian geometry admits $40$~variables;
and no doubt the number could be extended. The real reason for the $14$~potentials
is because, even admitting a geometry with $40$~variables, the
fundamental in\hyp{}tensor of the second rank has $14$~variables; and it is the
in\hyp{}tensor (a measure of the physical state of the world) not the world\hyp{}geometry
(an arbitrary graphical representation of it) which determines the phenomena.

The ``lumber'' which we have found can do no harm, If it does not affect
the structure of electrons or quanta, then we cannot be aware of it because
we are unprovided with appliances for detecting it, if it does affect their
structure then it is just as well to have discovered it. The important thing
is to keep it out of problems to which it is irrelevant, and this is easy since
$\Star{G}_{\mu\nu}$~extracts the gold from the dross. It is quite unnecessary to specialise
the possible relation\hyp{}structure of the world in such a way that the useless
variables have the fixed value zero; that loses sight of the interesting result
that the world will go on just the same if they are not zero.

We see that two points of view may be taken---

(1) Only those things \emph{exist} (in the physical meaning of the word) which
could be detected by conceivable experiments.

(2) We are only aware of a selection of the things which \emph{exist} (in an
extended meaning of the word), the selection being determined by the nature
of the apparatus available for exploring nature.

Both principles are valuable in their respective spheres. In the earlier
part of this book the first has been specially useful in purging physics from
metaphysical conceptions. But when we are inquiring why the structure of
the world is such that just~$g_{\mu\nu}$ and~$\kappa_{\mu}$ appear and nothing else, we cannot
ignore the fact that no structure of the world could make anything else
appear if we had no cognizance of the appliances necessary for detecting it.
Therefore there is no need to insert, and puzzle over the cause of, special
limitations on the world\hyp{}structure, intended to eliminate everything which
physics is unable to determine. The world\hyp{}structure is clearly not the place
in which the limitations arise.

\Section[General properties of world\hyp{}invariants]{100.}{Dynamical consequences of the general properties of
world\hyp{}invariants}

We shall apply the method of \SecRef{61} to world\hyp{}invariants containing the
electromagnetic variables. Let $\mf{K}$~be a scalar\hyp{}density which is a function of
$g_{\mu\nu}$, $F_{\mu\nu}$, $\kappa_{\mu}$, and their derivatives up to any order, so that for a given region
\[
\int \mf{K}\, d\tau\text{ is an invariant.}
\]

It would have been possible to express~$F_{\mu\nu}$ in terms of the derivatives of~$\kappa_{\mu}$;
\index{Hamiltonian derivative!of general world\hyp{}invariants}%
\index{World\hyp{}invariants, dynamical properties of}%
but in this investigation we keep it separate, because special attention will
be directed to the case in which $\mf{K}$~does not contain the $\kappa_{\mu}$ themselves but
only their curl, so that it depends on $g_{\mu\nu}$ and $F_{\mu\nu}$~only.

By partial integration we obtain as in \SecRef{61}
\[
\delta \int \mf{K}\, d\tau
= \int(\mf{P}^{\mu\nu}\, \delta g_{\mu\nu}
     - \mf{H}^{\mu\nu}\, \delta F_{\mu\nu}
     + \mf{Q}^{\mu}\, \delta \kappa_{\mu})\, d\tau,
\Tag{(100.1)}
\]
for variations which vanish at the boundary of the region. Here
\[
P^{\mu\nu} = \frac{\Ham\Kappa}{\Ham g_{\mu\nu}},\quad
H^{\mu\nu} = -\frac{\Ham\Kappa}{\Ham F_{\mu\nu}},\quad
Q^{\mu} = \frac{\Ham\Kappa}{\Ham \kappa_{\mu}},
\Tag{(100.2)}
\]
and $P^{\mu\nu}$~is a symmetrical tensor, $H^{\mu\nu}$~an antisymmetrical tensor.

We have
\begin{align*}
  \mf{H}^{\mu\nu}\, \delta F_{\mu\nu}
  &= \mf{H}^{\mu\nu} \left(\frac{\dd(\delta\kappa_{\mu})}{\dd x_{\nu}}
  - \frac{\dd(\delta\kappa_{\nu})}{\dd x_{\mu}}\right) \\
  &= 2\mf{H}^{\mu\nu}\, \frac{\dd(\delta\kappa_{\mu})}{\dd x_{\nu}} \\
  &= -\frac{2\, \dd\mf{H}^{\mu\nu}}{\dd x_{\nu}}\, \delta\kappa_{\mu}
\intertext{rejecting a complete differential,}
  &= -2 \mf{H}_{\nu}^{\mu\nu}\, \delta\kappa_{\mu}\quad\text{by~\Eq{(51.52)}.}
\end{align*}

Hence
\[
\delta\int \mf{K}\, d\tau
= \int \bigl\{\mf{P}^{\mu\nu}\, \delta g_{\mu\nu}
+ (2\mf{H}_{\nu}^{\mu\nu} + \mf{Q}^{\mu})\, \delta\kappa_{\mu}\bigr\}\, d\tau.
\Tag{(100.3)}
\]

Now suppose that the $\delta g_{\mu\nu}$ and $\delta\kappa_{\mu}$ arise solely from arbitrary variations~$\delta x_{\alpha}$
of the coordinate\hyp{}system in accordance with the laws of transformation
of tensors and vectors. The invariant will not be affected, so that its variation
vanishes. By the same process as in obtaining~\Eq{(61.3)} we find that the change
of~$\delta\kappa_{\mu}$, for a comparison of points having the same coordinates~$x_{\alpha}$ in both the
original and varied systems, is
\[
-\delta\kappa_{\mu}
= \kappa_{\alpha}\, \frac{\dd(\delta x_{\alpha})}{\dd x_{\mu}}
+ \frac{\dd\kappa_{\mu}}{\dd x_{\alpha}}\, \delta x_{\alpha}.
\]
Hence
\[
-(\mf{Q}^{\mu} + 2\mf{H}_{\nu}^{\mu\nu})\, \delta\kappa_{\mu}
= \left\{\frac{\dd\kappa_{\mu}}{\dd x_{\alpha}} (\mf{Q}^{\mu} + 2\mf{H}_{\nu}^{\mu\nu})
- \frac{\dd}{\dd x_{\mu}} \bigl\{\kappa_{\alpha} (\mf{Q}^{\mu} + 2\mf{H}_{\nu}^{\mu\nu})\bigr\}\right\} \delta x_{\alpha}
\]
rejecting a complete differential. Since $\dd\mf{H}_{\nu}^{\mu\nu}/\dd x_{\mu} \equiv 0$ \Eq{(73.76)}, this becomes
\[
\bigl\{F_{\mu\alpha} (\mf{Q}^{\mu} + 2\mf{H}_{\nu}^{\mu\nu}) - \kappa_{\alpha} \mf{Q}_{\mu}^{\mu}\bigr\}.
\]
Using the previous reduction for $\delta g_{\mu\nu}$ \Eq{(61.4)}, our equation~\Eq{(100.3)} reduces to
\[
0 = \int \bigl\{2 \mf{P}_{\alpha\nu}^{\nu}
- F_{\mu\alpha} (\mf{Q}^{\mu} + 2\mf{H}_{\nu}^{\mu\nu})
+ \kappa_{\alpha} \mf{Q}_{\mu}^{\mu}\bigr\}\, \delta x_{\alpha}\, d\tau
\Tag{(100.41)}
\]
for all arbitrary variations $\delta x_{\alpha}$ which vanish at the boundary of the region.
Accordingly we must have identically
\[
\mf{P}_{\alpha\nu}^{\nu}
= F_{\mu\alpha} \mf{H}_{\nu}^{\mu\nu}
+ \tfrac{1}{2} F_{\mu\alpha} \mf{Q}^{\mu}
- \tfrac{1}{2} \kappa_{\alpha} \mf{Q}_{\mu}^{\mu}
\]
or, dividing by~$\sqrt{-g}$, and changing dummy suffixes,
\[
P_{\mu\nu}^{\nu}
= -F_{\mu\nu} H_{\sigma}^{\nu\sigma}
- \tfrac{1}{2} (F_{\mu\nu} Q^{\nu} + \kappa_{\mu} Q_{\nu}^{\nu}).
\Tag{(100.42)}
\]

First consider the case when $\mf{K}$~is a function of $g_{\mu\nu}$ and $F_{\mu\nu}$ only, so that
$Q^{\mu} = 0$. The equation
\[
P_{\mu\nu}^{\nu} = -F_{\mu\nu} H_{\sigma}^{\nu\sigma}
\Tag{(100.43)}
\]
at once suggests the equations of the mechanical force of an electromagnetic
\index{Force!mechanical force due to}%
\index{Mechanical force of electromagnetic field!general theory of}%
field
\[
M_{\mu\nu}^{\nu} = -h_{\mu} = -F_{\mu\nu} J^{\nu} = -F_{\mu\nu} F_{\sigma}^{\nu\sigma}.
\]

It has already become plain that anything recognised in physics as an
energy\hyp{}tensor must be of the nature of a Hamiltonian derivative of some
invariant with respect to~$g_{\mu\nu}$; and the property of conservation has been
shown to depend on this fact. \emph{We now see that the general theory of invariants
also predicts the type of the reaction of any such derived tensor to the
electromagnetic field,} viz.\ that its conservation is disturbed by a ponderomotive
force of the type $F_{\mu\nu} H_{\sigma}^{\nu\sigma}$.
\index{Hamiltonian derivative!creative aspect of}%

If we identify $P_{\mu}^{\nu}$ with the material energy\hyp{}tensor, $H_{\nu}^{\mu\nu}$~must be identified
\index{Charge\hyp{}and\hyp{}current vector!general existence of}%
\index{Energy\hyp{}tensor of matter!obtained by Hamiltonian differentiation}%
with the charge\hyp{}and\hyp{}current vector\footnotemark,\footnotetext
  {This definition of electric charge through the mechanical effects experienced by charged
  bodies corresponds exactly to the definition employed in practice. Our previous definition of it
  as~$F_{\nu}^{\mu\nu}$ corresponded to a measure of the strength of the singularity in the electromagnetic field.}
so that
\[
J^{\mu} = H_{\nu}^{\mu\nu},
\Tag{(100.44)}
\]
which is the general equation given in~\Eq{(82.2)}. It follows without any further
specialisation that electric charge must be conserved ($J_{\mu}^{\mu} = 0$).

The foregoing investigation shows that the antisymmetric part of the
principal world\hyp{}tensor will manifest itself in our experience by producing
the effects of a force. This force will act on a certain stream\hyp{}vector (in
the manner that electromagnetic force acts on a charge and current); and
further this stream\hyp{}vector represents the flow of something permanently conserved.
The existence of electricity and the qualitative nature of electrical
phenomena are thus predicted.

In considering the results of substituting a particular function for~$\Kappa$, it
has to be remembered that the equation~\Eq{(100.42)} is an identity. We shall
not obtain from it any fresh law connecting $g_{\mu\nu}$ and~$\kappa_{\mu}$. The final result after
making the substitutions will probably be quite puerile and unworthy of the
powerful general method employed. The interest lies not in the identity
itself but in the general process of which it is the result. We have seen
reason to believe that the process of Hamiltonian differentiation is actually
the process of creation of the perceptual world around us, so that in this
\index{Creation of the physical world}%
investigation we are discovering the laws of physics by examining the mode
in which the physical world is created. The identities expressing these
laws may be trivial from the mathematical point of view when separated
from the context; but the present mode of derivation gives the clue to their
significance in our experience as fundamental laws of nature\footnotemark.\footnotetext
  {The definitive development of the theory ends at this point. From here to the end of \SecRef{102}
  we discuss certain possibilities which may be on the track of further progress; but there is no
  certain guidance, and it may be suspected that the right clue is still lacking.}

To agree with Maxwell's theory it is necessary to have $H^{\mu\nu} = F^{\mu\nu}$. Accordingly
by~\Eq{(100.2)} the invariant~$\Kappa$ should contain the term $-\frac{1}{2}F^{\mu\nu} F_{\mu\nu}$.
The only natural way in which this can be combined linearly with other
terms not containing~$F_{\mu\nu}$ is in one of the invariants $\frac{1}{2}\, \Star{G}_{\mu\nu}\, \Star{G}^{\nu\mu}$ or  $-\frac{1}{2}\, \Star{G}_{\mu\nu}\, \Star{G}^{\mu\nu}$.
We take
\begin{align*}
\Kappa &= \tfrac{1}{2}\, \Star{G}_{\mu\nu}\, \Star{G}^{\nu\mu} \\
&= \tfrac{1}{2} (R_{\mu\nu} + F_{\mu\nu})(R^{\nu\mu} + F^{\nu\mu}) \\
&= \tfrac{1}{2} (R_{\mu\nu} R^{\mu\nu} - F_{\mu\nu} F^{\mu\nu})
\Tag{(100.5)}
\end{align*}
by the antisymmetric properties of~$F_{\mu\nu}$.

The quantity~$R_{\mu\nu}$ can be expressed as a function of the variables in two
ways, either by the gauging\hyp{}equation
\[
R_{\mu\nu} = \lambda g_{\mu\nu}
\]
or by the general expressions \Eq{(87.5)} and~\Eq{(94.61)}. If the first form is adopted
we obtain an identity, which, however, is clearly not the desired relation of
energy.

If we adopt the more general expression some care is required. Presumably
$\mf{K}$~should be an in\hyp{}invariant\hyp{}density if it has the fundamental
importance supposed. As written it is not formally in\hyp{}invariant in our
generalised theory though it is in Weyl's theory. We can make it in\hyp{}invariant
by writing $R_{\mu\nu} R^{\mu\nu} \sqrt{-g}$ in the form
\[
g^{\mu\alpha} g^{\nu\beta} R_{\mu\nu} R_{\alpha\beta} \sqrt{-g},
\]
where the~$g^{\mu\nu}$ are to have the values for the natural gauge, but in the in\hyp{}tensor~$R_{\mu\nu}$
the general values for any gauge may be used. The general theory
becomes highly complicated, and we shall content ourselves with the partially
generalised expression in Weyl's theory, which will sufficiently illustrate the
procedure. In this case $R_{\mu\nu} = \lambda g_{\mu\nu}$, but $\lambda$~is a variable function of position.
Accordingly $R_{\mu\nu} R^{\mu\nu} = 4\lambda^2 = \frac{1}{4}\, \Star{G}^{2}$, so that
\[
\mf{K} = \tfrac{1}{8}(\Star{G}^{2} - 4F_{\mu\nu} F^{\mu\nu}) \sqrt{-g}.
\Tag{(100.6)}
\]
Comparing with~\Eq{(90.1)} we see that $\mf{K}$~is equivalent to the action adopted by
\index{Action, material or gravitational!Weyl's formula}%
Weyl.

This appears to throw light on the meaning of the combination of $\Star{G}^{2}$
with~$F_{\mu\nu} F^{\mu\nu}$ which we have recognised in~\Eq{(90.1)} as having an important
significance. It is the degenerate form in Weyl's gauge of the natural combination
$\Star{G}_{\mu\nu}\, \Star{G}^{\nu\mu}$. The alternation of the suffixes is primarily adopted as a
trick to obtain the required sign, but is perhaps justifiable.

If this view of the origin of~\Eq{(90.1)} is correct, the constant~$\alpha$ must be
equal to~$4$. Accordingly $\beta = 1/2\lambda$, and by~\Eq{(90.51)} the whole energy\hyp{}tensor
and the electromagnetic energy\hyp{}tensor are reduced to the same units in the
expressions
\[
E^{\mu\nu},\quad
8\pi\lambda T^{\mu\nu}.
\Tag{(100.7)}
\]
The numerical results obtainable from this conclusion will be discussed in
\SecRef{102}.

In the discussion of \SecRef{90} it was assumed that $P^{\mu\nu}$ ($= \Ham\Kappa/\Ham g_{\mu\nu}$) vanished.
I do not think there is any good reason for introducing an arbitrary action\hyp{}principle
of this kind, and it seems more likely that $P^{\mu\nu}$~will be a non\hyp{}vanishing
energy\hyp{}tensor.

This seems to leave a superfluity of energy\hyp{}tensors, because owing to the
non\hyp{}vanishing coefficient~$Q^{\mu}$ we have the term $(\kappa^{\mu} \kappa^{\nu} - \frac{1}{2} g^{\mu\nu} \kappa_{\alpha} \kappa^{\alpha})$ in~\Eq{(90.51)}
which has to play some rôle. In \SecRef{90} this was supposed to be the material
energy\hyp{}tensor, but I am inclined to think that it has another interpretation.
In order to liberate material energy we must relax the binding forces of the
electrons, allowing them to expand. Suppose that we make a small virtual
change of this kind. In addition to the material energy liberated by the
process there will be another consequential change in the energy of the
region. The electron furnishes the standard of length, so that all the gravitational
energy will now have to be re\hyp{}gauged. It seems likely that the
function of the term $(\kappa^{\mu} \kappa^{\nu} - \frac{1}{2} g^{\mu\nu} \kappa_{\alpha} \kappa^{\alpha})$ is to provide for this change. If so,
nothing hinders us from identifying $P^{\mu\nu}$ with the true material energy\hyp{}tensor.

An attractive development of the theory has recently been published by Einstein
(\Title{Berlin. Sitzungsberichte, 1923, pp.~32, 76, 137}).
 This development may be regarded as a substitute for Weyl's action\hyp{}theory
discussed in~\SecRef{90}, which aims at relating the field-laws and field\hyp{}tensors to a single regional invariant.
The theory is intensely formal as indeed all such action\hyp{}theories must be, and I cannot avoid the suspicion
that the mathematical elegance is obtained by a short cut which does not lead along the direct route of real physical
progress.
From a recent conversation with Einstein I learn that he is of much the same opinion.
Nevertheless, where the path of progress is uncertain, it would be unwise to ignore advance along any open route,
and we shall give an account of Einstein's results, which appeal very strongly to those who take a view of the
problem before us slightly different from that adopted by the author.

Let~$\mf{K}$ be an invariant\hyp{}density which is a function only of the~$\Star{G}_{\mu\nu}$, and
let~$\mf{P}^{\mu\nu} = \dd\mf{K}/\dd\Star{G_{\mu\nu}}$, so that
\[
\delta\mf{K} = \mf{P}^{\mu\nu}\delta\Star{G}_{\mu\nu}.
\Tag{(100.8)}
\]
Einstein supposes that there exists an action~$\mf{K}\,d\tau$ which has the stationary property for all variations
of the affine connection described by the coefficients~$\Gamma^\alpha_{\mu\nu}$.
Writing\footnotemark\footnotetext
     {In a non\hyp{}metrical geometry there is no fixed association of~$K$ and~$\mf{K}$, and we have to introduce
      directly the Hamiltonian derivative of an \emph{invariant\hyp{}density,} which was not provided for in
      the original definition~\Eq{(60.43)}.
      Equation~\Eq{(100.9)} gives the definition; to remove ambiguity it should also be explicitly stated as part
      of the definition that the Hamiltonian derivative with respect to a symmetrical quantity is symmetrical and
      with respect to an antisymmetrical quantity is antisymmetrical.
      I find that Prof.~de~Donder had already introduced the name ``Hamiltonian of~$\mf{K}$'' for what I have called
      the Hamiltonian derivative.}
\[
\delta\int\mf{K}\,d\tau = \int\frac{\Ham\mf{K}}{\Ham\Gamma^\alpha_{\mu\nu}}\delta\Gamma^\alpha_{\mu\nu}\,d\tau,
\Tag{(100.9)}
\]
the stationary condition is
\[
\frac{\Ham\mf{K}}{\Ham\Gamma^\alpha_{\mu\nu}} = 0.
\]

Inserting~\Eq{(92.42)} in~\Eq{(100.8)} and rejecting the complete differential after the usual partial integration
\begin{multline*}
\delta\mf{K}  = \frac{\dd\mf{P}^{\mu\nu}}{\dd x_\alpha}\delta\Gamma^\alpha_{\mu\nu} -
               \frac{\dd\mf{P}^{\mu\nu}}{\dd x_\nu}\delta\Gamma^\alpha_{\alpha\mu} +\\
    \mf{P}^{\mu\nu}(\Gamma^\alpha_{\beta\mu}\delta\Gamma^\beta_{\alpha\nu} +
                   \Gamma^\beta_{\alpha\nu}\delta\Gamma^\alpha_{\beta\mu} -
                   \Gamma^\alpha_{\mu\nu}\delta\Gamma^\beta_{\beta\alpha} -
                   \Gamma^\beta_{\beta\alpha}\delta\Gamma^\alpha_{\mu\nu})\\
             = \delta\Gamma^\alpha_{\mu\nu}
  \left(
    \frac{\dd\mf{P}^{\mu\nu}}{\dd x_\alpha} - \delta^\nu_\alpha\frac{\dd\mf{P}^{\mu\sigma}}{\dd x_\sigma} +
    \Gamma^\mu_{\alpha\epsilon}\mf{P}^{\epsilon\nu} +
    \Gamma^\nu_{\alpha\epsilon}\mf{P}^{\mu\epsilon} -
    \Gamma^\beta_{\beta\alpha}\mf{P}^{\mu\nu} -
    \delta^\nu_\alpha\Gamma^\mu_{\sigma\tau}\mf{P}^{\sigma\tau}
  \right),
\end{multline*}
by changing the dummy suffixes.
This may be written
\[
\delta\mf{K} = \delta\Gamma^\alpha_{\mu\nu}\{(\mf{P}^{\mu\nu})_\alpha - \delta^\nu_\alpha(\mf{P}^{\mu\sigma})_\sigma\},
\Tag{(100.10)}
\]
where~$(\mf{P}^{\mu\nu})_\alpha$ is the affine derivative, viz.
\[
(\mf{P}^{\mu\nu})_\alpha = \dd\mf{P}^{\mu\nu}/\dd x_\alpha
       + \Gamma^{\mu}_{\alpha\epsilon}\mf{P}^{\epsilon\nu}
       + \Gamma^{\nu}_{\alpha\epsilon}\mf{P}^{\mu\epsilon}
       - \Gamma^{\beta}_{\beta\alpha}\mf{P}^{\mu\nu}.
\Tag{(100.11)}
\]

Since~$\Gamma^\alpha_{\mu\nu}\equiv\delta\Gamma^\alpha_{\nu\mu}$, we must not make the coefficients of these vanish
independently but must set the sum of the two coefficients equal to zero.
The stationary property is accordingly expressed by
\[
(\mf{P}^{\mu\nu}+\mf{P}^{\nu\mu})_\alpha
    - \delta^\nu_\alpha(\mf{P}^{\mu\tau})_\sigma - \delta^\mu_\alpha(\mf{P}^{\nu\sigma})_\sigma = 0.
\Tag{(100.12)}
\]

Einstein identifies~$\mf{P}^{\mu\nu}$ with the sum of the tensor\hyp{}densities of the metrical and electromagnetic
fields, these constituting respectively its symmetrical and antisymmetrical parts; that is to say,
\[
\mf{P}^{\mu\nu} = \mf{g}^{\mu\nu} + \mf{F}^{\mu\nu}.
\Tag{(100.13a)}
\]
It will be seen that this is a departure from the author's identification of~$\Star{G}_{\mu\nu}$
with~$g_{\mu\nu}+F_{\mu\nu}$, although closely analogous.
Since~$F_{\mu\nu}$ is now identified through~\Eq{(100.13a)} we must use another symbol for the antisymmetrical
part of~$\Star{G}_{\mu\nu}$, viz.
\[
\Star{G}_{\mu\nu} = R_{\mu\nu} + \Phi_{\mu\nu}.
\Tag{(100.13b)}
\]

By~\Eq{(100.11)} we see that for an antisymmetrical tensor
\[
(\mf{F}^{\mu\sigma})_\sigma = \dd\mf{F}^{\mu\sigma}/\dd x_\sigma = \mf{J}^\mu.
\]
Hence~\Eq{(100.12)} becomes
\[
2(\mf{g}^{\mu\nu})_\alpha
  - \delta^\nu_\alpha(\mf{g}^{\mu\sigma})_\sigma
  - \delta^\mu_\alpha(\mf{g}^{\nu\sigma})_\sigma
  - \delta^\nu_\alpha\mf{J}^\mu
  - \delta^\mu_\alpha\mf{J}^\nu = 0.
\Tag{(100.14)}
\]
Contract by setting~$\nu=\alpha$; we obtain
\[
-3(\mf{g}^{\mu\alpha})_\alpha - 3\mf{J}^\mu = 0,
\]
so that~\Eq{(100.14)} simplifies to
\[
(\mf{g}^{\mu\nu})_\alpha + \tfrac{1}{3}\delta^\nu_\alpha\mf{J}^\mu + \tfrac{1}{3}\delta^\mu_\alpha\mf{J}^\nu = 0.
\Tag{(100.15)}
\]

From the comparison of covariant and affine derivatives in~\Eq{(93.91)} and~\Eq{(93.92)}, we have
\[
(\mf{g}^{\mu\nu})_\alpha - \mf{g}^{\mu\nu}_\alpha = S^\mu_{\alpha\epsilon}\mf{g}^{\epsilon\nu} +
                                                    S^\nu_{\alpha\epsilon}\mf{g}^{\mu\epsilon} -
                                                    S^\epsilon_{\alpha\epsilon}\mf{g}^{\mu\nu},
\Tag{(100.16)}
\]
where
\[
S^\alpha_{\mu\nu}=\Gamma^\alpha_{\mu\nu} - \{\mu\nu,\alpha\},
\Tag{(100.17)}
\]
as in~\Eq{(93.6)}.
Since the covariant derivative of~$\mf{g}^{\mu\nu}$ vanishes we have from~\Eq{(100.15)} and~\Eq{(100.16)},
by lowering suffixes and removing the density factor,
\[
S_{\alpha\nu,\mu} + S_{\alpha\mu,\nu} - g_{\mu\nu} S^\epsilon_{\alpha\epsilon} + \tfrac{1}{3}g_{\alpha\nu}J_\mu +
  \tfrac{1}{3}g_{\alpha\mu}J_\nu = 0,
\]
whence multiplying by~$g^{\mu\nu}$
\[
S^\epsilon_{\alpha\epsilon} = \tfrac{1}{3}J_\alpha.
\]
Accordingly
\[
S_{\alpha\nu,\mu} + S_{\alpha\mu,\nu} - \tfrac{1}{3}g_{\mu\nu}J_\alpha
 + \tfrac{1}{3}g_{\alpha\nu}J_\mu + \tfrac{1}{3}g_{\alpha\mu}J_\nu = 0.
\]
Solving these equations for the~$S$\hyp{}tensor
\[
S_{\mu\nu,\alpha} = \tfrac{1}{6}g_{\alpha\nu}J_\mu + \tfrac{1}{6}g_{\alpha\mu}J_\nu - \tfrac{1}{2}g_{\mu\nu}J_\alpha.
\]
By~\Eq{(100.17)} we now determine the coefficients of affine connection in terms of familiar physical quantities
\[
\Gamma^\alpha_{\mu\nu} = \{\mu\nu,\alpha\}
             + \tfrac{1}{6}g^\alpha_\nu J_\mu + \tfrac{1}{6}g^\alpha_\mu J_\nu
             - \tfrac{1}{2}g_{\mu\nu}J^\alpha.
\Tag{(100.18)}
\]
Using this value in~\Eq{(92.42)} or~\Eq{(94.3)} we find after a little reduction
\[
\left.
\begin{gathered}
R_{\mu\nu} = G_{\mu\nu} + \tfrac{1}{6}J_\mu J_\nu\\
\Phi_{\mu\nu} = \frac{1}{6}\left(\frac{\dd J_\mu}{\dd x_\nu}-\frac{\dd J_\nu}{\dd x_\mu}\right).
\end{gathered}
\right\}
\Tag{(100.19)}
\]
This solves the problem of determining~$\Star{G}_{\mu\nu}$ in terms of ordinary physical variables identified
by~\Eq{(100.13a)}.

Further progress will depend on assuming a special form of~$\mf{K}$.
It will be remembered that Weyl's action\hyp{}principle led to two laws of nature expressed by~\Eq{(90.61)}
and~\Eq{(90.71)}.
If we think it probable that these are the actual laws of the world we shall seek to identify~$\mf{K}$
in such a manner that the same laws will result from the present theory.
The first step is to connect Weyl's laws with the symbols here employed.
Setting~$\beta=4\pi$ as indicated by~\Eq{(90.71)}, the laws are
\[
k_\mu = -\frac{4\pi}{3}J_\mu.
\Tag{(100.20)}
\]
\[
M_{\mu\nu} = \frac{3}{4\pi}(k_\mu k_\nu - \tfrac{1}{2}g_{\mu\nu}k_\alpha k^\alpha) =
         \frac{4\pi}{3}(J_\mu J_\nu - \tfrac{1}{2}g_{\mu\nu}J_\alpha J^\alpha).
\Tag{(100.21)}
\]
Hence
\begin{align*}
T_{\mu\nu}-M_{\mu\nu} & = -8\pi\{(G_{\mu\nu} - \tfrac{1}{2}g_{\mu\nu}(G-2\lambda))+\tfrac{1}{6}(J_\mu J_\nu -\tfrac{1}{2}g_{\mu\nu}J_\alpha J^\alpha)\}\\
                      & = -8\pi (R_{\mu\nu} - \tfrac{1}{2}g_{\mu\nu}(R-2\lambda)) \quad \text{by~\Eq{(100.19)}}.
\end{align*}

This difference between the whole energy\hyp{}tensor and the electronic energy\hyp{}tensor must represent the
Maxwellian energy\hyp{}tensor~$E_{\mu\nu}$.
Hence~$E_{\mu\nu}=-8\pi(R_{\mu\nu}-\tfrac{1}{2}g_{\mu\nu}(R-2\lambda))$.
Since~$E=0$, we obtain by contraction~$R=4\lambda$.
Again by~\Eq{(100.19)} and~\Eq{(100.20)}
\[
\Phi_{\mu\nu}=-\frac{1}{8\pi}\left(\frac{\dd k_\mu}{\dd x_\nu}-\frac{\dd k_\nu}{\dd x_\mu}\right)
             =-\frac{1}{8\pi}F_{\mu\nu}.
\]
Hence Weyl's laws correspond to the equations
\[
\left.
\begin{gathered}
E_{\mu\nu} = -8\pi (R_{\mu\nu} - \lambda g_{\mu\nu})\\
F_{\mu\nu} = -8\pi \Phi_{\mu\nu}.
\end{gathered}
\right\}
\Tag{(100.22)}
\]

Now
\begin{multline*}% XXX fix this alignment
\delta(F_{\mu\nu}\mf{F}^{\mu\nu}) =
   (g_{\mu\alpha}g_{\nu\beta}/\sqrt{-g})\delta(\mf{F}^{\mu\nu}\mf{F}^{\alpha\beta})
   + \mf{F}^{\mu\nu}\mf{F}^{\alpha\beta}\delta(g_{\mu\alpha}g_{\nu\beta}/\sqrt{-g})\\
 = F_{\mu\nu}\delta\mf{F}^{\mu\nu} + F_{\alpha\beta}\delta\mf{F}^{\alpha\beta}\\
     +\mf{F}^{\mu\nu}\mf{F}^{\alpha\beta}(-g_{\mu\alpha}g_{\nu\sigma}g_{\beta\tau}\delta g^{\sigma\tau}
      -g_{\nu\beta}g_{\mu\sigma}g_{\alpha\tau}\delta g^{\sigma\tau}
 + \tfrac{1}{2}g_{\mu\alpha}g_{\nu\beta}g_{\sigma\tau}\delta g^{\sigma\tau})/\sqrt{-g},
\end{multline*}
by~\Eq{(35.12)} and~\Eq{(35.3)}
\[
= 2(F_{\mu\nu}\delta\mf{F}^{\mu\nu} + \mf{E}_{\sigma\tau}\delta g^{\sigma\tau})\quad\text{by~\Eq{(77.2)}}.
\]
But $E_{\sigma\tau}\delta\mf{g}^{\sigma\tau}=E_{\sigma\tau}\sqrt{-g}\delta g^{\sigma\tau}+E_{\sigma\tau}g^{\sigma\tau}\delta(\sqrt{-g})=\mf{E}_{\sigma\tau}\delta g^{\sigma\tau}$, since~$E=0$.
Hence
\begin{align*}
-\frac{1}{16\pi}\delta(F_{\mu\nu}\mf{F}^{\mu\nu}) & = -\frac{1}{8\pi}(F_{\mu\nu}\delta\mf{F}^{\mu\nu} +
    E_{\mu\nu}\delta\mf{g}^{\mu\nu})\\
   & = (R_{\mu\nu}\delta\mf{g}^{\mu\nu} + \Phi_{\mu\nu}\delta\mf{F}^{\mu\nu})-\lambda g_{\mu\nu}\delta\mf{g}^{\mu\nu}.
\end{align*}
Since~$g_{\mu\nu}\delta\mf{g}^{\mu\nu}=2\delta\sqrt{-g}$ this becomes (writing~$\alpha=32\pi\lambda$)
\[
-\frac{1}{16\pi}\delta(-\alpha\sqrt{-g}+F_{\mu\nu}\mf{F}^{\mu\nu}) = \Star{G}_{\mu\nu}\delta\mf{P}^{\mu\nu}.
\Tag{(100.23)}
\]

Einstein notes that associated with any density~$\mf{K}$ there will be another density~$\mf{K}'$ given by
\begin{align*}
\mf{K}' & = \mf{K} - \Star{G}_{\mu\nu} \frac{\dd\mf{K}}{\dd\Star{G}_{\mu\nu}}\\
        & = \mf{K} - \Star{G}_{\mu\nu} \mf{P}^{\mu\nu}.
\Tag{(100.24)}
\end{align*}
This ``modified'' density is scarcely less fundamental than the original density.
We have by~\Eq{(100.8)} and~\Eq{(100.24)}
\[
\left.
\begin{gathered}
  \delta\mf{K}   = \mf{P}^{\mu\nu}   \delta\Star{G}_{\mu\nu}\\
- \delta\mf{K}'  = \Star{G}_{\mu\nu} \delta\mf{P}_{\mu\nu}
\end{gathered}
\right\}
\Tag{(100.25)}
\]
Comparing~\Eq{(100.23)} and~\Eq{(100.25)} it follows that
\[
\mf{K}' = \frac{1}{16\pi}(-\alpha\sqrt{-g} + F_{\mu\nu}\mf{F}^{\mu\nu}).
\]
This is Einstein's conclusion.

The~$\alpha$\hyp{}term in~$\mf{K}'$, which seenis to be a rather unnatural complication of the expression,
arises from the cosmical~$\lambda$\hyp{}term in the energy.
It would be simpler if we could dispense with the cosmical term, reverting to Einstein's original unclosed space.
Since~$\lambda=\tfrac{1}{4}R$, this would involve~$R=0$.

Hamiltonian derivatives with respect to~$\Gamma^\alpha_{\mu\nu}$ may be expected to represent quantities of
fundamental significance in regard to the structure of the world;
derivatives with respect to~$g_{\mu\nu}$ and~$k_\mu$ should rather yield quantities which spring into prominence
in our perception of the world as the activity of an electromagnetic process~$k_\mu$ working in a passive
metric~$g_{\mu\nu}$.
Einstein has used the variation appropriate to his object---the formulation of a controlling law of
world\hyp{}structure;
the author used the other variation since his aim was to discover quantities conspicuous in current physics.

It is difficult to regard any invariant\hyp{}density other than the generalised volume~\Eq{(101.13)} as ideally simple.
We may therefore inquire what would be the result if~$\mf{K}$ is taken to
be~$\sqrt{-|\Star{G}_{\mu\nu}|}$\footnotemark.\footnotetext
         {Einstein originally started with the generalised volume and reached equations~\Eq{(100.19)};
          but in his third paper he proved the same equations independently of the form of~$\mf{K}$.
          Thus the generalised volume is not actually used in his discussion.}

Since~$\mf{K}$ is then a homogeneous quadratic (but irrational) function of the~$\Star{G}_{\mu\nu}$ we have
by~\Eq{(100.24)}
\[
\mf{K}' = \mf{K} - 2\mf{K} = -\mf{K}.
\Tag{(100.26)}
\]

Also writing~$\Delta=|\Star{G}_{\mu\nu}|$
\[
\mf{P}^{\mu\nu}=-\frac{1}{2\sqrt{-\Delta}}\times(\text{minor of}\,\Star{G}_{\mu\nu}).
\]
By the properties of determinants the determinant of the minors is equal to~$\Delta^3$.
Hence
\[
|\mf{P}^{\mu\nu}| = \frac{\Delta^3}{16\Delta^2} = \frac{1}{16}\Delta.
\]
Hence
\begin{align*}
\tfrac{1}{4}\mf{K} = \sqrt{-|\mf{P}^{\mu\nu}|} & = \sqrt{-|\mf{g}^{\mu\nu} + \mf{F}^{\mu\nu}|}\\
                                               & = \sqrt{-|g_{\mu\nu} + F_{\mu\nu}|}.
\end{align*}
The last step follows as in~\Eq{(49.11)}.
Thus notwithstanding the different identifications of the physical tensors by Einstein and the author the generalised
volume has the same identification in both discussions (except for the numerical factor) and its evaluation
in~\SecRef{101} is still applicable.
We shall insert the factor~$\lambda$ in order that current units of length may be used instead of the unknown natural
unit; then if fourth powers of~$F_{\mu\nu}$ are neglected, we have by~\Eq{(101.31)}
\begin{align*}
\tfrac{1}{4}\mf{K} & = (\lambda^2 + \tfrac{1}{4}F_{\mu\nu}F^{\mu\nu})\sqrt{-g},\\
\text{or}\quad -\mf{K}' & = 4\lambda^2\sqrt{-g} + F_{\mu\nu}\mf{F}^{\mu\nu}.
\Tag{(100.27)}
\end{align*}
This differs from the previous identification of~$\mf{K}'$ only in the coefficient of~$\sqrt{-g}$;
and we have thus obtained quite naturally a term of the right form to supply the cosmical term of the energy.
But it is joined with the \emph{wrong sign}\footnotemark.\footnotetext
       {The difference of magnitude is of no consequence since it is absorbed in the choice of unit of~$F_{\mu\nu}$.}

The following is perhaps a permissible way of remedying this difficulty of sign.
The generalised volume is based on the determinant
\begin{multline*}
\left.
\begin{gathered}
(4!)\Delta = \mf{E}^{\alpha\beta\gamma\delta} \mf{E}^{\epsilon\zeta\eta\theta}
                  \Star{G}_{\alpha\epsilon} \Star{G}_{\beta\zeta} \Star{G}_{\gamma\eta} \Star{G}_{\delta\theta}\\
\text{Let}\quad (4!)\Delta' = \mf{E}^{\alpha\beta\gamma\delta} \mf{E}^{\epsilon\zeta\eta\theta}
                  \Star{G}_{\alpha\epsilon} \Star{G}_{\zeta\beta} \Star{G}_{\gamma\eta} \Star{G}_{\theta\delta}.
\end{gathered}
\right\}
\Tag{(100.28)}
\end{multline*}
which is also the square of an invariant\hyp{}density, but is not a determinant\footnotemark.\footnotetext
    {My attention was called to this expression by Prof.~B.~Weitzenb\"ock.}

The alternating expression~$\Delta'$ seems to be a no less natural combination than~$\Delta$.
To form a term which is quadratic in the~$F_{\mu\nu}$ we must pick out 2 of the 4
factors~$\Star{G}_{\alpha\epsilon}$, etc., to provide the~$F_{\mu\nu}$.
Clearly out of the 6 possible selections, 2 will have the same sign in~$\Delta$ and~$\Delta'$ and 4 will have
opposite signs.
Thus in the sum the quadratic term is joined with opposite signs in~$\Delta$ and~$\Delta'$, so that by
substituting~$\Delta'$ for~$\Delta$ we obtain the sign which we require.
We conclude that---

Provided the natural metrical and electromagnetic units are such that fourth powers of the electromagnetic force
are beyond the range of observation, the present system of field-laws (other than equations of definition) is
summed up in the condition that
\[
\delta \int \sqrt{-\Delta'}\,d\tau = 0,
\Tag{(100.29)}
\]
for all small variations of the affine connection which vanish at the bouridary of the region.

But, so far as I can see, the natural units do not satisfy the proviso~(\SecRef{102}), and the neglected fourth
powers may cause further trouble.

\Section{101.}{The generalised volume}
\index{Generalised volume}%
\index{Volume!generalised}%

Admitting that $\Star{G}_{\mu\nu}$~is the building\hyp{}material with which we have to construct
the physical world, let us examine what are the simplest invariants
that can be formed from it. The meaning of ``simple'' is ambiguous, and
depends to some extent on our outlook. I take the order of simplicity to be
the order in which the quantities appear in building the physical world from
the material~$\Star{G}_{\mu\nu}$. Before introducing the process of gauging by which we
obtain the~$g_{\mu\nu}$, and later (by a rather intricate use of determinants) the~$g^{\mu\nu}$,
we can form in\hyp{}invariants belonging respectively to a one\hyp{}dimensional, a two\hyp{}dimensional
\index{In\hyp{}invariants}%
and a four\hyp{}dimensional domain.

(1) For a line\hyp{}element $(dx)^{\mu}$, the simplest in\hyp{}invariant is
\[
\Star{G}_{\mu\nu}\, (dx)^{\mu}\, (dx)^{\nu},
\Tag{(101.11)}
\]
which appears physically as the square of the length.

(2) For a surface\hyp{}element~$dS^{\mu\nu}$, the simplest in\hyp{}invariant is
\index{Surface\hyp{}element!in\hyp{}invariant pertaining to}%
\[
\Star{G}_{\mu\nu}\, dS^{\mu\nu},
\Tag{(101.12)}
\]
which appears physically as the flux of electromagnetic force. It may be
\index{Flux!electromagnetic}%
remarked that this invariant, although formally pertaining to the surface\hyp{}element,
is actually a property of the bounding circuit only.

(3) For a volume\hyp{}element~$d\tau$, the simplest in\hyp{}invariant is
\[
V = \sqrt{-|\Star{G}_{\mu\nu}|}\, d\tau,
\Tag{(101.13)}
\]
which has been called the generalised volume, but has not yet received a
physical interpretation.

We shall first calculate $|\Star{G}_{\mu\nu}|$ for Galilean coordinates. Since
\[
\Star{G}_{\mu\nu} = \lambda g_{\mu\nu} + F_{\mu\nu}
\]
we have on inserting the Galilean values
\begin{multline*}
  |\Star{G}_{\mu\nu}|
  = \left\lvert
  \begin{array}{r@{\quad}r@{\quad}r@{\quad}c}
  -\lambda & -\gamma & \beta & -X \\
  \gamma & -\lambda & -\alpha & -Y \\
  -\beta & \alpha & -\lambda & -Z \\
  X & Y & Z & \lambda \\
  \end{array}\right\rvert \\
  = -\bigl\{\lambda^{4}
  + \lambda^2(\alpha^2 + \beta^2 + \gamma^2 - X^{2} - Y^{2} - Z^{2})
  - (\alpha X + \beta Y + \gamma Z)^{2}\bigr\}.
  \Tag{(101.2)}
\end{multline*}
The relation of the absolute unit of electromagnetic force (which is here being
used) to the practical unit is not yet known, but it seems likely that the fields
used in laboratory experiments correspond to small values of~$F_{\mu\nu}$\footnotemark.\footnotetext
  {This is doubtful, since the calculations in the next section do not bear it out.}
If this is
so we may neglect the fourth powers of~$F_{\mu\nu}$ and obtain approximately
\begin{align*}
  V = \sqrt{-|\Star{G}_{\mu\nu}|}\, d\tau
  &= \bigl\{\lambda^2 + \tfrac{1}{2}(\alpha^2 + \beta^2 + \gamma^2 - X^{2} - Y^{2} - Z^{2})\bigr\}\, d\tau \\
  &= (\lambda^2 + \tfrac{1}{4} F_{\mu\nu} F^{\mu\nu})\, d\tau
  \quad\text{by~\Eq{(77.3)}.}
\end{align*}
Since $V$~is an invariant we can at once write down the result for any other
coordinate\hyp{}system, viz.\
\[
V = (\lambda^2 + \tfrac{1}{4} F_{\mu\nu} F^{\mu\nu}) \sqrt{-g}\, d\tau,
\Tag{(101.31)}
\]
or in the natural gauge $R_{\mu\nu} = \lambda g_{\mu\nu}$, this can be written
\begin{align*}
V &= \tfrac{1}{4} (R_{\mu\nu} R^{\mu\nu} + F_{\mu\nu} F^{\mu\nu}) \sqrt{-g}\, d\tau \\
&= \tfrac{1}{4}\, \Star{G}_{\mu\nu}\, \Star{G}^{\mu\nu} \sqrt{-g}\, d\tau.
\Tag{(101.32)}
\end{align*}
Thus if the generalised volume is the fundamental in\hyp{}invariant from which
the dynamical laws arise, we may expect that our approximate experimental
laws will pertain to the invariant $\Star{G}_{\mu\nu}\, \Star{G}^{\mu\nu} \sqrt{-g}\, d\tau$, which is a close approximation
to it except in very intense electromagnetic fields.

In~\Eq{(100.5)} we took $\Kappa = \Star{G}_{\mu\nu}\, \Star{G}^{\nu\mu}$. The alternation of the suffixes seems to
be essential if $\Ham\Kappa/\Ham g_{\mu\nu}$ is to represent the material energy (or to be zero
according to Weyl's action\hyp{}principle). If we do not alternate the suffixes the
Hamiltonian derivative contains the whole energy\hyp{}tensor plus the electromagnetic
energy\hyp{}tensor, whereas we must naturally attach more significance
to the difference of these two tensors. It may, however, be noted that
\[
\Star{G}_{\mu\nu}\, \Star{G}^{\nu\mu}
= \Star{G}_{\mu\nu}\, \Star{G}^{\mu\nu}
- \kappa_{\mu\nu}\, \frac{\Ham}{\Ham\kappa_{\mu\nu}}(\Star{G}_{\mu\nu}\, \Star{G}^{\mu\nu})
\Tag{(101.33)}
\]
(variations of~$\kappa_{\mu}$ being ignored except in so far as they affect~$F_{\mu\nu}$). It would
\index{Ignoration of coordinates}%
seem therefore that the invariant~$\Kappa$ previously discussed arises from~$V$ by the
process of ignoration of the coordinates~$\kappa_{\mu}$. Equation~\Eq{(101.33)} represents
exactly the usual procedure for obtaining the modified Lagrangian function in
\index{Lagrangian function}%
dynamics.

If this view is correct, that the invariants which give the ordinary equations
adopted in physics are really approximations to more accurate expressions
based on the generalised volume, it becomes possible to predict the second\hyp{}order
terms which are needed to complete the equations currently used. It
will sufficiently illustrate this if we consider the corrections to Maxwell's
equations suggested by this method.

Whereas in~\Eq{(79.32)} we found that $J^{\mu}$~was the Hamiltonian derivative of
$\frac{1}{4} F^{\mu\nu} F_{\mu\nu} \cdot \sqrt{-g}\, d\tau$, we now suppose that it is more exactly the Hamiltonian
derivative of $\sqrt{-|\Star{G}_{\mu\nu}|}\, d\tau$ with respect to~$\kappa_{\mu}$\footnotemark.\footnotetext
  {We consider only the variations of~$\kappa_{\mu}$ as affecting~$F_{\mu\nu}$.}
We use Galilean (or natural)
coordinates; and it is convenient to use the notation of \SecRef{82} in which $(a, b, c)$
takes the place of $(\alpha, \beta, \gamma)$.

Let
\[
\Delta = -|\Star{G}_{\mu\nu}|
= \lambda^4 + \lambda^2 (a^2 + b^2 + c^2 - X^2 - Y^2 - Z^2) - S^2,
\]
where
\[
S = aX + bY + cZ.
\]
Then
\[
\delta(\sqrt{\Delta})
= \frac{1}{\sqrt{\Delta}}\, \lambda^2 \left\{
\left(a - \frac{SX}{\lambda^2}\right) \delta a + \cdots
- \left(X + \frac{Sa}{\lambda^2}\right) \delta X - \cdots\right\}.
\]
Take a permeability and specific inductive capacity given by
\index{Permeability, magnetic}%
\[
\mu = \frac{1}{\Kappa} = \frac{\sqrt{\Delta}}{\lambda^2},
\Tag{(101.41)}
\]
so that
\[
\alpha = \lambda^2a/\sqrt{\Delta},\quad
P = \lambda^2X/\sqrt{\Delta},
\]
and let
\[
S' = S/\sqrt{\Delta}
= (\alpha X + \beta Y + \gamma Z)/\lambda^2.
\Tag{(101.42)}
\]

Then
\begin{multline*}
  \delta(\sqrt{\Delta})
  = (\alpha - XS')\, \delta\left(\frac{\dd H}{\dd y} - \frac{\dd G}{\dd z}\right) + \cdots \\
  - (P + aS')\, \delta\left(-\frac{\dd F}{\dd t} - \frac{\dd \Phi}{\dd x}\right) - \cdots\displaybreak[0] \\
  = \left\{-\frac{\dd}{\dd t} (P + aS')
  - \frac{\dd}{\dd z} (\beta - YS')
  + \frac{\dd}{\dd y} (\gamma - ZS')\right\} \delta F + \cdots \\
  + \left\{\frac{\dd}{\dd x} (P + aS')
  + \frac{\dd}{\dd y} (Q + bS')
  + \frac{\dd}{\dd z} (R + cS')\right\} \delta(-\Phi),
\end{multline*}
rejecting a complete differential. Equating the coefficients to the charge\hyp{}and\hyp{}current
vector $(\sigma_{x}, \sigma_{y}, \sigma_{z}, \rho)$ we have
\begin{gather*}
  \sigma_{x} + \frac{\dd}{\dd t} (P + aS')
  = \frac{\dd}{\dd y} (\gamma - ZS') - \frac{\dd}{\dd z} (\beta - YS'), \\
  \rho = \frac{\dd}{\dd x} (P + aS')
  + \frac{\dd}{\dd y} (Q + bS')
  + \frac{\dd}{\dd z} (R + cS').
\end{gather*}
These reduce to the classical form
\[
\left.
\begin{gathered}
  \frac{\dd\gamma}{\dd y} - \frac{\dd\beta}{\dd z} = \frac{\dd P}{\dd t} + \sigma'\\
  \frac{\dd P}{\dd x} + \frac{\dd Q}{\dd y} + \frac{\dd R}{\dd z} = \rho'
\end{gathered}
\right\}
\Tag{(101.5)}
\]
provided that
\[
\left.
\begin{gathered}
  \sigma_{x}' = \sigma_{x}
  + \frac{\dd(aS')}{\dd t} + \frac{\dd(ZS')}{\dd y} - \frac{\dd(YS')}{\dd z}\\
  \rho' = \rho - \frac{\dd(aS')}{\dd x} - \frac{\dd(bS')}{\dd y} - \frac{\dd(cS')}{\dd z}
\end{gathered}
\right\}
\Tag{(101.6)}
\]
These at once reduce to
\[
\left.
\begin{gathered}
  \sigma_{x}' = \sigma_{x}
  + a\, \frac{\dd S'}{\dd t} + Z\, \frac{\dd S'}{\dd y} - Y\, \frac{\dd S'}{\dd z}\\
  \rho' = \rho - a\, \frac{\dd S'}{\dd x} - b\, \frac{\dd S'}{\dd y} - c\, \frac{\dd S'}{\dd z}
\end{gathered}
\right\}
\Tag{(101.7)}
\]
The effect of the second\hyp{}order terms is thus to make the aether appear to
\index{Maxwell's equations!second order corrections to}%
have a specific inductive capacity and permeability given by~\Eq{(101.41)} and
also to introduce a spurious charge and current given by~\Eq{(101.7)}.

This revision makes no difference whatever to the propagation of light.
Since $\sqrt{\mu\Kappa}$ is always unity, the velocity of propagation is unaltered; and no
spurious charge or current is produced because $S'$~vanishes when the magnetic
and electric forces are at right angles.

It would be interesting if all electric charges could be produced in this
way by the second\hyp{}order terms of the pure field equations, so that there would
be no need to introduce the extraneous charge and current $(\sigma_{x}, \sigma_{y}, \sigma_{z}, \rho)$.
I think, however, that this is scarcely possible. The total spurious charge in
a three\hyp{}dimensional region is equal to
\[
\iiint (\rho' - \rho)\, dx\, dy\, dz
= -\iint B_{n}S'\, dS\quad\text{by~\Eq{(101.6)},}
\]
where $B_{n}$~is the normal magnetic induction across the boundary. This requires
that $B_{n}S'$ in the field of an electron falls off only as the inverse square. It is
scarcely likely that the electron has the distant magnetic effects that are implied.

It is readily verified that the spurious charge is conserved independently
of the true charge.

It has seemed worth while to show in some detail the kind of amendment
to Maxwell's laws which may result from further progress of theory. Perhaps
the chief interest lies in the way in which the propagation of electromagnetic
waves is preserved entirely unchanged. But the present proposals are not
intended to be definitive.

\Section{102.}{Numerical values}

Our electromagnetic quantities have been expressed in terms of some
absolute unit whose relation to the \CGS\ system has hitherto been unknown.
It seems probable that we are now in a position to make this unit more
definite because we have found expressions believed to be physically significant
in which the whole energy\hyp{}tensor and electromagnetic energy\hyp{}tensor
occur in unforced combination. Thus according to~\Eq{(100.6)} Weyl's constant~$\alpha$
in \SecRef{90} is~$4$, so that $\beta = 1/2\lambda$. Accordingly in~\Eq{(90.51)} we have the combination
\[
8\pi T^{\mu\nu} - \frac{1}{\lambda} E^{\mu\nu},
\]
which can scarcely be significant unless it represents the difference of the two
tensors reduced to a common unit. It appears therefore that in an electromagnetic
field we must have
\[
E^{\mu\nu} = 8\pi\lambda T^{\mu\nu}
= -\lambda\bigl\{G^{\mu\nu} - \tfrac{1}{2} g^{\mu\nu}(G - 2\lambda)\bigr\},
\]
where $E^{\mu\nu}$~is expressed in terms of the natural unit involved in~$F_{\mu\nu}$. The
underlying hypothesis is that in~$\Star{G}_{\mu\nu}$ the metrical and electrical variables
occur in their natural combination.

The constant~$\lambda$, which determines the radius of curvature of the world, is
unknown; but since our knowledge of the stellar universe extends nearly to
$10^{25}$~cm., we shall adopt
\[
\lambda = 10^{-50} \text{ cm.}^{-2}.
\]
It may be much smaller.

Consider an electrostatic field of $1500$~volts per~cm., or $5$~electrostatic units.
The density of the energy is $5^2/8\pi$ or practically $1$~erg per cubic~cm. The
mass is obtained by dividing by the square of the velocity of light, viz.\
$1.1 \cdot 10^{-21}$~gm. We transform this into gravitational units by remembering
that the sun's mass, $1.99 \cdot 10^{33}$~gm., is equivalent to $1.47 \cdot 10^{5}$~cm. Hence we
find---

%[**TN: Block indent in the original]
The gravitational mass\hyp{}density~$T_{4}^{4}$ of an electric field of $1500$~volts
per~cm.\ is $8.4 \cdot 10^{-50}$~cm.\ per~c.c.

According to the equation $E^{\mu\nu} = 8\pi\lambda T^{\mu\nu}$ we shall have
\[
E_{4}^{4} = 2.1 \cdot 10^{-98} \text{ cm.}^{-4}.
\]

For an electrostatic field along the axis of~$x$ in Galilean coordinates we
have
\[
E_{4}^{4} = \tfrac{1}{2} F_{14}^{2},
\]
so that
\[
F_{14} = 2 \cdot 10^{-49}
\]
in terms of the centimetre. The centimetre is not directly concerned as a
gauge since $F_{14}$~is an in\hyp{}tensor; but the coordinates have been taken as
Galilean, and accordingly the centimetre is also the width of the unit mesh.

Hence an electric force of $1500$~volts per~cm.\ is expressed in natural
measure by the number $2 \cdot 10^{-49}$ referred to a Galilean coordinate\hyp{}system with
a centimetre mesh.

Let us take two rods of length~$l$ at a distance $\delta x_{1}$~cm.\ apart and maintain
them at a difference of potential~$\delta\kappa_{4}$ for a time~$\delta x_{4}$ (centimetres). Compare
their lengths at the beginning and end of the experiment. If they are all the
time subject to parallel displacement in space and time there should be a
discrepancy~$\delta l$ between the two comparisons, given by~\Eq{(84.4)}
\begin{align*}
  \frac{\delta l}{l}
  &= \tfrac{1}{2} F_{\mu\nu}\, dS^{\mu\nu} \\
  &= F_{41}\, \delta x_{1}\, \delta x_{4} \\
  &= \frac{\dd\kappa_{4}}{\dd x_{1}}\, \delta x_{1}\, \delta x_{4}
  = \delta\kappa_{4}\, \delta x_{4}.
\end{align*}
For example if our rods are of metre-length and maintained for a year
($1 \text{ light-year} = 10^{18} \text{ cm.}$)\ at a potential difference of $1\frac{1}{2}$~million volts, the
discrepancy is
\begin{align*}
  \delta l &= 10^2 \cdot 2 \cdot 10^{-49} \cdot 10^{3} \cdot 10^{18} \text{ cm.} \\
  &= 2 \cdot 10^{-26} \text{ cm.}
\end{align*}
We have already concluded that the length of a rod is not determined by
parallel displacement; but it would clearly be impossible to detect the discrepancy
experimentally if it were so determined.

The value of~$F_{14}$ depends on the unit mesh of the coordinate\hyp{}system. If
we take a mesh of width $10^{25}$~cm.\ and therefore comparable with the assumed
radius of the world the value must be multiplied by~$10^{50}$ in accordance with
the law of transformation of a covariant tensor. Hence referred to this natural
mesh\hyp{}system the natural unit of electric force is about $75$~volts per cm. The
result rests on our adopted radius of space, and the unit may well be less than
$75$~volts per~cm.\ but can scarcely be larger. It is puzzling to find that the
natural unit is of the size encountered in laboratory experiments; we should
\index{Unit!of action}%
have expected it to be of the order of the intensity at the boundary of an
electron. This difficulty raises some doubt as to whether we are quite on the
right track.

The result may be put in another form which is less open to doubt.
Imagine the whole spherical world filled with an electric field of about $75$~volts
per~cm.\ for the time during which a ray of light travels round the world. The
electromagnetic action is expressed by an invariant which is a pure number
\index{Action, material or gravitational!numerical values of}%
independent of gauge and coordinate systems; and the total amount of action
for this case is of the order of magnitude of the number~$1$. The natural unit
of action is evidently considerably larger than the quantum. With the radius
\index{Quantum!numerical value of}%
\index{Numerical value of quantum}%
of the world here used I find that it is $10^{115}$~quanta.

\Section{103.}{Conclusion}

We may now review the general physical results which have been established
or rendered plausible in the course of our work. The numbers in brackets
refer to the sections in which the points are discussed.

We offer no explanation of the occurrence of electrons or of quanta; but
in other respects the theory appears to cover fairly adequately the phenomena
of physics. The excluded domain forms a large part of modern physics, but it
is one in which all explanation has apparently been baffled hitherto. The
domain here surveyed covers a system of natural laws fairly complete in itself
and detachable from the excluded phenomena, although at one point difficulties
arise since it comes into close contact with the problem of the nature of the
electron.

We have been engaged in \emph{world\hyp{}building}---the construction of a world
which shall operate under the same laws as the natural world around us. The
most fundamental part of the problem falls under two heads, the building\hyp{}material
and the process of building.

\emph{The building\hyp{}material.} There is little satisfaction to the builder in the
mere assemblage of selected material already possessing the properties which
will appear in the finished structure. Our desire is to achieve the purpose with
unselected material. In the game of world\hyp{}building we lose a point whenever
we have to ask for extraordinary material specially prepared for the end in
view. Considering the most general kind of relation\hyp{}structure which we have
been able to imagine---provided always that it is \emph{a structure}---we have found
that there will always exist as building\hyp{}material an in\hyp{}tensor~$\Star{G}_{\mu\nu}$ consisting
of symmetrical and antisymmetrical parts $R_{\mu\nu}$ and~$F_{\mu\nu}$, the latter being the
curl of a vector (\SecNum{97}, \SecNum{98}). This is all that we shall require for the domain of
physics not excluded above.

\emph{The process of building.} Here from the nature of the case it is impossible
to avoid trespassing for a moment beyond the bounds of physics. The world
which we have to build from the crude material is the world of perception, and
the process of building must depend on the nature of the percipient. Many
things may be built out of~$\Star{G}_{\mu\nu}$, but they will only appear in the perceptual
\index{Percipient, determines natural laws by selection}%
world if the percipient is interested in them. We cannot exclude the consideration
of what kind of things are likely to appeal to the percipient. The
building process of the mathematical theory must keep step with that process
by which the mind of the percipient endows with vivid qualities certain
selected structural properties of the world. We have found reason to believe
that this creative action of the mind follows closely the mathematical process
of Hamiltonian differentiation of an invariant~(\SecNum{64}).

In one sense deductive theory is the enemy of experimental physics. The
latter is always striving to settle by crucial tests the nature of the fundamental
things; the former strives to minimise the successes obtained by showing how
wide a nature of things is compatible with all experimental results. We have
called on all the evidence available in an attempt to discover what is the exact
invariant whose Hamiltonian differentiation provides the principal quantities
recognised in physics. It is of great importance to determine it, since on it
depend the formulae for the law of gravitation, the mass, energy, and momentum
and other important quantities. It seems impossible to decide this
question without appeal to a perhaps dubious principle of simplicity; and it
\index{Measurement, principle of}%
\index{Principle!of measurement}%
has seemed a flaw in the argument that we have not been able to exclude
more definitely the complex alternatives~(\SecNum{62}). But is it not rather an unhoped
for success for the deductive theory that all the observed consequences follow
without requiring an arbitrary selection of a particular invariant?

We have shown that the physical things created by Hamiltonian differentiation
\index{Creation of the physical world}%
\index{Hamiltonian derivative!creative aspect of}%
must in virtue of mathematical identities have certain properties. When
the antisymmetric part~$F_{\mu\nu}$ of the in\hyp{}tensor is not taken into account, they
have the property of conservation or permanence; and it is thus that mass,
energy and momentum arise~(\SecNum{61}). When $F_{\mu\nu}$~is included, its modifying effect
on these mechanical phenomena shows that it will manifest itself after the
manner of electric and magnetic force acting respectively on the charge\hyp{}component
and current\hyp{}components of a stream\hyp{}vector~(\SecNum{100}). Thus the part played
by~$F_{\mu\nu}$ in the phenomena becomes assigned.

All relations of space and time are comprised in the in\hyp{}invariant $\Star{G}_{\mu\nu}\, dx_{\mu}\, dx_{\nu}$,
which expresses an absolute relation (the \emph{interval}) between two points with
coordinate differences~$dx_{\mu}$ (\SecNum{97}). To understand why this expresses space and
time, we have to examine the principles of measurement of space and time by
material or optical apparatus~(\SecNum{95}). It is shown that the conventions of measurement
introduce an isotropy and homogeneity into measured space which need
not originally have any counterpart in the relation\hyp{}structure which is being
surveyed. This isotropy and homogeneity is exactly expressed by Einstein's
law of gravitation~(\SecNum{66}).

The transition from the spatio\hyp{}temporal relation of interval to space and
time as a framework of location is made by choosing a coordinate\hyp{}frame such
that the quadratic form $\Star{G}_{\mu\nu}\, dx_{\mu}\, dx_{\nu}$ breaks up into the sum of four squares~(\SecNum{4}).
It is a property of the world, which we have had to leave unexplained, that
the sign of one of these squares is opposite to that of the other three~(\SecNum{9}); the
coordinate so distinguished is called time. Since the resolution into four squares
can be made in many ways, the space-time frame is necessarily indeterminate,
and the Lorentz transformation connecting the spaces and times of different
observers is immediately obtained~(\SecNum{5}). This gives rise to the \emph{special} theory of
relativity. It is a further consequence that there will exist a definite speed
which is absolute~(\SecNum{6}); and disturbances of the tensor~$F_{\mu\nu}$  (electromagnetic
waves) are propagated in vacuum with this speed~(\SecNum{74}). The resolution into
four squares is usually only possible in an infinitesimal region so that a world-wide
frame of space and time as strictly defined does not exist. Latitude is,
however, given by the concession that a space-time frame may be used which
does not fulfil the strict definition, observed discrepancies being then attributed
to a field of force~(\SecNum{16}). Owing to this latitude the space-time frame becomes
entirely indeterminate; any system of coordinates may be described as a frame
of space and time, and no one system can be considered superior since all alike
require a field of force to justify them. Hence arises the \emph{general} theory of
relativity.

The law of gravitation in continuous matter is most directly obtained from
the identification of the energy\hyp{}tensor of matter~(\SecNum{54}), and this gives again the
law for empty space as a particular case. This mode of approach is closely
connected with the previous deduction of the law in empty space from the
isotropic properties introduced by the processes of measurement, since the
components of the energy\hyp{}tensor are identified with coefficients of the quadric
of curvature~(\SecNum{65}). To deduce the field of a particle~(\SecNum{38}) or the motion of a
particle in the field~(\SecNum{56}), we have to postulate symmetrical properties of the
particle (or average particle); but these arise not from the particle itself but
because it provides the standard of symmetry in measurement~(\SecNum{66}). It is then
shown that the Newtonian attraction is accounted for~(\SecNum{39}); as well as the
refinements introduced by Einstein in calculating the perihelion of Mercury~(\SecNum{40})
and the deflection of light~(\SecNum{41}).

It is possible to discuss mechanics without electrodynamics but scarcely
possible to discuss electrodynamics without mechanics. Hence a certain difficulty
arises in our treatment of electricity, because the natural linking of the
two subjects is through the excluded domain of electron\hyp{}structure. In practice
electric and magnetic forces are defined through their mechanical effects on
charges and currents, and these mechanical effects have been investigated in
general terms~(\SecNum{100}) and with particular reference to the electron~(\SecNum{80}). One
half of Maxwell's equations is satisfied because $F_{\mu\nu}$~is the curl of a vector~(\SecNum{92}),
and the other half amounts to the identification of~$F_{\nu}^{\mu\nu}$ with the charge\hyp{}and\hyp{}current
vector~(\SecNum{73}). The electromagnetic energy\hyp{}tensor as deduced is found
to agree in Galilean coordinates with the classical formulae~(\SecNum{77}).

Since a field of force is relative to the frame of space-time which is used,
potential energy can no longer be treated on the same footing with kinetic
energy. It is not represented by a tensor~(\SecNum{59}) and becomes reduced to an
artificial expression appearing in a mathematical mode of treatment which is
no longer regarded as the simplest. Although the importance of ``action'' is
enhanced on account of its invariance, the principle of least action loses in
status since it is incapable of sufficiently wide generalisation (\SecNum{60}, \SecNum{63}).

In order that material bodies may be on a definite scale of size there must
be a curvature of the world in empty space. Whereas the differential equations
governing the form of the world are plainly indicated, the integrated form is
not definitely known since it depends on the unknown density of distribution
of matter. Two forms have been given~(\SecNum{67}), Einstein's involving a large quantity
of matter and de~Sitter's a small quantity~(\SecNum{69}); but whereas in the latter the
quantity of matter is regarded as accidental, in the former it is fixed in accordance
with a definite law~(\SecNum{71}). This law at present seems mysterious, but it is
perhaps not out of keeping with natural anticipations of future developments
of the theory. On the other hand the evidence of the spiral nebulae possibly
favours de~Sitter's form which dispenses with the mysterious law~(\SecNum{70}).

Can the theory of relativity ultimately be extended to account in the same
manner for the phenomena of the excluded domain of physics, to which the
laws of atomicity at present bar the entrance? On the one hand it would
seem an idle exaggeration to claim that the magnificent conception of Einstein
is necessarily the key to all the riddles of the universe; on the other hand we
have no reason to think that all the consequences of this conception have
become apparent in a few short years. It may be that the laws of atomicity
arise only in the presentation of the world to us, according to some extension
of the principles of identification and of measurement. But it is perhaps as
likely that, after the relativity theory has cleared away to the utmost the
superadded laws which arise solely in our mode of apprehension of the world
about us, there will be left an external world developing under specialised
laws of behaviour.

The physicist who explores nature conducts experiments. He handles
material structures, sends rays of light from point to point, marks coincidences,
and performs mathematical operations on the numbers which he obtains. His
result is a physical quantity, which, he believes, stands for something in the
condition of the world. In a sense this is true, for whatever is actually occurring
in the outside world is only accessible to our knowledge in so far as it
helps to determine the results of these experimental operations. But we must
not suppose that a law obeyed by the physical quantity necessarily has its seat
in the world\hyp{}condition which that quantity ``stands for''; its origin may be
disclosed by unravelling the series of operations of which the physical quantity
is the result. Results of measurement are the subject\hyp{}matter of physics; and
the moral of the theory of relativity is that we can only comprehend what the
physical quantities \emph{stand for} if we first comprehend what they \emph{are}.

If we could write down exactly the whole system of equations holding in the physical world,
these equations would themselves contain the definitions of all the quantities occurring in them.
For any definition---any statement as to the mode in which such a quantity is to be recognised and
measured---is itself capable of expression as one of the equations of physics.
If the number of independent equations is not greater than the number of quantities to be defined,
no governing law will be imposed on the substratum of the world, and those laws of nature which we
discover must be implicit in the definition of the quantities which obey them.
Such truisms appear remarkable because in actual experience we deal with mental images associated with
the physical quantities by processes which physics is unable to explore.
We shall assume, however, that the number of independent equations~$n$ exceeds the number of definitions~$m$.
In that case we can divide the equations into~$m$ equations of \emph{definition} and~$n-m$ equations of \emph{control}.

It is generally admitted that we are not in a position to formulate the real equations of control which belong
to the domain of electron structure and quantum activity, and that a radical alteration of our methods is needed
to deal with these successfully.
The field\hyp{}theory (which we have seen in~\SecRef{76} to be essentially macroscopic) skirts round the excluded domain
without however entirely avoiding it.
We find that the number of field\hyp{}equations exceeds the required number of definitions by one, so that one
equation of control is needed which (however disguised by macroscopic averaging) must arise in the
forbidden territory.
This equation of control in the present development is taken bo be the law of ponderomotive force of the
electromagnetic field.
(The choice of a particular equation is arbitrary, e.g.\ in elementary electrostatics the law of ponderomotive
force is the definition of electric charge, and some other equation then becomes the equation of control.)

Having satisfied ourselves that we have rightly discovered the~$m$ equations of definition and the single equation
of control it would appear that our task is at an end---until the excluded domain can be entered.
The task is left in an untidy state, but that is perhaps inevitable since it is admittedly unfinished.
Nevertheless a certain amount of preliminary tidying up is possible.
We can, of course, define anything we like; but in practice we define only certain things which have a
particular prominence---an importance from a certain aspect.
It is certainly a step forward to recognise that there is a uniformity in the aspect here referred to---that
``prominence in our survey of the world'' is a definite character capable of mathematical specification.
This introduces an orderliness into our equations of definition.
Probably the secret of this prominence lies in the excluded domain with its unknown equations of control;
or we may connect it with our faculties of perception, which perhaps comes ultimately to the same thing.
In the field\hyp{}theory we must be content with discovering the formal principle (of Hamiltonian differentiation)
and noting its unifying effect on our equations of definition.

We have divided physics into two parts, one of which we are competent to deal with by our methods of continuous
analysis and the other we are not competent to deal with.
We do not pretend to predict \Foreign{a priori} how nature will behave, but it is not impossible to set limits
to the behaviour of that part of nature with which we are competent to deal, if we know the limits of that competence.
So that provided that there exists a domain or structure (viz.\ the macroscopic field) which satisfies this condition,
provided also that we have a criterion of ``prominence'' for selecting the quantities important to study,
we may be able to predict the field properties.
That is what the affine theory in Chapter VII, Part II attempts to do, and it is I think partially successful.
The idea is that the affine connection is the most general structure coming within the scope of continuous analysis
(a contention not fully demonstrated) and may therefore be used as a basis of prediction.

The single law of control cannot be predicted \Foreign{a priori,} since it arises in the excluded region and is not
limited by the competence of particular mathematical analysis.
We might find that, without attempting to trace its microscopic origin, we could give it a simple formal expression
in macroscopic terms; and it is the object of the action\hyp{}theories of Weyl (\SecRef{90}) and Einstein (\SecRef{100})
to reduce this expression to the simplest form.
One difficulty is that any law of stationary action seems to lead necessarily to two equations of control,
corresponding to symmetrical and antisymmetrical components and the superfluous equation~\Eq{(90.71)}
has to be explained away as outside practical observation.
It appears to the author more profitable, instead of seeking a purely formal expression of the law of control,
to make a slight inroad into the excluded domain; then the law is seen to arise from a simple limitation of electron
structure, viz.\ that, a certain integral property, known to hold in the absence of an external field, is preserved
in all cases (\SecRef{80}).
The action\hyp{}principle is no doubt attractive in that it makes the field\hyp{}theory formally complete in itself without
reference to an excluded domain, but the attraction is somewhat dimmed by our knowledge that the completeness cannot
be more than formal.

\Matter{Bibliography}

The following is the classical series of papers leading up to the present state of the theory.

\Bibitem{B. Riemann.} Über die Hypothesen, welche der Geometrie zu Grunde liegen, \Title{Abhanglungen
d.\ K. Gesells.\ zu Göttingen,} 13, p.~133 (Habilitationsschrift, 1854).

\Bibitem{H. A. Lorentz.} Versuch einer Theorie der elektrischen und optischen Erscheinungen in
bewegten Körpern. (Leiden, 1895.)

\Bibitem{J. Larmor.} Aether and Matter, Chap.~\Vol{XI}. (Cambridge, 1900.)

\Bibitem{H. A. Lorentz.} Electromagnetic phenomena in a system moving with any velocity smaller
than that of light, \Title{Proc.\ Amsterdam Acad.,} 6, p.~809 (1904).

\Bibitem{A. Einstein.} Zur Electrodynamik bewegter Körpern, \Title{Ann.\ d.\ Physik,} 17, p.~891 (1905).

\Bibitem{H. Minkowski.} Raum und Zeit. (Lecture at Cologne, 21~September, 1908.)

\Bibitem{A. Einstein.} Über den Einfluss der Schwerkraft auf die Ausbreitung des Lichtes, \Title{Ann.\ d.\
Physik,} 35, p.~898 (1911).

\Bibitem{A. Einstein.} Die Grundlage der allgemeinen Relativitätstheorie, \Title{Ann.\ d.\ Physik,} 49, p.~769
(1916).

\Bibitem{A. Einstein.} Kosmologische Betrachtungen zur allgemeinen Relativitätstheorie, \Title{Berlin.\
Sitzungsberichte,} 1917, p.~142.

\Bibitem{H. Weyl.} Gravitation und Elektrizität, \Title{Berlin.\ Sitzungsberichte,} 1918, p.~465.
\medskip

The pure differential geometry used in the theory is based on
\Bibitem{M. M. G. Ricci {\upshape and} T. Levi\hyp{}Civita.} Méthodes de calcul différential absolu et leurs
applications, \Title{Math.\ Ann.,} 54, p.~125 (1901).

\Bibitem{T. Levi\hyp{}Civita.} Nozione di parallelismo in una varietà qualunque, \Title{Rend.\ del Circ.\ Mat.\ di
Palermo,} 42, p.~173 (1917).
\medskip

A useful review of the subject is given by J.~E. Wright, Invariants of Quadratic Differential
Forms (Camb.\ Math.\ Tracts, No.~9). Reference may also be made to J.~Knoblauch,
Differentialgeometrie (Teubner, Leipzig). W.~Blaschke, Vorlesungen über Differentialgeometrie
(Springer, Berlin), promises a comprehensive treatment with special reference
to Einstein's applications; but only the first volume has yet appeared.
\medskip

From the very numerous papers and books on the Theory of Relativity, I select the
following as most likely to be helpful on particular points, or as of importance in the historic
development. Where possible the subject\hyp{}matter is indicated by references to the sections
in this book chiefly concerned.

\Bibitem{E. Cunningham.} Relativity and the Electron Theory (Longmans, 1921). (Particularly
full treatment of experimental foundations of the theory.)

\Bibitem{T. de Donder.} La Gravifique Einsteinienne, \Title{Ann.\ de l'Obs.\ Royal de Belgique,} 3rd Ser., 1,
p.~75 (1921).

(Recommended as an example of treatment differing widely from that here chosen, but
with equivalent conclusions. See especially his electromagnetic theory in Chap~\Vol{V}.)

\Bibitem{J. Dröste.} The Field of $n$~moving centres on Einstein's Theory, \Title{Proc.\ Amsterdam Acad.,}
19, p.~447 (1916), \SecRef{44}.

\Bibitem{A. S. Eddington.} A generalisation of Weyl's Theory of the Electromagnetic and Gravitational
Fields, \Title{Proc.\ Roy.\ Soc.,} A~99, p.~104 (1921). \SecRefs{91}--\SecNum{97}.

\Bibitem{A. Einstein.} Über Gravitationswellen, \Title{Berlin.\ Sitzungsberichte,} 1918, p.~154, \SecRef{57}.

\Bibitem{L. P. Eisenhart {\upshape and} O. Veblen.} The Riemann Geometry and its Generalisation, \Title{Proc.\
Nat.\ Acad.\ Sci.,} 8, p.~19 (1922). \SecRefs{84},~\SecNum{91}.

\Bibitem{A. D. Fokker.} The Geodesic Precession; a consequence of Einstein's Theory of Gravitation,
\Title{Proc.\ Amsterdam Acad.,} 23, p.~729 (1921). \SecRef{44}.

\Bibitem{A. E. Harward.} The Identical Relations in Einstein's Theory, \Title{Phil.\ Mag.,} 44, p.~380
(1922). \SecRef{52}.

\Bibitem{G. Herglotz.} Über die Mechanik des deformierbaren Körpers vom Standpunkte der
Relativitätstheorie, \Title{Ann.\ d.\ Physik,} 36, p.~493 (1911). \SecRef{53}.

\Bibitem{D. Hilbert.} Die Grundlagen der Physik, \Title{Göttingen Nachrichten,} 1915, p.~395; 1917,
p.~53. \SecRef{61}.

\Bibitem{G. B. Jeffery.} The Field of an Electron on Einstein's Theory of Gravitation, \Title{Proc.\ Roy.\
Soc.,} A~99, p.~123 (1921). \SecRef{78}.

\Bibitem{E. Kasner.} Finite representation of the Solar Gravitational Field in Flat Space of Six
Dimensions, \Title{Amer.\ Journ.\ Mathematics,} 43, p.~130 (1921). \SecRef{65}.

\Bibitem{F. Klein.} Über die Integralform der Erhaltungssätze und die Theorie der räumlichgeschlossenen
Welt, \Title{Göttingen Nachrichten,} 1918, p.~394. \SecRefs{67},~\SecNum{70}.

\Bibitem{J. Larmor.} Questions in physical indetermination, \Title{C. R. du Congrès International, Strasbourg} (1920).

\Bibitem{T. Levi\hyp{}Civita.} Statica Einsteiniana, \Title{Rend.\ dei Lincei,} 26(1), p.~458 (1917).

\Bibitem{\Same} $ds^2$ Einsteiniani in Campi Newtoniani, \Title{Rend.\ dei Lincei,} 26(2), p.~307; 27(1), p.~3;
27(2), p.~183. etc. (1917--19).

\Bibitem{H. A. Lorentz.} On Einstein's Theory of Gravitation, \Title{Proc.\ Amsterdam Acad.,} 19, p.~1341,
20, p.~2 (1916).

\Bibitem{G. Mie.} Grundlagen einer Theorie der Materie, \Title{Ann.\ d.\ Physik,} 37, p.~511; 39, p.~1; 40,
p.~1 (1912--13).

\Bibitem{G. Nordström.} On the Energy of the Gravitational Field in Einstein's Theory, \Title{Proc.\
Amsterdam Acad.,} 20, p.~1238 (1918). \SecRefs{43}, \SecNum{59},~\SecNum{78}.

\Bibitem{\Same} Calculation of some special cases in Einstein's Theory of Gravitation, \Title{Proc.\ Amsterdam
Acad.,} 21, p.~68 (1918). \SecRef{72}.

\Bibitem{A. A. Robb.} A Theory of Time and Space (Cambridge, 1914). \SecRef{98}.

\Bibitem{J. A. Schouten.} Die direkte Analysis zur neueren Relativitätstheorie, \Title{Verhandelingen
Amsterdam Acad.,} 12 (1919).

\Bibitem{\Same} On the arising of a Precession\hyp{}motion owing to the non\hyp{}Euclidean linear element,
\Title{Proc.\ Amsterdam Acad.,} 21, p.~533 (1918). \SecRef{44}.

\Bibitem{K. Schwarzschild.} Über das Gravitationsfeld eines Massenpunktes nach der Einsteinschen
Theorie, \Title{Berlin.\ Sitzungsberichte,} 1916, p.~189. \SecRef{38}.

\Bibitem{\Same} Über das Gravitationsfeld einer Kugel aus incompressibler Flüssigkeit, \Title{Berlin.\
Sitzungsberichte,} 1916, p.~424. \SecRef{72}.

\Bibitem{W. de Sitter.} On Einstein's Theory of Gravitation and its Astronomical Consequences,
\Title{Monthly Notices, R.A.S.,} 76, p.~699; 77, p.~155; 78, p.~3 (1916--17). \SecRefs{44}, \SecNum{67}--\SecNum{70}.

\Bibitem{H. Weyl.} Über die physikalischen Grundlagen der erweiterten Relativitätstheorie, \Title{Phys.\
Zeits.,} 22, p. 473 (1921).

\Bibitem{\Same} Feld und Materie, \Title{Ann.\ d.\ Physik,} 65, p. 541 (1921).

\Bibitem{\Same} Die Einzigartigkeit der Pythagoreischen Massbestimmung, \Title{Math.\ Zeits.,} 12, p.~114
(1922). \SecRef{97}.
\medskip

On two of these papers received whilst this book was in the press I may specially
comment. Harward's paper contains a direct proof of ``the four identities'' more elegant
than my proof in \SecRef{52}. The paper of Eisenhart and Veblen suggests that, instead of basing
the geometry of a continuum on Levi\hyp{}Civita's conception of parallel displacement, we should
base it on a specification of continuous tracks in all directions. This leads to Weyl's geometry
(\emph{not} the generalisation of Chap.~\Vol{VII}, Part~\Vol{II}). This would seem to be the most logical mode
of approach to Weyl's theory, revealing clearly that it is essentially an analysis of physical
phenomena as related to a reference\hyp{}frame consisting of the tracks of moving particles and
light\hyp{}pulses---two of the most universal methods of practical exploration. Einstein's theory
on the other hand is an analysis of the phenomena as related to a metrical frame marked
out by transport of material objects. In both theories the phenomena are studied in relation
to certain experimental avenues of exploration; but the possible existence of such means
of exploration, being (directly or indirectly) a fundamental postulate of these theories,
cannot be further elucidated by them. It is here that the generalised theory of \SecRef{91} adds
its contribution, showing that the most general type of relation\hyp{}structure yet formulated
will necessarily contain within itself both Einstein's metric and Weyl's track\hyp{}framework

\printindex
\end{document}
